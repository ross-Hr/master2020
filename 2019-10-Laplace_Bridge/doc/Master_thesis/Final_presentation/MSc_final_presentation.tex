% \documentclass[10pt,usepdftitle=false,aspectratio=169]{beamer}
\documentclass[10pt,usepdftitle=false,aspectratio=169,handout]{beamer}
\usepackage{subfig}
\usepackage{graphicx}
%!TEX root = talk.tex

% serif fonts for mathematics
\usefonttheme[onlymath]{serif}

\usepackage{animate}
\usepackage{textcase,marvosym,cancel,booktabs,colortbl,nicefrac,ifsym}
\usepackage{fourier-orns}
\usepackage{amsmath,amssymb,graphicx,colonequals,mathtools} % MnSymbol
\usepackage{tikz,pgfplots,pgfplotstable,standalone}
\usepackage{booktabs,multirow,multicol}
\usepackage{pifont} % for the dingbats in the line search plots
%\usepackage{subcaption}
% \usepackage{stix} % for \sumint

\usepackage{multido}


%% not used with xetex
\usepackage{microtype}
% \usepackage[utf8]{inputenc}
\usepackage[utf8x]{inputenc} 

%%%% fiddling with lists:
% \usepackage{enumitem}
% try: \begin{description}[leftmargin=\parindent,labelindent=\parindent] ...
\let\olditemize\itemize
\renewcommand\itemize{\olditemize\addtolength{\itemsep}{0.3\baselineskip}}%
\let\oldenumerate\enumerate
\renewcommand\enumerate{\oldenumerate\addtolength{\itemsep}{0.3\baselineskip}}%



% somehow the minus causes trouble
\DeclareUnicodeCharacter{2212}{-}

% a few colours, for general use.
\definecolor{lred}{RGB}{200,0,0}
\definecolor{dred}{RGB}{130,0,0} 
\definecolor{dblu}{RGB}{0,0,130}
\definecolor{dgre}{RGB}{0,130,0} 
\definecolor{dgra}{RGB}{50,50,50}
\definecolor{mgra}{RGB}{221,222,214}
\definecolor{lgra}{RGB}{238,238,234}
\definecolor{MPG}{RGB}{000,125,122}
\definecolor{lMPG}{RGB}{000,190,189}
\definecolor{ora}{HTML}{FF9933} %EI orange
\definecolor{lblu}{HTML}{7DA7D9}%PS blue

% Color scheme of the Eberhard-Karls University
\definecolor{TUred}{RGB}{141,45,57}
\definecolor{TUdark}{RGB}{55,65,74}
\definecolor{TUgold}{RGB}{174,159,109}
\definecolor{TUgray}{RGB}{175,179,183}

% IMPRS color scheme
\definecolor{imprs}{RGB}{35, 127, 154} % IMPRS blue
\definecolor{imprsgra}{RGB}{153, 153, 153} % IMPRS gray

\definecolor{ERC_ora}{RGB}{233,93,15}

\setlength{\parindent}{0pt}

\newcommand{\mpg}[1]{{\color{MPG} #1}}   % highlight command 1
\newcommand{\dre}[1]{{\color{TUred} #1}}   % highlight command 1
\newcommand{\blu}[1]{{\color{dblu} #1}}   % highlight command 1
\newcommand{\ora}[1]{{\color{ora} #1}}   % highlight command 1
\newcommand{\gra}[1]{{\color{mgra} #1}}   % highlight command 1
\newcommand{\gold}[1]{{\color{TUgold} #1}}   % highlight command 1
\newcommand{\imprs}[1]{{\color{imprs} #1}} 
\newcommand{\imprsgra}[1]{{\color{imprsgra} #1}} 
\setbeamercolor{alerted text}{fg = TUred} % highlight command 2

\setbeamercolor{normal text}{fg=black,bg=white}
\setbeamercolor{structure}{fg=TUred}

\setbeamercolor{item projected}{use=item,fg=black,bg = TUred}

\setbeamercolor*{palette primary}{fg=white,bg=TUred}
\setbeamercolor*{palette secondary}{parent=palette primary,use=palette
primary,bg=dblu} 
\setbeamercolor*{palette tertiary}{parent=palette
primary,use=palette primary,fg=white,bg=dgre} 
\setbeamercolor*{palette
quaternary}{parent=palette primary,use=palette primary,bg=dgre}

\setbeamercolor*{block body}{bg=TUgray!80, fg =black}
\setbeamercolor*{block title}{parent=structure,bg=TUdark,fg=white}
\setbeamercolor{block body alerted}{bg=TUgray!90,fg=TUred}
\setbeamercolor{block title example}{fg=MPG,bg=white}
\setbeamercolor{block body example}{fg=black,bg=white}

\setbeamercolor{frametitle}{bg=TUdark,fg=white}
\setbeamercolor{frametitle right}{bg=white}
\setbeamercolor{framesubtitle}{fg=TUgray}

% \setbeamerfont{framesubtitle}{size*={8}{12}}

\setbeamercolor{title}{fg=TUred}
\setbeamercolor{subtitle}{fg=black} \setbeamercolor{author}{fg=TUdark}
\setbeamercolor{date}{fg=TUdark}

\setbeamercolor*{titlelike}{parent=structure}

\setbeamertemplate{navigation symbols}{}
\setbeamertemplate{bibliography item}[triangle]

\setbeamerfont{frametitle}{}

%%%% REDEFINE \emph to COLORS.
\let\emph\relax % there's no \RedeclareTextFontCommand
\DeclareTextFontCommand{\emph}{\color{TUred}\bfseries} % changed from \em

% \setbeamertemplate{itemize items}{\color{TUgold}$\blacktriangleright$}
\setbeamertemplate{itemize items}{\starredbullet}

\setbeamertemplate{footline}{\hfill\color{TUdark}{\insertframenumber}\hspace{2ex}\null\newline\vspace{2mm}}

\arrayrulecolor{TUdark}

\newcommand{\filltotal}{\hspace{0pt plus 1 filll}}

\newcommand{\titlemark}[1]{
  \begin{tikzpicture}[remember picture, overlay]
    \node[draw=none,text=TUgray,anchor=north east,yshift=-.8259cm] at (current
    page.north east) {\footnotesize{#1}};
  \end{tikzpicture}
}

\newcommand{\graybox}[1]{
  \tikzexternaldisable%
  \begin{center}%
    \tikz{\node[fill=TUdark,text width=\textwidth]{%
        \begin{minipage}{1.0\linewidth}%
        \color{white}
          \begin{center}%
            #1
          \end{center}%
        \end{minipage}};}%
  \end{center}%
  \tikzexternalenable 
}

\newcommand{\lightgraybox}[1]{
  \tikzexternaldisable%
  \begin{center}%
    \tikz{\node[fill=TUgray!80,text width=\textwidth]{%
        \begin{minipage}{1.0\linewidth}%
%         \color{white}
          \begin{center}%
            #1
          \end{center}%
        \end{minipage}};}%
  \end{center}%
  \tikzexternalenable 
}

\newcommand{\divider}{\noindent\makebox[\linewidth]{\rule{\paperwidth}{.4pt}}  }

\newcommand{\paperwhite}{\includegraphics[height=.6\baselineskip]{../text-document_white.png}\hspace{.5em}}
\newcommand{\paperblack}{\includegraphics[height=.6\baselineskip]{../text-document.png}\hspace{.5em}}

\newcommand{\bookwhite}{\includegraphics[height=.6\baselineskip]{../book-white.png}\hspace{.5em}}
\newcommand{\bookblack}{\includegraphics[height=.6\baselineskip]{../book-black.png}\hspace{.5em}}

\setbeamercolor{ribboncolor}{fg=black,bg=lgra}
\newcommand{\ribbon}[1]{
    \begin{beamercolorbox}[wd=\paperwidth,colsep*=.3em,center]{ribboncolor}
    \setbeamertemplate{itemize items}{\color{black}\starredbullet}
    \setbeamercolor{structure}{fg=black}
    \begin{minipage}{1.0\textwidth}%
            #1
        \end{minipage}
    \end{beamercolorbox}
}
\setbeamercolor{whiteribboncolor}{fg=black,bg=white}
\newcommand{\whiteribbon}[1]{
    \begin{beamercolorbox}[wd=\paperwidth,colsep*=.5em,center,text width=\textwidth]{whiteribboncolor}
    \setbeamertemplate{itemize items}{\color{black}\starredbullet}
    \setbeamercolor{structure}{fg=black}
    \begin{minipage}{\textwidth}%
            #1
        \end{minipage}
    \end{beamercolorbox}
}

\newcommand{\bfa}[1]{
  \begin{beamercolorbox}[wd=\paperwidth,colsep*=.5em,center]{white}
    \setbeamertemplate{itemize items}{\color{black}\starredbullet}
    \tikzexternaldisable
    \begin{tikzpicture}
    \node[signal,minimum width=\paperwidth,draw=lgra,fill=lgra,text=black,text width=\textwidth]{
    \begin{minipage}{\textwidth}%
            #1
        \end{minipage}};
    \end{tikzpicture}
    \tikzexternalenable
    \end{beamercolorbox}
}

\newcommand{\bfi}[1]{
  \begin{beamercolorbox}[wd=\paperwidth,colsep*=.5em,center,text width=\textwidth]{white}
    \setbeamertemplate{itemize items}{\color{black}\starredbullet}
    \tikzexternaldisable
    \begin{tikzpicture}
    \node[signal,signal from=west,signal to=nowhere,minimum width=\linewidth,draw=TUgold,fill=TUgold,text=black,signal pointer angle=140]{
    \begin{minipage}{\textwidth}%
            #1
        \end{minipage}};
    \end{tikzpicture}\hspace{2.5mm}
    \tikzexternalenable
    \end{beamercolorbox}
}

\newcommand{\blackslide}{
{\setbeamercolor{background canvas}{bg=black}
  \begin{frame}[plain]
    \null
  \end{frame}
}}

\newcommand{\blackslidetext}[1]{
{\setbeamercolor{background canvas}{bg=TUdark}%
  \setbeamercolor{item}{fg=white}%
  \setbeamercolor{structure}{fg=white}%
  \setbeamercolor{normal text}{fg=white}%
  \setbeamercolor{body}{fg=white}%
  \setbeamertemplate{itemize items}{\color{white}\starredbullet}%
  \begin{frame}
    \color{white} #1
  \end{frame}
}}


%%%% FONTS

% \usepackage{inconsolata}
% \usepackage[defaultsans]{droidsans}
% \usepackage[sfdefault,light,condensed]{roboto}
% \renewcommand*\ttdefault{cmvtt} % use with roboto condensed
% \usepackage[sfdefault]{universalis}
% \usepackage[default]{lato}
% \usepackage{lmodern}
% \usepackage{paratype}
% \usepackage{cmbright}
% \usepackage[default,osfigures,scale=0.95]{opensans}
% \usepackage{sfmath}
% \renewcommand*\familydefault{\sfdefault}
\usepackage[T1]{fontenc}

% special \vneq that works with sfmath
\makeatletter
\newcommand*{\vneq}{%
  \mathrel{%
    \mathpalette\@vneq{=}%
  }%
}
\newcommand*{\@vneq}[2]{%
  % #1: math style (\displaystyle, \textstyle, ...)
  % #2: symbol (=, ...)
  \sbox0{\raisebox{\depth}{$#1\neq$}}%
  \sbox2{\raisebox{\depth}{$#1|\m@th$}}%
  \ifdim\ht2>\ht0 %
    \sbox2{\resizebox{\vneqxscale\width}{\vneqyscale\ht0}{\unhbox2}}%
  \fi
  \sbox2{$\m@th#1\vcenter{\copy2}$}%
  \ooalign{%
    \hfil\phantom{\copy2}\hfil\cr
    \hfil$#1#2\m@th$\hfil\cr
    \hfil\copy2\hfil\cr
  }%
}
\newcommand*{\vneqxscale}{1}
\newcommand*{\vneqyscale}{.8}
\makeatother

%tables
\newcolumntype{C}[1]{>{\centering\arraybackslash}p{#1}}



% logo in the upper right corner
% \usepackage{eso-pic}
% \newcommand\AtPagemyUpperLeft[1]{\AtPageLowerLeft{%
% \put(\LenToUnit{0.83\paperwidth},\LenToUnit{0.915\paperheight}){#1}}}
% \AddToShipoutPictureFG{
%   \AtPagemyUpperLeft{{\includegraphics[width=2.5cm,keepaspectratio]{assets/UT_WBMW_Weiss_1C.pdf}}}
% }%

%%%% REDEFINE \emph to COLORS.
\let\emph\relax % there's no \RedeclareTextFontCommand
\DeclareTextFontCommand{\emph}{\color{TUred}\bfseries} % } changed from \em
\DeclareTextFontCommand{\emphgold}{\color{TUgold}\bfseries} % } changed from \em


% Paper-specific definitions
\newcommand{\gp}{\text{\textsc{gp}}}
\newcommand{\ep}{\text{\textsc{ep}}}
\newcommand{\ess}{\text{\sc ess}} % Elliptical slice sampling
\newcommand{\liness}{\text{\sc lin-ess}} % Elliptical slice sampling
\newcommand{\mcmc}{\text{\sc mcmc}} % MCMC
\newcommand{\hdr}{\text{\sc hdr}} % Holmes Diaconis Ross
\newcommand{\dingens}{\text{\sc dingens}} % Holmes Diaconis Ross
\newcommand{\suppmat}{supp.~mat.}


\newcommand{\ie}{i.e.,} % 
\newcommand{\eg}{e.g.,} % 
\newcommand{\g}{\,|\,} 
\newcommand{\de}{\partial}
\newcommand{\e}{\operatorname{e}}
\renewcommand{\d}{\:d}  %sometimes overwritten, watch out!
\newcommand{\eps}{\epsilon}
\newcommand{\dd}{\:\mathrm{d}}
\newcommand{\ddd}{\mathsf{d}}
\newcommand{\dx}{\,\mathrm{d}\mathbf{x}}
\newcommand{\DD}{\mathsf{D}}
\newcommand{\Exp}{\mathbb{E}}
\newcommand{\Ent}{\mathbb{H}}
\newcommand{\Cov}{\operatorname{cov}} 
\newcommand{\cov}{\operatorname{cov}} 
\newcommand{\erf}{\operatorname{erf}} 
\newcommand{\F}{\mathcal{F}}
\renewcommand{\H}{\mathcal{H}}
\newcommand{\K}{\mathcal{K}} 
\newcommand{\KL}{\text{KL}} 
\renewcommand{\Re}{\mathbb{R}}
\newcommand{\Co}{\mathbb{C}}
\newcommand{\one}{\mathbf{1}}
\newcommand{\indicator}{\mathbb{I}}
\newcommand{\const}{\text{const.}}
\newcommand{\diag}{\operatorname{diag}}
\newcommand{\Dcal}{\mathcal{D}} 
\newcommand{\N}{\mathcal{N}} 
\newcommand{\W}{\mathcal{W}} 
\newcommand{\cS}{\mathcal{S}} 
\newcommand{\sT}{\mathsf{T}} 
\newcommand{\B}{\mathcal{B}} 
\newcommand{\C}{\mathcal{C}}
\renewcommand{\L}{\mathcal{L}}
\newcommand{\inv}{^{-1}} 
\newcommand{\Trans}{^{\intercal}} 
\newcommand{\argmin}{\operatorname*{arg\:min}}
\newcommand{\argmax}{\operatorname*{arg\:max}}
\newcommand{\eq}[1]{\stackrel{#1}{=}}
\newcommand{\spa}{\operatorname{span}} 
\newcommand{\ce}{\colonequals}
\newcommand{\ec}{\equalscolon}
\renewcommand{\det}{\operatorname{det}}
\newcommand{\var}{\operatorname{var}}
\newcommand{\V}{\mathbb{V}}
\renewcommand{\=}{\operatorname*{=}}
\newcommand{\myexp}[1]{\exp{\left[ #1 \right] }}
\newcommand{\expmap}{\mathrm{Exp}}
\newcommand{\logmap}{\mathrm{Log}}
\newcommand{\dbyd}[2]{\frac{\operatorname{d}{#1}}{\operatorname{d}{#2}}}

\newcommand{\q}{\quad}
\newcommand{\qq}{\qquad}
\newcommand{\qqq}{\quad\qquad}
\newcommand{\qqqq}{\qquad\qquad}

\renewcommand{\vec}{\boldsymbol} 
\newcommand{\mb}{\mathbf} 
\newcommand{\vect}[1]{\overrightarrow{#1}}
\newcommand{\vest}[1]{\overrightharpoon{#1}}
\newcommand{\fun}[1]{\mathsf{#1}}
\newcommand{\logm}{\operatorname{Log}}
\newcommand{\expm}{\operatorname{Exp}}
\renewcommand{\O}{\mathcal{O}} 
\newcommand{\G}{\mathcal{G}} 
\newcommand{\GP}{\mathcal{GP}}
\newcommand{\Id}{\boldsymbol{{I}}}
\newcommand{\zero}{\vec{0}}
\newcommand{\tr}{\operatorname{tr}}
\newcommand{\rk}{\operatorname{rk}}
\newcommand{\II}{\mathbb{I}}
\renewcommand{\L}{\mathcal{L}}

\newcommand{\bm}{\boldsymbol{m}}
\newcommand{\bgamma}{\boldsymbol{\gamma}}
\newcommand{\bmu}{\boldsymbol{\mu}}
\newcommand{\bnu}{\boldsymbol{\nu}}
\newcommand{\bSigma}{\boldsymbol{\Sigma}}
\newcommand{\bPhi}{\boldsymbol{\Phi}}
\newcommand{\bphi}{\boldsymbol{\phi}}
\newcommand{\balpha}{\boldsymbol{\alpha}}
\newcommand{\bbeta}{\boldsymbol{\beta}}
\newcommand{\btheta}{\boldsymbol{\theta}}

\renewcommand{\a}{\boldsymbol{\mathsf{a}}}
\renewcommand{\b}{\boldsymbol{\mathsf{b}}}
\newcommand{\f}{\boldsymbol{\mathsf{f}}}
\newcommand{\h}{\boldsymbol{\mathsf{h}}}
\newcommand{\kvec}{\boldsymbol{\mathsf{k}}}
\newcommand{\m}{\boldsymbol{\mathsf{m}}}
\renewcommand{\u}{u}
\renewcommand{\v}{v}
\newcommand{\w}{w}
\newcommand{\x}{x}
\newcommand{\y}{y}
\newcommand{\z}{z}

\newcommand{\sA}{\boldsymbol{\mathsf{A}}}
\newcommand{\sB}{\boldsymbol{\mathsf{B}}}
\newcommand{\sC}{\boldsymbol{\mathsf{C}}}
\newcommand{\sF}{\boldsymbol{\mathsf{F}}}
\newcommand{\sG}{\boldsymbol{\mathsf{G}}}
\newcommand{\sH}{\boldsymbol{\mathsf{H}}}
\newcommand{\sI}{\boldsymbol{\mathsf{I}}}
\newcommand{\sK}{\boldsymbol{\mathsf{K}}}
\newcommand{\sL}{\boldsymbol{\mathsf{L}}}
\newcommand{\sM}{\boldsymbol{\mathsf{M}}}
\newcommand{\sR}{\boldsymbol{\mathsf{R}}}
\newcommand{\sS}{\boldsymbol{\mathsf{S}}}
\newcommand{\sU}{\boldsymbol{\mathsf{U}}}
\newcommand{\sV}{\boldsymbol{\mathsf{V}}}
\newcommand{\sW}{\boldsymbol{\mathsf{W}}}
\newcommand{\sX}{\boldsymbol{\mathsf{X}}}
\newcommand{\sY}{\boldsymbol{\mathsf{Y}}}
\newcommand{\sSigma}{\boldsymbol{\mathsf{\Sigma}}}
\newcommand{\sLambda}{\boldsymbol{\mathsf{\Lambda}}}
\newcommand{\sXi}{\boldsymbol{\mathsf{\Xi}}}

\newcommand{\X}{\vec{X}}
\newcommand{\bX}{\mathbb{X}}

\newcommand{\up}{\textasciicircum}


\newcommand*\circled[1]{\tikzexternaldisable\tikz[baseline=(char.base)]{
            \node[shape=circle,draw,inner sep=2pt] (char) {#1};}\tikzexternalenable}

%%%%%%%% TiKZ %%%%%%%%
\usetikzlibrary{arrows,shapes,plotmarks,decorations.pathmorphing,matrix}
\usetikzlibrary{backgrounds,calc,positioning,fadings}

\tikzset{every picture/.style={node distance=2cm}}
\tikzset{>=stealth'} 
\tikzstyle{graphnode} = 
   [circle,draw=TUdark,minimum size=22pt,text centered,text
     width=22pt,inner sep=0pt] 
\tikzstyle{var}   =[graphnode,fill=white]
\tikzstyle{obs}   =[graphnode,fill=TUdark,text=white]
\tikzstyle{act}   =[rectangle,draw=TUdark,text=white,minimum
size=22pt,text centered, text width=22pt,inner sep=0pt]
\tikzstyle{fac}   =[rectangle,draw=TUdark,fill=TUgold,minimum size=5pt]
\tikzstyle{facprior} =[rectangle,draw=TUdark,fill=TUdark,text=white,minimum size=5pt]
\tikzstyle{edge}  =[draw=white,double=TUdark,thick,-]
\tikzstyle{prior} =[rectangle, draw=TUdark, fill=TUdark, minimum size=
5pt, inner sep=0pt]
\tikzstyle{dirprior} = [circle, draw=TUdark, fill=TUdark, minimum
size=5pt, inner sep=0pt]

\tikzfading[name=fade top,bottom color=transparent!0,top color=transparent!75]

% to avoid warnings, copy only two symbols from stmaryrd
\DeclareSymbolFont{stmry}{U}{stmry}{m}{n}
\DeclareMathSymbol\leftarrowtriangle\mathrel{stmry}{"5E}
\DeclareMathSymbol\rightarrowtriangle\mathrel{stmry}{"5F}
\DeclareMathSymbol\sslash\mathrel{stmry}{"0C}
\DeclareMathSymbol\obar\mathrel{stmry}{"3A}
\DeclareMathSymbol\otimes\mathrel{stmry}{"0F}
\DeclareMathSymbol\ominus\mathrel{stmry}{"17}
\DeclareMathSymbol\minuso\mathrel{stmry}{"0A}
\renewcommand{\gets}{\operatorname*{\leftarrowtriangle}}
\renewcommand{\to}{\operatorname*{\rightarrowtriangle}}

\usetikzlibrary{arrows,shapes,plotmarks,pgfplots.colormaps,backgrounds}
\usetikzlibrary{pgfplots.groupplots,fillbetween,patterns}
\pgfplotsset{compat=newest}
\pgfplotsset{
  every axis legend/.append style =
    {
      cells = { anchor = east },
      draw  = none
    },
}  

% \usepackage[skins]{tcolorbox}

\makeatletter
\pgfplotsset{ 
    range frame/.style={
        tick align = outside,
        axis line style={opacity=0},
        after end axis/.code={
            \draw ({rel axis cs:0,0}-|{axis cs:\pgfplots@data@xmin,0}) -- ({rel axis cs:0,0}-|{axis cs:\pgfplots@data@xmax,0});
            \draw ({rel axis cs:0,0}|-{axis cs:0,\pgfplots@data@ymin}) -- ({rel axis cs:0,0}|-{axis cs:0,\pgfplots@data@ymax});
        }
    }
}
\makeatother

\pgfkeys{/pgfplots/mystyle/.style={
  % semithick,
  % tick style={major tick length=4pt,semithick,gray},
  xtick align = inside,
  ytick align = inside
  }}

\pgfkeys{/pgfplots/mytuftestyle/.style={
  semithick,
  tick style={major tick length=4pt,semithick,black},
  separate axis lines,
  axis x line*=bottom,
  axis x line shift=5pt,
  xlabel shift=0pt,
  axis y line*=left,
  tick align = outside,
  axis y line shift=5pt,
  ylabel shift=0pt}}


%%%% THEOREM environments
\newtheorem{defi}{Definition}[section]
\newtheorem{theo}[defi]{Theorem}
\newtheorem{cor}[defi]{Corollary}
\newtheorem{lemm}[defi]{Lemma}
\newtheorem{prob}[defi]{Problem}

\newlength{\figureheight}
\newlength{\figurewidth}
\newlength{\figheight}
\newlength{\figwidth}

% algorithms
\usepackage{algorithm}
\usepackage{algpseudocode}
\algrenewcommand{\algorithmiccomment}[1]{\hfill {$\sslash$ \scriptsize #1}}
\algrenewcommand\alglinenumber[1]{\tiny #1}


% %%%%%% adding vertical bars to algorithmicx. Based on a response at 
% % http://tex.stackexchange.com/questions/52473/is-it-possible-to-have-connecting-loop-lines-like-algorithm2e-in-algorithmic/52778#52778
\makeatletter
% This is the vertical rule that is inserted
\def\therule{\makebox[\algorithmicindent][l]{\hspace*{.5em}\vrule height .75\baselineskip depth .25\baselineskip}}%

\newtoks\therules% Contains rules
\therules={}% Start with empty token list
\def\appendto#1#2{\expandafter#1\expandafter{\the#1#2}}% Append to token list
\def\gobblefirst#1{% Remove (first) from token list
  #1\expandafter\expandafter\expandafter{\expandafter\@gobble\the#1}}%
\def\LState{\State\unskip\the\therules}% New line-state
\def\pushindent{\appendto\therules\therule}%
\def\popindent{\gobblefirst\therules}%
\def\printindent{\unskip\the\therules}%
\def\printandpush{\printindent\pushindent}%
\def\popandprint{\popindent\printindent}%

%      ***      DECLARED LOOPS      ***
% (from algpseudocode.sty)
\algdef{SE}[WHILE]{While}{EndWhile}[1]
  {\printandpush\algorithmicwhile\ #1\ \algorithmicdo}
  {\popandprint\algorithmicend\ \algorithmicwhile}%
\algdef{SE}[FOR]{For}{EndFor}[1]
  {\printandpush\algorithmicfor\ #1\ \algorithmicdo}
  {\popandprint\algorithmicend\ \algorithmicfor}%
\algdef{S}[FOR]{ForAll}[1]
  {\printindent\algorithmicforall\ #1\ \algorithmicdo}%
\algdef{SE}[LOOP]{Loop}{EndLoop}
  {\printandpush\algorithmicloop}
  {\popandprint\algorithmicend\ \algorithmicloop}%
\algdef{SE}[REPEAT]{Repeat}{Until}
  {\printandpush\algorithmicrepeat}[1]
  {\popandprint\algorithmicuntil\ #1}%
\algdef{SE}[IF]{If}{EndIf}[1]
  {\printandpush\algorithmicif\ #1\ \algorithmicthen}
  {\popandprint\algorithmicend\ \algorithmicif}%
\algdef{C}[IF]{IF}{ElsIf}[1]
  {\popandprint\pushindent\algorithmicelse\ \algorithmicif\ #1\ \algorithmicthen}%
\algdef{Ce}[ELSE]{IF}{Else}{EndIf}
  {\popandprint\pushindent\algorithmicelse}%
\algdef{SE}[PROCEDURE]{Procedure}{EndProcedure}[2]
   {\printandpush\algorithmicprocedure\ \textproc{#1}\ifthenelse{\equal{#2}{}}{}{(#2)}}%
   {\popandprint\algorithmicend\ \algorithmicprocedure}%
\algdef{SE}[FUNCTION]{Function}{EndFunction}[2]
   {\printandpush\algorithmicfunction\ \textproc{#1}\ifthenelse{\equal{#2}{}}{}{(#2)}}%
   {\popandprint\algorithmicend\ \algorithmicfunction}%
\makeatother




% \input{../preamble_slides}
\usepackage{tcolorbox}
\usepackage{changepage}


% Assets included in preamble
\newcommand{\assetsDIR}{../figures}

% Add speaker notes to talk
% \setbeameroption{hide notes} % Only slides
%\setbeameroption{show only notes} % Only notes
% \setbeameroption{show notes on second screen}

\usetikzlibrary{external}
% \tikzexternalize[mode=list and make]
% use  make -j 4 -f talk.makefile to compile with 8 parallel threads (this is what it takes to max out the machine, depsite it having 4 cores)
\tikzset{external/force remake=false}
\tikzsetexternalprefix{external/}

\input{math_commands}
\newtheorem{proposition}{Proposition}


\begin{document}

\tikzexternaldisable
\begin{frame}
  \title{{\bf Master Presentation Marius Hobbhahn}\newline {\color{dgra}  Start: 14:30 \newline Fast Predictive Uncertainty for Classification with Bayesian Deep Networks}
  \vspace*{-.7cm}}
  \author{Marius Hobbhahn \vspace{-1cm}} \date{30 June 2020}

  \vspace{-1.5cm}
  \maketitle 
  \vspace{-1.0cm}

  \begin{columns} 
    \column{0.45\textwidth}
    \includegraphics[width=\textwidth]{\assetsDIR/UT_WBMW_Rot_RGB.pdf}\hfill
    % \column{.45\textwidth}
    \column{0.45\textwidth}
    \dre{Faculty of Science\\
    Department of Computer Science\\
    {\small Chair for the Methods of Machine Learning}}
    % \includegraphics[width=.8\textwidth]{\assetsDIR/MPI-IS-WortBildMarke.png}\\
    % \includegraphics[width=.8\textwidth]{\assetsDIR/imprs-is-logo.pdf}\\
    % \parbox[c]{.2\textwidth}{\centering\includegraphics[height=1.2cm]{\assetsDIR/ERC.png}}
    % \parbox{.75\textwidth}{\scriptsize \color{ERC_ora}some of the presented work is supported\newline by the European Research Council.}
  \end{columns}

  \thispagestyle{empty}
  \setcounter{framenumber}{0}

  %%%%%%%%%%%%%%%% ANIMATED LOGO %%%%%%%%%%%%%%%%
  % \tikzifexternalizing{}{%
  % \begin{tikzpicture}[remember picture,overlay]
  % \node[anchor=south,yshift=-5mm] at (current page.south) 
  % {\animategraphics[width=0.9995\paperwidth,autoplay,loop]{36}{\assetsDIR/logo_TU_169_}{0}{39}};
  %   \end{tikzpicture}%
  % }%
  %%%%%%%%%%%%%% END OF ANIMATED LOGO %%%%%%%%%%%

  %%%%%%%%%%%%%%%% STATIC LOGO %%%%%%%%%%%%%%%%%%
  \tikzifexternalizing{}{%
  \begin{tikzpicture}[remember picture,overlay]
   \node[anchor=south,yshift=-5mm] at (current page.south)
  {\includegraphics[width=0.9995\paperwidth]{\assetsDIR/logo_TU_169_1.pdf}};
  \end{tikzpicture}%
  }%
  %%%%%%%%%%%%%% END OF STATIC LOGO %%%%%%%%%%%%%

\end{frame}
\tikzexternalenable

\setlength{\figurewidth}{.9\textwidth}
\setlength{\figureheight}{.6\textheight}


%%%%%%%%%%%%%%%%
%   0 - Motivation and context
%%%%%%%%%%%%%%%%
\begin{frame}\frametitle{Motivation}
    \framesubtitle{Why do we need fast uncertainty in neural networks?}
    \begin{columns}
    	\column{0.55\textwidth}
		\begin{itemize}
			\item safety-critical applications e.g. self-driving cars
			\item trade-off between accuracy and speed
			\item out-of-distribution detection
		\end{itemize}
    	\column{0.45\textwidth}
    	\includegraphics[width=\textwidth]{../figures/self-driving_car.jpg}
    \end{columns}
\end{frame}

%%%%%%%%%%%%%%%

\begin{frame}\frametitle{Context}
	\framesubtitle{What's our new contribution?}
	\begin{figure}
		\centering
		\includegraphics[height=0.4\textheight]{../figures/GaussNN_classic_slim.pdf}\\
		\vspace{4pt}
		\includegraphics[height=0.4\textheight]{../figures/GaussNN_LaplaceBridge_slim.pdf}
	\end{figure}
\end{frame}


%%%%%%%%%%%%%%%%%%%%%%%%%
%     Theory
%%%%%%%%%%%%%%%%%%%%%%%%%
\blackslidetext{\center{\large\textbf{Theory}}}

\begin{frame}\frametitle{Background}
    \framesubtitle{Change of variable for PDFs}
	\subsection*{Change of Variable for pdf} 
	Let $\rvx$ be an $n$-dimensional continuous random variable with joint density function $p_\rvx$. If $\rvy = g(\rvx)$, where $g$ is a differentiable function, then $\rvy$ has density $p_\rvy$:
	\begin{equation}
	p_\rvy(\mathbf{y}) = p_\rvx\left(g^{-1}(\mathbf{\rvy})\right)\left\vert \det\left[\frac{dg^{-1}(\mathbf{\rvy})}{d\mathbf{\rvy}} \right]\right \vert
	\end{equation}
	where the differential is the Jacobian of the inverse of $g$ evaluated at $\rvy$. 
\end{frame}

%%%%%%%%%%%%%%%%%%%%%%%%%

\begin{frame}\frametitle{A new basis for the Dirichlet}
	\framesubtitle{The math}
	\begin{itemize}
		\item \begin{equation}\label{eq:dirichlet}
		\mathrm{Dir}(\vpi | \valpha) := \frac{1}{B(\alpha)}\prod_{k=1}^K \pi_k^{\alpha_k-1}
		\end{equation}
		\item \begin{equation}
		\pi_k(\vz) := \frac{\exp(z_k)}{\sum_{l=1}^K \exp(z_l)} \, ,
		\end{equation}
		\item \begin{equation}\label{eq:dirichlet_softmax}
		\mathrm{Dir}_{\vz}(\vpi(\vz) | \valpha) := \frac{1}{B(\alpha)}\prod_{k=1}^K \pi_k(\vz)^{\alpha_k} \, ,
		\end{equation}
	\end{itemize}
\end{frame}

%%%%%%%%%%%%%%%%%%%%%%%%%

\setlength{\figwidth}{0.33\textwidth}
\setlength{\figheight}{0.7\textheight}

\begin{frame}\frametitle{A new basis for the Dirichlet}
	\framesubtitle{In pictures}
	\begin{figure}
		\centering
		\scriptsize
		\input{../figures/BetaVizTransformation_new}
	\end{figure}
\end{frame}

%%%%%%%%%%%%%%%%%%%%%%%%%

\begin{frame}\frametitle{The Laplace Bridge}
	\framesubtitle{A bridge between the parameters of the Dirichlet and Gaussian}
	\begin{align}
	\alpha_k &= \frac{1}{\Sigma_{kk}}\left(1 - \frac{2}{K} + \frac{e^{\mu_k}}{K^2}\sum_l^K e^{-\mu_l} \right)
	\label{eq:alpha_transform}
	\\
	\mu_k &= \log \alpha_k  - \frac{1}{K} \sum_{l=1}^{K} \log \alpha_l
	\\
	\Sigma_{kl} &= \delta_{kl} \frac{1}{\alpha_k} - \frac{1}{K} \left[\frac{1}{\alpha_k} + \frac{1}{\alpha_l} - \frac{1}{K} \sum_{u=1}^{K} \frac{1}{\alpha_u}\right]
	\end{align}
\end{frame}

%%%%%%%%%%%%%%%%%%%%%%%%%

\begin{frame}\frametitle{The Laplace Bridge}
	\framesubtitle{Summary}
	\begin{figure}
		\includegraphics[width=0.7\textwidth]{../figures/Laplace_Bridge_sketch.pdf}
	\end{figure}
	\begin{itemize}
		\item The Dirichlet in the inverse softmax basis approximates a Gaussian
		\item Via the Laplace approximation in the transformed basis we can create a closed-form transformation $\alpha \rightarrow (\mu, \Sigma)$.
		\item We can also construct an inverse of this transformation $(\mu, \Sigma) \rightarrow \alpha$
		\item In total, we have a \textbf{fast} way to transform between the parameters of a Dirichlet and a Gaussian
	\end{itemize}
\end{frame}

%%%%%%%%%%%%%%%%%%%%%%%%%

\begin{frame}\frametitle{The Laplace Bridge}
	\framesubtitle{Application to Neural Networks}
	\begin{figure}
		\includegraphics[width=\textwidth]{../figures/GaussNN_LaplaceBridge.pdf}
	\end{figure}
\end{frame}

%%%%%%%%%%%%%%%%%%%%%%%%%
%     Experiments
%%%%%%%%%%%%%%%%%%%%%%%%%
\blackslidetext{\center{\large\textbf{Experiments}}}

\begin{frame}\frametitle{A sanity check}
	\framesubtitle{Samples from a 3D Gaussian + Softmax vs. Dirichlet}
	\begin{figure}
		\centering
		\subfloat{\includegraphics[width=0.18\textwidth]{../figures/sMAP/sMAP_Gaussian_coolwarm_0.png}}
		\subfloat{\includegraphics[width=0.18\textwidth]{../figures/sMAP/sMAP_Gaussian_coolwarm_1.png}}
		\subfloat{\includegraphics[width=0.18\textwidth]{../figures/Uncertainty/Uncertainty_Gaussian_coolwarm_0.png}}
		\subfloat{\includegraphics[width=0.18\textwidth]{../figures/Uncertainty/Uncertainty_Gaussian_coolwarm_1.png}}
		\subfloat{\includegraphics[width=0.18\textwidth]{../figures/Uncertainty/Uncertainty_Gaussian_coolwarm_2.png}}
		
		\vspace{10pt}
		\setcounter{subfigure}{0}
		
		\subfloat{\includegraphics[width=0.18\textwidth]{../figures/sMAP/sMAP_Dirichlet_coolwarm_0.png}}
		\subfloat{\includegraphics[width=0.18\textwidth]{../figures/sMAP/sMAP_Dirichlet_coolwarm_1.png}}
		\subfloat{\includegraphics[width=0.18\textwidth]{../figures/Uncertainty/Uncertainty_Dirichlet_coolwarm_0.png}}
		\subfloat{\includegraphics[width=0.18\textwidth]{../figures/Uncertainty/Uncertainty_Dirichlet_coolwarm_1.png}}
		\subfloat{\includegraphics[width=0.18\textwidth]{../figures/Uncertainty/Uncertainty_Dirichlet_coolwarm_2.png}}
	\end{figure}
\end{frame}

%%%%%%%%%%%%%%%%%%%%%%%%%

\setlength{\figwidth}{0.9\textwidth}
\setlength{\figheight}{0.7\textheight}

\begin{frame}\frametitle{MNIST}
	\framesubtitle{Train on 0,1,2; test on 0-9}
	\begin{figure}
		%\includegraphics[width=0.9\textwidth]{imagefile}
		% This file was created by tikzplotlib v0.8.2.
\begin{tikzpicture}

\begin{axis}[
height=\figheight,
legend cell align={left},
legend pos=north west,
legend style={draw=white!80.0!black},
tick align=outside,
tick pos=both,
width=\figwidth,
x grid style={white!69.01960784313725!black},
xlabel={MNIST Class},
xmin=-0.45, xmax=9.45,
xtick align=outside,
xtick pos=left,
xtick style={color=black},
y grid style={white!69.01960784313725!black},
ylabel={Mean Variance},
ymin=-0.00546239636314567, ymax=0.12091085826687,
ytick align=outside,
ytick pos=left,
ytick style={color=black},
ytick={-0.02,0,0.02,0.04,0.06,0.08,0.1,0.12,0.14},
yticklabels={−0.02,0.00,0.02,0.04,0.06,0.08,0.10,0.12,0.14}
]
\addplot [semithick, red, mark=*, mark size=3, mark options={solid}, only marks]
table {%
0 0.000485597585793585
1 0.00123227143194526
2 0.000281842483673245
};
\addlegendentry{In Dist}
\addplot [semithick, blue, mark=*, mark size=3, mark options={solid}, only marks]
table {%
3 0.0408942848443985
4 0.115166619420052
5 0.0661759749054909
6 0.0397674925625324
7 0.0582835488021374
8 0.0391143336892128
9 0.0806220844388008
};
\addlegendentry{Out Dist}
\end{axis}

\end{tikzpicture}
	\end{figure}
\end{frame}

%%%%%%%%%%%%%%%%%%%%%%%%%

\begin{frame}\frametitle{Speedtest - I}
	\framesubtitle{KL divergence vs. number of samples}
	\begin{figure}
		%\includegraphics[width=0.9\textwidth]{imagefile}
		% This file was created by tikzplotlib v0.8.2.
\begin{tikzpicture}

\definecolor{color0}{rgb}{1,0.647058823529412,0}
\definecolor{color1}{rgb}{0.501960784313725,0,0.501960784313725}
\definecolor{color2}{rgb}{0.647058823529412,0.164705882352941,0.164705882352941}

\begin{axis}[
height=\figheight,
legend cell align={left},
legend pos=north east,
legend style={draw=white!80.0!black},
log basis x={10},
tick pos=both,
width=\figwidth,
x grid style={white!69.01960784313725!black},
xlabel={Number of Samples},
xmin=0.562341325190349, xmax=177827.941003892,
xmode=log,
xtick align=inside,
xtick pos=left,
xtick style={color=black},
xtick={0.01,0.1,1,10,100,1000,10000,100000,1000000,10000000},
xticklabels={\(\displaystyle {10^{-2}}\),\(\displaystyle {10^{-1}}\),\(\displaystyle {10^{0}}\),\(\displaystyle {10^{1}}\),\(\displaystyle {10^{2}}\),\(\displaystyle {10^{3}}\),\(\displaystyle {10^{4}}\),\(\displaystyle {10^{5}}\),\(\displaystyle {10^{6}}\),\(\displaystyle {10^{7}}\)},
y grid style={white!69.01960784313725!black},
ylabel={KL Divergence},
ymin=-2.15182824953863, ymax=45.1883932403112,
ytick align=inside,
ytick pos=left,
ytick style={color=black}
]
\path [draw=blue, very thick]
(axis cs:1,41.3982701206161)
--(axis cs:1,41.3982701206161);

\path [draw=blue, very thick]
(axis cs:5,38.8507543346251)
--(axis cs:5,39.0741843320037);

\path [draw=blue, very thick]
(axis cs:10,35.2864009355057)
--(axis cs:10,36.9114764535222);

\path [draw=blue, very thick]
(axis cs:25,29.1974087952039)
--(axis cs:25,31.5444868369423);

\path [draw=blue, very thick]
(axis cs:50,21.2385057852551)
--(axis cs:50,24.0839878754606);

\path [draw=blue, very thick]
(axis cs:75,16.5836595601802)
--(axis cs:75,17.6562791071983);

\path [draw=blue, very thick]
(axis cs:100,12.4266156862388)
--(axis cs:100,15.3071127260996);

\path [draw=blue, very thick]
(axis cs:250,5.64300059313468)
--(axis cs:250,6.24192339554636);

\path [draw=blue, very thick]
(axis cs:500,2.17084407823034)
--(axis cs:500,3.26238974458838);

\path [draw=blue, very thick]
(axis cs:750,1.55585648184509)
--(axis cs:750,2.02968745684394);

\path [draw=blue, very thick]
(axis cs:1000,1.06204505905847)
--(axis cs:1000,1.23981532367538);

\path [draw=blue, very thick]
(axis cs:2500,0.342035635436652)
--(axis cs:2500,0.477450992689462);

\path [draw=blue, very thick]
(axis cs:5000,0.218322482920984)
--(axis cs:5000,0.268408819681547);

\path [draw=blue, very thick]
(axis cs:7500,0.13068971731337)
--(axis cs:7500,0.162126643314091);

\path [draw=blue, very thick]
(axis cs:10000,0.0779525844713338)
--(axis cs:10000,0.103310209035486);

\path [draw=blue, very thick]
(axis cs:25000,0.0294258984006188)
--(axis cs:25000,0.0365894729729816);

\path [draw=blue, very thick]
(axis cs:50000,0.0153424426022053)
--(axis cs:50000,0.0206531839975509);

\path [draw=blue, very thick]
(axis cs:75000,0.0106506508870555)
--(axis cs:75000,0.0136986975478188);

\path [draw=blue, very thick]
(axis cs:100000,0.00730743331987499)
--(axis cs:100000,0.0110392743663015);

\path [draw=color0, very thick]
(axis cs:1,43.0365649907725)
--(axis cs:1,43.0365649907725);

\path [draw=color0, very thick]
(axis cs:5,33.0201579263759)
--(axis cs:5,38.5375789923562);

\path [draw=color0, very thick]
(axis cs:10,29.4207509996895)
--(axis cs:10,34.4540493212375);

\path [draw=color0, very thick]
(axis cs:25,18.2002537726077)
--(axis cs:25,23.7228144750167);

\path [draw=color0, very thick]
(axis cs:50,14.2079535270124)
--(axis cs:50,16.8178855815704);

\path [draw=color0, very thick]
(axis cs:75,10.7978424905023)
--(axis cs:75,12.621459585656);

\path [draw=color0, very thick]
(axis cs:100,9.65565014199387)
--(axis cs:100,10.4393281103967);

\path [draw=color0, very thick]
(axis cs:250,5.76202105533096)
--(axis cs:250,6.9592094137994);

\path [draw=color0, very thick]
(axis cs:500,3.8780405161802)
--(axis cs:500,4.39747176429878);

\path [draw=color0, very thick]
(axis cs:750,2.59710480487391)
--(axis cs:750,3.429512175518);

\path [draw=color0, very thick]
(axis cs:1000,2.27680257720257)
--(axis cs:1000,2.48691127817604);

\path [draw=color0, very thick]
(axis cs:2500,1.1031769857544)
--(axis cs:2500,1.28354880529062);

\path [draw=color0, very thick]
(axis cs:5000,0.509876781796327)
--(axis cs:5000,0.557208452909373);

\path [draw=color0, very thick]
(axis cs:7500,0.335537450129714)
--(axis cs:7500,0.39203101419392);

\path [draw=color0, very thick]
(axis cs:10000,0.220824229571785)
--(axis cs:10000,0.296936046364272);

\path [draw=color0, very thick]
(axis cs:25000,0.0797656998927399)
--(axis cs:25000,0.105835417169598);

\path [draw=color0, very thick]
(axis cs:50000,0.0348721700873487)
--(axis cs:50000,0.0407328414074566);

\path [draw=color0, very thick]
(axis cs:75000,0.0204050997678521)
--(axis cs:75000,0.0233974307740536);

\path [draw=color0, very thick]
(axis cs:100000,0.0135524483867114)
--(axis cs:100000,0.0179733356724642);

\path [draw=green!50.19607843137255!black, very thick]
(axis cs:1,42.9305846157211)
--(axis cs:1,42.9305846157211);

\path [draw=green!50.19607843137255!black, very thick]
(axis cs:5,37.0611658478545)
--(axis cs:5,38.4449378413652);

\path [draw=green!50.19607843137255!black, very thick]
(axis cs:10,34.2813512152306)
--(axis cs:10,35.975071352106);

\path [draw=green!50.19607843137255!black, very thick]
(axis cs:25,24.4005624791702)
--(axis cs:25,26.5692601804098);

\path [draw=green!50.19607843137255!black, very thick]
(axis cs:50,18.7118986152758)
--(axis cs:50,19.7576581143128);

\path [draw=green!50.19607843137255!black, very thick]
(axis cs:75,10.7435568709726)
--(axis cs:75,14.7752725815288);

\path [draw=green!50.19607843137255!black, very thick]
(axis cs:100,7.72596014341502)
--(axis cs:100,9.58686687383905);

\path [draw=green!50.19607843137255!black, very thick]
(axis cs:250,3.00653165881441)
--(axis cs:250,4.27282174432397);

\path [draw=green!50.19607843137255!black, very thick]
(axis cs:500,0.849290756433662)
--(axis cs:500,1.25090202549443);

\path [draw=green!50.19607843137255!black, very thick]
(axis cs:750,0.572699964690209)
--(axis cs:750,0.886085814215126);

\path [draw=green!50.19607843137255!black, very thick]
(axis cs:1000,0.292419473332981)
--(axis cs:1000,0.558853801330423);

\path [draw=green!50.19607843137255!black, very thick]
(axis cs:2500,0.164470506245009)
--(axis cs:2500,0.23014714501126);

\path [draw=green!50.19607843137255!black, very thick]
(axis cs:5000,0.0713618791929187)
--(axis cs:5000,0.125350884327049);

\path [draw=green!50.19607843137255!black, very thick]
(axis cs:7500,0.0440501484025761)
--(axis cs:7500,0.0546591076357259);

\path [draw=green!50.19607843137255!black, very thick]
(axis cs:10000,0.0354904508147201)
--(axis cs:10000,0.0589326299877341);

\path [draw=green!50.19607843137255!black, very thick]
(axis cs:25000,0.0130415843450368)
--(axis cs:25000,0.0200767160525845);

\path [draw=green!50.19607843137255!black, very thick]
(axis cs:50000,0.00530030358239454)
--(axis cs:50000,0.00768632301500861);

\path [draw=green!50.19607843137255!black, very thick]
(axis cs:75000,0.00457916087173392)
--(axis cs:75000,0.0079657724518956);

\path [draw=green!50.19607843137255!black, very thick]
(axis cs:100000,0.00287229403113529)
--(axis cs:100000,0.00594796808345979);

\path [draw=blue, draw opacity=0.7, semithick]
(axis cs:10000,0)
--(axis cs:10000,7);

\path [draw=color0, draw opacity=0.7, semithick]
(axis cs:5000,0)
--(axis cs:5000,7);

\path [draw=green!50.19607843137255!black, draw opacity=0.7, semithick]
(axis cs:500,0)
--(axis cs:500,7);

\addplot [semithick, red, mark=*, mark size=3, mark options={solid}, only marks]
table {%
1 0.0907225415781931
};
\addlegendentry{Laplace Bridge}
\addplot [semithick, color1, mark=*, mark size=3, mark options={solid}, only marks, forget plot]
table {%
1 0.704045819785254
};
\addplot [semithick, color2, mark=*, mark size=3, mark options={solid}, only marks, forget plot]
table {%
1 1.76507413738839
};
\addplot [very thick, red, opacity=0.5, dash pattern=on 1pt off 3pt on 3pt off 3pt, forget plot]
table {%
1 0.0907225415781931
10000 0.0907225415781931
};
\addplot [very thick, color1, opacity=0.5, dash pattern=on 1pt off 3pt on 3pt off 3pt, forget plot]
table {%
1 0.704045819785254
5000 0.704045819785254
};
\addplot [very thick, color2, opacity=0.5, dash pattern=on 1pt off 3pt on 3pt off 3pt, forget plot]
table {%
1 1.76507413738839
500 1.76507413738839
};
\addplot [very thick, blue]
table {%
1 41.3982701206161
5 38.9624693333144
10 36.098938694514
25 30.3709478160731
50 22.6612468303579
75 17.1199693336892
100 13.8668642061692
250 5.94246199434052
500 2.71661691140936
750 1.79277196934452
1000 1.15093019136693
2500 0.409743314063057
5000 0.243365651301265
7500 0.146408180313731
10000 0.0906313967534098
25000 0.0330076856868002
50000 0.0179978132998781
75000 0.0121746742174372
100000 0.00917335384308824
};
\addlegendentry{Monte Carlo}
\addplot [very thick, color0]
table {%
1 43.0365649907725
5 35.778868459366
10 31.9374001604635
25 20.9615341238122
50 15.5129195542914
75 11.7096510380792
100 10.0474891261953
250 6.36061523456518
500 4.13775614023949
750 3.01330849019595
1000 2.3818569276893
2500 1.19336289552251
5000 0.53354261735285
7500 0.363784232161817
10000 0.258880137968029
25000 0.092800558531169
50000 0.0378025057474026
75000 0.0219012652709528
100000 0.0157628920295878
};
\addlegendentry{Monte Carlo}
\addplot [very thick, green!50.19607843137255!black]
table {%
1 42.9305846157211
5 37.7530518446098
10 35.1282112836683
25 25.48491132979
50 19.2347783647943
75 12.7594147262507
100 8.65641350862704
250 3.63967670156919
500 1.05009639096405
750 0.729392889452668
1000 0.425636637331702
2500 0.197308825628134
5000 0.0983563817599838
7500 0.049354628019151
10000 0.0472115404012271
25000 0.0165591501988106
50000 0.00649331329870158
75000 0.00627246666181476
100000 0.00441013105729754
};
\addlegendentry{Monte Carlo}
\end{axis}

\end{tikzpicture}
	\end{figure}
\end{frame}

%%%%%%%%%%%%%%%%%%%%%%%%%

\begin{frame}\frametitle{Speedtest - II}
	\framesubtitle{KL divergence vs. wall-clock time}
	\begin{figure}
		%\includegraphics[width=0.9\textwidth]{imagefile}
		% This file was created by tikzplotlib v0.8.2.
\begin{tikzpicture}

\definecolor{color0}{rgb}{1,0.647058823529412,0}
\definecolor{color1}{rgb}{0.501960784313725,0,0.501960784313725}
\definecolor{color2}{rgb}{0.647058823529412,0.164705882352941,0.164705882352941}

\begin{axis}[
height=\figheight,
legend cell align={left},
legend pos=north east,
legend style={draw=white!80.0!black},
log basis x={10},
tick pos=both,
width=\figwidth,
x grid style={white!69.01960784313725!black},
xlabel={Time in s},
xmin=7.66454204975669e-05, xmax=2.31707187675955,
xmode=log,
xtick align=inside,
xtick pos=left,
xtick style={color=black},
xtick={1e-06,1e-05,0.0001,0.001,0.01,0.1,1,10,100},
xticklabels={\(\displaystyle {10^{-6}}\),\(\displaystyle {10^{-5}}\),\(\displaystyle {10^{-4}}\),\(\displaystyle {10^{-3}}\),\(\displaystyle {10^{-2}}\),\(\displaystyle {10^{-1}}\),\(\displaystyle {10^{0}}\),\(\displaystyle {10^{1}}\),\(\displaystyle {10^{2}}\)},
y grid style={white!69.01960784313725!black},
ylabel={KL Divergence},
ymin=-2.15182824953863, ymax=45.1883932403112,
ytick align=inside,
ytick pos=left,
ytick style={color=black}
]
\path [draw=blue, very thick]
(axis cs:0.000479772000016965,41.3982701206161)
--(axis cs:0.000479772000016965,41.3982701206161);

\path [draw=blue, very thick]
(axis cs:0.000462709400039785,38.8507543346251)
--(axis cs:0.000462709400039785,39.0741843320037);

\path [draw=blue, very thick]
(axis cs:0.000490144399987003,35.2864009355057)
--(axis cs:0.000490144399987003,36.9114764535222);

\path [draw=blue, very thick]
(axis cs:0.000664097199978642,29.1974087952039)
--(axis cs:0.000664097199978642,31.5444868369423);

\path [draw=blue, very thick]
(axis cs:0.000871186800009127,21.2385057852551)
--(axis cs:0.000871186800009127,24.0839878754606);

\path [draw=blue, very thick]
(axis cs:0.00112354140001116,16.5836595601802)
--(axis cs:0.00112354140001116,17.6562791071983);

\path [draw=blue, very thick]
(axis cs:0.00152032119999603,12.4266156862388)
--(axis cs:0.00152032119999603,15.3071127260996);

\path [draw=blue, very thick]
(axis cs:0.00299396859998069,5.64300059313468)
--(axis cs:0.00299396859998069,6.24192339554636);

\path [draw=blue, very thick]
(axis cs:0.00561123019997467,2.17084407823034)
--(axis cs:0.00561123019997467,3.26238974458838);

\path [draw=blue, very thick]
(axis cs:0.0100978610000084,1.55585648184509)
--(axis cs:0.0100978610000084,2.02968745684394);

\path [draw=blue, very thick]
(axis cs:0.0126740359999985,1.06204505905847)
--(axis cs:0.0126740359999985,1.23981532367538);

\path [draw=blue, very thick]
(axis cs:0.0308647542000017,0.342035635436652)
--(axis cs:0.0308647542000017,0.477450992689462);

\path [draw=blue, very thick]
(axis cs:0.0561433891999968,0.218322482920984)
--(axis cs:0.0561433891999968,0.268408819681547);

\path [draw=blue, very thick]
(axis cs:0.179833333199986,0.13068971731337)
--(axis cs:0.179833333199986,0.162126643314091);

\path [draw=blue, very thick]
(axis cs:0.230520355399972,0.0779525844713338)
--(axis cs:0.230520355399972,0.103310209035486);

\path [draw=blue, very thick]
(axis cs:0.506458102200008,0.0294258984006188)
--(axis cs:0.506458102200008,0.0365894729729816);

\path [draw=blue, very thick]
(axis cs:0.824743553000008,0.0153424426022053)
--(axis cs:0.824743553000008,0.0206531839975509);

\path [draw=blue, very thick]
(axis cs:1.10083765720001,0.0106506508870555)
--(axis cs:1.10083765720001,0.0136986975478188);

\path [draw=blue, very thick]
(axis cs:1.34734060660001,0.00730743331987499)
--(axis cs:1.34734060660001,0.0110392743663015);

\path [draw=color0, very thick]
(axis cs:0.000437038800032497,43.0365649907725)
--(axis cs:0.000437038800032497,43.0365649907725);

\path [draw=color0, very thick]
(axis cs:0.000435702200047672,33.0201579263759)
--(axis cs:0.000435702200047672,38.5375789923562);

\path [draw=color0, very thick]
(axis cs:0.000535914600027354,29.4207509996895)
--(axis cs:0.000535914600027354,34.4540493212375);

\path [draw=color0, very thick]
(axis cs:0.000777945000004365,18.2002537726077)
--(axis cs:0.000777945000004365,23.7228144750167);

\path [draw=color0, very thick]
(axis cs:0.00119840080003542,14.2079535270124)
--(axis cs:0.00119840080003542,16.8178855815704);

\path [draw=color0, very thick]
(axis cs:0.00127340500002902,10.7978424905023)
--(axis cs:0.00127340500002902,12.621459585656);

\path [draw=color0, very thick]
(axis cs:0.00141027140002734,9.65565014199387)
--(axis cs:0.00141027140002734,10.4393281103967);

\path [draw=color0, very thick]
(axis cs:0.00329267580000305,5.76202105533096)
--(axis cs:0.00329267580000305,6.9592094137994);

\path [draw=color0, very thick]
(axis cs:0.00617591600000651,3.8780405161802)
--(axis cs:0.00617591600000651,4.39747176429878);

\path [draw=color0, very thick]
(axis cs:0.0100022406000107,2.59710480487391)
--(axis cs:0.0100022406000107,3.429512175518);

\path [draw=color0, very thick]
(axis cs:0.0114433376000079,2.27680257720257)
--(axis cs:0.0114433376000079,2.48691127817604);

\path [draw=color0, very thick]
(axis cs:0.0318198007999626,1.1031769857544)
--(axis cs:0.0318198007999626,1.28354880529062);

\path [draw=color0, very thick]
(axis cs:0.0589909576000309,0.509876781796327)
--(axis cs:0.0589909576000309,0.557208452909373);

\path [draw=color0, very thick]
(axis cs:0.165210962800006,0.335537450129714)
--(axis cs:0.165210962800006,0.39203101419392);

\path [draw=color0, very thick]
(axis cs:0.245892468600005,0.220824229571785)
--(axis cs:0.245892468600005,0.296936046364272);

\path [draw=color0, very thick]
(axis cs:0.479274818599993,0.0797656998927399)
--(axis cs:0.479274818599993,0.105835417169598);

\path [draw=color0, very thick]
(axis cs:0.799617780599988,0.0348721700873487)
--(axis cs:0.799617780599988,0.0407328414074566);

\path [draw=color0, very thick]
(axis cs:1.08390964980003,0.0204050997678521)
--(axis cs:1.08390964980003,0.0233974307740536);

\path [draw=color0, very thick]
(axis cs:1.36421115500004,0.0135524483867114)
--(axis cs:1.36421115500004,0.0179733356724642);

\path [draw=green!50.19607843137255!black, very thick]
(axis cs:0.000531081000008271,42.9305846157211)
--(axis cs:0.000531081000008271,42.9305846157211);

\path [draw=green!50.19607843137255!black, very thick]
(axis cs:0.000426537399948757,37.0611658478545)
--(axis cs:0.000426537399948757,38.4449378413652);

\path [draw=green!50.19607843137255!black, very thick]
(axis cs:0.000685418600005505,34.2813512152306)
--(axis cs:0.000685418600005505,35.975071352106);

\path [draw=green!50.19607843137255!black, very thick]
(axis cs:0.000784627799976079,24.4005624791702)
--(axis cs:0.000784627799976079,26.5692601804098);

\path [draw=green!50.19607843137255!black, very thick]
(axis cs:0.00110269080000762,18.7118986152758)
--(axis cs:0.00110269080000762,19.7576581143128);

\path [draw=green!50.19607843137255!black, very thick]
(axis cs:0.00127357879994179,10.7435568709726)
--(axis cs:0.00127357879994179,14.7752725815288);

\path [draw=green!50.19607843137255!black, very thick]
(axis cs:0.00193130859995563,7.72596014341502)
--(axis cs:0.00193130859995563,9.58686687383905);

\path [draw=green!50.19607843137255!black, very thick]
(axis cs:0.00295731420001175,3.00653165881441)
--(axis cs:0.00295731420001175,4.27282174432397);

\path [draw=green!50.19607843137255!black, very thick]
(axis cs:0.00722474840003997,0.849290756433662)
--(axis cs:0.00722474840003997,1.25090202549443);

\path [draw=green!50.19607843137255!black, very thick]
(axis cs:0.00921846979999827,0.572699964690209)
--(axis cs:0.00921846979999827,0.886085814215126);

\path [draw=green!50.19607843137255!black, very thick]
(axis cs:0.0124544410000226,0.292419473332981)
--(axis cs:0.0124544410000226,0.558853801330423);

\path [draw=green!50.19607843137255!black, very thick]
(axis cs:0.0281576174000293,0.164470506245009)
--(axis cs:0.0281576174000293,0.23014714501126);

\path [draw=green!50.19607843137255!black, very thick]
(axis cs:0.0554622965999897,0.0713618791929187)
--(axis cs:0.0554622965999897,0.125350884327049);

\path [draw=green!50.19607843137255!black, very thick]
(axis cs:0.164105063199986,0.0440501484025761)
--(axis cs:0.164105063199986,0.0546591076357259);

\path [draw=green!50.19607843137255!black, very thick]
(axis cs:0.234555237199993,0.0354904508147201)
--(axis cs:0.234555237199993,0.0589326299877341);

\path [draw=green!50.19607843137255!black, very thick]
(axis cs:0.504778742999997,0.0130415843450368)
--(axis cs:0.504778742999997,0.0200767160525845);

\path [draw=green!50.19607843137255!black, very thick]
(axis cs:0.775621087800005,0.00530030358239454)
--(axis cs:0.775621087800005,0.00768632301500861);

\path [draw=green!50.19607843137255!black, very thick]
(axis cs:1.08139886579997,0.00457916087173392)
--(axis cs:1.08139886579997,0.0079657724518956);

\path [draw=green!50.19607843137255!black, very thick]
(axis cs:1.44971468479998,0.00287229403113529)
--(axis cs:1.44971468479998,0.00594796808345979);

\path [draw=blue, draw opacity=0.7, semithick]
(axis cs:0.230520355399972,0)
--(axis cs:0.230520355399972,7);

\path [draw=color0, draw opacity=0.7, semithick]
(axis cs:0.0589909576000309,0)
--(axis cs:0.0589909576000309,7);

\path [draw=green!50.19607843137255!black, draw opacity=0.7, semithick]
(axis cs:0.00722474840003997,0)
--(axis cs:0.00722474840003997,7);

\addplot [semithick, red, mark=*, mark size=3, mark options={solid}, only marks]
table {%
0.00018578000000069 0.0907225415781931
};
\addlegendentry{Laplace Bridge}
\addplot [semithick, color1, mark=*, mark size=3, mark options={solid}, only marks, forget plot]
table {%
0.000126700000000035 0.704045819785254
};
\addplot [semithick, color2, mark=*, mark size=3, mark options={solid}, only marks, forget plot]
table {%
0.000122501999999969 1.76507413738839
};
\addplot [very thick, red, opacity=0.5, dash pattern=on 1pt off 3pt on 3pt off 3pt, forget plot]
table {%
0.00018578000000069 0.0907225415781931
0.230520355399972 0.0907225415781931
};
\addplot [very thick, color1, opacity=0.5, dash pattern=on 1pt off 3pt on 3pt off 3pt, forget plot]
table {%
0.000126700000000035 0.704045819785254
0.0589909576000309 0.704045819785254
};
\addplot [very thick, color2, opacity=0.5, dash pattern=on 1pt off 3pt on 3pt off 3pt, forget plot]
table {%
0.000122501999999969 1.76507413738839
0.00722474840003997 1.76507413738839
};
\addplot [very thick, blue]
table {%
0.000479772000016965 41.3982701206161
0.000462709400039785 38.9624693333144
0.000490144399987003 36.098938694514
0.000664097199978642 30.3709478160731
0.000871186800009127 22.6612468303579
0.00112354140001116 17.1199693336892
0.00152032119999603 13.8668642061692
0.00299396859998069 5.94246199434052
0.00561123019997467 2.71661691140936
0.0100978610000084 1.79277196934452
0.0126740359999985 1.15093019136693
0.0308647542000017 0.409743314063057
0.0561433891999968 0.243365651301265
0.179833333199986 0.146408180313731
0.230520355399972 0.0906313967534098
0.506458102200008 0.0330076856868002
0.824743553000008 0.0179978132998781
1.10083765720001 0.0121746742174372
1.34734060660001 0.00917335384308824
};
\addlegendentry{Monte Carlo}
\addplot [very thick, color0]
table {%
0.000437038800032497 43.0365649907725
0.000435702200047672 35.778868459366
0.000535914600027354 31.9374001604635
0.000777945000004365 20.9615341238122
0.00119840080003542 15.5129195542914
0.00127340500002902 11.7096510380792
0.00141027140002734 10.0474891261953
0.00329267580000305 6.36061523456518
0.00617591600000651 4.13775614023949
0.0100022406000107 3.01330849019595
0.0114433376000079 2.3818569276893
0.0318198007999626 1.19336289552251
0.0589909576000309 0.53354261735285
0.165210962800006 0.363784232161817
0.245892468600005 0.258880137968029
0.479274818599993 0.092800558531169
0.799617780599988 0.0378025057474026
1.08390964980003 0.0219012652709528
1.36421115500004 0.0157628920295878
};
\addlegendentry{Monte Carlo}
\addplot [very thick, green!50.19607843137255!black]
table {%
0.000531081000008271 42.9305846157211
0.000426537399948757 37.7530518446098
0.000685418600005505 35.1282112836683
0.000784627799976079 25.48491132979
0.00110269080000762 19.2347783647943
0.00127357879994179 12.7594147262507
0.00193130859995563 8.65641350862704
0.00295731420001175 3.63967670156919
0.00722474840003997 1.05009639096405
0.00921846979999827 0.729392889452668
0.0124544410000226 0.425636637331702
0.0281576174000293 0.197308825628134
0.0554622965999897 0.0983563817599838
0.164105063199986 0.049354628019151
0.234555237199993 0.0472115404012271
0.504778742999997 0.0165591501988106
0.775621087800005 0.00649331329870158
1.08139886579997 0.00627246666181476
1.44971468479998 0.00441013105729754
};
\addlegendentry{Monte Carlo}
\end{axis}

\end{tikzpicture}
	\end{figure}
\end{frame}

%%%%%%%%%%%%%%%%%%%%%%%%%

\setlength{\figwidth}{0.95\textwidth}
\setlength{\figheight}{0.34\textheight}

\begin{frame}\frametitle{Imagenet}
	\framesubtitle{Using the properties of the Dirichlet - The marginal of a Dirichlet is a Dirichlet}
	\begin{figure}
		\centering
		\includegraphics[width=\figwidth,height=\figheight]{../figures/imagenet_images.pdf} \\
		\includegraphics[width=\figwidth,height=\figheight]{../figures/imagenet_marginal_betas.pdf}
	\end{figure}
	We can use the overlap of the distributions to create an uncertainty-aware top-k ranking.
\end{frame}

%%%%%%%%%%%%%%%%%%%%%%%%%

\setlength{\figwidth}{0.95\textwidth}
\setlength{\figheight}{0.6\textheight}

\begin{frame}\frametitle{Imagenet - II}
	\framesubtitle{How good is the flexible top-k ranking?}
	\begin{figure}
		\centering
		\includegraphics[width=\figwidth,height=\figheight]{../figures/output-figure4.pdf} 
	\end{figure}
	\begin{itemize}
		\item The original top-$1$ accuracy of DenseNet on ImageNet is $0.744$ and top-$5$ accuracy is $0.919$
		\item The uncertainty-aware top-$k$ accuracy is $0.797$, where $k$ is on average $1.688$
	\end{itemize}
\end{frame}

%%%%%%%%%%%%%%%%%%%%%%%%%

\begin{frame}\frametitle{Out-of-distribution Detection}
	\framesubtitle{Looking at the numbers}
	\resizebox{\textwidth}{!}{% use resizebox with textwidth
		\begin{tabular}{l  l || c c c c | c  c  c  c | c c}
			\toprule
			& & \multicolumn{2}{c}{\textbf{Diag Sampling}} & \multicolumn{2}{c}{\textbf{Diag LB}} &\multicolumn{2}{c}{\textbf{KFAC Sampling}} &  \multicolumn{2}{c}{\textbf{KFAC LB}} & \multicolumn{2}{c}{\textbf{Time in s} $\downarrow$} \\
			\textbf{Train} & \textbf{Test} & \textbf{MMC} $\downarrow$ & \textbf{AUROC} $\uparrow$& \textbf{MMC} $\downarrow$ & \textbf{AUROC} $\uparrow$& \textbf{MMC} $\downarrow$& \textbf{AUROC} $\uparrow$& \textbf{MMC}$\downarrow$ & \textbf{AUROC} $\uparrow$& \textbf{Sampling} & \textbf{LB} \\
			\midrule
			MNIST & MNIST & 0.942 $\pm$ 0.007 & - &  \textbf{0.987} $\pm$ 0.000 & -  & - & - & - & - & 26.8 & \textbf{0.062}\\
			MNIST & FMNIST & 0.397 $\pm$ 0.001 & 0.992 $\pm$ 0.000 & \textbf{0.363} $\pm$ 0.000 & \textbf{0.996} $\pm$ 0.000 & - & - & - & - & 26.8 & \textbf{0.062}\\
			MNIST & notMNIST & \textbf{0.543} $\pm$ 0.000 & 0.960 $\pm$ 0.000 & 0.649 $\pm$ 0.000 & \textbf{0.961} $\pm$ 0.000 & - & - & - & - & 50.3 & \textbf{0.117}\\
			MNIST & KMNIST & \textbf{0.513} $\pm$ 0.001 & \textbf{0.974} $\pm$ 0.000 & 0.637 $\pm$ 0.000 & 0.973 $\pm$ 0.000 & - & - & - & - & 26.9 & \textbf{0.062}\\
			\midrule
			CIFAR-10 & CIFAR-10  & 0.948 $\pm$ 0.000 & -     & \textbf{0.966} $\pm$ 0.000 & -  & 0.857 $\pm$ 0.003 & - & \textbf{0.966} $\pm$ 0.000 & - & 6.58 & \textbf{0.017}\\
			CIFAR-10 & CIFAR-100 & \textbf{0.708} $\pm$ 0.000 & \textbf{0.889} $\pm$ 0.000 & 0.742 $\pm$ 0.000 & 0.866 $\pm$ 0.000 & \textbf{0.562} $\pm$ 0.003 & \textbf{0.880} $\pm$ 0.012 & 0.741 $\pm$ 0.000 & 0.866 $\pm$ 0.000 & 6.59 & \textbf{0.016}\\
			CIFAR-10 & SVHN      & \textbf{0.643}$\pm$ 0.000 & 0.933 $\pm$ 0.000 & 0.647 $\pm$ 0.000 & \textbf{0.934} $\pm$ 0.000& \textbf{0.484} $\pm$ 0.004 & \textbf{0.939} $\pm$ 0.001 & 0.648 $\pm$ 0.003 & 0.934 $\pm$ 0.001 & 17.0 & \textbf{0.040}\\
			\midrule
			SVHN & SVHN       & 0.986 $\pm$ 0.000 &   -   &  \textbf{0.993} $\pm$ 0.000& -     & 0.947 $\pm$ 0.002 & -             & \textbf{0.993}   $\pm$ 0.000          & -     & 17.1 & \textbf{0.042}\\
			SVHN & CIFAR-100  & 0.595 $\pm$ 0.000 & 0.984 $\pm$ 0.000 &  \textbf{0.526} $\pm$ 0.000 & \textbf{0.985} $\pm$ 0.000& \textbf{0.460} $\pm$ 0.004  & \textbf{0.986} $\pm$ 0.001 & 0.527 $\pm$ 0.002 & 0.985 $\pm$ 0.000 & 6.62 & \textbf{0.017}\\
			SVHN & CIFAR-10   & 0.593 $\pm$ 0.000 & 0.984 $\pm$ 0.000&  \textbf{0.520} $\pm$ 0.000 & \textbf{0.987} $\pm$ 0.000 & \textbf{0.458} $\pm$ 0.004  & 0.986 $\pm$ 0.001 & 0.520 $\pm$ 0.002 & \textbf{0.987} $\pm$ 0.000 & 6.62 & \textbf{0.017}\\
			\midrule
			CIFAR-100 & CIFAR-100 & \textbf{0.762} $\pm$ 0.000& - & 0.590 $\pm$ 0.000& - &  0.404 $\pm$ 0.000& - & \textbf{0.593} $\pm$ 0.000& - & 6.76 & \textbf{0.016}\\
			CIFAR-100 & CIFAR-10  & 0.467 $\pm$ 0.000& 0.788 $\pm$ 0.000& \textbf{0.206} $\pm$ 0.000& \textbf{0.791} $\pm$ 0.000& 0.213 $\pm$ 0.000& 0.788 $\pm$ 0.000& \textbf{0.209} $\pm$ 0.000& \textbf{0.791} $\pm$ 0.000 & 6.71 & \textbf{0.017}\\
			CIFAR-100 & SVHN      & 0.461 $\pm$ 0.000& 0.795 $\pm$ 0.000& \textbf{0.170} $\pm$ 0.000& \textbf{0.815} $\pm$ 0.000& 0.180 $\pm$ 0.001 & \textbf{0.838} $\pm$ 0.001 & \textbf{0.173} $\pm$ 0.000 & 0.815 $\pm$ 0.000 & 17.3 & \textbf{0.041}\\
			\bottomrule
		\end{tabular}
	}
	\begin{itemize}
		\item The Laplace Bridge seems to be have better MMC and AUROC compared to sampling from a diagonal Gaussian approximation
		\item The Laplace Bridge is as good as a KFAC approximation
		\item The Laplace Bridge is around 400 times faster on average
	\end{itemize}
\end{frame}

%%%%%%%%%%%%%%%%%%%%%%%%%

\begin{frame}\frametitle{Conclusions}
	\framesubtitle{What can or can't the Laplace Bridge achieve in the context of BNNs?}
	\begin{itemize}
		\item The Laplace Bridge improves an important part of Bayesian Neural Network inference for classification (fast \& non-invasive)
		\item The Dirichlet distribution has some additional interesting use cases (e.g. the top-k ranking)
		\item It will not revolutionize BNNs; it is just one piece in the larger puzzle
	\end{itemize}
\end{frame}

\blackslidetext{\center{\large\textbf{Questions?}}}

%%%%%%%%%%%%%%%%%%%%%%%%%
%     Future 
%%%%%%%%%%%%%%%%%%%%%%%%%

\blackslidetext{\center{\large\textbf{Future}}}

\begin{frame}\frametitle{The generalized Laplace Bridge}
	\framesubtitle{Looking at the larger pattern}
	\begin{itemize}
		\item Similar ``Bridges'' can be found for all exponential families.
		\item Develop a general theoretically grounded framework for the general Laplace Bridge
		\item Compute KL-divergences in the different basis
	\end{itemize}
\end{frame}

%%%%%%%%%%%%%%%%%%%%%%%%%

\begin{frame}\frametitle{The generalized Laplace Bridge}
	\framesubtitle{So what?}
	\textbf{Implications:} (with a small error)
	\begin{itemize}
		\item All exponential families can be transformed to Gaussians
		\item All exponential families can be transformed to each other
		\item All exponential families are conjugate priors for each other
	\end{itemize}
\end{frame}

%%%%%%%%%%%%%%%%%%%%%%%%%
%     Backup
%%%%%%%%%%%%%%%%%%%%%%%%%

\blackslidetext{\center{\textbf{Backup}}}

\begin{frame}\frametitle{Backup}
	\framesubtitle{Laplace approximations of a neural network}
	\begin{equation}
	p(c | x) = \mathcal{N}(x; f(x,w_\text{MAP}), J(x)^T H^{-1} J(x))
	\label{eq:LANN}
	\end{equation}
	\begin{itemize}
		\item $f(x; w_\text{MAP})$ is the network output induced by the MAP estimate $w_\text{MAP}$.
		\item $J(x) = \frac{\partial f(x, w_{\text{MAP}})}{\partial w} \in \mathbb{R}^{K\times P}$ is the Jacobian of the network 
		\item $H_{ij} = \frac{\partial^2 \mathcal{L}(f(x), y)}{\partial w_i \partial w_j} \in \mathbb{R}^{P \times P}$ its Hessian. 
		\item $K, P$ are the number of classes and parameters of the network respectively.
	\end{itemize}
\end{frame}

%%%%%%%%%%%%%%%%%%%%%%%%%

\begin{frame}\frametitle{Backup}
	\framesubtitle{A theoretical bound for the transformation}
	\begin{proposition} \label{prop:dir_var_from_gaussian}
		Let $\mathrm{Dir}(\vpi | \valpha)$ be obtained via the Laplace Bridge from a Gaussian distribution $\N(\vz | \vmu, \mSigma)$ over $\R^K$. Then, for each $k = 1, \dots, K$, letting $\alpha_{\neq k} := \sum_{l \neq k} \alpha_l$, if
		%
		\begin{equation*}
		\alpha_k > \frac{1}{4} \left(\sqrt{9\alpha_{\neq k}^2 + 10\alpha_{\neq k} + 1} - \alpha_{\neq k} - 1\right) \, ,
		\end{equation*}
		%
		then the variance $\mathrm{Var}(\pi_k | \valpha)$ of the $k$-th component of $\vpi$ is increasing in $\mSigma_{kk}$.
	\end{proposition}
\end{frame}

%%%%%%%%%%%%%%%%%%%%%%%%%

\begin{frame}\frametitle{Backup}
	\framesubtitle{Computing the Hessian}
	First, we consider the special case where $\vpi$ is confined to a $I-1$ dimensional subspace satisfying $\sum_i \vpi_i = c$. In this subspace we can represent $\vpi$ by an $I - 1$ dimensional vector $\va$ such that 
	
	\begin{align}
	\pi_i &= a_i \quad i,...,I-1 \\
	\pi_I &= c - \sum_i^{I-1} a_i
	\end{align}
	
	and similarly we can represent $\vz$ by an $I-1$ dimensional vector $\vvarrho$:
	
	\begin{align}
	z_i &= \varrho_i \quad i,...,I-1 \\
	z_I &= 1 - \sum_i^{I-1}\varrho_i
	\end{align}
\end{frame}

%%%%%%%%%%%%%%%%%%%%%%%%%%%

\begin{frame}\frametitle{Backup}
	\framesubtitle{Computing the Hessian - II}
	then we can find the density over $\vvarrho$ (which is proportional to the required density over $\rz$)
	from the density over $\vpi$ (which is proportional to the given density over $\vpi$) by finding the
	determinant of the $(I - 1) \times (I - 1)$ Jacobian $\mJ$ given by
	
	\begin{align}
	J_{ik} &= \frac{\partial \varrho_i}{\partial a_i} = \sum_j^{I} \frac{\partial z_i}{\partial \pi_j}\frac{\partial \pi_j}{\partial a_k} \\
	&= \delta_{ik}\rvz_i - \rvz_i\rvz_k + \rvz_i\rvz_I =  \rvz_i(\delta_{ik} - (\rvz_k - \rvz_I))
	\end{align}
\end{frame}

%%%%%%%%%%%%%%%%%%%%%%%%%%%%

\begin{frame}\frametitle{Backup}
	\framesubtitle{Computing the Hessian - III}
	We define two additional $I-1$ dimensional helper vectors $\rvz_k^+ := \rvz_k - \rvz_I$ and $n_k := 1$, and use $\det(I - xy^T) = 1 - x \cdot y$ from linear algebra. It follows that
	\begin{align}
	\det J &= \prod_{i=1}^{I-1} \rvz_i \times \det[I-n\rvz^{+^T}]  \\
	&= \prod_{i=1}^{I-1} \rvz_i \times (1 - n \cdot \rvz^{+})  \\
	&= \prod_{i=1}^{I-1} \rvz_i \times \left(1 - \sum_k \rvz_k^{+} \right) = I \prod_{i=1}^I \rvz_i 
	\end{align}
	
\end{frame}


% \begin{frame}\frametitle{Theory question}
%     \framesubtitle{$P(B\mid A)\ge P(B)$}
% \begin{block}{Block title}
% a block
% \end{block}
% \note{Note on second screen}
% \note[item]{Another note}

% \ribbon{a ribbon across the page (for big takeaways).}
% \end{frame}


\end{document}

