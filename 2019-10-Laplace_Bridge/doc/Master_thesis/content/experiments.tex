
We conduct five experiments. In \Cref{subsec:exp1_MNIST}, we analyze the approximation quality of the Laplace Bridge applied to a BNN on the MNIST \cite{MNIST2010} dataset. Then, we compare the Laplace Bridge to the MC-integral in terms of the out-of-distribution (OOD) detection performance in \Cref{subsec:exp2_numbers}. Their computational costs are compared in \Cref{subsec:exp3_time}. In \Cref{subsec:exp4_toy_dataset} we visualize some properties of the Laplace Bridge and compare it to sampling-based methods. Finally, in \Cref{subsec:exp5_imagenet}, we present analysis on ImageNet \cite{ImageNet2015} to demonstrate the scalability of the Laplace Bridge and the advantage of having a full Dirichlet distribution over softmax outputs.


\setlength{\figwidth}{0.8\textwidth}
\setlength{\figheight}{0.3\textheight}
\begin{figure}[t]
    \centering
    \scriptsize
    % This file was created by tikzplotlib v0.8.2.
\begin{tikzpicture}

\begin{axis}[
height=\figheight,
legend cell align={left},
legend pos=north west,
legend style={draw=white!80.0!black},
tick align=outside,
tick pos=both,
width=\figwidth,
x grid style={white!69.01960784313725!black},
xlabel={MNIST Class},
xmin=-0.45, xmax=9.45,
xtick align=outside,
xtick pos=left,
xtick style={color=black},
y grid style={white!69.01960784313725!black},
ylabel={Mean Variance},
ymin=-0.00546239636314567, ymax=0.12091085826687,
ytick align=outside,
ytick pos=left,
ytick style={color=black},
ytick={-0.02,0,0.02,0.04,0.06,0.08,0.1,0.12,0.14},
yticklabels={−0.02,0.00,0.02,0.04,0.06,0.08,0.10,0.12,0.14}
]
\addplot [semithick, red, mark=*, mark size=3, mark options={solid}, only marks]
table {%
0 0.000485597585793585
1 0.00123227143194526
2 0.000281842483673245
};
\addlegendentry{In Dist}
\addplot [semithick, blue, mark=*, mark size=3, mark options={solid}, only marks]
table {%
3 0.0408942848443985
4 0.115166619420052
5 0.0661759749054909
6 0.0397674925625324
7 0.0582835488021374
8 0.0391143336892128
9 0.0806220844388008
};
\addlegendentry{Out Dist}
\end{axis}

\end{tikzpicture}
    \captionsetup{skip=0pt}
    \caption{Average variance of the Dirichlet distributions of each MNIST class. The in-distribution uncertainty (variance) is nearly nil, while out-of-distribution variance is high.}
    \label{subfig:MNIST_uncertainty}
\end{figure}

\section{Uncertainty estimates on MNIST}
\label{subsec:exp1_MNIST}

We empirically investigate the approximation quality of the Laplace Bridge in a ``real-world'' BNN on the MNIST dataset. A convolutional network with 2 convolutional and 2 fully-connected layers is trained on the first three digits of MNIST (the digits $0$, $1$, and $2$).
Adam optimizer with learning rate $1$e-$3$ and weight decay $5$e-$4$ is used. The batch size is 128.
To obtain the posterior over the weights of this network, we perform a full (all-layer) Laplace approximation using BackPACK \citep{dangel2020backpack} to get the diagonal Hessian. The network is then evaluated on the full test set of MNIST (containing all ten classes).

We present the results in \Cref{subfig:MNIST_uncertainty}. We show for each $k = 1, \dots, K$, the average variance $\frac{1}{D_k} \sum_{i=1}^{D_k} \mathrm{Var}(\pi_k(f_\vtheta(\vx_i)))$ of the resulting Dirichlet distribution over the softmax outputs, where $D_k$ is the number of test points predicted with label $k$. The results show that the variance of the Dirichlet distribution obtained via the Laplace Bridge is useful for uncertainty quantification: The mean variance of the first three classes is close to zero, while that of the other classes is higher. Therefore, these variances are informative for detecting OOD data.
Samples of the in- and out-of-distribution sets reflect this difference in uncertainty, as shown in Figure \ref{fig:MNIST_ID_OOD}. While these results could also be obtained via sampling, the Laplace Bridge provides a computationally lightweight alternative for estimating predictive uncertainty.

\begin{figure}[htb]
    \centering
    \scriptsize

    \captionsetup[subfigure]{labelformat=empty}


    % \setcounter{subfigure}{0}

    \subfloat[In-distribution predictions]{
        \subfloat{\includegraphics[width=0.3\textwidth]{figures/MNIST_cool/MNIST_3Classes_in_dist_coolwarm_0.png}}
        \subfloat{\includegraphics[width=0.3\textwidth]{figures/MNIST_cool/MNIST_3Classes_in_dist_coolwarm_1.png}}
        \subfloat{\includegraphics[width=0.3\textwidth]{figures/MNIST_cool/MNIST_3Classes_in_dist_coolwarm_2.png}}
    }

    \vspace{-1em}

    \subfloat[Out-of-distribution predictions]{
        \subfloat{\includegraphics[width=0.3\textwidth]{figures/MNIST_cool/MNIST_3Classes_out_dist_coolwarm_0.png}}
        \subfloat{\includegraphics[width=0.3\textwidth]{figures/MNIST_cool/MNIST_3Classes_out_dist_coolwarm_1.png}}
        \subfloat{\includegraphics[width=0.3\textwidth]{figures/MNIST_cool/MNIST_3Classes_out_dist_coolwarm_2.png}}
    }

    \caption{\textbf{Top:} In-distribution pdfs. All probability mass is concentrated in the corner of the respective correct class. \textbf{Bottom:} Out-of-distribution pdfs. The probability mass is distributed more equally since the networks' uncertainty about is higher.}
    \label{fig:MNIST_ID_OOD}
\end{figure}\


\begin{table*}[h!]
	\scriptsize
    \centering
    \begin{tabular}{l  l || c c  c | c  c  c}
	     \toprule
         & & \multicolumn{3}{c}{\textbf{Diag Sampling}} &  \multicolumn{3}{c}{\textbf{Laplace Bridge (mean)}}\\
         \textbf{Train} & \textbf{Test} & \textbf{MMC} & \textbf{AUROC} & \textbf{Time} & \textbf{MMC} & \textbf{AUROC} &  \textbf{Time}\\
         \midrule
         MNIST & MNIST & 0.932 $\pm$ 0.007 & - & 6.6 & \textbf{0.987} $\pm$ 0.001 & - & \textbf{0.016} \\
         MNIST & FMNIST & 0.407 $\pm$ 0.010 & 0.989 $\pm$ 0.002 & 6.6 & \textbf{0.377} $\pm$ 0.019 & \textbf{0.994} $\pm$ 0.002 &  \textbf{0.016}\\
         MNIST & notMNIST & \textbf{0.535} $\pm$ 0.018 & 0.958 $\pm$ 0.006 & 12.3 & 0.630 $\pm$ 0.018 & \textbf{0.962} $\pm$ 0.007 &  \textbf{0.029}\\
         MNIST & KMNIST & \textbf{0.500} $\pm$ 0.014 & 0.974 $\pm$ 0.005 & 6.6 & 0.630 $\pm$ 0.018 & \textbf{0.975} $\pm$ 0.004 & \textbf{0.016} \\
         \midrule
         CIFAR-10 & CIFAR-10 & 0.949 $\pm$ 0.001 & - & 6.6 & \textbf{0.969} $\pm$ 0.002 & - & \textbf{0.017} \\
         CIFAR-10 & CIFAR-100 & \textbf{0.724} $\pm$ 0.002 & \textbf{0.884} $\pm$ 0.004 & 6.6 & 0.774 $\pm$ 0.003 & 0.858 $\pm$ 0.004 & \textbf{0.016} \\
         CIFAR-10 & SVHN & \textbf{0.659} $\pm$ 0.028 & \textbf{0.931} $\pm$ 0.007 & 17.0 & 0.704 $\pm$ 0.036 & 0.923 $\pm$ 0.008 & \textbf{0.041} \\
         \midrule
         SVHN & SVHN & 0.986 $\pm$ 0.000 & - & 17.1 & \textbf{0.991} $\pm$ 0.000 & - & \textbf{0.040} \\
         SVHN & CIFAR-10 & 0.537 $\pm$ 0.012 & 0.995 $\pm$ 0.000 & 6.61 & \textbf{0.392} $\pm$ 0.016 & \textbf{0.996} $\pm$ 0.000 & \textbf{0.169} \\
         SVHN & CIFAR-100 & 0.543 $\pm$ 0.009 & 0.994 $\pm$ 0.000 & 6.61 & \textbf{0.400} $\pm$ 0.013 & \textbf{0.996} $\pm$ 0.000 & \textbf{0.016} \\
         \midrule
         CIFAR-100 & CIFAR-100 & \textbf{0.527}s $\pm$ 0.004 & - & 6.68 & 0.263 $\pm$ 0.003 & - & \textbf{0.017} \\
         CIFAR-100 & CIFAR-10 & 0.276 $\pm$ 0.004 & \textbf{0.707} $\pm$ 0.004 & 6.67 & \textbf{0.068} $\pm$ 0.003 & 0.703 $\pm$ 0.003 & \textbf{0.018} \\
         CIFAR-100 & SVHN  & 0.348 $\pm$ 0.014 & 0.647 $\pm$ 0.011 & 17.2 & \textbf{0.074} $\pm$ 0.012 & \textbf{0.661} $\pm$ 0.013 & \textbf{0.040} \\
         \bottomrule
    \end{tabular}
    \caption{OOD detection results. Optimally, the MMC for OOD data is low and the AUROC is high. While there is arguable no clear winner when it comes to discriminating in- and out-distribution data w.r.t. both metrics, the Laplace Bridge is around 400 times faster on average. Time is measured in seconds. Five runs with different seeds per experiment were conducted. 1000 samples were drawn from the Gaussian over the outputs. The (F-, K-, not-)MNIST experiments were done with a Laplace approximation of the entire network while the others only used the last layer.}
    \label{tab:experiments_table}
\end{table*}

\vspace{-0.5em}
\section{OOD detection}
\label{subsec:exp2_numbers}

We compare the performance of the Laplace Bridge to the MC-integral on a standard OOD detection benchmark suite, to test whether the Laplace Bridge gives similar results to the MC sampling method and compare their computational overhead. Following prior literature, we use the standard mean-maximum-confidence (MMC) and area under the ROC-curve (AUROC) metrics \citep{HendycksOODBaseline}. For an in-distribution dataset, a higher MMC value is desirable while for the OOD dataset we want a lower MMC value (optimally, $1/K$ in $K$-class classification problems). For the AUROC metric, the higher the better, since it represents how good a method is for distinguishing in- and out-of-distribution datasets.


The test scenarios are as follows: (i) The same convolutional network as in \Cref{subsec:exp1_MNIST} is trained on the MNIST dataset. To approximate the posterior over the parameter of this network, a full (all-layer) Laplace approximation with the exact Hessian is employed. The OOD datasets for this case are FMNIST \cite{FMNIST2017}, notMNIST \cite{notMNIST2011}, and KMNIST \cite{KMNIST2018}. (ii) For larger datasets, i.e.~CIFAR-10 \cite{CIFAR2009}, SVHN \cite{SVHN2011}, and CIFAR-100 \cite{CIFAR2009}, we use a ResNet-18 network \citep{2015_ResNet}. Since this network is large, \eqref{eq:logit_dist} in conjunction with a full Laplace approximation is too costly. We, therefore, use a last-layer Laplace approximation to obtain the approximate diagonal Gaussian posterior. The OOD datasets for CIFAR-10, SVHN, and CIFAR-100 are SVHN and CIFAR100; CIFAR-10 and CIFAR-100; and SVHN and CIFAR-10, respectively. In all scenarios, the networks are well-trained with $99\%$ accuracy on MNIST, $95.4\%$ on CIFAR-10, $76.6\%$ on CIFAR-100 and $100\%$ on SVHN. For the sampling baseline, we use $1000$ posterior samples to compute the predictive distribution. We use the mean of the Dirichlet to obtain a comparable approximation to the MC-integral. Experiments comparing the Laplace Bridge to a KFAC approximation of the last layer and sampling from all weights of the network can be found in the appendix.

%accuracies: 95% cifar10, 100% SVHN, 59% cifar100
% \begin{table*}[h!]
% 	\scriptsize
%     \centering
%     \begin{tabular}{l  l || c c c c  c  c  c  c}
% 	     \toprule
%          & & \multicolumn{2}{c}{\textbf{MAP}} & \multicolumn{3}{c}{\textbf{Diag Sampling}} &  \multicolumn{3}{c}{\textbf{Dirichlet mode}}\\
%          \textbf{Train} & \textbf{Test} & \textbf{MMC} & \textbf{AUROC} & \textbf{MMC} & \textbf{AUROC} & \textbf{Time} & \textbf{MMC} & \textbf{AUROC} &  \textbf{Time}\\
%          \midrule
%          MNIST & MNIST & \textbf{0.989} $\pm$ 0.001 & - & 0.932 $\pm$ 0.007 & - & 6.6 & 0.987 $\pm$ 0.001 & - & \textbf{0.016} \\
%          MNIST & FMNIST & 0.538 $\pm$ 0.022 & 0.990 $\pm$ 0.001 & 0.407 $\pm$ 0.010 & 0.989 $\pm$ 0.002 & 6.6 & \textbf{0.377} $\pm$ 0.019 & \textbf{0.994} $\pm$ 0.002 &  \textbf{0.016}\\
%          MNIST & notMNIST & 0.706 $\pm$ 0.014 & 0.954 $\pm$ 0.007 & \textbf{0.535} $\pm$ 0.018 & 0.958 $\pm$ 0.006 & 12.3 & 0.630 $\pm$ 0.018 & \textbf{0.962} $\pm$ 0.007 &  \textbf{0.029}\\
%          MNIST & KMNIST & 0.684 $\pm$ 0.015 & 0.974 $\pm$ 0.005 & \textbf{0.500} $\pm$ 0.014 & 0.974 $\pm$ 0.005 & 6.6 & 0.630 $\pm$ 0.018 & \textbf{0.975} $\pm$ 0.004 & \textbf{0.016} \\
%          \midrule
%          CIFAR-10 & CIFAR-10 & \textbf{0.978} $\pm$ 0.001 & - & 0.949 $\pm$ 0.001 & - & 6.6 & 0.969 $\pm$ 0.002 & - & \textbf{0.017} \\
%          CIFAR-10 & CIFAR-100 & 0.828 $\pm$ 0.001 & 0.872 $\pm$ 0.004 & \textbf{0.724} $\pm$ 0.002 & \textbf{0.884} $\pm$ 0.004 & 6.6 & 0.774 $\pm$ 0.003 & 0.858 $\pm$ 0.004 & \textbf{0.016} \\
%          CIFAR-10 & SVHN & 0.777 $\pm$ 0.030 & 0.925 $\pm$ 0.008 & \textbf{0.659} $\pm$ 0.028 & \textbf{0.931} $\pm$ 0.007 & 17.0 & 0.704 $\pm$ 0.036 & 0.923 $\pm$ 0.008 & \textbf{0.041} \\
%          \midrule
%          SVHN & SVHN & \textbf{0.999} $\pm$ 0.000 & - & 0.986 $\pm$ 0.000 & - & 17.1 & 0.991 $\pm$ 0.000 & - & \textbf{0.040} \\
%          SVHN & CIFAR-10 & 0.616 $\pm$ 0.013 & \textbf{0.996} $\pm$ 0.000 & 0.537 $\pm$ 0.012 & 0.995 $\pm$ 0.000 & 6.61 & \textbf{0.392} $\pm$ 0.016 & \textbf{0.996} $\pm$ 0.000 & \textbf{0.169} \\
%          SVHN & CIFAR-100 & 0.621 $\pm$ 0.010 & \textbf{0.996} $\pm$ 0.000 & 0.543 $\pm$ 0.009 & 0.994 $\pm$ 0.000 & 6.61 & \textbf{0.400} $\pm$ 0.013 & \textbf{0.996} $\pm$ 0.000 & \textbf{0.016} \\
%          \midrule
%          CIFAR-100 & CIFAR-100 & \textbf{0.564} $\pm$ 0.018 & - & 0.527 $\pm$ 0.004 & - & 6.68 & 0.263 $\pm$ 0.003 & - & \textbf{0.017} \\
%          CIFAR-100 & CIFAR-10 & 0.298 $\pm$ 0.004 & 0.706 $\pm$ 0.003 & 0.276 $\pm$ 0.004 & \textbf{0.707} $\pm$ 0.004 & 6.67 & \textbf{0.068} $\pm$ 0.003 & 0.703 $\pm$ 0.003 & \textbf{0.018} \\
%          CIFAR-100 & SVHN & 0.372 $\pm$ 0.015 & 0.649 $\pm$ 0.012 & 0.348 $\pm$ 0.014 & 0.647 $\pm$ 0.011 & 17.2 & \textbf{0.074} $\pm$ 0.012 & \textbf{0.661} $\pm$ 0.013 & \textbf{0.040} \\
%          \bottomrule
%     \end{tabular}
%     \caption{Out-of-distribution detection results. A network has been trained on the data set in the \textbf{train} column and is tested on the \textbf{test} column. Optimally, the MMC for out of distribution data is low and the AUROC is high. There is no clear winner when it comes to discriminating in and OOD w.r.t. both metrics. However, the Laplace Bridge is around 400 times faster on average. Time is measured in seconds. Five runs with different seeds per experiment were conducted.}
%     \label{tab:experiments_table}
% \end{table*}


The results are presented in Table \ref{tab:experiments_table}. The Laplace Bridge is competitive to the baseline in terms of the MMC and AUROC metrics. In the case of MNIST and SVHN the Bridge is better than the MC-integral w.r.t. the AUROC metric. Moreover, the Laplace Bridge is also better than the sampling baseline in terms of the MMC metric in the SVHN and CIFAR-100 datasets. The key observation, however, is that the Bridge is on average around $400$ times faster than the sampling baseline, while returning at least competitive, if not even improved fidelity.


\section{Time comparison}
\label{subsec:exp3_time}


We compare the computational cost of the density-estimated $p_\text{sample}$ distribution via sampling and the Dirichlet distribution obtained from the Laplace Bridge $p_\text{LB}$ for approximating the true distribution $p_\text{true}$ over softmax-Gaussian samples\footnote{I.e. samples are obtained by first sampling from a Gaussian and transforming it via the softmax function.}. Different amounts of samples are drawn from the Gaussian, the softmax is applied and the KL divergence between the histogram of the samples with the true distribution is computed. We use KL-divergences $D_\text{KL}(p_\text{true} \Vert p_\text{sample})$ and $D_\text{KL}(p_\text{true} \Vert p_\text{LB})$, respectively, to measure similarity between the approximations and ground truth while the number of samples for $p_\text{sample}$ is increased on a logarithmic scale. The true distribution $p_\text{true}$ is constructed via Monte Carlo with $100$k samples. The experiment is conducted for three different Gaussian distributions over $\R^3$. Since the softmax applied to a Gaussian does not have a closed-form analytic solution, the calculation of the approximation error is not possible and an empirical evaluation via sampling is the best option. The fact that there is no analytic solution is part of the justification for using the Laplace Bridge in the first place.

\Cref{fig:KL_div_samples} suggests that the number of samples required such that the distribution $p_\text{sample}$ is approximating the true distribution $p_\text{true}$ as good as the Dirichlet distribution obtained via the Laplace Bridge is large, i.e. somewhere between $500$ and $10000$. This translates to a wall-clock time advantage of at least a factor of $100$ before sampling becomes competitive in quality with the Laplace Bridge.


\setlength{\figwidth}{0.8\textwidth}
\setlength{\figheight}{0.3\textheight}

\begin{figure}[h!]
    \scriptsize

    \hspace{2em}
    % This file was created by tikzplotlib v0.8.2.
\begin{tikzpicture}

\definecolor{color0}{rgb}{1,0.647058823529412,0}
\definecolor{color1}{rgb}{0.501960784313725,0,0.501960784313725}
\definecolor{color2}{rgb}{0.647058823529412,0.164705882352941,0.164705882352941}

\begin{axis}[
height=\figheight,
legend cell align={left},
legend pos=north east,
legend style={draw=white!80.0!black},
log basis x={10},
tick pos=both,
width=\figwidth,
x grid style={white!69.01960784313725!black},
xlabel={Number of Samples},
xmin=0.562341325190349, xmax=177827.941003892,
xmode=log,
xtick align=inside,
xtick pos=left,
xtick style={color=black},
xtick={0.01,0.1,1,10,100,1000,10000,100000,1000000,10000000},
xticklabels={\(\displaystyle {10^{-2}}\),\(\displaystyle {10^{-1}}\),\(\displaystyle {10^{0}}\),\(\displaystyle {10^{1}}\),\(\displaystyle {10^{2}}\),\(\displaystyle {10^{3}}\),\(\displaystyle {10^{4}}\),\(\displaystyle {10^{5}}\),\(\displaystyle {10^{6}}\),\(\displaystyle {10^{7}}\)},
y grid style={white!69.01960784313725!black},
ylabel={KL Divergence},
ymin=-2.15182824953863, ymax=45.1883932403112,
ytick align=inside,
ytick pos=left,
ytick style={color=black}
]
\path [draw=blue, very thick]
(axis cs:1,41.3982701206161)
--(axis cs:1,41.3982701206161);

\path [draw=blue, very thick]
(axis cs:5,38.8507543346251)
--(axis cs:5,39.0741843320037);

\path [draw=blue, very thick]
(axis cs:10,35.2864009355057)
--(axis cs:10,36.9114764535222);

\path [draw=blue, very thick]
(axis cs:25,29.1974087952039)
--(axis cs:25,31.5444868369423);

\path [draw=blue, very thick]
(axis cs:50,21.2385057852551)
--(axis cs:50,24.0839878754606);

\path [draw=blue, very thick]
(axis cs:75,16.5836595601802)
--(axis cs:75,17.6562791071983);

\path [draw=blue, very thick]
(axis cs:100,12.4266156862388)
--(axis cs:100,15.3071127260996);

\path [draw=blue, very thick]
(axis cs:250,5.64300059313468)
--(axis cs:250,6.24192339554636);

\path [draw=blue, very thick]
(axis cs:500,2.17084407823034)
--(axis cs:500,3.26238974458838);

\path [draw=blue, very thick]
(axis cs:750,1.55585648184509)
--(axis cs:750,2.02968745684394);

\path [draw=blue, very thick]
(axis cs:1000,1.06204505905847)
--(axis cs:1000,1.23981532367538);

\path [draw=blue, very thick]
(axis cs:2500,0.342035635436652)
--(axis cs:2500,0.477450992689462);

\path [draw=blue, very thick]
(axis cs:5000,0.218322482920984)
--(axis cs:5000,0.268408819681547);

\path [draw=blue, very thick]
(axis cs:7500,0.13068971731337)
--(axis cs:7500,0.162126643314091);

\path [draw=blue, very thick]
(axis cs:10000,0.0779525844713338)
--(axis cs:10000,0.103310209035486);

\path [draw=blue, very thick]
(axis cs:25000,0.0294258984006188)
--(axis cs:25000,0.0365894729729816);

\path [draw=blue, very thick]
(axis cs:50000,0.0153424426022053)
--(axis cs:50000,0.0206531839975509);

\path [draw=blue, very thick]
(axis cs:75000,0.0106506508870555)
--(axis cs:75000,0.0136986975478188);

\path [draw=blue, very thick]
(axis cs:100000,0.00730743331987499)
--(axis cs:100000,0.0110392743663015);

\path [draw=color0, very thick]
(axis cs:1,43.0365649907725)
--(axis cs:1,43.0365649907725);

\path [draw=color0, very thick]
(axis cs:5,33.0201579263759)
--(axis cs:5,38.5375789923562);

\path [draw=color0, very thick]
(axis cs:10,29.4207509996895)
--(axis cs:10,34.4540493212375);

\path [draw=color0, very thick]
(axis cs:25,18.2002537726077)
--(axis cs:25,23.7228144750167);

\path [draw=color0, very thick]
(axis cs:50,14.2079535270124)
--(axis cs:50,16.8178855815704);

\path [draw=color0, very thick]
(axis cs:75,10.7978424905023)
--(axis cs:75,12.621459585656);

\path [draw=color0, very thick]
(axis cs:100,9.65565014199387)
--(axis cs:100,10.4393281103967);

\path [draw=color0, very thick]
(axis cs:250,5.76202105533096)
--(axis cs:250,6.9592094137994);

\path [draw=color0, very thick]
(axis cs:500,3.8780405161802)
--(axis cs:500,4.39747176429878);

\path [draw=color0, very thick]
(axis cs:750,2.59710480487391)
--(axis cs:750,3.429512175518);

\path [draw=color0, very thick]
(axis cs:1000,2.27680257720257)
--(axis cs:1000,2.48691127817604);

\path [draw=color0, very thick]
(axis cs:2500,1.1031769857544)
--(axis cs:2500,1.28354880529062);

\path [draw=color0, very thick]
(axis cs:5000,0.509876781796327)
--(axis cs:5000,0.557208452909373);

\path [draw=color0, very thick]
(axis cs:7500,0.335537450129714)
--(axis cs:7500,0.39203101419392);

\path [draw=color0, very thick]
(axis cs:10000,0.220824229571785)
--(axis cs:10000,0.296936046364272);

\path [draw=color0, very thick]
(axis cs:25000,0.0797656998927399)
--(axis cs:25000,0.105835417169598);

\path [draw=color0, very thick]
(axis cs:50000,0.0348721700873487)
--(axis cs:50000,0.0407328414074566);

\path [draw=color0, very thick]
(axis cs:75000,0.0204050997678521)
--(axis cs:75000,0.0233974307740536);

\path [draw=color0, very thick]
(axis cs:100000,0.0135524483867114)
--(axis cs:100000,0.0179733356724642);

\path [draw=green!50.19607843137255!black, very thick]
(axis cs:1,42.9305846157211)
--(axis cs:1,42.9305846157211);

\path [draw=green!50.19607843137255!black, very thick]
(axis cs:5,37.0611658478545)
--(axis cs:5,38.4449378413652);

\path [draw=green!50.19607843137255!black, very thick]
(axis cs:10,34.2813512152306)
--(axis cs:10,35.975071352106);

\path [draw=green!50.19607843137255!black, very thick]
(axis cs:25,24.4005624791702)
--(axis cs:25,26.5692601804098);

\path [draw=green!50.19607843137255!black, very thick]
(axis cs:50,18.7118986152758)
--(axis cs:50,19.7576581143128);

\path [draw=green!50.19607843137255!black, very thick]
(axis cs:75,10.7435568709726)
--(axis cs:75,14.7752725815288);

\path [draw=green!50.19607843137255!black, very thick]
(axis cs:100,7.72596014341502)
--(axis cs:100,9.58686687383905);

\path [draw=green!50.19607843137255!black, very thick]
(axis cs:250,3.00653165881441)
--(axis cs:250,4.27282174432397);

\path [draw=green!50.19607843137255!black, very thick]
(axis cs:500,0.849290756433662)
--(axis cs:500,1.25090202549443);

\path [draw=green!50.19607843137255!black, very thick]
(axis cs:750,0.572699964690209)
--(axis cs:750,0.886085814215126);

\path [draw=green!50.19607843137255!black, very thick]
(axis cs:1000,0.292419473332981)
--(axis cs:1000,0.558853801330423);

\path [draw=green!50.19607843137255!black, very thick]
(axis cs:2500,0.164470506245009)
--(axis cs:2500,0.23014714501126);

\path [draw=green!50.19607843137255!black, very thick]
(axis cs:5000,0.0713618791929187)
--(axis cs:5000,0.125350884327049);

\path [draw=green!50.19607843137255!black, very thick]
(axis cs:7500,0.0440501484025761)
--(axis cs:7500,0.0546591076357259);

\path [draw=green!50.19607843137255!black, very thick]
(axis cs:10000,0.0354904508147201)
--(axis cs:10000,0.0589326299877341);

\path [draw=green!50.19607843137255!black, very thick]
(axis cs:25000,0.0130415843450368)
--(axis cs:25000,0.0200767160525845);

\path [draw=green!50.19607843137255!black, very thick]
(axis cs:50000,0.00530030358239454)
--(axis cs:50000,0.00768632301500861);

\path [draw=green!50.19607843137255!black, very thick]
(axis cs:75000,0.00457916087173392)
--(axis cs:75000,0.0079657724518956);

\path [draw=green!50.19607843137255!black, very thick]
(axis cs:100000,0.00287229403113529)
--(axis cs:100000,0.00594796808345979);

\path [draw=blue, draw opacity=0.7, semithick]
(axis cs:10000,0)
--(axis cs:10000,7);

\path [draw=color0, draw opacity=0.7, semithick]
(axis cs:5000,0)
--(axis cs:5000,7);

\path [draw=green!50.19607843137255!black, draw opacity=0.7, semithick]
(axis cs:500,0)
--(axis cs:500,7);

\addplot [semithick, red, mark=*, mark size=3, mark options={solid}, only marks]
table {%
1 0.0907225415781931
};
\addlegendentry{Laplace Bridge}
\addplot [semithick, color1, mark=*, mark size=3, mark options={solid}, only marks, forget plot]
table {%
1 0.704045819785254
};
\addplot [semithick, color2, mark=*, mark size=3, mark options={solid}, only marks, forget plot]
table {%
1 1.76507413738839
};
\addplot [very thick, red, opacity=0.5, dash pattern=on 1pt off 3pt on 3pt off 3pt, forget plot]
table {%
1 0.0907225415781931
10000 0.0907225415781931
};
\addplot [very thick, color1, opacity=0.5, dash pattern=on 1pt off 3pt on 3pt off 3pt, forget plot]
table {%
1 0.704045819785254
5000 0.704045819785254
};
\addplot [very thick, color2, opacity=0.5, dash pattern=on 1pt off 3pt on 3pt off 3pt, forget plot]
table {%
1 1.76507413738839
500 1.76507413738839
};
\addplot [very thick, blue]
table {%
1 41.3982701206161
5 38.9624693333144
10 36.098938694514
25 30.3709478160731
50 22.6612468303579
75 17.1199693336892
100 13.8668642061692
250 5.94246199434052
500 2.71661691140936
750 1.79277196934452
1000 1.15093019136693
2500 0.409743314063057
5000 0.243365651301265
7500 0.146408180313731
10000 0.0906313967534098
25000 0.0330076856868002
50000 0.0179978132998781
75000 0.0121746742174372
100000 0.00917335384308824
};
\addlegendentry{Monte Carlo}
\addplot [very thick, color0]
table {%
1 43.0365649907725
5 35.778868459366
10 31.9374001604635
25 20.9615341238122
50 15.5129195542914
75 11.7096510380792
100 10.0474891261953
250 6.36061523456518
500 4.13775614023949
750 3.01330849019595
1000 2.3818569276893
2500 1.19336289552251
5000 0.53354261735285
7500 0.363784232161817
10000 0.258880137968029
25000 0.092800558531169
50000 0.0378025057474026
75000 0.0219012652709528
100000 0.0157628920295878
};
\addlegendentry{Monte Carlo}
\addplot [very thick, green!50.19607843137255!black]
table {%
1 42.9305846157211
5 37.7530518446098
10 35.1282112836683
25 25.48491132979
50 19.2347783647943
75 12.7594147262507
100 8.65641350862704
250 3.63967670156919
500 1.05009639096405
750 0.729392889452668
1000 0.425636637331702
2500 0.197308825628134
5000 0.0983563817599838
7500 0.049354628019151
10000 0.0472115404012271
25000 0.0165591501988106
50000 0.00649331329870158
75000 0.00627246666181476
100000 0.00441013105729754
};
\addlegendentry{Monte Carlo}
\end{axis}

\end{tikzpicture}%

    \vspace{2em}

    \hspace{2em}
    % This file was created by tikzplotlib v0.8.2.
\begin{tikzpicture}

\definecolor{color0}{rgb}{1,0.647058823529412,0}
\definecolor{color1}{rgb}{0.501960784313725,0,0.501960784313725}
\definecolor{color2}{rgb}{0.647058823529412,0.164705882352941,0.164705882352941}

\begin{axis}[
height=\figheight,
legend cell align={left},
legend pos=north east,
legend style={draw=white!80.0!black},
log basis x={10},
tick pos=both,
width=\figwidth,
x grid style={white!69.01960784313725!black},
xlabel={Time in s},
xmin=7.66454204975669e-05, xmax=2.31707187675955,
xmode=log,
xtick align=inside,
xtick pos=left,
xtick style={color=black},
xtick={1e-06,1e-05,0.0001,0.001,0.01,0.1,1,10,100},
xticklabels={\(\displaystyle {10^{-6}}\),\(\displaystyle {10^{-5}}\),\(\displaystyle {10^{-4}}\),\(\displaystyle {10^{-3}}\),\(\displaystyle {10^{-2}}\),\(\displaystyle {10^{-1}}\),\(\displaystyle {10^{0}}\),\(\displaystyle {10^{1}}\),\(\displaystyle {10^{2}}\)},
y grid style={white!69.01960784313725!black},
ylabel={KL Divergence},
ymin=-2.15182824953863, ymax=45.1883932403112,
ytick align=inside,
ytick pos=left,
ytick style={color=black}
]
\path [draw=blue, very thick]
(axis cs:0.000479772000016965,41.3982701206161)
--(axis cs:0.000479772000016965,41.3982701206161);

\path [draw=blue, very thick]
(axis cs:0.000462709400039785,38.8507543346251)
--(axis cs:0.000462709400039785,39.0741843320037);

\path [draw=blue, very thick]
(axis cs:0.000490144399987003,35.2864009355057)
--(axis cs:0.000490144399987003,36.9114764535222);

\path [draw=blue, very thick]
(axis cs:0.000664097199978642,29.1974087952039)
--(axis cs:0.000664097199978642,31.5444868369423);

\path [draw=blue, very thick]
(axis cs:0.000871186800009127,21.2385057852551)
--(axis cs:0.000871186800009127,24.0839878754606);

\path [draw=blue, very thick]
(axis cs:0.00112354140001116,16.5836595601802)
--(axis cs:0.00112354140001116,17.6562791071983);

\path [draw=blue, very thick]
(axis cs:0.00152032119999603,12.4266156862388)
--(axis cs:0.00152032119999603,15.3071127260996);

\path [draw=blue, very thick]
(axis cs:0.00299396859998069,5.64300059313468)
--(axis cs:0.00299396859998069,6.24192339554636);

\path [draw=blue, very thick]
(axis cs:0.00561123019997467,2.17084407823034)
--(axis cs:0.00561123019997467,3.26238974458838);

\path [draw=blue, very thick]
(axis cs:0.0100978610000084,1.55585648184509)
--(axis cs:0.0100978610000084,2.02968745684394);

\path [draw=blue, very thick]
(axis cs:0.0126740359999985,1.06204505905847)
--(axis cs:0.0126740359999985,1.23981532367538);

\path [draw=blue, very thick]
(axis cs:0.0308647542000017,0.342035635436652)
--(axis cs:0.0308647542000017,0.477450992689462);

\path [draw=blue, very thick]
(axis cs:0.0561433891999968,0.218322482920984)
--(axis cs:0.0561433891999968,0.268408819681547);

\path [draw=blue, very thick]
(axis cs:0.179833333199986,0.13068971731337)
--(axis cs:0.179833333199986,0.162126643314091);

\path [draw=blue, very thick]
(axis cs:0.230520355399972,0.0779525844713338)
--(axis cs:0.230520355399972,0.103310209035486);

\path [draw=blue, very thick]
(axis cs:0.506458102200008,0.0294258984006188)
--(axis cs:0.506458102200008,0.0365894729729816);

\path [draw=blue, very thick]
(axis cs:0.824743553000008,0.0153424426022053)
--(axis cs:0.824743553000008,0.0206531839975509);

\path [draw=blue, very thick]
(axis cs:1.10083765720001,0.0106506508870555)
--(axis cs:1.10083765720001,0.0136986975478188);

\path [draw=blue, very thick]
(axis cs:1.34734060660001,0.00730743331987499)
--(axis cs:1.34734060660001,0.0110392743663015);

\path [draw=color0, very thick]
(axis cs:0.000437038800032497,43.0365649907725)
--(axis cs:0.000437038800032497,43.0365649907725);

\path [draw=color0, very thick]
(axis cs:0.000435702200047672,33.0201579263759)
--(axis cs:0.000435702200047672,38.5375789923562);

\path [draw=color0, very thick]
(axis cs:0.000535914600027354,29.4207509996895)
--(axis cs:0.000535914600027354,34.4540493212375);

\path [draw=color0, very thick]
(axis cs:0.000777945000004365,18.2002537726077)
--(axis cs:0.000777945000004365,23.7228144750167);

\path [draw=color0, very thick]
(axis cs:0.00119840080003542,14.2079535270124)
--(axis cs:0.00119840080003542,16.8178855815704);

\path [draw=color0, very thick]
(axis cs:0.00127340500002902,10.7978424905023)
--(axis cs:0.00127340500002902,12.621459585656);

\path [draw=color0, very thick]
(axis cs:0.00141027140002734,9.65565014199387)
--(axis cs:0.00141027140002734,10.4393281103967);

\path [draw=color0, very thick]
(axis cs:0.00329267580000305,5.76202105533096)
--(axis cs:0.00329267580000305,6.9592094137994);

\path [draw=color0, very thick]
(axis cs:0.00617591600000651,3.8780405161802)
--(axis cs:0.00617591600000651,4.39747176429878);

\path [draw=color0, very thick]
(axis cs:0.0100022406000107,2.59710480487391)
--(axis cs:0.0100022406000107,3.429512175518);

\path [draw=color0, very thick]
(axis cs:0.0114433376000079,2.27680257720257)
--(axis cs:0.0114433376000079,2.48691127817604);

\path [draw=color0, very thick]
(axis cs:0.0318198007999626,1.1031769857544)
--(axis cs:0.0318198007999626,1.28354880529062);

\path [draw=color0, very thick]
(axis cs:0.0589909576000309,0.509876781796327)
--(axis cs:0.0589909576000309,0.557208452909373);

\path [draw=color0, very thick]
(axis cs:0.165210962800006,0.335537450129714)
--(axis cs:0.165210962800006,0.39203101419392);

\path [draw=color0, very thick]
(axis cs:0.245892468600005,0.220824229571785)
--(axis cs:0.245892468600005,0.296936046364272);

\path [draw=color0, very thick]
(axis cs:0.479274818599993,0.0797656998927399)
--(axis cs:0.479274818599993,0.105835417169598);

\path [draw=color0, very thick]
(axis cs:0.799617780599988,0.0348721700873487)
--(axis cs:0.799617780599988,0.0407328414074566);

\path [draw=color0, very thick]
(axis cs:1.08390964980003,0.0204050997678521)
--(axis cs:1.08390964980003,0.0233974307740536);

\path [draw=color0, very thick]
(axis cs:1.36421115500004,0.0135524483867114)
--(axis cs:1.36421115500004,0.0179733356724642);

\path [draw=green!50.19607843137255!black, very thick]
(axis cs:0.000531081000008271,42.9305846157211)
--(axis cs:0.000531081000008271,42.9305846157211);

\path [draw=green!50.19607843137255!black, very thick]
(axis cs:0.000426537399948757,37.0611658478545)
--(axis cs:0.000426537399948757,38.4449378413652);

\path [draw=green!50.19607843137255!black, very thick]
(axis cs:0.000685418600005505,34.2813512152306)
--(axis cs:0.000685418600005505,35.975071352106);

\path [draw=green!50.19607843137255!black, very thick]
(axis cs:0.000784627799976079,24.4005624791702)
--(axis cs:0.000784627799976079,26.5692601804098);

\path [draw=green!50.19607843137255!black, very thick]
(axis cs:0.00110269080000762,18.7118986152758)
--(axis cs:0.00110269080000762,19.7576581143128);

\path [draw=green!50.19607843137255!black, very thick]
(axis cs:0.00127357879994179,10.7435568709726)
--(axis cs:0.00127357879994179,14.7752725815288);

\path [draw=green!50.19607843137255!black, very thick]
(axis cs:0.00193130859995563,7.72596014341502)
--(axis cs:0.00193130859995563,9.58686687383905);

\path [draw=green!50.19607843137255!black, very thick]
(axis cs:0.00295731420001175,3.00653165881441)
--(axis cs:0.00295731420001175,4.27282174432397);

\path [draw=green!50.19607843137255!black, very thick]
(axis cs:0.00722474840003997,0.849290756433662)
--(axis cs:0.00722474840003997,1.25090202549443);

\path [draw=green!50.19607843137255!black, very thick]
(axis cs:0.00921846979999827,0.572699964690209)
--(axis cs:0.00921846979999827,0.886085814215126);

\path [draw=green!50.19607843137255!black, very thick]
(axis cs:0.0124544410000226,0.292419473332981)
--(axis cs:0.0124544410000226,0.558853801330423);

\path [draw=green!50.19607843137255!black, very thick]
(axis cs:0.0281576174000293,0.164470506245009)
--(axis cs:0.0281576174000293,0.23014714501126);

\path [draw=green!50.19607843137255!black, very thick]
(axis cs:0.0554622965999897,0.0713618791929187)
--(axis cs:0.0554622965999897,0.125350884327049);

\path [draw=green!50.19607843137255!black, very thick]
(axis cs:0.164105063199986,0.0440501484025761)
--(axis cs:0.164105063199986,0.0546591076357259);

\path [draw=green!50.19607843137255!black, very thick]
(axis cs:0.234555237199993,0.0354904508147201)
--(axis cs:0.234555237199993,0.0589326299877341);

\path [draw=green!50.19607843137255!black, very thick]
(axis cs:0.504778742999997,0.0130415843450368)
--(axis cs:0.504778742999997,0.0200767160525845);

\path [draw=green!50.19607843137255!black, very thick]
(axis cs:0.775621087800005,0.00530030358239454)
--(axis cs:0.775621087800005,0.00768632301500861);

\path [draw=green!50.19607843137255!black, very thick]
(axis cs:1.08139886579997,0.00457916087173392)
--(axis cs:1.08139886579997,0.0079657724518956);

\path [draw=green!50.19607843137255!black, very thick]
(axis cs:1.44971468479998,0.00287229403113529)
--(axis cs:1.44971468479998,0.00594796808345979);

\path [draw=blue, draw opacity=0.7, semithick]
(axis cs:0.230520355399972,0)
--(axis cs:0.230520355399972,7);

\path [draw=color0, draw opacity=0.7, semithick]
(axis cs:0.0589909576000309,0)
--(axis cs:0.0589909576000309,7);

\path [draw=green!50.19607843137255!black, draw opacity=0.7, semithick]
(axis cs:0.00722474840003997,0)
--(axis cs:0.00722474840003997,7);

\addplot [semithick, red, mark=*, mark size=3, mark options={solid}, only marks]
table {%
0.00018578000000069 0.0907225415781931
};
\addlegendentry{Laplace Bridge}
\addplot [semithick, color1, mark=*, mark size=3, mark options={solid}, only marks, forget plot]
table {%
0.000126700000000035 0.704045819785254
};
\addplot [semithick, color2, mark=*, mark size=3, mark options={solid}, only marks, forget plot]
table {%
0.000122501999999969 1.76507413738839
};
\addplot [very thick, red, opacity=0.5, dash pattern=on 1pt off 3pt on 3pt off 3pt, forget plot]
table {%
0.00018578000000069 0.0907225415781931
0.230520355399972 0.0907225415781931
};
\addplot [very thick, color1, opacity=0.5, dash pattern=on 1pt off 3pt on 3pt off 3pt, forget plot]
table {%
0.000126700000000035 0.704045819785254
0.0589909576000309 0.704045819785254
};
\addplot [very thick, color2, opacity=0.5, dash pattern=on 1pt off 3pt on 3pt off 3pt, forget plot]
table {%
0.000122501999999969 1.76507413738839
0.00722474840003997 1.76507413738839
};
\addplot [very thick, blue]
table {%
0.000479772000016965 41.3982701206161
0.000462709400039785 38.9624693333144
0.000490144399987003 36.098938694514
0.000664097199978642 30.3709478160731
0.000871186800009127 22.6612468303579
0.00112354140001116 17.1199693336892
0.00152032119999603 13.8668642061692
0.00299396859998069 5.94246199434052
0.00561123019997467 2.71661691140936
0.0100978610000084 1.79277196934452
0.0126740359999985 1.15093019136693
0.0308647542000017 0.409743314063057
0.0561433891999968 0.243365651301265
0.179833333199986 0.146408180313731
0.230520355399972 0.0906313967534098
0.506458102200008 0.0330076856868002
0.824743553000008 0.0179978132998781
1.10083765720001 0.0121746742174372
1.34734060660001 0.00917335384308824
};
\addlegendentry{Monte Carlo}
\addplot [very thick, color0]
table {%
0.000437038800032497 43.0365649907725
0.000435702200047672 35.778868459366
0.000535914600027354 31.9374001604635
0.000777945000004365 20.9615341238122
0.00119840080003542 15.5129195542914
0.00127340500002902 11.7096510380792
0.00141027140002734 10.0474891261953
0.00329267580000305 6.36061523456518
0.00617591600000651 4.13775614023949
0.0100022406000107 3.01330849019595
0.0114433376000079 2.3818569276893
0.0318198007999626 1.19336289552251
0.0589909576000309 0.53354261735285
0.165210962800006 0.363784232161817
0.245892468600005 0.258880137968029
0.479274818599993 0.092800558531169
0.799617780599988 0.0378025057474026
1.08390964980003 0.0219012652709528
1.36421115500004 0.0157628920295878
};
\addlegendentry{Monte Carlo}
\addplot [very thick, green!50.19607843137255!black]
table {%
0.000531081000008271 42.9305846157211
0.000426537399948757 37.7530518446098
0.000685418600005505 35.1282112836683
0.000784627799976079 25.48491132979
0.00110269080000762 19.2347783647943
0.00127357879994179 12.7594147262507
0.00193130859995563 8.65641350862704
0.00295731420001175 3.63967670156919
0.00722474840003997 1.05009639096405
0.00921846979999827 0.729392889452668
0.0124544410000226 0.425636637331702
0.0281576174000293 0.197308825628134
0.0554622965999897 0.0983563817599838
0.164105063199986 0.049354628019151
0.234555237199993 0.0472115404012271
0.504778742999997 0.0165591501988106
0.775621087800005 0.00649331329870158
1.08139886579997 0.00627246666181476
1.44971468479998 0.00441013105729754
};
\addlegendentry{Monte Carlo}
\end{axis}

\end{tikzpicture}%

	\centering
	\caption{KL-divergence plotted against the number of samples (top) and wall-clock time (bottom). Monte Carlo density estimation becomes as good as the Laplace Bridge after around $750$ to $10000$ samples and takes at least $100$ times longer. The three lines represent three different samples.}
	\label{fig:KL_div_samples}
\end{figure}


\section{Toy dataset}
\label{subsec:exp4_toy_dataset}

To understand the properties of the Laplace Bridge we visualize its predictions on a toy dataset. The dataset is generated by drawing from four different 2D Gaussians and the task is for a neural network to classify them. The network is a simple four-layer network with ReLU activations and 100 units per layer. A visualization is created by using different methods for calculating predictive uncertainty for all points on a two-dimensional grid. There are four methods to predict uncertainty that are independent of the Laplace Bridge: the MAP estimate, a diagonal approximation of the Hessian, a Kronecker-factorized approximation of the Hessian and the exact Hessian. Their respective predictive entropy can be found on the left column of Figure \ref{fig:toy_data}. This is compared to the MAP prediction of the Laplace Bridge, its predictive entropy, the variance of the MAP estimate of the Dirichlet and a MAP estimate that is weighted by it's respective variance. These can be found in the right column of Figure \ref{fig:toy_data}. The estimates in the left column get increasingly better since we include more information about the uncertainty. We conclude that the entropy and the variance of the Dirichlet are only marginally better than the original MAP estimate. Reweighing the estimate by the variance improves it slightly. However, the Laplace Bridge is not able to produce a similarly good estimate as a Kronecker-factorized or exact Hessian. 


\newgeometry{top=20mm, bottom=20mm}

\begin{figure}[htb]
	\centering
	\scriptsize
	
	\captionsetup[subfigure]{labelformat=empty}
	
	\subfloat{\includegraphics[width=0.4\textwidth]{figures/toy_data/ent_toy_2d_nn_multiclass_map.pdf}}
	\subfloat{\includegraphics[width=0.4\textwidth]{figures/toy_data/ent_toy_2d_nn_multiclass_LPB_alpha_max.pdf}} \\ [-5ex]	
	\subfloat{\includegraphics[width=0.4\textwidth]{figures/toy_data/ent_toy_2d_nn_multiclass_laplace_diag.pdf}}
	\subfloat{\includegraphics[width=0.4\textwidth]{figures/toy_data/ent_toy_2d_nn_multiclass_LPB_entropy_mode.pdf}} \\ [-5ex]
	\subfloat{\includegraphics[width=0.4\textwidth]{figures/toy_data/ent_toy_2d_nn_multiclass_laplace_kf.pdf}}
	\subfloat{\includegraphics[width=0.4\textwidth]{figures/toy_data/ent_toy_2d_nn_multiclass_LPB_variance_norm_dir.pdf}} \\ [-5ex]	
	\subfloat{\includegraphics[width=0.4\textwidth]{figures/toy_data/ent_toy_2d_nn_multiclass_laplace_exact.pdf}}
	\subfloat{\includegraphics[width=0.4\textwidth]{figures/toy_data/ent_toy_2d_nn_multiclass_LPB_variance_alphas_norm_dir.pdf}} 

	\caption{\textbf{Left:} Entropy of the MAP estimate, a diagonal approximation of the Hessian, a Kronecker-factorized approximation, and the exact Hessian.
	\textbf{Right:} MAP prediction of the Dirichlet coming from the Laplace Bridge, its predictive entropy, the variance of the Dirichlet, and a MAP estimate weighed by its variance. We find that the Laplace Bridge entropy and variance are only marginally better than the MAP estimate but the reweighed version improves it.}
	\label{fig:toy_data}
\end{figure}

\restoregeometry

\section{Uncertainty-aware output ranking on ImageNet}
\label{subsec:exp5_imagenet}

\setlength{\figwidth}{1\textwidth}
\setlength{\figheight}{0.18\textheight}

\begin{figure*}[t]
	\centering
	\includegraphics[width=\figwidth,height=\figheight]{figures/imagenet_images.pdf}
	\includegraphics[width=\figwidth,height=\figheight]{figures/imagenet_marginal_betas.pdf}
	%% This file was created by tikzplotlib v0.9.0.
\begin{tikzpicture}

\definecolor{color0}{rgb}{0.12156862745098,0.466666666666667,0.705882352941177}
\definecolor{color1}{rgb}{1,0.498039215686275,0.0549019607843137}
\definecolor{color2}{rgb}{0.172549019607843,0.627450980392157,0.172549019607843}
\definecolor{color3}{rgb}{0.83921568627451,0.152941176470588,0.156862745098039}
\definecolor{color4}{rgb}{0.580392156862745,0.403921568627451,0.741176470588235}
\definecolor{color5}{rgb}{0.549019607843137,0.337254901960784,0.294117647058824}
\definecolor{color6}{rgb}{0.890196078431372,0.466666666666667,0.76078431372549}
\definecolor{color7}{rgb}{0.737254901960784,0.741176470588235,0.133333333333333}
\definecolor{color8}{rgb}{0.0901960784313725,0.745098039215686,0.811764705882353}

\begin{groupplot}[group style={group size=4 by 1}]
\nextgroupplot[
height=\figheight,
legend cell align={left},
legend style={fill opacity=0.8, draw opacity=1, text opacity=1, draw=white!80!black},
tick align=outside,
tick pos=both,
width=\figwidth,
x grid style={white!69.0196078431373!black},
xmin=-0.0012, xmax=0.0252,
xtick style={color=black},
xtick={-0.005,0,0.005,0.01,0.015,0.02,0.025,0.03},
xticklabels={−0.005,0.000,0.005,0.010,0.015,0.020,0.025,0.030},
y grid style={white!69.0196078431373!black},
ymin=-19.5670691114727, ymax=410.908451340926,
ytick style={color=black}
]
\path [draw=none, fill=blue, fill opacity=0.5]
(axis cs:0,0)
--(axis cs:0.0002,4.22795300245434e-13)
--(axis cs:0.0004,1.37496114174896e-08)
--(axis cs:0.0006,4.44349918357242e-06)
--(axis cs:0.0008,0.000217352763044482)
--(axis cs:0.001,0.00377803209216558)
--(axis cs:0.0012,0.0341292040287655)
--(axis cs:0.0014,0.196252804450894)
--(axis cs:0.0016,0.81078219972159)
--(axis cs:0.0018,2.60195337937075)
--(axis cs:0.002,6.84154460221035)
--(axis cs:0.0022,15.3102330091327)
--(axis cs:0.0024,29.989931796765)
--(axis cs:0.0026,52.5284673552818)
--(axis cs:0.0028,83.6444501303574)
--(axis cs:0.003,122.689795384513)
--(axis cs:0.0032,167.536087950985)
--(axis cs:0.0034,214.832099146605)
--(axis cs:0.0036,260.553774989619)
--(axis cs:0.0038,300.686244763745)
--(axis cs:0.004,331.861083501153)
--(axis cs:0.0042,350.659364364021)
--(axis cs:0.0044,316.671390945087)
--(axis cs:0.0046,278.132151246613)
--(axis cs:0.0048,238.141886324419)
--(axis cs:0.005,199.186771160928)
--(axis cs:0.0052,163.048755076709)
--(axis cs:0.0054,130.83056941505)
--(axis cs:0.0056,103.053206760269)
--(axis cs:0.0058,79.7873803369428)
--(axis cs:0.006,60.7897332615976)
--(axis cs:0.0062,45.6250166719352)
--(axis cs:0.0064,33.764694951832)
--(axis cs:0.0066,24.6593694570613)
--(axis cs:0.0068,17.7868661496403)
--(axis cs:0.007,12.6801601962024)
--(axis cs:0.0072,8.94007607834724)
--(axis cs:0.0074,6.23747618667221)
--(axis cs:0.0076,4.30890608686193)
--(axis cs:0.0078,2.94873796599512)
--(axis cs:0.008,1.99995481107366)
--(axis cs:0.0082,1.34495499920915)
--(axis cs:0.0084,0.897168396988933)
--(axis cs:0.0086,0.59385652774837)
--(axis cs:0.0088,0.390195217766181)
--(axis cs:0.009,0.254575414817491)
--(axis cs:0.0092,0.164974583259305)
--(axis cs:0.0094,0.106220200791715)
--(axis cs:0.0096,0.0679676620931031)
--(axis cs:0.0098,0.0432325808603513)
--(axis cs:0.01,0.0273423296596845)
--(axis cs:0.0102,0.0171977344356047)
--(axis cs:0.0104,0.0107599089132325)
--(axis cs:0.0106,0.00669780790315936)
--(axis cs:0.0108,0.00414881285588331)
--(axis cs:0.011,0.00255774928923221)
--(axis cs:0.0112,0.00156966462258963)
--(axis cs:0.0114,0.000959043980204529)
--(axis cs:0.0116,0.000583468440526314)
--(axis cs:0.0118,0.000353512309656768)
--(axis cs:0.012,0.000213333020725685)
--(axis cs:0.0122,0.000128242850643948)
--(axis cs:0.0124,7.68037201326851e-05)
--(axis cs:0.0126,4.58305761506692e-05)
--(axis cs:0.0128,2.72520977192749e-05)
--(axis cs:0.013,1.61495898117959e-05)
--(axis cs:0.0132,9.53857042332152e-06)
--(axis cs:0.0134,5.6157357557516e-06)
--(axis cs:0.0136,3.29587795741631e-06)
--(axis cs:0.0138,1.92847688063577e-06)
--(axis cs:0.014,1.12505239925148e-06)
--(axis cs:0.0142,6.54456317675556e-07)
--(axis cs:0.0144,3.79639444120029e-07)
--(axis cs:0.0146,2.19622342593566e-07)
--(axis cs:0.0148,1.26714616937668e-07)
--(axis cs:0.015,7.29207563822728e-08)
--(axis cs:0.0152,4.18579500369891e-08)
--(axis cs:0.0154,2.39681255123372e-08)
--(axis cs:0.0156,1.36913240004885e-08)
--(axis cs:0.0158,7.8025585154859e-09)
--(axis cs:0.016,4.43642182094435e-09)
--(axis cs:0.0162,2.51684199065801e-09)
--(axis cs:0.0164,1.4247171282791e-09)
--(axis cs:0.0166,8.04771081382379e-10)
--(axis cs:0.0168,4.53636274462889e-10)
--(axis cs:0.017,2.55184812810522e-10)
--(axis cs:0.0172,1.43262573263436e-10)
--(axis cs:0.0174,8.02712667488462e-11)
--(axis cs:0.0176,4.48905435239373e-11)
--(axis cs:0.0178,2.50573023493042e-11)
--(axis cs:0.018,1.3960957590044e-11)
--(axis cs:0.0182,7.76450428095243e-12)
--(axis cs:0.0184,4.31067763326576e-12)
--(axis cs:0.0186,2.3890530370788e-12)
--(axis cs:0.0188,1.32181029046682e-12)
--(axis cs:0.019,7.30112445413e-13)
--(axis cs:0.0192,4.02625619692298e-13)
--(axis cs:0.0194,2.21675354345489e-13)
--(axis cs:0.0196,1.21857053967287e-13)
--(axis cs:0.0198,6.68826864553411e-14)
--(axis cs:0.02,3.66537700284644e-14)
--(axis cs:0.0202,2.00575217613878e-14)
--(axis cs:0.0204,1.0959758809364e-14)
--(axis cs:0.0206,5.97999627347004e-15)
--(axis cs:0.0208,3.25827538993974e-15)
--(axis cs:0.021,1.77285035089687e-15)
--(axis cs:0.0212,9.63305278865802e-16)
--(axis cs:0.0214,5.22724971007102e-16)
--(axis cs:0.0216,2.83275877689995e-16)
--(axis cs:0.0218,1.53314174574747e-16)
--(axis cs:0.022,8.28705967052991e-17)
--(axis cs:0.0222,4.47376214030539e-17)
--(axis cs:0.0224,2.41217142550633e-17)
--(axis cs:0.0226,1.29901647606133e-17)
--(axis cs:0.0228,6.98715695218015e-18)
--(axis cs:0.023,3.75382142728391e-18)
--(axis cs:0.0232,2.01438141289907e-18)
--(axis cs:0.0234,1.07972288043767e-18)
--(axis cs:0.0236,5.78086370487335e-19)
--(axis cs:0.0238,3.09164855713053e-19)
--(axis cs:0.024,1.65162493477036e-19)
--cycle;
\path [draw=none, fill=blue, fill opacity=0.5]
(axis cs:0,0)
--(axis cs:0.0002,7.61192645735247e-14)
--(axis cs:0.0004,3.61092349697281e-09)
--(axis cs:0.0006,1.45540906761335e-06)
--(axis cs:0.0008,8.32729233244913e-05)
--(axis cs:0.001,0.00163463517706023)
--(axis cs:0.0012,0.016309637005717)
--(axis cs:0.0014,0.102008198661971)
--(axis cs:0.0016,0.453261075131758)
--(axis cs:0.0018,1.5511170267273)
--(axis cs:0.002,4.31978738912353)
--(axis cs:0.0022,10.1830168367104)
--(axis cs:0.0024,20.9167445288748)
--(axis cs:0.0026,38.2727217377937)
--(axis cs:0.0028,63.4606200393569)
--(axis cs:0.003,96.6581704221682)
--(axis cs:0.0032,136.725048683837)
--(axis cs:0.0034,181.227009291322)
--(axis cs:0.0036,226.769934038948)
--(axis cs:0.0038,269.547320550091)
--(axis cs:0.004,305.953317774446)
--(axis cs:0.0042,333.115658709357)
--(axis cs:0.0044,316.671390945087)
--(axis cs:0.0046,278.132151246613)
--(axis cs:0.0048,238.141886324419)
--(axis cs:0.005,199.186771160928)
--(axis cs:0.0052,163.048755076709)
--(axis cs:0.0054,130.83056941505)
--(axis cs:0.0056,103.053206760269)
--(axis cs:0.0058,79.7873803369428)
--(axis cs:0.006,60.7897332615976)
--(axis cs:0.0062,45.6250166719352)
--(axis cs:0.0064,33.764694951832)
--(axis cs:0.0066,24.6593694570613)
--(axis cs:0.0068,17.7868661496403)
--(axis cs:0.007,12.6801601962024)
--(axis cs:0.0072,8.94007607834724)
--(axis cs:0.0074,6.23747618667221)
--(axis cs:0.0076,4.30890608686193)
--(axis cs:0.0078,2.94873796599512)
--(axis cs:0.008,1.99995481107366)
--(axis cs:0.0082,1.34495499920915)
--(axis cs:0.0084,0.897168396988933)
--(axis cs:0.0086,0.59385652774837)
--(axis cs:0.0088,0.390195217766181)
--(axis cs:0.009,0.254575414817491)
--(axis cs:0.0092,0.164974583259305)
--(axis cs:0.0094,0.106220200791715)
--(axis cs:0.0096,0.0679676620931031)
--(axis cs:0.0098,0.0432325808603513)
--(axis cs:0.01,0.0273423296596845)
--(axis cs:0.0102,0.0171977344356047)
--(axis cs:0.0104,0.0107599089132325)
--(axis cs:0.0106,0.00669780790315936)
--(axis cs:0.0108,0.00414881285588331)
--(axis cs:0.011,0.00255774928923221)
--(axis cs:0.0112,0.00156966462258963)
--(axis cs:0.0114,0.000959043980204529)
--(axis cs:0.0116,0.000583468440526314)
--(axis cs:0.0118,0.000353512309656768)
--(axis cs:0.012,0.000213333020725685)
--(axis cs:0.0122,0.000128242850643948)
--(axis cs:0.0124,7.68037201326851e-05)
--(axis cs:0.0126,4.58305761506692e-05)
--(axis cs:0.0128,2.72520977192749e-05)
--(axis cs:0.013,1.61495898117959e-05)
--(axis cs:0.0132,9.53857042332152e-06)
--(axis cs:0.0134,5.6157357557516e-06)
--(axis cs:0.0136,3.29587795741631e-06)
--(axis cs:0.0138,1.92847688063577e-06)
--(axis cs:0.014,1.12505239925148e-06)
--(axis cs:0.0142,6.54456317675556e-07)
--(axis cs:0.0144,3.79639444120029e-07)
--(axis cs:0.0146,2.19622342593566e-07)
--(axis cs:0.0148,1.26714616937668e-07)
--(axis cs:0.015,7.29207563822728e-08)
--(axis cs:0.0152,4.18579500369891e-08)
--(axis cs:0.0154,2.39681255123372e-08)
--(axis cs:0.0156,1.36913240004885e-08)
--(axis cs:0.0158,7.8025585154859e-09)
--(axis cs:0.016,4.43642182094435e-09)
--(axis cs:0.0162,2.51684199065801e-09)
--(axis cs:0.0164,1.4247171282791e-09)
--(axis cs:0.0166,8.04771081382379e-10)
--(axis cs:0.0168,4.53636274462889e-10)
--(axis cs:0.017,2.55184812810522e-10)
--(axis cs:0.0172,1.43262573263436e-10)
--(axis cs:0.0174,8.02712667488462e-11)
--(axis cs:0.0176,4.48905435239373e-11)
--(axis cs:0.0178,2.50573023493042e-11)
--(axis cs:0.018,1.3960957590044e-11)
--(axis cs:0.0182,7.76450428095243e-12)
--(axis cs:0.0184,4.31067763326576e-12)
--(axis cs:0.0186,2.3890530370788e-12)
--(axis cs:0.0188,1.32181029046682e-12)
--(axis cs:0.019,7.30112445413e-13)
--(axis cs:0.0192,4.02625619692298e-13)
--(axis cs:0.0194,2.21675354345489e-13)
--(axis cs:0.0196,1.21857053967287e-13)
--(axis cs:0.0198,6.68826864553411e-14)
--(axis cs:0.02,3.66537700284644e-14)
--(axis cs:0.0202,2.00575217613878e-14)
--(axis cs:0.0204,1.0959758809364e-14)
--(axis cs:0.0206,5.97999627347004e-15)
--(axis cs:0.0208,3.25827538993974e-15)
--(axis cs:0.021,1.77285035089687e-15)
--(axis cs:0.0212,9.63305278865802e-16)
--(axis cs:0.0214,5.22724971007102e-16)
--(axis cs:0.0216,2.83275877689995e-16)
--(axis cs:0.0218,1.53314174574747e-16)
--(axis cs:0.022,8.28705967052991e-17)
--(axis cs:0.0222,4.47376214030539e-17)
--(axis cs:0.0224,2.41217142550633e-17)
--(axis cs:0.0226,1.29901647606133e-17)
--(axis cs:0.0228,6.98715695218015e-18)
--(axis cs:0.023,3.75382142728391e-18)
--(axis cs:0.0232,2.01438141289907e-18)
--(axis cs:0.0234,1.07972288043767e-18)
--(axis cs:0.0236,5.78086370487335e-19)
--(axis cs:0.0238,3.09164855713053e-19)
--(axis cs:0.024,1.65162493477036e-19)
--cycle;
\path [draw=none, fill=blue, fill opacity=0.5]
(axis cs:0,0)
--(axis cs:0.0002,1.63304653410572e-16)
--(axis cs:0.0004,2.90980802338031e-11)
--(axis cs:0.0006,2.54391471865381e-08)
--(axis cs:0.0008,2.52149359912999e-06)
--(axis cs:0.001,7.58076561312621e-05)
--(axis cs:0.0012,0.00107160766687709)
--(axis cs:0.0014,0.0089985803413974)
--(axis cs:0.0016,0.0516110463947514)
--(axis cs:0.0018,0.221227821059011)
--(axis cs:0.002,0.753634925751084)
--(axis cs:0.0022,2.13180511178777)
--(axis cs:0.0024,5.17199814367324)
--(axis cs:0.0026,11.0298671973361)
--(axis cs:0.0028,21.0757198851318)
--(axis cs:0.003,36.6331090584455)
--(axis cs:0.0032,58.6338009969209)
--(axis cs:0.0034,87.2857386482926)
--(axis cs:0.0036,121.857708878618)
--(axis cs:0.0038,160.652447408685)
--(axis cs:0.004,201.183753056193)
--(axis cs:0.0042,240.515622550635)
--(axis cs:0.0044,275.68210296399)
--(axis cs:0.0046,278.132151246613)
--(axis cs:0.0048,238.141886324419)
--(axis cs:0.005,199.186771160928)
--(axis cs:0.0052,163.048755076709)
--(axis cs:0.0054,130.83056941505)
--(axis cs:0.0056,103.053206760269)
--(axis cs:0.0058,79.7873803369428)
--(axis cs:0.006,60.7897332615976)
--(axis cs:0.0062,45.6250166719352)
--(axis cs:0.0064,33.764694951832)
--(axis cs:0.0066,24.6593694570613)
--(axis cs:0.0068,17.7868661496403)
--(axis cs:0.007,12.6801601962024)
--(axis cs:0.0072,8.94007607834724)
--(axis cs:0.0074,6.23747618667221)
--(axis cs:0.0076,4.30890608686193)
--(axis cs:0.0078,2.94873796599512)
--(axis cs:0.008,1.99995481107366)
--(axis cs:0.0082,1.34495499920915)
--(axis cs:0.0084,0.897168396988933)
--(axis cs:0.0086,0.59385652774837)
--(axis cs:0.0088,0.390195217766181)
--(axis cs:0.009,0.254575414817491)
--(axis cs:0.0092,0.164974583259305)
--(axis cs:0.0094,0.106220200791715)
--(axis cs:0.0096,0.0679676620931031)
--(axis cs:0.0098,0.0432325808603513)
--(axis cs:0.01,0.0273423296596845)
--(axis cs:0.0102,0.0171977344356047)
--(axis cs:0.0104,0.0107599089132325)
--(axis cs:0.0106,0.00669780790315936)
--(axis cs:0.0108,0.00414881285588331)
--(axis cs:0.011,0.00255774928923221)
--(axis cs:0.0112,0.00156966462258963)
--(axis cs:0.0114,0.000959043980204529)
--(axis cs:0.0116,0.000583468440526314)
--(axis cs:0.0118,0.000353512309656768)
--(axis cs:0.012,0.000213333020725685)
--(axis cs:0.0122,0.000128242850643948)
--(axis cs:0.0124,7.68037201326851e-05)
--(axis cs:0.0126,4.58305761506692e-05)
--(axis cs:0.0128,2.72520977192749e-05)
--(axis cs:0.013,1.61495898117959e-05)
--(axis cs:0.0132,9.53857042332152e-06)
--(axis cs:0.0134,5.6157357557516e-06)
--(axis cs:0.0136,3.29587795741631e-06)
--(axis cs:0.0138,1.92847688063577e-06)
--(axis cs:0.014,1.12505239925148e-06)
--(axis cs:0.0142,6.54456317675556e-07)
--(axis cs:0.0144,3.79639444120029e-07)
--(axis cs:0.0146,2.19622342593566e-07)
--(axis cs:0.0148,1.26714616937668e-07)
--(axis cs:0.015,7.29207563822728e-08)
--(axis cs:0.0152,4.18579500369891e-08)
--(axis cs:0.0154,2.39681255123372e-08)
--(axis cs:0.0156,1.36913240004885e-08)
--(axis cs:0.0158,7.8025585154859e-09)
--(axis cs:0.016,4.43642182094435e-09)
--(axis cs:0.0162,2.51684199065801e-09)
--(axis cs:0.0164,1.4247171282791e-09)
--(axis cs:0.0166,8.04771081382379e-10)
--(axis cs:0.0168,4.53636274462889e-10)
--(axis cs:0.017,2.55184812810522e-10)
--(axis cs:0.0172,1.43262573263436e-10)
--(axis cs:0.0174,8.02712667488462e-11)
--(axis cs:0.0176,4.48905435239373e-11)
--(axis cs:0.0178,2.50573023493042e-11)
--(axis cs:0.018,1.3960957590044e-11)
--(axis cs:0.0182,7.76450428095243e-12)
--(axis cs:0.0184,4.31067763326576e-12)
--(axis cs:0.0186,2.3890530370788e-12)
--(axis cs:0.0188,1.32181029046682e-12)
--(axis cs:0.019,7.30112445413e-13)
--(axis cs:0.0192,4.02625619692298e-13)
--(axis cs:0.0194,2.21675354345489e-13)
--(axis cs:0.0196,1.21857053967287e-13)
--(axis cs:0.0198,6.68826864553411e-14)
--(axis cs:0.02,3.66537700284644e-14)
--(axis cs:0.0202,2.00575217613878e-14)
--(axis cs:0.0204,1.0959758809364e-14)
--(axis cs:0.0206,5.97999627347004e-15)
--(axis cs:0.0208,3.25827538993974e-15)
--(axis cs:0.021,1.77285035089687e-15)
--(axis cs:0.0212,9.63305278865802e-16)
--(axis cs:0.0214,5.22724971007102e-16)
--(axis cs:0.0216,2.83275877689995e-16)
--(axis cs:0.0218,1.53314174574747e-16)
--(axis cs:0.022,8.28705967052991e-17)
--(axis cs:0.0222,4.47376214030539e-17)
--(axis cs:0.0224,2.41217142550633e-17)
--(axis cs:0.0226,1.29901647606133e-17)
--(axis cs:0.0228,6.98715695218015e-18)
--(axis cs:0.023,3.75382142728391e-18)
--(axis cs:0.0232,2.01438141289907e-18)
--(axis cs:0.0234,1.07972288043767e-18)
--(axis cs:0.0236,5.78086370487335e-19)
--(axis cs:0.0238,3.09164855713053e-19)
--(axis cs:0.024,1.65162493477036e-19)
--cycle;
\path [draw=none, fill=blue, fill opacity=0.5]
(axis cs:0,0)
--(axis cs:0.0002,2.87061962533969e-17)
--(axis cs:0.0004,7.38215500264892e-12)
--(axis cs:0.0006,7.99925233557908e-09)
--(axis cs:0.0008,9.23345890210269e-07)
--(axis cs:0.001,3.12426163997331e-05)
--(axis cs:0.0012,0.000486425585682615)
--(axis cs:0.0014,0.00443227584242255)
--(axis cs:0.0016,0.0272853176447845)
--(axis cs:0.0018,0.124491716808887)
--(axis cs:0.002,0.448457426929982)
--(axis cs:0.0022,1.33431227744341)
--(axis cs:0.0024,3.39008775333182)
--(axis cs:0.0026,7.54334486722043)
--(axis cs:0.0028,14.9917839314935)
--(axis cs:0.003,27.0300203986886)
--(axis cs:0.0032,44.7711936223266)
--(axis cs:0.0034,68.829161876082)
--(axis cs:0.0036,99.0522136164206)
--(axis cs:0.0038,134.390978721141)
--(axis cs:0.004,172.945707968733)
--(axis cs:0.0042,212.187129650739)
--(axis cs:0.0044,249.299540863043)
--(axis cs:0.0046,278.132151246613)
--(axis cs:0.0048,238.141886324419)
--(axis cs:0.005,199.186771160928)
--(axis cs:0.0052,163.048755076709)
--(axis cs:0.0054,130.83056941505)
--(axis cs:0.0056,103.053206760269)
--(axis cs:0.0058,79.7873803369428)
--(axis cs:0.006,60.7897332615976)
--(axis cs:0.0062,45.6250166719352)
--(axis cs:0.0064,33.764694951832)
--(axis cs:0.0066,24.6593694570613)
--(axis cs:0.0068,17.7868661496403)
--(axis cs:0.007,12.6801601962024)
--(axis cs:0.0072,8.94007607834724)
--(axis cs:0.0074,6.23747618667221)
--(axis cs:0.0076,4.30890608686193)
--(axis cs:0.0078,2.94873796599512)
--(axis cs:0.008,1.99995481107366)
--(axis cs:0.0082,1.34495499920915)
--(axis cs:0.0084,0.897168396988933)
--(axis cs:0.0086,0.59385652774837)
--(axis cs:0.0088,0.390195217766181)
--(axis cs:0.009,0.254575414817491)
--(axis cs:0.0092,0.164974583259305)
--(axis cs:0.0094,0.106220200791715)
--(axis cs:0.0096,0.0679676620931031)
--(axis cs:0.0098,0.0432325808603513)
--(axis cs:0.01,0.0273423296596845)
--(axis cs:0.0102,0.0171977344356047)
--(axis cs:0.0104,0.0107599089132325)
--(axis cs:0.0106,0.00669780790315936)
--(axis cs:0.0108,0.00414881285588331)
--(axis cs:0.011,0.00255774928923221)
--(axis cs:0.0112,0.00156966462258963)
--(axis cs:0.0114,0.000959043980204529)
--(axis cs:0.0116,0.000583468440526314)
--(axis cs:0.0118,0.000353512309656768)
--(axis cs:0.012,0.000213333020725685)
--(axis cs:0.0122,0.000128242850643948)
--(axis cs:0.0124,7.68037201326851e-05)
--(axis cs:0.0126,4.58305761506692e-05)
--(axis cs:0.0128,2.72520977192749e-05)
--(axis cs:0.013,1.61495898117959e-05)
--(axis cs:0.0132,9.53857042332152e-06)
--(axis cs:0.0134,5.6157357557516e-06)
--(axis cs:0.0136,3.29587795741631e-06)
--(axis cs:0.0138,1.92847688063577e-06)
--(axis cs:0.014,1.12505239925148e-06)
--(axis cs:0.0142,6.54456317675556e-07)
--(axis cs:0.0144,3.79639444120029e-07)
--(axis cs:0.0146,2.19622342593566e-07)
--(axis cs:0.0148,1.26714616937668e-07)
--(axis cs:0.015,7.29207563822728e-08)
--(axis cs:0.0152,4.18579500369891e-08)
--(axis cs:0.0154,2.39681255123372e-08)
--(axis cs:0.0156,1.36913240004885e-08)
--(axis cs:0.0158,7.8025585154859e-09)
--(axis cs:0.016,4.43642182094435e-09)
--(axis cs:0.0162,2.51684199065801e-09)
--(axis cs:0.0164,1.4247171282791e-09)
--(axis cs:0.0166,8.04771081382379e-10)
--(axis cs:0.0168,4.53636274462889e-10)
--(axis cs:0.017,2.55184812810522e-10)
--(axis cs:0.0172,1.43262573263436e-10)
--(axis cs:0.0174,8.02712667488462e-11)
--(axis cs:0.0176,4.48905435239373e-11)
--(axis cs:0.0178,2.50573023493042e-11)
--(axis cs:0.018,1.3960957590044e-11)
--(axis cs:0.0182,7.76450428095243e-12)
--(axis cs:0.0184,4.31067763326576e-12)
--(axis cs:0.0186,2.3890530370788e-12)
--(axis cs:0.0188,1.32181029046682e-12)
--(axis cs:0.019,7.30112445413e-13)
--(axis cs:0.0192,4.02625619692298e-13)
--(axis cs:0.0194,2.21675354345489e-13)
--(axis cs:0.0196,1.21857053967287e-13)
--(axis cs:0.0198,6.68826864553411e-14)
--(axis cs:0.02,3.66537700284644e-14)
--(axis cs:0.0202,2.00575217613878e-14)
--(axis cs:0.0204,1.0959758809364e-14)
--(axis cs:0.0206,5.97999627347004e-15)
--(axis cs:0.0208,3.25827538993974e-15)
--(axis cs:0.021,1.77285035089687e-15)
--(axis cs:0.0212,9.63305278865802e-16)
--(axis cs:0.0214,5.22724971007102e-16)
--(axis cs:0.0216,2.83275877689995e-16)
--(axis cs:0.0218,1.53314174574747e-16)
--(axis cs:0.022,8.28705967052991e-17)
--(axis cs:0.0222,4.47376214030539e-17)
--(axis cs:0.0224,2.41217142550633e-17)
--(axis cs:0.0226,1.29901647606133e-17)
--(axis cs:0.0228,6.98715695218015e-18)
--(axis cs:0.023,3.75382142728391e-18)
--(axis cs:0.0232,2.01438141289907e-18)
--(axis cs:0.0234,1.07972288043767e-18)
--(axis cs:0.0236,5.78086370487335e-19)
--(axis cs:0.0238,3.09164855713053e-19)
--(axis cs:0.024,1.65162493477036e-19)
--cycle;
\path [draw=none, fill=blue, fill opacity=0.5]
(axis cs:0,0)
--(axis cs:0.0002,4.77632767978611e-17)
--(axis cs:0.0004,1.10350468092225e-11)
--(axis cs:0.0006,1.12309897522798e-08)
--(axis cs:0.0008,1.23997965836469e-06)
--(axis cs:0.001,4.05331104838323e-05)
--(axis cs:0.0012,0.000613522820835944)
--(axis cs:0.0014,0.00545862533187201)
--(axis cs:0.0016,0.032916327874361)
--(axis cs:0.0018,0.147470611530503)
--(axis cs:0.002,0.522639852402313)
--(axis cs:0.0022,1.53224950990862)
--(axis cs:0.0024,3.84087832479369)
--(axis cs:0.0026,8.44109535473997)
--(axis cs:0.0028,16.5844717211536)
--(axis cs:0.003,29.5836281445369)
--(axis cs:0.0032,48.5131391024827)
--(axis cs:0.0034,73.884182008032)
--(axis cs:0.0036,105.388702355709)
--(axis cs:0.0038,141.794147211553)
--(axis cs:0.004,181.026617480686)
--(axis cs:0.0042,220.426658502137)
--(axis cs:0.0044,257.117315904375)
--(axis cs:0.0046,278.132151246613)
--(axis cs:0.0048,238.141886324419)
--(axis cs:0.005,199.186771160928)
--(axis cs:0.0052,163.048755076709)
--(axis cs:0.0054,130.83056941505)
--(axis cs:0.0056,103.053206760269)
--(axis cs:0.0058,79.7873803369428)
--(axis cs:0.006,60.7897332615976)
--(axis cs:0.0062,45.6250166719352)
--(axis cs:0.0064,33.764694951832)
--(axis cs:0.0066,24.6593694570613)
--(axis cs:0.0068,17.7868661496403)
--(axis cs:0.007,12.6801601962024)
--(axis cs:0.0072,8.94007607834724)
--(axis cs:0.0074,6.23747618667221)
--(axis cs:0.0076,4.30890608686193)
--(axis cs:0.0078,2.94873796599512)
--(axis cs:0.008,1.99995481107366)
--(axis cs:0.0082,1.34495499920915)
--(axis cs:0.0084,0.897168396988933)
--(axis cs:0.0086,0.59385652774837)
--(axis cs:0.0088,0.390195217766181)
--(axis cs:0.009,0.254575414817491)
--(axis cs:0.0092,0.164974583259305)
--(axis cs:0.0094,0.106220200791715)
--(axis cs:0.0096,0.0679676620931031)
--(axis cs:0.0098,0.0432325808603513)
--(axis cs:0.01,0.0273423296596845)
--(axis cs:0.0102,0.0171977344356047)
--(axis cs:0.0104,0.0107599089132325)
--(axis cs:0.0106,0.00669780790315936)
--(axis cs:0.0108,0.00414881285588331)
--(axis cs:0.011,0.00255774928923221)
--(axis cs:0.0112,0.00156966462258963)
--(axis cs:0.0114,0.000959043980204529)
--(axis cs:0.0116,0.000583468440526314)
--(axis cs:0.0118,0.000353512309656768)
--(axis cs:0.012,0.000213333020725685)
--(axis cs:0.0122,0.000128242850643948)
--(axis cs:0.0124,7.68037201326851e-05)
--(axis cs:0.0126,4.58305761506692e-05)
--(axis cs:0.0128,2.72520977192749e-05)
--(axis cs:0.013,1.61495898117959e-05)
--(axis cs:0.0132,9.53857042332152e-06)
--(axis cs:0.0134,5.6157357557516e-06)
--(axis cs:0.0136,3.29587795741631e-06)
--(axis cs:0.0138,1.92847688063577e-06)
--(axis cs:0.014,1.12505239925148e-06)
--(axis cs:0.0142,6.54456317675556e-07)
--(axis cs:0.0144,3.79639444120029e-07)
--(axis cs:0.0146,2.19622342593566e-07)
--(axis cs:0.0148,1.26714616937668e-07)
--(axis cs:0.015,7.29207563822728e-08)
--(axis cs:0.0152,4.18579500369891e-08)
--(axis cs:0.0154,2.39681255123372e-08)
--(axis cs:0.0156,1.36913240004885e-08)
--(axis cs:0.0158,7.8025585154859e-09)
--(axis cs:0.016,4.43642182094435e-09)
--(axis cs:0.0162,2.51684199065801e-09)
--(axis cs:0.0164,1.4247171282791e-09)
--(axis cs:0.0166,8.04771081382379e-10)
--(axis cs:0.0168,4.53636274462889e-10)
--(axis cs:0.017,2.55184812810522e-10)
--(axis cs:0.0172,1.43262573263436e-10)
--(axis cs:0.0174,8.02712667488462e-11)
--(axis cs:0.0176,4.48905435239373e-11)
--(axis cs:0.0178,2.50573023493042e-11)
--(axis cs:0.018,1.3960957590044e-11)
--(axis cs:0.0182,7.76450428095243e-12)
--(axis cs:0.0184,4.31067763326576e-12)
--(axis cs:0.0186,2.3890530370788e-12)
--(axis cs:0.0188,1.32181029046682e-12)
--(axis cs:0.019,7.30112445413e-13)
--(axis cs:0.0192,4.02625619692298e-13)
--(axis cs:0.0194,2.21675354345489e-13)
--(axis cs:0.0196,1.21857053967287e-13)
--(axis cs:0.0198,6.68826864553411e-14)
--(axis cs:0.02,3.66537700284644e-14)
--(axis cs:0.0202,2.00575217613878e-14)
--(axis cs:0.0204,1.0959758809364e-14)
--(axis cs:0.0206,5.97999627347004e-15)
--(axis cs:0.0208,3.25827538993974e-15)
--(axis cs:0.021,1.77285035089687e-15)
--(axis cs:0.0212,9.63305278865802e-16)
--(axis cs:0.0214,5.22724971007102e-16)
--(axis cs:0.0216,2.83275877689995e-16)
--(axis cs:0.0218,1.53314174574747e-16)
--(axis cs:0.022,8.28705967052991e-17)
--(axis cs:0.0222,4.47376214030539e-17)
--(axis cs:0.0224,2.41217142550633e-17)
--(axis cs:0.0226,1.29901647606133e-17)
--(axis cs:0.0228,6.98715695218015e-18)
--(axis cs:0.023,3.75382142728391e-18)
--(axis cs:0.0232,2.01438141289907e-18)
--(axis cs:0.0234,1.07972288043767e-18)
--(axis cs:0.0236,5.78086370487335e-19)
--(axis cs:0.0238,3.09164855713053e-19)
--(axis cs:0.024,1.65162493477036e-19)
--cycle;
\path [draw=none, fill=blue, fill opacity=0.5]
(axis cs:0,0)
--(axis cs:0.0002,2.50033281924293e-17)
--(axis cs:0.0004,6.61920264526063e-12)
--(axis cs:0.0006,7.29531684613714e-09)
--(axis cs:0.0008,8.52296949347681e-07)
--(axis cs:0.001,2.91093689969302e-05)
--(axis cs:0.0012,0.000456687181093467)
--(axis cs:0.0014,0.00418826597701668)
--(axis cs:0.0016,0.025927869467974)
--(axis cs:0.0018,0.118883729553988)
--(axis cs:0.002,0.43015163399355)
--(axis cs:0.0022,1.28497176113775)
--(axis cs:0.0024,3.27666446986953)
--(axis cs:0.0026,7.31548767742108)
--(axis cs:0.0028,14.5842153965101)
--(axis cs:0.003,26.3714255968948)
--(axis cs:0.0032,43.7988227032249)
--(axis cs:0.0034,67.5058869162191)
--(axis cs:0.0036,97.3813437364043)
--(axis cs:0.0038,132.424374144004)
--(axis cs:0.004,170.782523087049)
--(axis cs:0.0042,209.96311643966)
--(axis cs:0.0044,247.169302110898)
--(axis cs:0.0046,278.132151246613)
--(axis cs:0.0048,238.141886324419)
--(axis cs:0.005,199.186771160928)
--(axis cs:0.0052,163.048755076709)
--(axis cs:0.0054,130.83056941505)
--(axis cs:0.0056,103.053206760269)
--(axis cs:0.0058,79.7873803369428)
--(axis cs:0.006,60.7897332615976)
--(axis cs:0.0062,45.6250166719352)
--(axis cs:0.0064,33.764694951832)
--(axis cs:0.0066,24.6593694570613)
--(axis cs:0.0068,17.7868661496403)
--(axis cs:0.007,12.6801601962024)
--(axis cs:0.0072,8.94007607834724)
--(axis cs:0.0074,6.23747618667221)
--(axis cs:0.0076,4.30890608686193)
--(axis cs:0.0078,2.94873796599512)
--(axis cs:0.008,1.99995481107366)
--(axis cs:0.0082,1.34495499920915)
--(axis cs:0.0084,0.897168396988933)
--(axis cs:0.0086,0.59385652774837)
--(axis cs:0.0088,0.390195217766181)
--(axis cs:0.009,0.254575414817491)
--(axis cs:0.0092,0.164974583259305)
--(axis cs:0.0094,0.106220200791715)
--(axis cs:0.0096,0.0679676620931031)
--(axis cs:0.0098,0.0432325808603513)
--(axis cs:0.01,0.0273423296596845)
--(axis cs:0.0102,0.0171977344356047)
--(axis cs:0.0104,0.0107599089132325)
--(axis cs:0.0106,0.00669780790315936)
--(axis cs:0.0108,0.00414881285588331)
--(axis cs:0.011,0.00255774928923221)
--(axis cs:0.0112,0.00156966462258963)
--(axis cs:0.0114,0.000959043980204529)
--(axis cs:0.0116,0.000583468440526314)
--(axis cs:0.0118,0.000353512309656768)
--(axis cs:0.012,0.000213333020725685)
--(axis cs:0.0122,0.000128242850643948)
--(axis cs:0.0124,7.68037201326851e-05)
--(axis cs:0.0126,4.58305761506692e-05)
--(axis cs:0.0128,2.72520977192749e-05)
--(axis cs:0.013,1.61495898117959e-05)
--(axis cs:0.0132,9.53857042332152e-06)
--(axis cs:0.0134,5.6157357557516e-06)
--(axis cs:0.0136,3.29587795741631e-06)
--(axis cs:0.0138,1.92847688063577e-06)
--(axis cs:0.014,1.12505239925148e-06)
--(axis cs:0.0142,6.54456317675556e-07)
--(axis cs:0.0144,3.79639444120029e-07)
--(axis cs:0.0146,2.19622342593566e-07)
--(axis cs:0.0148,1.26714616937668e-07)
--(axis cs:0.015,7.29207563822728e-08)
--(axis cs:0.0152,4.18579500369891e-08)
--(axis cs:0.0154,2.39681255123372e-08)
--(axis cs:0.0156,1.36913240004885e-08)
--(axis cs:0.0158,7.8025585154859e-09)
--(axis cs:0.016,4.43642182094435e-09)
--(axis cs:0.0162,2.51684199065801e-09)
--(axis cs:0.0164,1.4247171282791e-09)
--(axis cs:0.0166,8.04771081382379e-10)
--(axis cs:0.0168,4.53636274462889e-10)
--(axis cs:0.017,2.55184812810522e-10)
--(axis cs:0.0172,1.43262573263436e-10)
--(axis cs:0.0174,8.02712667488462e-11)
--(axis cs:0.0176,4.48905435239373e-11)
--(axis cs:0.0178,2.50573023493042e-11)
--(axis cs:0.018,1.3960957590044e-11)
--(axis cs:0.0182,7.76450428095243e-12)
--(axis cs:0.0184,4.31067763326576e-12)
--(axis cs:0.0186,2.3890530370788e-12)
--(axis cs:0.0188,1.32181029046682e-12)
--(axis cs:0.019,7.30112445413e-13)
--(axis cs:0.0192,4.02625619692298e-13)
--(axis cs:0.0194,2.21675354345489e-13)
--(axis cs:0.0196,1.21857053967287e-13)
--(axis cs:0.0198,6.68826864553411e-14)
--(axis cs:0.02,3.66537700284644e-14)
--(axis cs:0.0202,2.00575217613878e-14)
--(axis cs:0.0204,1.0959758809364e-14)
--(axis cs:0.0206,5.97999627347004e-15)
--(axis cs:0.0208,3.25827538993974e-15)
--(axis cs:0.021,1.77285035089687e-15)
--(axis cs:0.0212,9.63305278865802e-16)
--(axis cs:0.0214,5.22724971007102e-16)
--(axis cs:0.0216,2.83275877689995e-16)
--(axis cs:0.0218,1.53314174574747e-16)
--(axis cs:0.022,8.28705967052991e-17)
--(axis cs:0.0222,4.47376214030539e-17)
--(axis cs:0.0224,2.41217142550633e-17)
--(axis cs:0.0226,1.29901647606133e-17)
--(axis cs:0.0228,6.98715695218015e-18)
--(axis cs:0.023,3.75382142728391e-18)
--(axis cs:0.0232,2.01438141289907e-18)
--(axis cs:0.0234,1.07972288043767e-18)
--(axis cs:0.0236,5.78086370487335e-19)
--(axis cs:0.0238,3.09164855713053e-19)
--(axis cs:0.024,1.65162493477036e-19)
--cycle;
\path [draw=none, fill=blue, fill opacity=0.5]
(axis cs:0,0)
--(axis cs:0.0002,4.38630231113233e-32)
--(axis cs:0.0004,9.42164546775413e-24)
--(axis cs:0.0006,5.22996289695702e-19)
--(axis cs:0.0008,9.86216555458081e-16)
--(axis cs:0.001,2.9143627354464e-13)
--(axis cs:0.0012,2.66669889722348e-11)
--(axis cs:0.0014,1.08650456690335e-09)
--(axis cs:0.0016,2.44844210199571e-08)
--(axis cs:0.0018,3.50988320156314e-07)
--(axis cs:0.002,3.52140105748224e-06)
--(axis cs:0.0022,2.64692151139118e-05)
--(axis cs:0.0024,0.000156751672867648)
--(axis cs:0.0026,0.000759830128964595)
--(axis cs:0.0028,0.00310561691820713)
--(axis cs:0.003,0.0109579090659142)
--(axis cs:0.0032,0.034016993111393)
--(axis cs:0.0034,0.0943609589409247)
--(axis cs:0.0036,0.236919036551997)
--(axis cs:0.0038,0.544246235965627)
--(axis cs:0.004,1.15434405660579)
--(axis cs:0.0042,2.27822422319553)
--(axis cs:0.0044,4.21196189035521)
--(axis cs:0.0046,7.33697307600828)
--(axis cs:0.0048,12.1029173241866)
--(axis cs:0.005,18.9902597620454)
--(axis cs:0.0052,28.4537592035236)
--(axis cs:0.0054,40.8529986597383)
--(axis cs:0.0056,56.3802313963309)
--(axis cs:0.0058,74.9980614577203)
--(axis cs:0.006,60.7897332615976)
--(axis cs:0.0062,45.6250166719352)
--(axis cs:0.0064,33.764694951832)
--(axis cs:0.0066,24.6593694570613)
--(axis cs:0.0068,17.7868661496403)
--(axis cs:0.007,12.6801601962024)
--(axis cs:0.0072,8.94007607834724)
--(axis cs:0.0074,6.23747618667221)
--(axis cs:0.0076,4.30890608686193)
--(axis cs:0.0078,2.94873796599512)
--(axis cs:0.008,1.99995481107366)
--(axis cs:0.0082,1.34495499920915)
--(axis cs:0.0084,0.897168396988933)
--(axis cs:0.0086,0.59385652774837)
--(axis cs:0.0088,0.390195217766181)
--(axis cs:0.009,0.254575414817491)
--(axis cs:0.0092,0.164974583259305)
--(axis cs:0.0094,0.106220200791715)
--(axis cs:0.0096,0.0679676620931031)
--(axis cs:0.0098,0.0432325808603513)
--(axis cs:0.01,0.0273423296596845)
--(axis cs:0.0102,0.0171977344356047)
--(axis cs:0.0104,0.0107599089132325)
--(axis cs:0.0106,0.00669780790315936)
--(axis cs:0.0108,0.00414881285588331)
--(axis cs:0.011,0.00255774928923221)
--(axis cs:0.0112,0.00156966462258963)
--(axis cs:0.0114,0.000959043980204529)
--(axis cs:0.0116,0.000583468440526314)
--(axis cs:0.0118,0.000353512309656768)
--(axis cs:0.012,0.000213333020725685)
--(axis cs:0.0122,0.000128242850643948)
--(axis cs:0.0124,7.68037201326851e-05)
--(axis cs:0.0126,4.58305761506692e-05)
--(axis cs:0.0128,2.72520977192749e-05)
--(axis cs:0.013,1.61495898117959e-05)
--(axis cs:0.0132,9.53857042332152e-06)
--(axis cs:0.0134,5.6157357557516e-06)
--(axis cs:0.0136,3.29587795741631e-06)
--(axis cs:0.0138,1.92847688063577e-06)
--(axis cs:0.014,1.12505239925148e-06)
--(axis cs:0.0142,6.54456317675556e-07)
--(axis cs:0.0144,3.79639444120029e-07)
--(axis cs:0.0146,2.19622342593566e-07)
--(axis cs:0.0148,1.26714616937668e-07)
--(axis cs:0.015,7.29207563822728e-08)
--(axis cs:0.0152,4.18579500369891e-08)
--(axis cs:0.0154,2.39681255123372e-08)
--(axis cs:0.0156,1.36913240004885e-08)
--(axis cs:0.0158,7.8025585154859e-09)
--(axis cs:0.016,4.43642182094435e-09)
--(axis cs:0.0162,2.51684199065801e-09)
--(axis cs:0.0164,1.4247171282791e-09)
--(axis cs:0.0166,8.04771081382379e-10)
--(axis cs:0.0168,4.53636274462889e-10)
--(axis cs:0.017,2.55184812810522e-10)
--(axis cs:0.0172,1.43262573263436e-10)
--(axis cs:0.0174,8.02712667488462e-11)
--(axis cs:0.0176,4.48905435239373e-11)
--(axis cs:0.0178,2.50573023493042e-11)
--(axis cs:0.018,1.3960957590044e-11)
--(axis cs:0.0182,7.76450428095243e-12)
--(axis cs:0.0184,4.31067763326576e-12)
--(axis cs:0.0186,2.3890530370788e-12)
--(axis cs:0.0188,1.32181029046682e-12)
--(axis cs:0.019,7.30112445413e-13)
--(axis cs:0.0192,4.02625619692298e-13)
--(axis cs:0.0194,2.21675354345489e-13)
--(axis cs:0.0196,1.21857053967287e-13)
--(axis cs:0.0198,6.68826864553411e-14)
--(axis cs:0.02,3.66537700284644e-14)
--(axis cs:0.0202,2.00575217613878e-14)
--(axis cs:0.0204,1.0959758809364e-14)
--(axis cs:0.0206,5.97999627347004e-15)
--(axis cs:0.0208,3.25827538993974e-15)
--(axis cs:0.021,1.77285035089687e-15)
--(axis cs:0.0212,9.63305278865802e-16)
--(axis cs:0.0214,5.22724971007102e-16)
--(axis cs:0.0216,2.83275877689995e-16)
--(axis cs:0.0218,1.53314174574747e-16)
--(axis cs:0.022,8.28705967052991e-17)
--(axis cs:0.0222,4.47376214030539e-17)
--(axis cs:0.0224,2.41217142550633e-17)
--(axis cs:0.0226,1.29901647606133e-17)
--(axis cs:0.0228,6.98715695218015e-18)
--(axis cs:0.023,3.75382142728391e-18)
--(axis cs:0.0232,2.01438141289907e-18)
--(axis cs:0.0234,1.07972288043767e-18)
--(axis cs:0.0236,5.78086370487335e-19)
--(axis cs:0.0238,3.09164855713053e-19)
--(axis cs:0.024,1.65162493477036e-19)
--cycle;
\path [draw=none, fill=blue, fill opacity=0.5]
(axis cs:0,0)
--(axis cs:0.0002,4.28964223063707e-25)
--(axis cs:0.0004,4.23925669893049e-18)
--(axis cs:0.0006,3.8843035174309e-14)
--(axis cs:0.0008,2.03980960361959e-11)
--(axis cs:0.001,2.23578413477403e-09)
--(axis cs:0.0012,9.09612932925116e-08)
--(axis cs:0.0014,1.86740983355346e-06)
--(axis cs:0.0016,2.32374652780952e-05)
--(axis cs:0.0018,0.00019726676995222)
--(axis cs:0.002,0.00123850159771812)
--(axis cs:0.0022,0.00609150384571705)
--(axis cs:0.0024,0.0244907822733581)
--(axis cs:0.0026,0.0831291604329364)
--(axis cs:0.0028,0.244273624310176)
--(axis cs:0.003,0.633888506344497)
--(axis cs:0.0032,1.47614150045633)
--(axis cs:0.0034,3.12549923236712)
--(axis cs:0.0036,6.08284419664121)
--(axis cs:0.0038,10.9810360144135)
--(axis cs:0.004,18.5299350809066)
--(axis cs:0.0042,29.4207126030825)
--(axis cs:0.0044,44.2015281358291)
--(axis cs:0.0046,63.1472686753469)
--(axis cs:0.0048,86.1512421387252)
--(axis cs:0.005,112.664536610035)
--(axis cs:0.0052,141.699695822165)
--(axis cs:0.0054,130.83056941505)
--(axis cs:0.0056,103.053206760269)
--(axis cs:0.0058,79.7873803369428)
--(axis cs:0.006,60.7897332615976)
--(axis cs:0.0062,45.6250166719352)
--(axis cs:0.0064,33.764694951832)
--(axis cs:0.0066,24.6593694570613)
--(axis cs:0.0068,17.7868661496403)
--(axis cs:0.007,12.6801601962024)
--(axis cs:0.0072,8.94007607834724)
--(axis cs:0.0074,6.23747618667221)
--(axis cs:0.0076,4.30890608686193)
--(axis cs:0.0078,2.94873796599512)
--(axis cs:0.008,1.99995481107366)
--(axis cs:0.0082,1.34495499920915)
--(axis cs:0.0084,0.897168396988933)
--(axis cs:0.0086,0.59385652774837)
--(axis cs:0.0088,0.390195217766181)
--(axis cs:0.009,0.254575414817491)
--(axis cs:0.0092,0.164974583259305)
--(axis cs:0.0094,0.106220200791715)
--(axis cs:0.0096,0.0679676620931031)
--(axis cs:0.0098,0.0432325808603513)
--(axis cs:0.01,0.0273423296596845)
--(axis cs:0.0102,0.0171977344356047)
--(axis cs:0.0104,0.0107599089132325)
--(axis cs:0.0106,0.00669780790315936)
--(axis cs:0.0108,0.00414881285588331)
--(axis cs:0.011,0.00255774928923221)
--(axis cs:0.0112,0.00156966462258963)
--(axis cs:0.0114,0.000959043980204529)
--(axis cs:0.0116,0.000583468440526314)
--(axis cs:0.0118,0.000353512309656768)
--(axis cs:0.012,0.000213333020725685)
--(axis cs:0.0122,0.000128242850643948)
--(axis cs:0.0124,7.68037201326851e-05)
--(axis cs:0.0126,4.58305761506692e-05)
--(axis cs:0.0128,2.72520977192749e-05)
--(axis cs:0.013,1.61495898117959e-05)
--(axis cs:0.0132,9.53857042332152e-06)
--(axis cs:0.0134,5.6157357557516e-06)
--(axis cs:0.0136,3.29587795741631e-06)
--(axis cs:0.0138,1.92847688063577e-06)
--(axis cs:0.014,1.12505239925148e-06)
--(axis cs:0.0142,6.54456317675556e-07)
--(axis cs:0.0144,3.79639444120029e-07)
--(axis cs:0.0146,2.19622342593566e-07)
--(axis cs:0.0148,1.26714616937668e-07)
--(axis cs:0.015,7.29207563822728e-08)
--(axis cs:0.0152,4.18579500369891e-08)
--(axis cs:0.0154,2.39681255123372e-08)
--(axis cs:0.0156,1.36913240004885e-08)
--(axis cs:0.0158,7.8025585154859e-09)
--(axis cs:0.016,4.43642182094435e-09)
--(axis cs:0.0162,2.51684199065801e-09)
--(axis cs:0.0164,1.4247171282791e-09)
--(axis cs:0.0166,8.04771081382379e-10)
--(axis cs:0.0168,4.53636274462889e-10)
--(axis cs:0.017,2.55184812810522e-10)
--(axis cs:0.0172,1.43262573263436e-10)
--(axis cs:0.0174,8.02712667488462e-11)
--(axis cs:0.0176,4.48905435239373e-11)
--(axis cs:0.0178,2.50573023493042e-11)
--(axis cs:0.018,1.3960957590044e-11)
--(axis cs:0.0182,7.76450428095243e-12)
--(axis cs:0.0184,4.31067763326576e-12)
--(axis cs:0.0186,2.3890530370788e-12)
--(axis cs:0.0188,1.32181029046682e-12)
--(axis cs:0.019,7.30112445413e-13)
--(axis cs:0.0192,4.02625619692298e-13)
--(axis cs:0.0194,2.21675354345489e-13)
--(axis cs:0.0196,1.21857053967287e-13)
--(axis cs:0.0198,6.68826864553411e-14)
--(axis cs:0.02,3.66537700284644e-14)
--(axis cs:0.0202,2.00575217613878e-14)
--(axis cs:0.0204,1.0959758809364e-14)
--(axis cs:0.0206,5.97999627347004e-15)
--(axis cs:0.0208,3.25827538993974e-15)
--(axis cs:0.021,1.77285035089687e-15)
--(axis cs:0.0212,9.63305278865802e-16)
--(axis cs:0.0214,5.22724971007102e-16)
--(axis cs:0.0216,2.83275877689995e-16)
--(axis cs:0.0218,1.53314174574747e-16)
--(axis cs:0.022,8.28705967052991e-17)
--(axis cs:0.0222,4.47376214030539e-17)
--(axis cs:0.0224,2.41217142550633e-17)
--(axis cs:0.0226,1.29901647606133e-17)
--(axis cs:0.0228,6.98715695218015e-18)
--(axis cs:0.023,3.75382142728391e-18)
--(axis cs:0.0232,2.01438141289907e-18)
--(axis cs:0.0234,1.07972288043767e-18)
--(axis cs:0.0236,5.78086370487335e-19)
--(axis cs:0.0238,3.09164855713053e-19)
--(axis cs:0.024,1.65162493477036e-19)
--cycle;
\path [draw=none, fill=blue, fill opacity=0.5]
(axis cs:0,0)
--(axis cs:0.0002,3.00097034650853e-39)
--(axis cs:0.0004,1.33180595158039e-29)
--(axis cs:0.0006,4.34794883594433e-24)
--(axis cs:0.0008,2.88281813447308e-20)
--(axis cs:0.001,2.25959278886682e-17)
--(axis cs:0.0012,4.58849753621502e-15)
--(axis cs:0.0014,3.66862376248184e-13)
--(axis cs:0.0016,1.4826057132876e-11)
--(axis cs:0.0018,3.55812394866283e-10)
--(axis cs:0.002,5.6607416830461e-09)
--(axis cs:0.0022,6.45750577460221e-08)
--(axis cs:0.0024,5.59706827273519e-07)
--(axis cs:0.0026,3.85186816583851e-06)
--(axis cs:0.0028,2.17797182193631e-05)
--(axis cs:0.003,0.000103963098101098)
--(axis cs:0.0032,0.000428198943529838)
--(axis cs:0.0034,0.00154923143069868)
--(axis cs:0.0036,0.00499716303584501)
--(axis cs:0.0038,0.0145497409226953)
--(axis cs:0.004,0.0386429091930658)
--(axis cs:0.0042,0.0944621505149839)
--(axis cs:0.0044,0.214173948050834)
--(axis cs:0.0046,0.453416129407434)
--(axis cs:0.0048,0.901527460031158)
--(axis cs:0.005,1.69212633933122)
--(axis cs:0.0052,3.01173851254532)
--(axis cs:0.0054,5.10351436468107)
--(axis cs:0.0056,8.26297420559593)
--(axis cs:0.0058,12.8233899369747)
--(axis cs:0.006,19.1298809719443)
--(axis cs:0.0062,27.5033666828899)
--(axis cs:0.0064,33.764694951832)
--(axis cs:0.0066,24.6593694570613)
--(axis cs:0.0068,17.7868661496403)
--(axis cs:0.007,12.6801601962024)
--(axis cs:0.0072,8.94007607834724)
--(axis cs:0.0074,6.23747618667221)
--(axis cs:0.0076,4.30890608686193)
--(axis cs:0.0078,2.94873796599512)
--(axis cs:0.008,1.99995481107366)
--(axis cs:0.0082,1.34495499920915)
--(axis cs:0.0084,0.897168396988933)
--(axis cs:0.0086,0.59385652774837)
--(axis cs:0.0088,0.390195217766181)
--(axis cs:0.009,0.254575414817491)
--(axis cs:0.0092,0.164974583259305)
--(axis cs:0.0094,0.106220200791715)
--(axis cs:0.0096,0.0679676620931031)
--(axis cs:0.0098,0.0432325808603513)
--(axis cs:0.01,0.0273423296596845)
--(axis cs:0.0102,0.0171977344356047)
--(axis cs:0.0104,0.0107599089132325)
--(axis cs:0.0106,0.00669780790315936)
--(axis cs:0.0108,0.00414881285588331)
--(axis cs:0.011,0.00255774928923221)
--(axis cs:0.0112,0.00156966462258963)
--(axis cs:0.0114,0.000959043980204529)
--(axis cs:0.0116,0.000583468440526314)
--(axis cs:0.0118,0.000353512309656768)
--(axis cs:0.012,0.000213333020725685)
--(axis cs:0.0122,0.000128242850643948)
--(axis cs:0.0124,7.68037201326851e-05)
--(axis cs:0.0126,4.58305761506692e-05)
--(axis cs:0.0128,2.72520977192749e-05)
--(axis cs:0.013,1.61495898117959e-05)
--(axis cs:0.0132,9.53857042332152e-06)
--(axis cs:0.0134,5.6157357557516e-06)
--(axis cs:0.0136,3.29587795741631e-06)
--(axis cs:0.0138,1.92847688063577e-06)
--(axis cs:0.014,1.12505239925148e-06)
--(axis cs:0.0142,6.54456317675556e-07)
--(axis cs:0.0144,3.79639444120029e-07)
--(axis cs:0.0146,2.19622342593566e-07)
--(axis cs:0.0148,1.26714616937668e-07)
--(axis cs:0.015,7.29207563822728e-08)
--(axis cs:0.0152,4.18579500369891e-08)
--(axis cs:0.0154,2.39681255123372e-08)
--(axis cs:0.0156,1.36913240004885e-08)
--(axis cs:0.0158,7.8025585154859e-09)
--(axis cs:0.016,4.43642182094435e-09)
--(axis cs:0.0162,2.51684199065801e-09)
--(axis cs:0.0164,1.4247171282791e-09)
--(axis cs:0.0166,8.04771081382379e-10)
--(axis cs:0.0168,4.53636274462889e-10)
--(axis cs:0.017,2.55184812810522e-10)
--(axis cs:0.0172,1.43262573263436e-10)
--(axis cs:0.0174,8.02712667488462e-11)
--(axis cs:0.0176,4.48905435239373e-11)
--(axis cs:0.0178,2.50573023493042e-11)
--(axis cs:0.018,1.3960957590044e-11)
--(axis cs:0.0182,7.76450428095243e-12)
--(axis cs:0.0184,4.31067763326576e-12)
--(axis cs:0.0186,2.3890530370788e-12)
--(axis cs:0.0188,1.32181029046682e-12)
--(axis cs:0.019,7.30112445413e-13)
--(axis cs:0.0192,4.02625619692298e-13)
--(axis cs:0.0194,2.21675354345489e-13)
--(axis cs:0.0196,1.21857053967287e-13)
--(axis cs:0.0198,6.68826864553411e-14)
--(axis cs:0.02,3.66537700284644e-14)
--(axis cs:0.0202,2.00575217613878e-14)
--(axis cs:0.0204,1.0959758809364e-14)
--(axis cs:0.0206,5.97999627347004e-15)
--(axis cs:0.0208,3.25827538993974e-15)
--(axis cs:0.021,1.77285035089687e-15)
--(axis cs:0.0212,9.63305278865802e-16)
--(axis cs:0.0214,5.22724971007102e-16)
--(axis cs:0.0216,2.83275877689995e-16)
--(axis cs:0.0218,1.53314174574747e-16)
--(axis cs:0.022,8.28705967052991e-17)
--(axis cs:0.0222,4.47376214030539e-17)
--(axis cs:0.0224,2.41217142550633e-17)
--(axis cs:0.0226,1.29901647606133e-17)
--(axis cs:0.0228,6.98715695218015e-18)
--(axis cs:0.023,3.75382142728391e-18)
--(axis cs:0.0232,2.01438141289907e-18)
--(axis cs:0.0234,1.07972288043767e-18)
--(axis cs:0.0236,5.78086370487335e-19)
--(axis cs:0.0238,3.09164855713053e-19)
--(axis cs:0.024,1.65162493477036e-19)
--cycle;
\path [draw=none, fill=blue, fill opacity=0.5]
(axis cs:0,0)
--(axis cs:0.0002,7.61192645735247e-14)
--(axis cs:0.0004,3.61092349697281e-09)
--(axis cs:0.0006,1.45540906761335e-06)
--(axis cs:0.0008,8.32729233244913e-05)
--(axis cs:0.001,0.00163463517706023)
--(axis cs:0.0012,0.016309637005717)
--(axis cs:0.0014,0.102008198661971)
--(axis cs:0.0016,0.453261075131758)
--(axis cs:0.0018,1.5511170267273)
--(axis cs:0.002,4.31978738912353)
--(axis cs:0.0022,10.1830168367104)
--(axis cs:0.0024,20.9167445288748)
--(axis cs:0.0026,38.2727217377937)
--(axis cs:0.0028,63.4606200393569)
--(axis cs:0.003,96.6581704221682)
--(axis cs:0.0032,136.725048683837)
--(axis cs:0.0034,181.227009291322)
--(axis cs:0.0036,226.769934038948)
--(axis cs:0.0038,269.547320550091)
--(axis cs:0.004,305.953317774446)
--(axis cs:0.0042,333.115658709357)
--(axis cs:0.0044,349.246132216938)
--(axis cs:0.0046,353.767577642377)
--(axis cs:0.0048,340.892444776181)
--(axis cs:0.005,317.870126226892)
--(axis cs:0.0052,288.849553341082)
--(axis cs:0.0054,256.286017343453)
--(axis cs:0.0056,222.412439864701)
--(axis cs:0.0058,189.079722158915)
--(axis cs:0.006,157.683074976465)
--(axis cs:0.0062,129.15840845009)
--(axis cs:0.0064,104.027759779785)
--(axis cs:0.0066,82.4728224869533)
--(axis cs:0.0068,64.418951169431)
--(axis cs:0.007,49.6167625224316)
--(axis cs:0.0072,37.7133085450788)
--(axis cs:0.0074,28.3089543469283)
--(axis cs:0.0076,20.9991936459158)
--(axis cs:0.0078,15.4026440241404)
--(axis cs:0.008,11.1775340891288)
--(axis cs:0.0082,8.02935931856564)
--(axis cs:0.0084,5.71228280177854)
--(axis cs:0.0086,4.02649685568888)
--(axis cs:0.0088,2.81329566877787)
--(axis cs:0.009,1.94913981490007)
--(axis cs:0.0092,1.33957890477063)
--(axis cs:0.0094,0.913564437606188)
--(axis cs:0.0096,0.618435189640704)
--(axis cs:0.0098,0.415683987125745)
--(axis cs:0.01,0.277503574526014)
--(axis cs:0.0102,0.184045366929219)
--(axis cs:0.0104,0.121294291427453)
--(axis cs:0.0106,0.0794542187269003)
--(axis cs:0.0108,0.0517429197120304)
--(axis cs:0.011,0.0335068200368587)
--(axis cs:0.0112,0.0215799201029034)
--(axis cs:0.0114,0.0138255445023038)
--(axis cs:0.0116,0.00881267294525788)
--(axis cs:0.0118,0.00558983901571188)
--(axis cs:0.012,0.00352879748888552)
--(axis cs:0.0122,0.00221746133601891)
--(axis cs:0.0124,0.00138723530792914)
--(axis cs:0.0126,0.000864109583526985)
--(axis cs:0.0128,0.000536005372954958)
--(axis cs:0.013,0.00033113529045748)
--(axis cs:0.0132,0.000203765039925337)
--(axis cs:0.0134,0.000124908300098191)
--(axis cs:0.0136,7.62847405416888e-05)
--(axis cs:0.0138,4.6420953790526e-05)
--(axis cs:0.014,2.81490511547575e-05)
--(axis cs:0.0142,1.70109369999016e-05)
--(axis cs:0.0144,1.02458314352702e-05)
--(axis cs:0.0146,6.15118439560684e-06)
--(axis cs:0.0148,3.68128443355421e-06)
--(axis cs:0.015,2.1963631010631e-06)
--(axis cs:0.0152,1.30649181636633e-06)
--(axis cs:0.0154,7.74888778617601e-07)
--(axis cs:0.0156,4.5828247881716e-07)
--(axis cs:0.0158,2.70282688558907e-07)
--(axis cs:0.016,1.58972888238179e-07)
--(axis cs:0.0162,9.3255712797952e-08)
--(axis cs:0.0164,5.45634777566251e-08)
--(axis cs:0.0166,3.18440689554753e-08)
--(axis cs:0.0168,1.85387204875035e-08)
--(axis cs:0.017,1.07666238287212e-08)
--(axis cs:0.0172,6.238080275349e-09)
--(axis cs:0.0174,3.60592150732689e-09)
--(axis cs:0.0176,2.07968212190716e-09)
--(axis cs:0.0178,1.19677852625666e-09)
--(axis cs:0.018,6.87205808105705e-10)
--(axis cs:0.0182,3.93763499329262e-10)
--(axis cs:0.0184,2.25153370260175e-10)
--(axis cs:0.0186,1.28479536596914e-10)
--(axis cs:0.0188,7.3167725019549e-11)
--(axis cs:0.019,4.15864867650185e-11)
--(axis cs:0.0192,2.3591115560581e-11)
--(axis cs:0.0194,1.33574712820726e-11)
--(axis cs:0.0196,7.54909716595647e-12)
--(axis cs:0.0198,4.2586875071307e-12)
--(axis cs:0.02,2.39817608909909e-12)
--(axis cs:0.0202,1.34810900327012e-12)
--(axis cs:0.0204,7.56521520562321e-13)
--(axis cs:0.0206,4.23821702237897e-13)
--(axis cs:0.0208,2.37041085325845e-13)
--(axis cs:0.021,1.32359502443409e-13)
--(axis cs:0.0212,7.37887008057146e-14)
--(axis cs:0.0214,4.10714156450754e-14)
--(axis cs:0.0216,2.28252767651452e-14)
--(axis cs:0.0218,1.26657288252302e-14)
--(axis cs:0.022,7.0176700806092e-15)
--(axis cs:0.0222,3.88253085379576e-15)
--(axis cs:0.0224,2.14489671943879e-15)
--(axis cs:0.0226,1.18325234325703e-15)
--(axis cs:0.0228,6.51834935557492e-16)
--(axis cs:0.023,3.58588750638783e-16)
--(axis cs:0.0232,1.96998857004142e-16)
--(axis cs:0.0234,1.08080608000636e-16)
--(axis cs:0.0236,5.92185349401817e-17)
--(axis cs:0.0238,3.2404244611104e-17)
--(axis cs:0.024,1.77087884307283e-17)
--cycle;
\path [draw=none, fill=blue, fill opacity=0.5]
(axis cs:0,0)
--(axis cs:0.0002,1.63304653410572e-16)
--(axis cs:0.0004,2.90980802338031e-11)
--(axis cs:0.0006,2.54391471865381e-08)
--(axis cs:0.0008,2.52149359912999e-06)
--(axis cs:0.001,7.58076561312621e-05)
--(axis cs:0.0012,0.00107160766687709)
--(axis cs:0.0014,0.0089985803413974)
--(axis cs:0.0016,0.0516110463947514)
--(axis cs:0.0018,0.221227821059011)
--(axis cs:0.002,0.753634925751084)
--(axis cs:0.0022,2.13180511178777)
--(axis cs:0.0024,5.17199814367324)
--(axis cs:0.0026,11.0298671973361)
--(axis cs:0.0028,21.0757198851318)
--(axis cs:0.003,36.6331090584455)
--(axis cs:0.0032,58.6338009969209)
--(axis cs:0.0034,87.2857386482926)
--(axis cs:0.0036,121.857708878618)
--(axis cs:0.0038,160.652447408685)
--(axis cs:0.004,201.183753056193)
--(axis cs:0.0042,240.515622550635)
--(axis cs:0.0044,275.68210296399)
--(axis cs:0.0046,304.095014317866)
--(axis cs:0.0048,323.861262245946)
--(axis cs:0.005,317.870126226892)
--(axis cs:0.0052,288.849553341082)
--(axis cs:0.0054,256.286017343453)
--(axis cs:0.0056,222.412439864701)
--(axis cs:0.0058,189.079722158915)
--(axis cs:0.006,157.683074976465)
--(axis cs:0.0062,129.15840845009)
--(axis cs:0.0064,104.027759779785)
--(axis cs:0.0066,82.4728224869533)
--(axis cs:0.0068,64.418951169431)
--(axis cs:0.007,49.6167625224316)
--(axis cs:0.0072,37.7133085450788)
--(axis cs:0.0074,28.3089543469283)
--(axis cs:0.0076,20.9991936459158)
--(axis cs:0.0078,15.4026440241404)
--(axis cs:0.008,11.1775340891288)
--(axis cs:0.0082,8.02935931856564)
--(axis cs:0.0084,5.71228280177854)
--(axis cs:0.0086,4.02649685568888)
--(axis cs:0.0088,2.81329566877787)
--(axis cs:0.009,1.94913981490007)
--(axis cs:0.0092,1.33957890477063)
--(axis cs:0.0094,0.913564437606188)
--(axis cs:0.0096,0.618435189640704)
--(axis cs:0.0098,0.415683987125745)
--(axis cs:0.01,0.277503574526014)
--(axis cs:0.0102,0.184045366929219)
--(axis cs:0.0104,0.121294291427453)
--(axis cs:0.0106,0.0794542187269003)
--(axis cs:0.0108,0.0517429197120304)
--(axis cs:0.011,0.0335068200368587)
--(axis cs:0.0112,0.0215799201029034)
--(axis cs:0.0114,0.0138255445023038)
--(axis cs:0.0116,0.00881267294525788)
--(axis cs:0.0118,0.00558983901571188)
--(axis cs:0.012,0.00352879748888552)
--(axis cs:0.0122,0.00221746133601891)
--(axis cs:0.0124,0.00138723530792914)
--(axis cs:0.0126,0.000864109583526985)
--(axis cs:0.0128,0.000536005372954958)
--(axis cs:0.013,0.00033113529045748)
--(axis cs:0.0132,0.000203765039925337)
--(axis cs:0.0134,0.000124908300098191)
--(axis cs:0.0136,7.62847405416888e-05)
--(axis cs:0.0138,4.6420953790526e-05)
--(axis cs:0.014,2.81490511547575e-05)
--(axis cs:0.0142,1.70109369999016e-05)
--(axis cs:0.0144,1.02458314352702e-05)
--(axis cs:0.0146,6.15118439560684e-06)
--(axis cs:0.0148,3.68128443355421e-06)
--(axis cs:0.015,2.1963631010631e-06)
--(axis cs:0.0152,1.30649181636633e-06)
--(axis cs:0.0154,7.74888778617601e-07)
--(axis cs:0.0156,4.5828247881716e-07)
--(axis cs:0.0158,2.70282688558907e-07)
--(axis cs:0.016,1.58972888238179e-07)
--(axis cs:0.0162,9.3255712797952e-08)
--(axis cs:0.0164,5.45634777566251e-08)
--(axis cs:0.0166,3.18440689554753e-08)
--(axis cs:0.0168,1.85387204875035e-08)
--(axis cs:0.017,1.07666238287212e-08)
--(axis cs:0.0172,6.238080275349e-09)
--(axis cs:0.0174,3.60592150732689e-09)
--(axis cs:0.0176,2.07968212190716e-09)
--(axis cs:0.0178,1.19677852625666e-09)
--(axis cs:0.018,6.87205808105705e-10)
--(axis cs:0.0182,3.93763499329262e-10)
--(axis cs:0.0184,2.25153370260175e-10)
--(axis cs:0.0186,1.28479536596914e-10)
--(axis cs:0.0188,7.3167725019549e-11)
--(axis cs:0.019,4.15864867650185e-11)
--(axis cs:0.0192,2.3591115560581e-11)
--(axis cs:0.0194,1.33574712820726e-11)
--(axis cs:0.0196,7.54909716595647e-12)
--(axis cs:0.0198,4.2586875071307e-12)
--(axis cs:0.02,2.39817608909909e-12)
--(axis cs:0.0202,1.34810900327012e-12)
--(axis cs:0.0204,7.56521520562321e-13)
--(axis cs:0.0206,4.23821702237897e-13)
--(axis cs:0.0208,2.37041085325845e-13)
--(axis cs:0.021,1.32359502443409e-13)
--(axis cs:0.0212,7.37887008057146e-14)
--(axis cs:0.0214,4.10714156450754e-14)
--(axis cs:0.0216,2.28252767651452e-14)
--(axis cs:0.0218,1.26657288252302e-14)
--(axis cs:0.022,7.0176700806092e-15)
--(axis cs:0.0222,3.88253085379576e-15)
--(axis cs:0.0224,2.14489671943879e-15)
--(axis cs:0.0226,1.18325234325703e-15)
--(axis cs:0.0228,6.51834935557492e-16)
--(axis cs:0.023,3.58588750638783e-16)
--(axis cs:0.0232,1.96998857004142e-16)
--(axis cs:0.0234,1.08080608000636e-16)
--(axis cs:0.0236,5.92185349401817e-17)
--(axis cs:0.0238,3.2404244611104e-17)
--(axis cs:0.024,1.77087884307283e-17)
--cycle;
\path [draw=none, fill=blue, fill opacity=0.5]
(axis cs:0,0)
--(axis cs:0.0002,2.87061962533969e-17)
--(axis cs:0.0004,7.38215500264892e-12)
--(axis cs:0.0006,7.99925233557908e-09)
--(axis cs:0.0008,9.23345890210269e-07)
--(axis cs:0.001,3.12426163997331e-05)
--(axis cs:0.0012,0.000486425585682615)
--(axis cs:0.0014,0.00443227584242255)
--(axis cs:0.0016,0.0272853176447845)
--(axis cs:0.0018,0.124491716808887)
--(axis cs:0.002,0.448457426929982)
--(axis cs:0.0022,1.33431227744341)
--(axis cs:0.0024,3.39008775333182)
--(axis cs:0.0026,7.54334486722043)
--(axis cs:0.0028,14.9917839314935)
--(axis cs:0.003,27.0300203986886)
--(axis cs:0.0032,44.7711936223266)
--(axis cs:0.0034,68.829161876082)
--(axis cs:0.0036,99.0522136164206)
--(axis cs:0.0038,134.390978721141)
--(axis cs:0.004,172.945707968733)
--(axis cs:0.0042,212.187129650739)
--(axis cs:0.0044,249.299540863043)
--(axis cs:0.0046,281.568493665345)
--(axis cs:0.0048,306.733195142688)
--(axis cs:0.005,317.870126226892)
--(axis cs:0.0052,288.849553341082)
--(axis cs:0.0054,256.286017343453)
--(axis cs:0.0056,222.412439864701)
--(axis cs:0.0058,189.079722158915)
--(axis cs:0.006,157.683074976465)
--(axis cs:0.0062,129.15840845009)
--(axis cs:0.0064,104.027759779785)
--(axis cs:0.0066,82.4728224869533)
--(axis cs:0.0068,64.418951169431)
--(axis cs:0.007,49.6167625224316)
--(axis cs:0.0072,37.7133085450788)
--(axis cs:0.0074,28.3089543469283)
--(axis cs:0.0076,20.9991936459158)
--(axis cs:0.0078,15.4026440241404)
--(axis cs:0.008,11.1775340891288)
--(axis cs:0.0082,8.02935931856564)
--(axis cs:0.0084,5.71228280177854)
--(axis cs:0.0086,4.02649685568888)
--(axis cs:0.0088,2.81329566877787)
--(axis cs:0.009,1.94913981490007)
--(axis cs:0.0092,1.33957890477063)
--(axis cs:0.0094,0.913564437606188)
--(axis cs:0.0096,0.618435189640704)
--(axis cs:0.0098,0.415683987125745)
--(axis cs:0.01,0.277503574526014)
--(axis cs:0.0102,0.184045366929219)
--(axis cs:0.0104,0.121294291427453)
--(axis cs:0.0106,0.0794542187269003)
--(axis cs:0.0108,0.0517429197120304)
--(axis cs:0.011,0.0335068200368587)
--(axis cs:0.0112,0.0215799201029034)
--(axis cs:0.0114,0.0138255445023038)
--(axis cs:0.0116,0.00881267294525788)
--(axis cs:0.0118,0.00558983901571188)
--(axis cs:0.012,0.00352879748888552)
--(axis cs:0.0122,0.00221746133601891)
--(axis cs:0.0124,0.00138723530792914)
--(axis cs:0.0126,0.000864109583526985)
--(axis cs:0.0128,0.000536005372954958)
--(axis cs:0.013,0.00033113529045748)
--(axis cs:0.0132,0.000203765039925337)
--(axis cs:0.0134,0.000124908300098191)
--(axis cs:0.0136,7.62847405416888e-05)
--(axis cs:0.0138,4.6420953790526e-05)
--(axis cs:0.014,2.81490511547575e-05)
--(axis cs:0.0142,1.70109369999016e-05)
--(axis cs:0.0144,1.02458314352702e-05)
--(axis cs:0.0146,6.15118439560684e-06)
--(axis cs:0.0148,3.68128443355421e-06)
--(axis cs:0.015,2.1963631010631e-06)
--(axis cs:0.0152,1.30649181636633e-06)
--(axis cs:0.0154,7.74888778617601e-07)
--(axis cs:0.0156,4.5828247881716e-07)
--(axis cs:0.0158,2.70282688558907e-07)
--(axis cs:0.016,1.58972888238179e-07)
--(axis cs:0.0162,9.3255712797952e-08)
--(axis cs:0.0164,5.45634777566251e-08)
--(axis cs:0.0166,3.18440689554753e-08)
--(axis cs:0.0168,1.85387204875035e-08)
--(axis cs:0.017,1.07666238287212e-08)
--(axis cs:0.0172,6.238080275349e-09)
--(axis cs:0.0174,3.60592150732689e-09)
--(axis cs:0.0176,2.07968212190716e-09)
--(axis cs:0.0178,1.19677852625666e-09)
--(axis cs:0.018,6.87205808105705e-10)
--(axis cs:0.0182,3.93763499329262e-10)
--(axis cs:0.0184,2.25153370260175e-10)
--(axis cs:0.0186,1.28479536596914e-10)
--(axis cs:0.0188,7.3167725019549e-11)
--(axis cs:0.019,4.15864867650185e-11)
--(axis cs:0.0192,2.3591115560581e-11)
--(axis cs:0.0194,1.33574712820726e-11)
--(axis cs:0.0196,7.54909716595647e-12)
--(axis cs:0.0198,4.2586875071307e-12)
--(axis cs:0.02,2.39817608909909e-12)
--(axis cs:0.0202,1.34810900327012e-12)
--(axis cs:0.0204,7.56521520562321e-13)
--(axis cs:0.0206,4.23821702237897e-13)
--(axis cs:0.0208,2.37041085325845e-13)
--(axis cs:0.021,1.32359502443409e-13)
--(axis cs:0.0212,7.37887008057146e-14)
--(axis cs:0.0214,4.10714156450754e-14)
--(axis cs:0.0216,2.28252767651452e-14)
--(axis cs:0.0218,1.26657288252302e-14)
--(axis cs:0.022,7.0176700806092e-15)
--(axis cs:0.0222,3.88253085379576e-15)
--(axis cs:0.0224,2.14489671943879e-15)
--(axis cs:0.0226,1.18325234325703e-15)
--(axis cs:0.0228,6.51834935557492e-16)
--(axis cs:0.023,3.58588750638783e-16)
--(axis cs:0.0232,1.96998857004142e-16)
--(axis cs:0.0234,1.08080608000636e-16)
--(axis cs:0.0236,5.92185349401817e-17)
--(axis cs:0.0238,3.2404244611104e-17)
--(axis cs:0.024,1.77087884307283e-17)
--cycle;
\path [draw=none, fill=blue, fill opacity=0.5]
(axis cs:0,0)
--(axis cs:0.0002,4.77632767978611e-17)
--(axis cs:0.0004,1.10350468092225e-11)
--(axis cs:0.0006,1.12309897522798e-08)
--(axis cs:0.0008,1.23997965836469e-06)
--(axis cs:0.001,4.05331104838323e-05)
--(axis cs:0.0012,0.000613522820835944)
--(axis cs:0.0014,0.00545862533187201)
--(axis cs:0.0016,0.032916327874361)
--(axis cs:0.0018,0.147470611530503)
--(axis cs:0.002,0.522639852402313)
--(axis cs:0.0022,1.53224950990862)
--(axis cs:0.0024,3.84087832479369)
--(axis cs:0.0026,8.44109535473997)
--(axis cs:0.0028,16.5844717211536)
--(axis cs:0.003,29.5836281445369)
--(axis cs:0.0032,48.5131391024827)
--(axis cs:0.0034,73.884182008032)
--(axis cs:0.0036,105.388702355709)
--(axis cs:0.0038,141.794147211553)
--(axis cs:0.004,181.026617480686)
--(axis cs:0.0042,220.426658502137)
--(axis cs:0.0044,257.117315904375)
--(axis cs:0.0046,288.401464242004)
--(axis cs:0.0048,312.107862563584)
--(axis cs:0.005,317.870126226892)
--(axis cs:0.0052,288.849553341082)
--(axis cs:0.0054,256.286017343453)
--(axis cs:0.0056,222.412439864701)
--(axis cs:0.0058,189.079722158915)
--(axis cs:0.006,157.683074976465)
--(axis cs:0.0062,129.15840845009)
--(axis cs:0.0064,104.027759779785)
--(axis cs:0.0066,82.4728224869533)
--(axis cs:0.0068,64.418951169431)
--(axis cs:0.007,49.6167625224316)
--(axis cs:0.0072,37.7133085450788)
--(axis cs:0.0074,28.3089543469283)
--(axis cs:0.0076,20.9991936459158)
--(axis cs:0.0078,15.4026440241404)
--(axis cs:0.008,11.1775340891288)
--(axis cs:0.0082,8.02935931856564)
--(axis cs:0.0084,5.71228280177854)
--(axis cs:0.0086,4.02649685568888)
--(axis cs:0.0088,2.81329566877787)
--(axis cs:0.009,1.94913981490007)
--(axis cs:0.0092,1.33957890477063)
--(axis cs:0.0094,0.913564437606188)
--(axis cs:0.0096,0.618435189640704)
--(axis cs:0.0098,0.415683987125745)
--(axis cs:0.01,0.277503574526014)
--(axis cs:0.0102,0.184045366929219)
--(axis cs:0.0104,0.121294291427453)
--(axis cs:0.0106,0.0794542187269003)
--(axis cs:0.0108,0.0517429197120304)
--(axis cs:0.011,0.0335068200368587)
--(axis cs:0.0112,0.0215799201029034)
--(axis cs:0.0114,0.0138255445023038)
--(axis cs:0.0116,0.00881267294525788)
--(axis cs:0.0118,0.00558983901571188)
--(axis cs:0.012,0.00352879748888552)
--(axis cs:0.0122,0.00221746133601891)
--(axis cs:0.0124,0.00138723530792914)
--(axis cs:0.0126,0.000864109583526985)
--(axis cs:0.0128,0.000536005372954958)
--(axis cs:0.013,0.00033113529045748)
--(axis cs:0.0132,0.000203765039925337)
--(axis cs:0.0134,0.000124908300098191)
--(axis cs:0.0136,7.62847405416888e-05)
--(axis cs:0.0138,4.6420953790526e-05)
--(axis cs:0.014,2.81490511547575e-05)
--(axis cs:0.0142,1.70109369999016e-05)
--(axis cs:0.0144,1.02458314352702e-05)
--(axis cs:0.0146,6.15118439560684e-06)
--(axis cs:0.0148,3.68128443355421e-06)
--(axis cs:0.015,2.1963631010631e-06)
--(axis cs:0.0152,1.30649181636633e-06)
--(axis cs:0.0154,7.74888778617601e-07)
--(axis cs:0.0156,4.5828247881716e-07)
--(axis cs:0.0158,2.70282688558907e-07)
--(axis cs:0.016,1.58972888238179e-07)
--(axis cs:0.0162,9.3255712797952e-08)
--(axis cs:0.0164,5.45634777566251e-08)
--(axis cs:0.0166,3.18440689554753e-08)
--(axis cs:0.0168,1.85387204875035e-08)
--(axis cs:0.017,1.07666238287212e-08)
--(axis cs:0.0172,6.238080275349e-09)
--(axis cs:0.0174,3.60592150732689e-09)
--(axis cs:0.0176,2.07968212190716e-09)
--(axis cs:0.0178,1.19677852625666e-09)
--(axis cs:0.018,6.87205808105705e-10)
--(axis cs:0.0182,3.93763499329262e-10)
--(axis cs:0.0184,2.25153370260175e-10)
--(axis cs:0.0186,1.28479536596914e-10)
--(axis cs:0.0188,7.3167725019549e-11)
--(axis cs:0.019,4.15864867650185e-11)
--(axis cs:0.0192,2.3591115560581e-11)
--(axis cs:0.0194,1.33574712820726e-11)
--(axis cs:0.0196,7.54909716595647e-12)
--(axis cs:0.0198,4.2586875071307e-12)
--(axis cs:0.02,2.39817608909909e-12)
--(axis cs:0.0202,1.34810900327012e-12)
--(axis cs:0.0204,7.56521520562321e-13)
--(axis cs:0.0206,4.23821702237897e-13)
--(axis cs:0.0208,2.37041085325845e-13)
--(axis cs:0.021,1.32359502443409e-13)
--(axis cs:0.0212,7.37887008057146e-14)
--(axis cs:0.0214,4.10714156450754e-14)
--(axis cs:0.0216,2.28252767651452e-14)
--(axis cs:0.0218,1.26657288252302e-14)
--(axis cs:0.022,7.0176700806092e-15)
--(axis cs:0.0222,3.88253085379576e-15)
--(axis cs:0.0224,2.14489671943879e-15)
--(axis cs:0.0226,1.18325234325703e-15)
--(axis cs:0.0228,6.51834935557492e-16)
--(axis cs:0.023,3.58588750638783e-16)
--(axis cs:0.0232,1.96998857004142e-16)
--(axis cs:0.0234,1.08080608000636e-16)
--(axis cs:0.0236,5.92185349401817e-17)
--(axis cs:0.0238,3.2404244611104e-17)
--(axis cs:0.024,1.77087884307283e-17)
--cycle;
\path [draw=none, fill=blue, fill opacity=0.5]
(axis cs:0,0)
--(axis cs:0.0002,2.50033281924293e-17)
--(axis cs:0.0004,6.61920264526063e-12)
--(axis cs:0.0006,7.29531684613714e-09)
--(axis cs:0.0008,8.52296949347681e-07)
--(axis cs:0.001,2.91093689969302e-05)
--(axis cs:0.0012,0.000456687181093467)
--(axis cs:0.0014,0.00418826597701668)
--(axis cs:0.0016,0.025927869467974)
--(axis cs:0.0018,0.118883729553988)
--(axis cs:0.002,0.43015163399355)
--(axis cs:0.0022,1.28497176113775)
--(axis cs:0.0024,3.27666446986953)
--(axis cs:0.0026,7.31548767742108)
--(axis cs:0.0028,14.5842153965101)
--(axis cs:0.003,26.3714255968948)
--(axis cs:0.0032,43.7988227032249)
--(axis cs:0.0034,67.5058869162191)
--(axis cs:0.0036,97.3813437364043)
--(axis cs:0.0038,132.424374144004)
--(axis cs:0.004,170.782523087049)
--(axis cs:0.0042,209.96311643966)
--(axis cs:0.0044,247.169302110898)
--(axis cs:0.0046,279.684632551789)
--(axis cs:0.0048,305.226638812366)
--(axis cs:0.005,317.870126226892)
--(axis cs:0.0052,288.849553341082)
--(axis cs:0.0054,256.286017343453)
--(axis cs:0.0056,222.412439864701)
--(axis cs:0.0058,189.079722158915)
--(axis cs:0.006,157.683074976465)
--(axis cs:0.0062,129.15840845009)
--(axis cs:0.0064,104.027759779785)
--(axis cs:0.0066,82.4728224869533)
--(axis cs:0.0068,64.418951169431)
--(axis cs:0.007,49.6167625224316)
--(axis cs:0.0072,37.7133085450788)
--(axis cs:0.0074,28.3089543469283)
--(axis cs:0.0076,20.9991936459158)
--(axis cs:0.0078,15.4026440241404)
--(axis cs:0.008,11.1775340891288)
--(axis cs:0.0082,8.02935931856564)
--(axis cs:0.0084,5.71228280177854)
--(axis cs:0.0086,4.02649685568888)
--(axis cs:0.0088,2.81329566877787)
--(axis cs:0.009,1.94913981490007)
--(axis cs:0.0092,1.33957890477063)
--(axis cs:0.0094,0.913564437606188)
--(axis cs:0.0096,0.618435189640704)
--(axis cs:0.0098,0.415683987125745)
--(axis cs:0.01,0.277503574526014)
--(axis cs:0.0102,0.184045366929219)
--(axis cs:0.0104,0.121294291427453)
--(axis cs:0.0106,0.0794542187269003)
--(axis cs:0.0108,0.0517429197120304)
--(axis cs:0.011,0.0335068200368587)
--(axis cs:0.0112,0.0215799201029034)
--(axis cs:0.0114,0.0138255445023038)
--(axis cs:0.0116,0.00881267294525788)
--(axis cs:0.0118,0.00558983901571188)
--(axis cs:0.012,0.00352879748888552)
--(axis cs:0.0122,0.00221746133601891)
--(axis cs:0.0124,0.00138723530792914)
--(axis cs:0.0126,0.000864109583526985)
--(axis cs:0.0128,0.000536005372954958)
--(axis cs:0.013,0.00033113529045748)
--(axis cs:0.0132,0.000203765039925337)
--(axis cs:0.0134,0.000124908300098191)
--(axis cs:0.0136,7.62847405416888e-05)
--(axis cs:0.0138,4.6420953790526e-05)
--(axis cs:0.014,2.81490511547575e-05)
--(axis cs:0.0142,1.70109369999016e-05)
--(axis cs:0.0144,1.02458314352702e-05)
--(axis cs:0.0146,6.15118439560684e-06)
--(axis cs:0.0148,3.68128443355421e-06)
--(axis cs:0.015,2.1963631010631e-06)
--(axis cs:0.0152,1.30649181636633e-06)
--(axis cs:0.0154,7.74888778617601e-07)
--(axis cs:0.0156,4.5828247881716e-07)
--(axis cs:0.0158,2.70282688558907e-07)
--(axis cs:0.016,1.58972888238179e-07)
--(axis cs:0.0162,9.3255712797952e-08)
--(axis cs:0.0164,5.45634777566251e-08)
--(axis cs:0.0166,3.18440689554753e-08)
--(axis cs:0.0168,1.85387204875035e-08)
--(axis cs:0.017,1.07666238287212e-08)
--(axis cs:0.0172,6.238080275349e-09)
--(axis cs:0.0174,3.60592150732689e-09)
--(axis cs:0.0176,2.07968212190716e-09)
--(axis cs:0.0178,1.19677852625666e-09)
--(axis cs:0.018,6.87205808105705e-10)
--(axis cs:0.0182,3.93763499329262e-10)
--(axis cs:0.0184,2.25153370260175e-10)
--(axis cs:0.0186,1.28479536596914e-10)
--(axis cs:0.0188,7.3167725019549e-11)
--(axis cs:0.019,4.15864867650185e-11)
--(axis cs:0.0192,2.3591115560581e-11)
--(axis cs:0.0194,1.33574712820726e-11)
--(axis cs:0.0196,7.54909716595647e-12)
--(axis cs:0.0198,4.2586875071307e-12)
--(axis cs:0.02,2.39817608909909e-12)
--(axis cs:0.0202,1.34810900327012e-12)
--(axis cs:0.0204,7.56521520562321e-13)
--(axis cs:0.0206,4.23821702237897e-13)
--(axis cs:0.0208,2.37041085325845e-13)
--(axis cs:0.021,1.32359502443409e-13)
--(axis cs:0.0212,7.37887008057146e-14)
--(axis cs:0.0214,4.10714156450754e-14)
--(axis cs:0.0216,2.28252767651452e-14)
--(axis cs:0.0218,1.26657288252302e-14)
--(axis cs:0.022,7.0176700806092e-15)
--(axis cs:0.0222,3.88253085379576e-15)
--(axis cs:0.0224,2.14489671943879e-15)
--(axis cs:0.0226,1.18325234325703e-15)
--(axis cs:0.0228,6.51834935557492e-16)
--(axis cs:0.023,3.58588750638783e-16)
--(axis cs:0.0232,1.96998857004142e-16)
--(axis cs:0.0234,1.08080608000636e-16)
--(axis cs:0.0236,5.92185349401817e-17)
--(axis cs:0.0238,3.2404244611104e-17)
--(axis cs:0.024,1.77087884307283e-17)
--cycle;
\path [draw=none, fill=blue, fill opacity=0.5]
(axis cs:0,0)
--(axis cs:0.0002,4.38630231113233e-32)
--(axis cs:0.0004,9.42164546775413e-24)
--(axis cs:0.0006,5.22996289695702e-19)
--(axis cs:0.0008,9.86216555458081e-16)
--(axis cs:0.001,2.9143627354464e-13)
--(axis cs:0.0012,2.66669889722348e-11)
--(axis cs:0.0014,1.08650456690335e-09)
--(axis cs:0.0016,2.44844210199571e-08)
--(axis cs:0.0018,3.50988320156314e-07)
--(axis cs:0.002,3.52140105748224e-06)
--(axis cs:0.0022,2.64692151139118e-05)
--(axis cs:0.0024,0.000156751672867648)
--(axis cs:0.0026,0.000759830128964595)
--(axis cs:0.0028,0.00310561691820713)
--(axis cs:0.003,0.0109579090659142)
--(axis cs:0.0032,0.034016993111393)
--(axis cs:0.0034,0.0943609589409247)
--(axis cs:0.0036,0.236919036551997)
--(axis cs:0.0038,0.544246235965627)
--(axis cs:0.004,1.15434405660579)
--(axis cs:0.0042,2.27822422319553)
--(axis cs:0.0044,4.21196189035521)
--(axis cs:0.0046,7.33697307600828)
--(axis cs:0.0048,12.1029173241866)
--(axis cs:0.005,18.9902597620454)
--(axis cs:0.0052,28.4537592035236)
--(axis cs:0.0054,40.8529986597383)
--(axis cs:0.0056,56.3802313963309)
--(axis cs:0.0058,74.9980614577203)
--(axis cs:0.006,96.399078434761)
--(axis cs:0.0062,119.996496092707)
--(axis cs:0.0064,104.027759779785)
--(axis cs:0.0066,82.4728224869533)
--(axis cs:0.0068,64.418951169431)
--(axis cs:0.007,49.6167625224316)
--(axis cs:0.0072,37.7133085450788)
--(axis cs:0.0074,28.3089543469283)
--(axis cs:0.0076,20.9991936459158)
--(axis cs:0.0078,15.4026440241404)
--(axis cs:0.008,11.1775340891288)
--(axis cs:0.0082,8.02935931856564)
--(axis cs:0.0084,5.71228280177854)
--(axis cs:0.0086,4.02649685568888)
--(axis cs:0.0088,2.81329566877787)
--(axis cs:0.009,1.94913981490007)
--(axis cs:0.0092,1.33957890477063)
--(axis cs:0.0094,0.913564437606188)
--(axis cs:0.0096,0.618435189640704)
--(axis cs:0.0098,0.415683987125745)
--(axis cs:0.01,0.277503574526014)
--(axis cs:0.0102,0.184045366929219)
--(axis cs:0.0104,0.121294291427453)
--(axis cs:0.0106,0.0794542187269003)
--(axis cs:0.0108,0.0517429197120304)
--(axis cs:0.011,0.0335068200368587)
--(axis cs:0.0112,0.0215799201029034)
--(axis cs:0.0114,0.0138255445023038)
--(axis cs:0.0116,0.00881267294525788)
--(axis cs:0.0118,0.00558983901571188)
--(axis cs:0.012,0.00352879748888552)
--(axis cs:0.0122,0.00221746133601891)
--(axis cs:0.0124,0.00138723530792914)
--(axis cs:0.0126,0.000864109583526985)
--(axis cs:0.0128,0.000536005372954958)
--(axis cs:0.013,0.00033113529045748)
--(axis cs:0.0132,0.000203765039925337)
--(axis cs:0.0134,0.000124908300098191)
--(axis cs:0.0136,7.62847405416888e-05)
--(axis cs:0.0138,4.6420953790526e-05)
--(axis cs:0.014,2.81490511547575e-05)
--(axis cs:0.0142,1.70109369999016e-05)
--(axis cs:0.0144,1.02458314352702e-05)
--(axis cs:0.0146,6.15118439560684e-06)
--(axis cs:0.0148,3.68128443355421e-06)
--(axis cs:0.015,2.1963631010631e-06)
--(axis cs:0.0152,1.30649181636633e-06)
--(axis cs:0.0154,7.74888778617601e-07)
--(axis cs:0.0156,4.5828247881716e-07)
--(axis cs:0.0158,2.70282688558907e-07)
--(axis cs:0.016,1.58972888238179e-07)
--(axis cs:0.0162,9.3255712797952e-08)
--(axis cs:0.0164,5.45634777566251e-08)
--(axis cs:0.0166,3.18440689554753e-08)
--(axis cs:0.0168,1.85387204875035e-08)
--(axis cs:0.017,1.07666238287212e-08)
--(axis cs:0.0172,6.238080275349e-09)
--(axis cs:0.0174,3.60592150732689e-09)
--(axis cs:0.0176,2.07968212190716e-09)
--(axis cs:0.0178,1.19677852625666e-09)
--(axis cs:0.018,6.87205808105705e-10)
--(axis cs:0.0182,3.93763499329262e-10)
--(axis cs:0.0184,2.25153370260175e-10)
--(axis cs:0.0186,1.28479536596914e-10)
--(axis cs:0.0188,7.3167725019549e-11)
--(axis cs:0.019,4.15864867650185e-11)
--(axis cs:0.0192,2.3591115560581e-11)
--(axis cs:0.0194,1.33574712820726e-11)
--(axis cs:0.0196,7.54909716595647e-12)
--(axis cs:0.0198,4.2586875071307e-12)
--(axis cs:0.02,2.39817608909909e-12)
--(axis cs:0.0202,1.34810900327012e-12)
--(axis cs:0.0204,7.56521520562321e-13)
--(axis cs:0.0206,4.23821702237897e-13)
--(axis cs:0.0208,2.37041085325845e-13)
--(axis cs:0.021,1.32359502443409e-13)
--(axis cs:0.0212,7.37887008057146e-14)
--(axis cs:0.0214,4.10714156450754e-14)
--(axis cs:0.0216,2.28252767651452e-14)
--(axis cs:0.0218,1.26657288252302e-14)
--(axis cs:0.022,7.0176700806092e-15)
--(axis cs:0.0222,3.88253085379576e-15)
--(axis cs:0.0224,2.14489671943879e-15)
--(axis cs:0.0226,1.18325234325703e-15)
--(axis cs:0.0228,6.51834935557492e-16)
--(axis cs:0.023,3.58588750638783e-16)
--(axis cs:0.0232,1.96998857004142e-16)
--(axis cs:0.0234,1.08080608000636e-16)
--(axis cs:0.0236,5.92185349401817e-17)
--(axis cs:0.0238,3.2404244611104e-17)
--(axis cs:0.024,1.77087884307283e-17)
--cycle;
\path [draw=none, fill=blue, fill opacity=0.5]
(axis cs:0,0)
--(axis cs:0.0002,4.28964223063707e-25)
--(axis cs:0.0004,4.23925669893049e-18)
--(axis cs:0.0006,3.8843035174309e-14)
--(axis cs:0.0008,2.03980960361959e-11)
--(axis cs:0.001,2.23578413477403e-09)
--(axis cs:0.0012,9.09612932925116e-08)
--(axis cs:0.0014,1.86740983355346e-06)
--(axis cs:0.0016,2.32374652780952e-05)
--(axis cs:0.0018,0.00019726676995222)
--(axis cs:0.002,0.00123850159771812)
--(axis cs:0.0022,0.00609150384571705)
--(axis cs:0.0024,0.0244907822733581)
--(axis cs:0.0026,0.0831291604329364)
--(axis cs:0.0028,0.244273624310176)
--(axis cs:0.003,0.633888506344497)
--(axis cs:0.0032,1.47614150045633)
--(axis cs:0.0034,3.12549923236712)
--(axis cs:0.0036,6.08284419664121)
--(axis cs:0.0038,10.9810360144135)
--(axis cs:0.004,18.5299350809066)
--(axis cs:0.0042,29.4207126030825)
--(axis cs:0.0044,44.2015281358291)
--(axis cs:0.0046,63.1472686753469)
--(axis cs:0.0048,86.1512421387252)
--(axis cs:0.005,112.664536610035)
--(axis cs:0.0052,141.699695822165)
--(axis cs:0.0054,171.902056209116)
--(axis cs:0.0056,201.678253673426)
--(axis cs:0.0058,189.079722158915)
--(axis cs:0.006,157.683074976465)
--(axis cs:0.0062,129.15840845009)
--(axis cs:0.0064,104.027759779785)
--(axis cs:0.0066,82.4728224869533)
--(axis cs:0.0068,64.418951169431)
--(axis cs:0.007,49.6167625224316)
--(axis cs:0.0072,37.7133085450788)
--(axis cs:0.0074,28.3089543469283)
--(axis cs:0.0076,20.9991936459158)
--(axis cs:0.0078,15.4026440241404)
--(axis cs:0.008,11.1775340891288)
--(axis cs:0.0082,8.02935931856564)
--(axis cs:0.0084,5.71228280177854)
--(axis cs:0.0086,4.02649685568888)
--(axis cs:0.0088,2.81329566877787)
--(axis cs:0.009,1.94913981490007)
--(axis cs:0.0092,1.33957890477063)
--(axis cs:0.0094,0.913564437606188)
--(axis cs:0.0096,0.618435189640704)
--(axis cs:0.0098,0.415683987125745)
--(axis cs:0.01,0.277503574526014)
--(axis cs:0.0102,0.184045366929219)
--(axis cs:0.0104,0.121294291427453)
--(axis cs:0.0106,0.0794542187269003)
--(axis cs:0.0108,0.0517429197120304)
--(axis cs:0.011,0.0335068200368587)
--(axis cs:0.0112,0.0215799201029034)
--(axis cs:0.0114,0.0138255445023038)
--(axis cs:0.0116,0.00881267294525788)
--(axis cs:0.0118,0.00558983901571188)
--(axis cs:0.012,0.00352879748888552)
--(axis cs:0.0122,0.00221746133601891)
--(axis cs:0.0124,0.00138723530792914)
--(axis cs:0.0126,0.000864109583526985)
--(axis cs:0.0128,0.000536005372954958)
--(axis cs:0.013,0.00033113529045748)
--(axis cs:0.0132,0.000203765039925337)
--(axis cs:0.0134,0.000124908300098191)
--(axis cs:0.0136,7.62847405416888e-05)
--(axis cs:0.0138,4.6420953790526e-05)
--(axis cs:0.014,2.81490511547575e-05)
--(axis cs:0.0142,1.70109369999016e-05)
--(axis cs:0.0144,1.02458314352702e-05)
--(axis cs:0.0146,6.15118439560684e-06)
--(axis cs:0.0148,3.68128443355421e-06)
--(axis cs:0.015,2.1963631010631e-06)
--(axis cs:0.0152,1.30649181636633e-06)
--(axis cs:0.0154,7.74888778617601e-07)
--(axis cs:0.0156,4.5828247881716e-07)
--(axis cs:0.0158,2.70282688558907e-07)
--(axis cs:0.016,1.58972888238179e-07)
--(axis cs:0.0162,9.3255712797952e-08)
--(axis cs:0.0164,5.45634777566251e-08)
--(axis cs:0.0166,3.18440689554753e-08)
--(axis cs:0.0168,1.85387204875035e-08)
--(axis cs:0.017,1.07666238287212e-08)
--(axis cs:0.0172,6.238080275349e-09)
--(axis cs:0.0174,3.60592150732689e-09)
--(axis cs:0.0176,2.07968212190716e-09)
--(axis cs:0.0178,1.19677852625666e-09)
--(axis cs:0.018,6.87205808105705e-10)
--(axis cs:0.0182,3.93763499329262e-10)
--(axis cs:0.0184,2.25153370260175e-10)
--(axis cs:0.0186,1.28479536596914e-10)
--(axis cs:0.0188,7.3167725019549e-11)
--(axis cs:0.019,4.15864867650185e-11)
--(axis cs:0.0192,2.3591115560581e-11)
--(axis cs:0.0194,1.33574712820726e-11)
--(axis cs:0.0196,7.54909716595647e-12)
--(axis cs:0.0198,4.2586875071307e-12)
--(axis cs:0.02,2.39817608909909e-12)
--(axis cs:0.0202,1.34810900327012e-12)
--(axis cs:0.0204,7.56521520562321e-13)
--(axis cs:0.0206,4.23821702237897e-13)
--(axis cs:0.0208,2.37041085325845e-13)
--(axis cs:0.021,1.32359502443409e-13)
--(axis cs:0.0212,7.37887008057146e-14)
--(axis cs:0.0214,4.10714156450754e-14)
--(axis cs:0.0216,2.28252767651452e-14)
--(axis cs:0.0218,1.26657288252302e-14)
--(axis cs:0.022,7.0176700806092e-15)
--(axis cs:0.0222,3.88253085379576e-15)
--(axis cs:0.0224,2.14489671943879e-15)
--(axis cs:0.0226,1.18325234325703e-15)
--(axis cs:0.0228,6.51834935557492e-16)
--(axis cs:0.023,3.58588750638783e-16)
--(axis cs:0.0232,1.96998857004142e-16)
--(axis cs:0.0234,1.08080608000636e-16)
--(axis cs:0.0236,5.92185349401817e-17)
--(axis cs:0.0238,3.2404244611104e-17)
--(axis cs:0.024,1.77087884307283e-17)
--cycle;
\path [draw=none, fill=blue, fill opacity=0.5]
(axis cs:0,0)
--(axis cs:0.0002,3.00097034650853e-39)
--(axis cs:0.0004,1.33180595158039e-29)
--(axis cs:0.0006,4.34794883594433e-24)
--(axis cs:0.0008,2.88281813447308e-20)
--(axis cs:0.001,2.25959278886682e-17)
--(axis cs:0.0012,4.58849753621502e-15)
--(axis cs:0.0014,3.66862376248184e-13)
--(axis cs:0.0016,1.4826057132876e-11)
--(axis cs:0.0018,3.55812394866283e-10)
--(axis cs:0.002,5.6607416830461e-09)
--(axis cs:0.0022,6.45750577460221e-08)
--(axis cs:0.0024,5.59706827273519e-07)
--(axis cs:0.0026,3.85186816583851e-06)
--(axis cs:0.0028,2.17797182193631e-05)
--(axis cs:0.003,0.000103963098101098)
--(axis cs:0.0032,0.000428198943529838)
--(axis cs:0.0034,0.00154923143069868)
--(axis cs:0.0036,0.00499716303584501)
--(axis cs:0.0038,0.0145497409226953)
--(axis cs:0.004,0.0386429091930658)
--(axis cs:0.0042,0.0944621505149839)
--(axis cs:0.0044,0.214173948050834)
--(axis cs:0.0046,0.453416129407434)
--(axis cs:0.0048,0.901527460031158)
--(axis cs:0.005,1.69212633933122)
--(axis cs:0.0052,3.01173851254532)
--(axis cs:0.0054,5.10351436468107)
--(axis cs:0.0056,8.26297420559593)
--(axis cs:0.0058,12.8233899369747)
--(axis cs:0.006,19.1298809719443)
--(axis cs:0.0062,27.5033666828899)
--(axis cs:0.0064,38.197769617239)
--(axis cs:0.0066,51.3557906719054)
--(axis cs:0.0068,64.418951169431)
--(axis cs:0.007,49.6167625224316)
--(axis cs:0.0072,37.7133085450788)
--(axis cs:0.0074,28.3089543469283)
--(axis cs:0.0076,20.9991936459158)
--(axis cs:0.0078,15.4026440241404)
--(axis cs:0.008,11.1775340891288)
--(axis cs:0.0082,8.02935931856564)
--(axis cs:0.0084,5.71228280177854)
--(axis cs:0.0086,4.02649685568888)
--(axis cs:0.0088,2.81329566877787)
--(axis cs:0.009,1.94913981490007)
--(axis cs:0.0092,1.33957890477063)
--(axis cs:0.0094,0.913564437606188)
--(axis cs:0.0096,0.618435189640704)
--(axis cs:0.0098,0.415683987125745)
--(axis cs:0.01,0.277503574526014)
--(axis cs:0.0102,0.184045366929219)
--(axis cs:0.0104,0.121294291427453)
--(axis cs:0.0106,0.0794542187269003)
--(axis cs:0.0108,0.0517429197120304)
--(axis cs:0.011,0.0335068200368587)
--(axis cs:0.0112,0.0215799201029034)
--(axis cs:0.0114,0.0138255445023038)
--(axis cs:0.0116,0.00881267294525788)
--(axis cs:0.0118,0.00558983901571188)
--(axis cs:0.012,0.00352879748888552)
--(axis cs:0.0122,0.00221746133601891)
--(axis cs:0.0124,0.00138723530792914)
--(axis cs:0.0126,0.000864109583526985)
--(axis cs:0.0128,0.000536005372954958)
--(axis cs:0.013,0.00033113529045748)
--(axis cs:0.0132,0.000203765039925337)
--(axis cs:0.0134,0.000124908300098191)
--(axis cs:0.0136,7.62847405416888e-05)
--(axis cs:0.0138,4.6420953790526e-05)
--(axis cs:0.014,2.81490511547575e-05)
--(axis cs:0.0142,1.70109369999016e-05)
--(axis cs:0.0144,1.02458314352702e-05)
--(axis cs:0.0146,6.15118439560684e-06)
--(axis cs:0.0148,3.68128443355421e-06)
--(axis cs:0.015,2.1963631010631e-06)
--(axis cs:0.0152,1.30649181636633e-06)
--(axis cs:0.0154,7.74888778617601e-07)
--(axis cs:0.0156,4.5828247881716e-07)
--(axis cs:0.0158,2.70282688558907e-07)
--(axis cs:0.016,1.58972888238179e-07)
--(axis cs:0.0162,9.3255712797952e-08)
--(axis cs:0.0164,5.45634777566251e-08)
--(axis cs:0.0166,3.18440689554753e-08)
--(axis cs:0.0168,1.85387204875035e-08)
--(axis cs:0.017,1.07666238287212e-08)
--(axis cs:0.0172,6.238080275349e-09)
--(axis cs:0.0174,3.60592150732689e-09)
--(axis cs:0.0176,2.07968212190716e-09)
--(axis cs:0.0178,1.19677852625666e-09)
--(axis cs:0.018,6.87205808105705e-10)
--(axis cs:0.0182,3.93763499329262e-10)
--(axis cs:0.0184,2.25153370260175e-10)
--(axis cs:0.0186,1.28479536596914e-10)
--(axis cs:0.0188,7.3167725019549e-11)
--(axis cs:0.019,4.15864867650185e-11)
--(axis cs:0.0192,2.3591115560581e-11)
--(axis cs:0.0194,1.33574712820726e-11)
--(axis cs:0.0196,7.54909716595647e-12)
--(axis cs:0.0198,4.2586875071307e-12)
--(axis cs:0.02,2.39817608909909e-12)
--(axis cs:0.0202,1.34810900327012e-12)
--(axis cs:0.0204,7.56521520562321e-13)
--(axis cs:0.0206,4.23821702237897e-13)
--(axis cs:0.0208,2.37041085325845e-13)
--(axis cs:0.021,1.32359502443409e-13)
--(axis cs:0.0212,7.37887008057146e-14)
--(axis cs:0.0214,4.10714156450754e-14)
--(axis cs:0.0216,2.28252767651452e-14)
--(axis cs:0.0218,1.26657288252302e-14)
--(axis cs:0.022,7.0176700806092e-15)
--(axis cs:0.0222,3.88253085379576e-15)
--(axis cs:0.0224,2.14489671943879e-15)
--(axis cs:0.0226,1.18325234325703e-15)
--(axis cs:0.0228,6.51834935557492e-16)
--(axis cs:0.023,3.58588750638783e-16)
--(axis cs:0.0232,1.96998857004142e-16)
--(axis cs:0.0234,1.08080608000636e-16)
--(axis cs:0.0236,5.92185349401817e-17)
--(axis cs:0.0238,3.2404244611104e-17)
--(axis cs:0.024,1.77087884307283e-17)
--cycle;
\path [draw=none, fill=blue, fill opacity=0.5]
(axis cs:0,0)
--(axis cs:0.0002,1.63304653410572e-16)
--(axis cs:0.0004,2.90980802338031e-11)
--(axis cs:0.0006,2.54391471865381e-08)
--(axis cs:0.0008,2.52149359912999e-06)
--(axis cs:0.001,7.58076561312621e-05)
--(axis cs:0.0012,0.00107160766687709)
--(axis cs:0.0014,0.0089985803413974)
--(axis cs:0.0016,0.0516110463947514)
--(axis cs:0.0018,0.221227821059011)
--(axis cs:0.002,0.753634925751084)
--(axis cs:0.0022,2.13180511178777)
--(axis cs:0.0024,5.17199814367324)
--(axis cs:0.0026,11.0298671973361)
--(axis cs:0.0028,21.0757198851318)
--(axis cs:0.003,36.6331090584455)
--(axis cs:0.0032,58.6338009969209)
--(axis cs:0.0034,87.2857386482926)
--(axis cs:0.0036,121.857708878618)
--(axis cs:0.0038,160.652447408685)
--(axis cs:0.004,201.183753056193)
--(axis cs:0.0042,240.515622550635)
--(axis cs:0.0044,275.68210296399)
--(axis cs:0.0046,304.095014317866)
--(axis cs:0.0048,323.861262245946)
--(axis cs:0.005,331.097097920626)
--(axis cs:0.0052,307.397169591144)
--(axis cs:0.0054,278.436077650686)
--(axis cs:0.0056,246.494649171655)
--(axis cs:0.0058,213.618758823852)
--(axis cs:0.006,181.486437029819)
--(axis cs:0.0062,151.350391217959)
--(axis cs:0.0064,124.041120785831)
--(axis cs:0.0066,100.011933662309)
--(axis cs:0.0068,79.4075693181014)
--(axis cs:0.007,62.1411541566471)
--(axis cs:0.0072,47.9683798653516)
--(axis cs:0.0074,36.5520078018336)
--(axis cs:0.0076,27.5133978528566)
--(axis cs:0.0078,20.4704445030645)
--(axis cs:0.008,15.063048139348)
--(axis cs:0.0082,10.9681854093018)
--(axis cs:0.0084,7.90696667177293)
--(axis cs:0.0086,5.64598607193122)
--(axis cs:0.0088,3.99495713920016)
--(axis cs:0.009,2.80221773170423)
--(axis cs:0.0092,1.94927234418957)
--(axis cs:0.0094,1.34516936751816)
--(axis cs:0.0096,0.9212096338105)
--(axis cs:0.0098,0.626255300784991)
--(axis cs:0.01,0.42274846574095)
--(axis cs:0.0102,0.283445185521903)
--(axis cs:0.0104,0.188809728248726)
--(axis cs:0.0106,0.124983681895028)
--(axis cs:0.0108,0.0822347618707651)
--(axis cs:0.011,0.0537928772674796)
--(axis cs:0.0112,0.0349905096527558)
--(axis cs:0.0114,0.0226368578650084)
--(axis cs:0.0116,0.0145680770183744)
--(axis cs:0.0118,0.0093279080114947)
--(axis cs:0.012,0.00594338868209651)
--(axis cs:0.0122,0.00376894854181817)
--(axis cs:0.0124,0.00237907146773072)
--(axis cs:0.0126,0.00149505738056734)
--(axis cs:0.0128,0.000935470911759743)
--(axis cs:0.013,0.000582882458416365)
--(axis cs:0.0132,0.000361712565739222)
--(axis cs:0.0134,0.0002235781404359)
--(axis cs:0.0136,0.000137666147229334)
--(axis cs:0.0138,8.44508158591391e-05)
--(axis cs:0.014,5.16183585156872e-05)
--(axis cs:0.0142,3.14391682796165e-05)
--(axis cs:0.0144,1.90829492356098e-05)
--(axis cs:0.0146,1.15442812485094e-05)
--(axis cs:0.0148,6.9610222268174e-06)
--(axis cs:0.015,4.18408115287677e-06)
--(axis cs:0.0152,2.50716509816976e-06)
--(axis cs:0.0154,1.49780389284948e-06)
--(axis cs:0.0156,8.92171408243447e-07)
--(axis cs:0.0158,5.29899951305346e-07)
--(axis cs:0.016,3.13849492572507e-07)
--(axis cs:0.0162,1.85378594792661e-07)
--(axis cs:0.0164,1.0920333976009e-07)
--(axis cs:0.0166,6.41618691818942e-08)
--(axis cs:0.0168,3.7601785096784e-08)
--(axis cs:0.017,2.19813819838646e-08)
--(axis cs:0.0172,1.2818595305834e-08)
--(axis cs:0.0174,7.45740627181704e-09)
--(axis cs:0.0176,4.32831953568279e-09)
--(axis cs:0.0178,2.5064349436483e-09)
--(axis cs:0.018,1.4481712293854e-09)
--(axis cs:0.0182,8.3489105423697e-10)
--(axis cs:0.0184,4.80292177084666e-10)
--(axis cs:0.0186,2.75718431925808e-10)
--(axis cs:0.0188,1.57953347408636e-10)
--(axis cs:0.019,9.03050845606961e-11)
--(axis cs:0.0192,5.15267728595561e-11)
--(axis cs:0.0194,2.93432056497533e-11)
--(axis cs:0.0196,1.66783099860174e-11)
--(axis cs:0.0198,9.46197732965061e-12)
--(axis cs:0.02,5.3581168652589e-12)
--(axis cs:0.0202,3.02870843139119e-12)
--(axis cs:0.0204,1.70896006831727e-12)
--(axis cs:0.0206,9.62607563091312e-13)
--(axis cs:0.0208,5.4128111253654e-13)
--(axis cs:0.021,3.03854508696917e-13)
--(axis cs:0.0212,1.70290450773551e-13)
--(axis cs:0.0214,9.52815773485272e-14)
--(axis cs:0.0216,5.32271859618645e-14)
--(axis cs:0.0218,2.96876323072844e-14)
--(axis cs:0.022,1.65327977068176e-14)
--(axis cs:0.0222,9.19299054581887e-15)
--(axis cs:0.0224,5.10408139634969e-15)
--(axis cs:0.0226,2.82969120950356e-15)
--(axis cs:0.0228,1.56650310711317e-15)
--(axis cs:0.023,8.65972411911608e-16)
--(axis cs:0.0232,4.78043037803117e-16)
--(axis cs:0.0234,2.63529598823739e-16)
--(axis cs:0.0236,1.45077594230184e-16)
--(axis cs:0.0238,7.97606489054128e-17)
--(axis cs:0.024,4.37928315113879e-17)
--cycle;
\path [draw=none, fill=blue, fill opacity=0.5]
(axis cs:0,0)
--(axis cs:0.0002,2.87061962533969e-17)
--(axis cs:0.0004,7.38215500264892e-12)
--(axis cs:0.0006,7.99925233557908e-09)
--(axis cs:0.0008,9.23345890210269e-07)
--(axis cs:0.001,3.12426163997331e-05)
--(axis cs:0.0012,0.000486425585682615)
--(axis cs:0.0014,0.00443227584242255)
--(axis cs:0.0016,0.0272853176447845)
--(axis cs:0.0018,0.124491716808887)
--(axis cs:0.002,0.448457426929982)
--(axis cs:0.0022,1.33431227744341)
--(axis cs:0.0024,3.39008775333182)
--(axis cs:0.0026,7.54334486722043)
--(axis cs:0.0028,14.9917839314935)
--(axis cs:0.003,27.0300203986886)
--(axis cs:0.0032,44.7711936223266)
--(axis cs:0.0034,68.829161876082)
--(axis cs:0.0036,99.0522136164206)
--(axis cs:0.0038,134.390978721141)
--(axis cs:0.004,172.945707968733)
--(axis cs:0.0042,212.187129650739)
--(axis cs:0.0044,249.299540863043)
--(axis cs:0.0046,281.568493665345)
--(axis cs:0.0048,306.733195142688)
--(axis cs:0.005,323.241784801446)
--(axis cs:0.0052,307.397169591144)
--(axis cs:0.0054,278.436077650686)
--(axis cs:0.0056,246.494649171655)
--(axis cs:0.0058,213.618758823852)
--(axis cs:0.006,181.486437029819)
--(axis cs:0.0062,151.350391217959)
--(axis cs:0.0064,124.041120785831)
--(axis cs:0.0066,100.011933662309)
--(axis cs:0.0068,79.4075693181014)
--(axis cs:0.007,62.1411541566471)
--(axis cs:0.0072,47.9683798653516)
--(axis cs:0.0074,36.5520078018336)
--(axis cs:0.0076,27.5133978528566)
--(axis cs:0.0078,20.4704445030645)
--(axis cs:0.008,15.063048139348)
--(axis cs:0.0082,10.9681854093018)
--(axis cs:0.0084,7.90696667177293)
--(axis cs:0.0086,5.64598607193122)
--(axis cs:0.0088,3.99495713920016)
--(axis cs:0.009,2.80221773170423)
--(axis cs:0.0092,1.94927234418957)
--(axis cs:0.0094,1.34516936751816)
--(axis cs:0.0096,0.9212096338105)
--(axis cs:0.0098,0.626255300784991)
--(axis cs:0.01,0.42274846574095)
--(axis cs:0.0102,0.283445185521903)
--(axis cs:0.0104,0.188809728248726)
--(axis cs:0.0106,0.124983681895028)
--(axis cs:0.0108,0.0822347618707651)
--(axis cs:0.011,0.0537928772674796)
--(axis cs:0.0112,0.0349905096527558)
--(axis cs:0.0114,0.0226368578650084)
--(axis cs:0.0116,0.0145680770183744)
--(axis cs:0.0118,0.0093279080114947)
--(axis cs:0.012,0.00594338868209651)
--(axis cs:0.0122,0.00376894854181817)
--(axis cs:0.0124,0.00237907146773072)
--(axis cs:0.0126,0.00149505738056734)
--(axis cs:0.0128,0.000935470911759743)
--(axis cs:0.013,0.000582882458416365)
--(axis cs:0.0132,0.000361712565739222)
--(axis cs:0.0134,0.0002235781404359)
--(axis cs:0.0136,0.000137666147229334)
--(axis cs:0.0138,8.44508158591391e-05)
--(axis cs:0.014,5.16183585156872e-05)
--(axis cs:0.0142,3.14391682796165e-05)
--(axis cs:0.0144,1.90829492356098e-05)
--(axis cs:0.0146,1.15442812485094e-05)
--(axis cs:0.0148,6.9610222268174e-06)
--(axis cs:0.015,4.18408115287677e-06)
--(axis cs:0.0152,2.50716509816976e-06)
--(axis cs:0.0154,1.49780389284948e-06)
--(axis cs:0.0156,8.92171408243447e-07)
--(axis cs:0.0158,5.29899951305346e-07)
--(axis cs:0.016,3.13849492572507e-07)
--(axis cs:0.0162,1.85378594792661e-07)
--(axis cs:0.0164,1.0920333976009e-07)
--(axis cs:0.0166,6.41618691818942e-08)
--(axis cs:0.0168,3.7601785096784e-08)
--(axis cs:0.017,2.19813819838646e-08)
--(axis cs:0.0172,1.2818595305834e-08)
--(axis cs:0.0174,7.45740627181704e-09)
--(axis cs:0.0176,4.32831953568279e-09)
--(axis cs:0.0178,2.5064349436483e-09)
--(axis cs:0.018,1.4481712293854e-09)
--(axis cs:0.0182,8.3489105423697e-10)
--(axis cs:0.0184,4.80292177084666e-10)
--(axis cs:0.0186,2.75718431925808e-10)
--(axis cs:0.0188,1.57953347408636e-10)
--(axis cs:0.019,9.03050845606961e-11)
--(axis cs:0.0192,5.15267728595561e-11)
--(axis cs:0.0194,2.93432056497533e-11)
--(axis cs:0.0196,1.66783099860174e-11)
--(axis cs:0.0198,9.46197732965061e-12)
--(axis cs:0.02,5.3581168652589e-12)
--(axis cs:0.0202,3.02870843139119e-12)
--(axis cs:0.0204,1.70896006831727e-12)
--(axis cs:0.0206,9.62607563091312e-13)
--(axis cs:0.0208,5.4128111253654e-13)
--(axis cs:0.021,3.03854508696917e-13)
--(axis cs:0.0212,1.70290450773551e-13)
--(axis cs:0.0214,9.52815773485272e-14)
--(axis cs:0.0216,5.32271859618645e-14)
--(axis cs:0.0218,2.96876323072844e-14)
--(axis cs:0.022,1.65327977068176e-14)
--(axis cs:0.0222,9.19299054581887e-15)
--(axis cs:0.0224,5.10408139634969e-15)
--(axis cs:0.0226,2.82969120950356e-15)
--(axis cs:0.0228,1.56650310711317e-15)
--(axis cs:0.023,8.65972411911608e-16)
--(axis cs:0.0232,4.78043037803117e-16)
--(axis cs:0.0234,2.63529598823739e-16)
--(axis cs:0.0236,1.45077594230184e-16)
--(axis cs:0.0238,7.97606489054128e-17)
--(axis cs:0.024,4.37928315113879e-17)
--cycle;
\path [draw=none, fill=blue, fill opacity=0.5]
(axis cs:0,0)
--(axis cs:0.0002,4.77632767978611e-17)
--(axis cs:0.0004,1.10350468092225e-11)
--(axis cs:0.0006,1.12309897522798e-08)
--(axis cs:0.0008,1.23997965836469e-06)
--(axis cs:0.001,4.05331104838323e-05)
--(axis cs:0.0012,0.000613522820835944)
--(axis cs:0.0014,0.00545862533187201)
--(axis cs:0.0016,0.032916327874361)
--(axis cs:0.0018,0.147470611530503)
--(axis cs:0.002,0.522639852402313)
--(axis cs:0.0022,1.53224950990862)
--(axis cs:0.0024,3.84087832479369)
--(axis cs:0.0026,8.44109535473997)
--(axis cs:0.0028,16.5844717211536)
--(axis cs:0.003,29.5836281445369)
--(axis cs:0.0032,48.5131391024827)
--(axis cs:0.0034,73.884182008032)
--(axis cs:0.0036,105.388702355709)
--(axis cs:0.0038,141.794147211553)
--(axis cs:0.004,181.026617480686)
--(axis cs:0.0042,220.426658502137)
--(axis cs:0.0044,257.117315904375)
--(axis cs:0.0046,288.401464242004)
--(axis cs:0.0048,312.107862563584)
--(axis cs:0.005,326.827472666344)
--(axis cs:0.0052,307.397169591144)
--(axis cs:0.0054,278.436077650686)
--(axis cs:0.0056,246.494649171655)
--(axis cs:0.0058,213.618758823852)
--(axis cs:0.006,181.486437029819)
--(axis cs:0.0062,151.350391217959)
--(axis cs:0.0064,124.041120785831)
--(axis cs:0.0066,100.011933662309)
--(axis cs:0.0068,79.4075693181014)
--(axis cs:0.007,62.1411541566471)
--(axis cs:0.0072,47.9683798653516)
--(axis cs:0.0074,36.5520078018336)
--(axis cs:0.0076,27.5133978528566)
--(axis cs:0.0078,20.4704445030645)
--(axis cs:0.008,15.063048139348)
--(axis cs:0.0082,10.9681854093018)
--(axis cs:0.0084,7.90696667177293)
--(axis cs:0.0086,5.64598607193122)
--(axis cs:0.0088,3.99495713920016)
--(axis cs:0.009,2.80221773170423)
--(axis cs:0.0092,1.94927234418957)
--(axis cs:0.0094,1.34516936751816)
--(axis cs:0.0096,0.9212096338105)
--(axis cs:0.0098,0.626255300784991)
--(axis cs:0.01,0.42274846574095)
--(axis cs:0.0102,0.283445185521903)
--(axis cs:0.0104,0.188809728248726)
--(axis cs:0.0106,0.124983681895028)
--(axis cs:0.0108,0.0822347618707651)
--(axis cs:0.011,0.0537928772674796)
--(axis cs:0.0112,0.0349905096527558)
--(axis cs:0.0114,0.0226368578650084)
--(axis cs:0.0116,0.0145680770183744)
--(axis cs:0.0118,0.0093279080114947)
--(axis cs:0.012,0.00594338868209651)
--(axis cs:0.0122,0.00376894854181817)
--(axis cs:0.0124,0.00237907146773072)
--(axis cs:0.0126,0.00149505738056734)
--(axis cs:0.0128,0.000935470911759743)
--(axis cs:0.013,0.000582882458416365)
--(axis cs:0.0132,0.000361712565739222)
--(axis cs:0.0134,0.0002235781404359)
--(axis cs:0.0136,0.000137666147229334)
--(axis cs:0.0138,8.44508158591391e-05)
--(axis cs:0.014,5.16183585156872e-05)
--(axis cs:0.0142,3.14391682796165e-05)
--(axis cs:0.0144,1.90829492356098e-05)
--(axis cs:0.0146,1.15442812485094e-05)
--(axis cs:0.0148,6.9610222268174e-06)
--(axis cs:0.015,4.18408115287677e-06)
--(axis cs:0.0152,2.50716509816976e-06)
--(axis cs:0.0154,1.49780389284948e-06)
--(axis cs:0.0156,8.92171408243447e-07)
--(axis cs:0.0158,5.29899951305346e-07)
--(axis cs:0.016,3.13849492572507e-07)
--(axis cs:0.0162,1.85378594792661e-07)
--(axis cs:0.0164,1.0920333976009e-07)
--(axis cs:0.0166,6.41618691818942e-08)
--(axis cs:0.0168,3.7601785096784e-08)
--(axis cs:0.017,2.19813819838646e-08)
--(axis cs:0.0172,1.2818595305834e-08)
--(axis cs:0.0174,7.45740627181704e-09)
--(axis cs:0.0176,4.32831953568279e-09)
--(axis cs:0.0178,2.5064349436483e-09)
--(axis cs:0.018,1.4481712293854e-09)
--(axis cs:0.0182,8.3489105423697e-10)
--(axis cs:0.0184,4.80292177084666e-10)
--(axis cs:0.0186,2.75718431925808e-10)
--(axis cs:0.0188,1.57953347408636e-10)
--(axis cs:0.019,9.03050845606961e-11)
--(axis cs:0.0192,5.15267728595561e-11)
--(axis cs:0.0194,2.93432056497533e-11)
--(axis cs:0.0196,1.66783099860174e-11)
--(axis cs:0.0198,9.46197732965061e-12)
--(axis cs:0.02,5.3581168652589e-12)
--(axis cs:0.0202,3.02870843139119e-12)
--(axis cs:0.0204,1.70896006831727e-12)
--(axis cs:0.0206,9.62607563091312e-13)
--(axis cs:0.0208,5.4128111253654e-13)
--(axis cs:0.021,3.03854508696917e-13)
--(axis cs:0.0212,1.70290450773551e-13)
--(axis cs:0.0214,9.52815773485272e-14)
--(axis cs:0.0216,5.32271859618645e-14)
--(axis cs:0.0218,2.96876323072844e-14)
--(axis cs:0.022,1.65327977068176e-14)
--(axis cs:0.0222,9.19299054581887e-15)
--(axis cs:0.0224,5.10408139634969e-15)
--(axis cs:0.0226,2.82969120950356e-15)
--(axis cs:0.0228,1.56650310711317e-15)
--(axis cs:0.023,8.65972411911608e-16)
--(axis cs:0.0232,4.78043037803117e-16)
--(axis cs:0.0234,2.63529598823739e-16)
--(axis cs:0.0236,1.45077594230184e-16)
--(axis cs:0.0238,7.97606489054128e-17)
--(axis cs:0.024,4.37928315113879e-17)
--cycle;
\path [draw=none, fill=blue, fill opacity=0.5]
(axis cs:0,0)
--(axis cs:0.0002,2.50033281924293e-17)
--(axis cs:0.0004,6.61920264526063e-12)
--(axis cs:0.0006,7.29531684613714e-09)
--(axis cs:0.0008,8.52296949347681e-07)
--(axis cs:0.001,2.91093689969302e-05)
--(axis cs:0.0012,0.000456687181093467)
--(axis cs:0.0014,0.00418826597701668)
--(axis cs:0.0016,0.025927869467974)
--(axis cs:0.0018,0.118883729553988)
--(axis cs:0.002,0.43015163399355)
--(axis cs:0.0022,1.28497176113775)
--(axis cs:0.0024,3.27666446986953)
--(axis cs:0.0026,7.31548767742108)
--(axis cs:0.0028,14.5842153965101)
--(axis cs:0.003,26.3714255968948)
--(axis cs:0.0032,43.7988227032249)
--(axis cs:0.0034,67.5058869162191)
--(axis cs:0.0036,97.3813437364043)
--(axis cs:0.0038,132.424374144004)
--(axis cs:0.004,170.782523087049)
--(axis cs:0.0042,209.96311643966)
--(axis cs:0.0044,247.169302110898)
--(axis cs:0.0046,279.684632551789)
--(axis cs:0.0048,305.226638812366)
--(axis cs:0.005,322.206786149517)
--(axis cs:0.0052,307.397169591144)
--(axis cs:0.0054,278.436077650686)
--(axis cs:0.0056,246.494649171655)
--(axis cs:0.0058,213.618758823852)
--(axis cs:0.006,181.486437029819)
--(axis cs:0.0062,151.350391217959)
--(axis cs:0.0064,124.041120785831)
--(axis cs:0.0066,100.011933662309)
--(axis cs:0.0068,79.4075693181014)
--(axis cs:0.007,62.1411541566471)
--(axis cs:0.0072,47.9683798653516)
--(axis cs:0.0074,36.5520078018336)
--(axis cs:0.0076,27.5133978528566)
--(axis cs:0.0078,20.4704445030645)
--(axis cs:0.008,15.063048139348)
--(axis cs:0.0082,10.9681854093018)
--(axis cs:0.0084,7.90696667177293)
--(axis cs:0.0086,5.64598607193122)
--(axis cs:0.0088,3.99495713920016)
--(axis cs:0.009,2.80221773170423)
--(axis cs:0.0092,1.94927234418957)
--(axis cs:0.0094,1.34516936751816)
--(axis cs:0.0096,0.9212096338105)
--(axis cs:0.0098,0.626255300784991)
--(axis cs:0.01,0.42274846574095)
--(axis cs:0.0102,0.283445185521903)
--(axis cs:0.0104,0.188809728248726)
--(axis cs:0.0106,0.124983681895028)
--(axis cs:0.0108,0.0822347618707651)
--(axis cs:0.011,0.0537928772674796)
--(axis cs:0.0112,0.0349905096527558)
--(axis cs:0.0114,0.0226368578650084)
--(axis cs:0.0116,0.0145680770183744)
--(axis cs:0.0118,0.0093279080114947)
--(axis cs:0.012,0.00594338868209651)
--(axis cs:0.0122,0.00376894854181817)
--(axis cs:0.0124,0.00237907146773072)
--(axis cs:0.0126,0.00149505738056734)
--(axis cs:0.0128,0.000935470911759743)
--(axis cs:0.013,0.000582882458416365)
--(axis cs:0.0132,0.000361712565739222)
--(axis cs:0.0134,0.0002235781404359)
--(axis cs:0.0136,0.000137666147229334)
--(axis cs:0.0138,8.44508158591391e-05)
--(axis cs:0.014,5.16183585156872e-05)
--(axis cs:0.0142,3.14391682796165e-05)
--(axis cs:0.0144,1.90829492356098e-05)
--(axis cs:0.0146,1.15442812485094e-05)
--(axis cs:0.0148,6.9610222268174e-06)
--(axis cs:0.015,4.18408115287677e-06)
--(axis cs:0.0152,2.50716509816976e-06)
--(axis cs:0.0154,1.49780389284948e-06)
--(axis cs:0.0156,8.92171408243447e-07)
--(axis cs:0.0158,5.29899951305346e-07)
--(axis cs:0.016,3.13849492572507e-07)
--(axis cs:0.0162,1.85378594792661e-07)
--(axis cs:0.0164,1.0920333976009e-07)
--(axis cs:0.0166,6.41618691818942e-08)
--(axis cs:0.0168,3.7601785096784e-08)
--(axis cs:0.017,2.19813819838646e-08)
--(axis cs:0.0172,1.2818595305834e-08)
--(axis cs:0.0174,7.45740627181704e-09)
--(axis cs:0.0176,4.32831953568279e-09)
--(axis cs:0.0178,2.5064349436483e-09)
--(axis cs:0.018,1.4481712293854e-09)
--(axis cs:0.0182,8.3489105423697e-10)
--(axis cs:0.0184,4.80292177084666e-10)
--(axis cs:0.0186,2.75718431925808e-10)
--(axis cs:0.0188,1.57953347408636e-10)
--(axis cs:0.019,9.03050845606961e-11)
--(axis cs:0.0192,5.15267728595561e-11)
--(axis cs:0.0194,2.93432056497533e-11)
--(axis cs:0.0196,1.66783099860174e-11)
--(axis cs:0.0198,9.46197732965061e-12)
--(axis cs:0.02,5.3581168652589e-12)
--(axis cs:0.0202,3.02870843139119e-12)
--(axis cs:0.0204,1.70896006831727e-12)
--(axis cs:0.0206,9.62607563091312e-13)
--(axis cs:0.0208,5.4128111253654e-13)
--(axis cs:0.021,3.03854508696917e-13)
--(axis cs:0.0212,1.70290450773551e-13)
--(axis cs:0.0214,9.52815773485272e-14)
--(axis cs:0.0216,5.32271859618645e-14)
--(axis cs:0.0218,2.96876323072844e-14)
--(axis cs:0.022,1.65327977068176e-14)
--(axis cs:0.0222,9.19299054581887e-15)
--(axis cs:0.0224,5.10408139634969e-15)
--(axis cs:0.0226,2.82969120950356e-15)
--(axis cs:0.0228,1.56650310711317e-15)
--(axis cs:0.023,8.65972411911608e-16)
--(axis cs:0.0232,4.78043037803117e-16)
--(axis cs:0.0234,2.63529598823739e-16)
--(axis cs:0.0236,1.45077594230184e-16)
--(axis cs:0.0238,7.97606489054128e-17)
--(axis cs:0.024,4.37928315113879e-17)
--cycle;
\path [draw=none, fill=blue, fill opacity=0.5]
(axis cs:0,0)
--(axis cs:0.0002,4.38630231113233e-32)
--(axis cs:0.0004,9.42164546775413e-24)
--(axis cs:0.0006,5.22996289695702e-19)
--(axis cs:0.0008,9.86216555458081e-16)
--(axis cs:0.001,2.9143627354464e-13)
--(axis cs:0.0012,2.66669889722348e-11)
--(axis cs:0.0014,1.08650456690335e-09)
--(axis cs:0.0016,2.44844210199571e-08)
--(axis cs:0.0018,3.50988320156314e-07)
--(axis cs:0.002,3.52140105748224e-06)
--(axis cs:0.0022,2.64692151139118e-05)
--(axis cs:0.0024,0.000156751672867648)
--(axis cs:0.0026,0.000759830128964595)
--(axis cs:0.0028,0.00310561691820713)
--(axis cs:0.003,0.0109579090659142)
--(axis cs:0.0032,0.034016993111393)
--(axis cs:0.0034,0.0943609589409247)
--(axis cs:0.0036,0.236919036551997)
--(axis cs:0.0038,0.544246235965627)
--(axis cs:0.004,1.15434405660579)
--(axis cs:0.0042,2.27822422319553)
--(axis cs:0.0044,4.21196189035521)
--(axis cs:0.0046,7.33697307600828)
--(axis cs:0.0048,12.1029173241866)
--(axis cs:0.005,18.9902597620454)
--(axis cs:0.0052,28.4537592035236)
--(axis cs:0.0054,40.8529986597383)
--(axis cs:0.0056,56.3802313963309)
--(axis cs:0.0058,74.9980614577203)
--(axis cs:0.006,96.399078434761)
--(axis cs:0.0062,119.996496092707)
--(axis cs:0.0064,124.041120785831)
--(axis cs:0.0066,100.011933662309)
--(axis cs:0.0068,79.4075693181014)
--(axis cs:0.007,62.1411541566471)
--(axis cs:0.0072,47.9683798653516)
--(axis cs:0.0074,36.5520078018336)
--(axis cs:0.0076,27.5133978528566)
--(axis cs:0.0078,20.4704445030645)
--(axis cs:0.008,15.063048139348)
--(axis cs:0.0082,10.9681854093018)
--(axis cs:0.0084,7.90696667177293)
--(axis cs:0.0086,5.64598607193122)
--(axis cs:0.0088,3.99495713920016)
--(axis cs:0.009,2.80221773170423)
--(axis cs:0.0092,1.94927234418957)
--(axis cs:0.0094,1.34516936751816)
--(axis cs:0.0096,0.9212096338105)
--(axis cs:0.0098,0.626255300784991)
--(axis cs:0.01,0.42274846574095)
--(axis cs:0.0102,0.283445185521903)
--(axis cs:0.0104,0.188809728248726)
--(axis cs:0.0106,0.124983681895028)
--(axis cs:0.0108,0.0822347618707651)
--(axis cs:0.011,0.0537928772674796)
--(axis cs:0.0112,0.0349905096527558)
--(axis cs:0.0114,0.0226368578650084)
--(axis cs:0.0116,0.0145680770183744)
--(axis cs:0.0118,0.0093279080114947)
--(axis cs:0.012,0.00594338868209651)
--(axis cs:0.0122,0.00376894854181817)
--(axis cs:0.0124,0.00237907146773072)
--(axis cs:0.0126,0.00149505738056734)
--(axis cs:0.0128,0.000935470911759743)
--(axis cs:0.013,0.000582882458416365)
--(axis cs:0.0132,0.000361712565739222)
--(axis cs:0.0134,0.0002235781404359)
--(axis cs:0.0136,0.000137666147229334)
--(axis cs:0.0138,8.44508158591391e-05)
--(axis cs:0.014,5.16183585156872e-05)
--(axis cs:0.0142,3.14391682796165e-05)
--(axis cs:0.0144,1.90829492356098e-05)
--(axis cs:0.0146,1.15442812485094e-05)
--(axis cs:0.0148,6.9610222268174e-06)
--(axis cs:0.015,4.18408115287677e-06)
--(axis cs:0.0152,2.50716509816976e-06)
--(axis cs:0.0154,1.49780389284948e-06)
--(axis cs:0.0156,8.92171408243447e-07)
--(axis cs:0.0158,5.29899951305346e-07)
--(axis cs:0.016,3.13849492572507e-07)
--(axis cs:0.0162,1.85378594792661e-07)
--(axis cs:0.0164,1.0920333976009e-07)
--(axis cs:0.0166,6.41618691818942e-08)
--(axis cs:0.0168,3.7601785096784e-08)
--(axis cs:0.017,2.19813819838646e-08)
--(axis cs:0.0172,1.2818595305834e-08)
--(axis cs:0.0174,7.45740627181704e-09)
--(axis cs:0.0176,4.32831953568279e-09)
--(axis cs:0.0178,2.5064349436483e-09)
--(axis cs:0.018,1.4481712293854e-09)
--(axis cs:0.0182,8.3489105423697e-10)
--(axis cs:0.0184,4.80292177084666e-10)
--(axis cs:0.0186,2.75718431925808e-10)
--(axis cs:0.0188,1.57953347408636e-10)
--(axis cs:0.019,9.03050845606961e-11)
--(axis cs:0.0192,5.15267728595561e-11)
--(axis cs:0.0194,2.93432056497533e-11)
--(axis cs:0.0196,1.66783099860174e-11)
--(axis cs:0.0198,9.46197732965061e-12)
--(axis cs:0.02,5.3581168652589e-12)
--(axis cs:0.0202,3.02870843139119e-12)
--(axis cs:0.0204,1.70896006831727e-12)
--(axis cs:0.0206,9.62607563091312e-13)
--(axis cs:0.0208,5.4128111253654e-13)
--(axis cs:0.021,3.03854508696917e-13)
--(axis cs:0.0212,1.70290450773551e-13)
--(axis cs:0.0214,9.52815773485272e-14)
--(axis cs:0.0216,5.32271859618645e-14)
--(axis cs:0.0218,2.96876323072844e-14)
--(axis cs:0.022,1.65327977068176e-14)
--(axis cs:0.0222,9.19299054581887e-15)
--(axis cs:0.0224,5.10408139634969e-15)
--(axis cs:0.0226,2.82969120950356e-15)
--(axis cs:0.0228,1.56650310711317e-15)
--(axis cs:0.023,8.65972411911608e-16)
--(axis cs:0.0232,4.78043037803117e-16)
--(axis cs:0.0234,2.63529598823739e-16)
--(axis cs:0.0236,1.45077594230184e-16)
--(axis cs:0.0238,7.97606489054128e-17)
--(axis cs:0.024,4.37928315113879e-17)
--cycle;
\path [draw=none, fill=blue, fill opacity=0.5]
(axis cs:0,0)
--(axis cs:0.0002,4.28964223063707e-25)
--(axis cs:0.0004,4.23925669893049e-18)
--(axis cs:0.0006,3.8843035174309e-14)
--(axis cs:0.0008,2.03980960361959e-11)
--(axis cs:0.001,2.23578413477403e-09)
--(axis cs:0.0012,9.09612932925116e-08)
--(axis cs:0.0014,1.86740983355346e-06)
--(axis cs:0.0016,2.32374652780952e-05)
--(axis cs:0.0018,0.00019726676995222)
--(axis cs:0.002,0.00123850159771812)
--(axis cs:0.0022,0.00609150384571705)
--(axis cs:0.0024,0.0244907822733581)
--(axis cs:0.0026,0.0831291604329364)
--(axis cs:0.0028,0.244273624310176)
--(axis cs:0.003,0.633888506344497)
--(axis cs:0.0032,1.47614150045633)
--(axis cs:0.0034,3.12549923236712)
--(axis cs:0.0036,6.08284419664121)
--(axis cs:0.0038,10.9810360144135)
--(axis cs:0.004,18.5299350809066)
--(axis cs:0.0042,29.4207126030825)
--(axis cs:0.0044,44.2015281358291)
--(axis cs:0.0046,63.1472686753469)
--(axis cs:0.0048,86.1512421387252)
--(axis cs:0.005,112.664536610035)
--(axis cs:0.0052,141.699695822165)
--(axis cs:0.0054,171.902056209116)
--(axis cs:0.0056,201.678253673426)
--(axis cs:0.0058,213.618758823852)
--(axis cs:0.006,181.486437029819)
--(axis cs:0.0062,151.350391217959)
--(axis cs:0.0064,124.041120785831)
--(axis cs:0.0066,100.011933662309)
--(axis cs:0.0068,79.4075693181014)
--(axis cs:0.007,62.1411541566471)
--(axis cs:0.0072,47.9683798653516)
--(axis cs:0.0074,36.5520078018336)
--(axis cs:0.0076,27.5133978528566)
--(axis cs:0.0078,20.4704445030645)
--(axis cs:0.008,15.063048139348)
--(axis cs:0.0082,10.9681854093018)
--(axis cs:0.0084,7.90696667177293)
--(axis cs:0.0086,5.64598607193122)
--(axis cs:0.0088,3.99495713920016)
--(axis cs:0.009,2.80221773170423)
--(axis cs:0.0092,1.94927234418957)
--(axis cs:0.0094,1.34516936751816)
--(axis cs:0.0096,0.9212096338105)
--(axis cs:0.0098,0.626255300784991)
--(axis cs:0.01,0.42274846574095)
--(axis cs:0.0102,0.283445185521903)
--(axis cs:0.0104,0.188809728248726)
--(axis cs:0.0106,0.124983681895028)
--(axis cs:0.0108,0.0822347618707651)
--(axis cs:0.011,0.0537928772674796)
--(axis cs:0.0112,0.0349905096527558)
--(axis cs:0.0114,0.0226368578650084)
--(axis cs:0.0116,0.0145680770183744)
--(axis cs:0.0118,0.0093279080114947)
--(axis cs:0.012,0.00594338868209651)
--(axis cs:0.0122,0.00376894854181817)
--(axis cs:0.0124,0.00237907146773072)
--(axis cs:0.0126,0.00149505738056734)
--(axis cs:0.0128,0.000935470911759743)
--(axis cs:0.013,0.000582882458416365)
--(axis cs:0.0132,0.000361712565739222)
--(axis cs:0.0134,0.0002235781404359)
--(axis cs:0.0136,0.000137666147229334)
--(axis cs:0.0138,8.44508158591391e-05)
--(axis cs:0.014,5.16183585156872e-05)
--(axis cs:0.0142,3.14391682796165e-05)
--(axis cs:0.0144,1.90829492356098e-05)
--(axis cs:0.0146,1.15442812485094e-05)
--(axis cs:0.0148,6.9610222268174e-06)
--(axis cs:0.015,4.18408115287677e-06)
--(axis cs:0.0152,2.50716509816976e-06)
--(axis cs:0.0154,1.49780389284948e-06)
--(axis cs:0.0156,8.92171408243447e-07)
--(axis cs:0.0158,5.29899951305346e-07)
--(axis cs:0.016,3.13849492572507e-07)
--(axis cs:0.0162,1.85378594792661e-07)
--(axis cs:0.0164,1.0920333976009e-07)
--(axis cs:0.0166,6.41618691818942e-08)
--(axis cs:0.0168,3.7601785096784e-08)
--(axis cs:0.017,2.19813819838646e-08)
--(axis cs:0.0172,1.2818595305834e-08)
--(axis cs:0.0174,7.45740627181704e-09)
--(axis cs:0.0176,4.32831953568279e-09)
--(axis cs:0.0178,2.5064349436483e-09)
--(axis cs:0.018,1.4481712293854e-09)
--(axis cs:0.0182,8.3489105423697e-10)
--(axis cs:0.0184,4.80292177084666e-10)
--(axis cs:0.0186,2.75718431925808e-10)
--(axis cs:0.0188,1.57953347408636e-10)
--(axis cs:0.019,9.03050845606961e-11)
--(axis cs:0.0192,5.15267728595561e-11)
--(axis cs:0.0194,2.93432056497533e-11)
--(axis cs:0.0196,1.66783099860174e-11)
--(axis cs:0.0198,9.46197732965061e-12)
--(axis cs:0.02,5.3581168652589e-12)
--(axis cs:0.0202,3.02870843139119e-12)
--(axis cs:0.0204,1.70896006831727e-12)
--(axis cs:0.0206,9.62607563091312e-13)
--(axis cs:0.0208,5.4128111253654e-13)
--(axis cs:0.021,3.03854508696917e-13)
--(axis cs:0.0212,1.70290450773551e-13)
--(axis cs:0.0214,9.52815773485272e-14)
--(axis cs:0.0216,5.32271859618645e-14)
--(axis cs:0.0218,2.96876323072844e-14)
--(axis cs:0.022,1.65327977068176e-14)
--(axis cs:0.0222,9.19299054581887e-15)
--(axis cs:0.0224,5.10408139634969e-15)
--(axis cs:0.0226,2.82969120950356e-15)
--(axis cs:0.0228,1.56650310711317e-15)
--(axis cs:0.023,8.65972411911608e-16)
--(axis cs:0.0232,4.78043037803117e-16)
--(axis cs:0.0234,2.63529598823739e-16)
--(axis cs:0.0236,1.45077594230184e-16)
--(axis cs:0.0238,7.97606489054128e-17)
--(axis cs:0.024,4.37928315113879e-17)
--cycle;
\path [draw=none, fill=blue, fill opacity=0.5]
(axis cs:0,0)
--(axis cs:0.0002,3.00097034650853e-39)
--(axis cs:0.0004,1.33180595158039e-29)
--(axis cs:0.0006,4.34794883594433e-24)
--(axis cs:0.0008,2.88281813447308e-20)
--(axis cs:0.001,2.25959278886682e-17)
--(axis cs:0.0012,4.58849753621502e-15)
--(axis cs:0.0014,3.66862376248184e-13)
--(axis cs:0.0016,1.4826057132876e-11)
--(axis cs:0.0018,3.55812394866283e-10)
--(axis cs:0.002,5.6607416830461e-09)
--(axis cs:0.0022,6.45750577460221e-08)
--(axis cs:0.0024,5.59706827273519e-07)
--(axis cs:0.0026,3.85186816583851e-06)
--(axis cs:0.0028,2.17797182193631e-05)
--(axis cs:0.003,0.000103963098101098)
--(axis cs:0.0032,0.000428198943529838)
--(axis cs:0.0034,0.00154923143069868)
--(axis cs:0.0036,0.00499716303584501)
--(axis cs:0.0038,0.0145497409226953)
--(axis cs:0.004,0.0386429091930658)
--(axis cs:0.0042,0.0944621505149839)
--(axis cs:0.0044,0.214173948050834)
--(axis cs:0.0046,0.453416129407434)
--(axis cs:0.0048,0.901527460031158)
--(axis cs:0.005,1.69212633933122)
--(axis cs:0.0052,3.01173851254532)
--(axis cs:0.0054,5.10351436468107)
--(axis cs:0.0056,8.26297420559593)
--(axis cs:0.0058,12.8233899369747)
--(axis cs:0.006,19.1298809719443)
--(axis cs:0.0062,27.5033666828899)
--(axis cs:0.0064,38.197769617239)
--(axis cs:0.0066,51.3557906719054)
--(axis cs:0.0068,66.9696976206799)
--(axis cs:0.007,62.1411541566471)
--(axis cs:0.0072,47.9683798653516)
--(axis cs:0.0074,36.5520078018336)
--(axis cs:0.0076,27.5133978528566)
--(axis cs:0.0078,20.4704445030645)
--(axis cs:0.008,15.063048139348)
--(axis cs:0.0082,10.9681854093018)
--(axis cs:0.0084,7.90696667177293)
--(axis cs:0.0086,5.64598607193122)
--(axis cs:0.0088,3.99495713920016)
--(axis cs:0.009,2.80221773170423)
--(axis cs:0.0092,1.94927234418957)
--(axis cs:0.0094,1.34516936751816)
--(axis cs:0.0096,0.9212096338105)
--(axis cs:0.0098,0.626255300784991)
--(axis cs:0.01,0.42274846574095)
--(axis cs:0.0102,0.283445185521903)
--(axis cs:0.0104,0.188809728248726)
--(axis cs:0.0106,0.124983681895028)
--(axis cs:0.0108,0.0822347618707651)
--(axis cs:0.011,0.0537928772674796)
--(axis cs:0.0112,0.0349905096527558)
--(axis cs:0.0114,0.0226368578650084)
--(axis cs:0.0116,0.0145680770183744)
--(axis cs:0.0118,0.0093279080114947)
--(axis cs:0.012,0.00594338868209651)
--(axis cs:0.0122,0.00376894854181817)
--(axis cs:0.0124,0.00237907146773072)
--(axis cs:0.0126,0.00149505738056734)
--(axis cs:0.0128,0.000935470911759743)
--(axis cs:0.013,0.000582882458416365)
--(axis cs:0.0132,0.000361712565739222)
--(axis cs:0.0134,0.0002235781404359)
--(axis cs:0.0136,0.000137666147229334)
--(axis cs:0.0138,8.44508158591391e-05)
--(axis cs:0.014,5.16183585156872e-05)
--(axis cs:0.0142,3.14391682796165e-05)
--(axis cs:0.0144,1.90829492356098e-05)
--(axis cs:0.0146,1.15442812485094e-05)
--(axis cs:0.0148,6.9610222268174e-06)
--(axis cs:0.015,4.18408115287677e-06)
--(axis cs:0.0152,2.50716509816976e-06)
--(axis cs:0.0154,1.49780389284948e-06)
--(axis cs:0.0156,8.92171408243447e-07)
--(axis cs:0.0158,5.29899951305346e-07)
--(axis cs:0.016,3.13849492572507e-07)
--(axis cs:0.0162,1.85378594792661e-07)
--(axis cs:0.0164,1.0920333976009e-07)
--(axis cs:0.0166,6.41618691818942e-08)
--(axis cs:0.0168,3.7601785096784e-08)
--(axis cs:0.017,2.19813819838646e-08)
--(axis cs:0.0172,1.2818595305834e-08)
--(axis cs:0.0174,7.45740627181704e-09)
--(axis cs:0.0176,4.32831953568279e-09)
--(axis cs:0.0178,2.5064349436483e-09)
--(axis cs:0.018,1.4481712293854e-09)
--(axis cs:0.0182,8.3489105423697e-10)
--(axis cs:0.0184,4.80292177084666e-10)
--(axis cs:0.0186,2.75718431925808e-10)
--(axis cs:0.0188,1.57953347408636e-10)
--(axis cs:0.019,9.03050845606961e-11)
--(axis cs:0.0192,5.15267728595561e-11)
--(axis cs:0.0194,2.93432056497533e-11)
--(axis cs:0.0196,1.66783099860174e-11)
--(axis cs:0.0198,9.46197732965061e-12)
--(axis cs:0.02,5.3581168652589e-12)
--(axis cs:0.0202,3.02870843139119e-12)
--(axis cs:0.0204,1.70896006831727e-12)
--(axis cs:0.0206,9.62607563091312e-13)
--(axis cs:0.0208,5.4128111253654e-13)
--(axis cs:0.021,3.03854508696917e-13)
--(axis cs:0.0212,1.70290450773551e-13)
--(axis cs:0.0214,9.52815773485272e-14)
--(axis cs:0.0216,5.32271859618645e-14)
--(axis cs:0.0218,2.96876323072844e-14)
--(axis cs:0.022,1.65327977068176e-14)
--(axis cs:0.0222,9.19299054581887e-15)
--(axis cs:0.0224,5.10408139634969e-15)
--(axis cs:0.0226,2.82969120950356e-15)
--(axis cs:0.0228,1.56650310711317e-15)
--(axis cs:0.023,8.65972411911608e-16)
--(axis cs:0.0232,4.78043037803117e-16)
--(axis cs:0.0234,2.63529598823739e-16)
--(axis cs:0.0236,1.45077594230184e-16)
--(axis cs:0.0238,7.97606489054128e-17)
--(axis cs:0.024,4.37928315113879e-17)
--cycle;
\path [draw=none, fill=blue, fill opacity=0.5]
(axis cs:0,0)
--(axis cs:0.0002,2.87061962533969e-17)
--(axis cs:0.0004,7.38215500264892e-12)
--(axis cs:0.0006,7.99925233557908e-09)
--(axis cs:0.0008,9.23345890210269e-07)
--(axis cs:0.001,3.12426163997331e-05)
--(axis cs:0.0012,0.000486425585682615)
--(axis cs:0.0014,0.00443227584242255)
--(axis cs:0.0016,0.0272853176447845)
--(axis cs:0.0018,0.124491716808887)
--(axis cs:0.002,0.448457426929982)
--(axis cs:0.0022,1.33431227744341)
--(axis cs:0.0024,3.39008775333182)
--(axis cs:0.0026,7.54334486722043)
--(axis cs:0.0028,14.9917839314935)
--(axis cs:0.003,27.0300203986886)
--(axis cs:0.0032,44.7711936223266)
--(axis cs:0.0034,68.829161876082)
--(axis cs:0.0036,99.0522136164206)
--(axis cs:0.0038,134.390978721141)
--(axis cs:0.004,172.945707968733)
--(axis cs:0.0042,212.187129650739)
--(axis cs:0.0044,249.299540863043)
--(axis cs:0.0046,281.568493665345)
--(axis cs:0.0048,306.733195142688)
--(axis cs:0.005,323.241784801446)
--(axis cs:0.0052,330.377055934632)
--(axis cs:0.0054,325.534799085384)
--(axis cs:0.0056,309.024310466083)
--(axis cs:0.0058,286.470438616187)
--(axis cs:0.006,259.748868967789)
--(axis cs:0.0062,230.696275128522)
--(axis cs:0.0064,200.958909646343)
--(axis cs:0.0066,171.896916227713)
--(axis cs:0.0068,144.541092554638)
--(axis cs:0.007,119.592670514931)
--(axis cs:0.0072,97.4538923817262)
--(axis cs:0.0074,78.2770485071028)
--(axis cs:0.0076,62.0212911535369)
--(axis cs:0.0078,48.5090753606369)
--(axis cs:0.008,37.4768080842784)
--(axis cs:0.0082,28.6167498138539)
--(axis cs:0.0084,21.6091651725825)
--(axis cs:0.0086,16.1450881904592)
--(axis cs:0.0088,11.9408949354873)
--(axis cs:0.009,8.74626266792109)
--(axis cs:0.0092,6.34716264087156)
--(axis cs:0.0094,4.56539895083454)
--(axis cs:0.0096,3.25596397229389)
--(axis cs:0.0098,2.30320217419976)
--(axis cs:0.01,1.61650454197908)
--(axis cs:0.0102,1.12602130592532)
--(axis cs:0.0104,0.77869215182818)
--(axis cs:0.0106,0.534751570883728)
--(axis cs:0.0108,0.364767719860701)
--(axis cs:0.011,0.247208643245249)
--(axis cs:0.0112,0.166491866350165)
--(axis cs:0.0114,0.111454712568036)
--(axis cs:0.0116,0.0741769407678494)
--(axis cs:0.0118,0.0490895356025638)
--(axis cs:0.012,0.0323101258225112)
--(axis cs:0.0122,0.0211540883083413)
--(axis cs:0.0124,0.0137793399025455)
--(axis cs:0.0126,0.00893120312896225)
--(axis cs:0.0128,0.00576109091603547)
--(axis cs:0.013,0.00369892387946485)
--(axis cs:0.0132,0.00236418677869029)
--(axis cs:0.0134,0.00150446130315117)
--(axis cs:0.0136,0.000953295817434555)
--(axis cs:0.0138,0.000601553104271344)
--(axis cs:0.014,0.000378067905717709)
--(axis cs:0.0142,0.000236680792640458)
--(axis cs:0.0144,0.0001476044276804)
--(axis cs:0.0146,9.17112954174405e-05)
--(axis cs:0.0148,5.67773940803756e-05)
--(axis cs:0.015,3.50265759457039e-05)
--(axis cs:0.0152,2.15341792001447e-05)
--(axis cs:0.0154,1.31948560323714e-05)
--(axis cs:0.0156,8.05865429334375e-06)
--(axis cs:0.0158,4.9061000592821e-06)
--(axis cs:0.016,2.97755266794665e-06)
--(axis cs:0.0162,1.8016217212e-06)
--(axis cs:0.0164,1.08687507048972e-06)
--(axis cs:0.0166,6.53788297169587e-07)
--(axis cs:0.0168,3.92161291128713e-07)
--(axis cs:0.017,2.34579429601707e-07)
--(axis cs:0.0172,1.39939183488036e-07)
--(axis cs:0.0174,8.32604236358792e-08)
--(axis cs:0.0176,4.94097379728479e-08)
--(axis cs:0.0178,2.92472529178579e-08)
--(axis cs:0.018,1.7269483325279e-08)
--(axis cs:0.0182,1.01722660889628e-08)
--(axis cs:0.0184,5.97753076019741e-09)
--(axis cs:0.0186,3.50439135698977e-09)
--(axis cs:0.0188,2.04979446219223e-09)
--(axis cs:0.019,1.19628479054174e-09)
--(axis cs:0.0192,6.96633567508887e-10)
--(axis cs:0.0194,4.04797826285059e-10)
--(axis cs:0.0196,2.34721972140453e-10)
--(axis cs:0.0198,1.35821466411226e-10)
--(axis cs:0.02,7.84330231220916e-11)
--(axis cs:0.0202,4.52024149086559e-11)
--(axis cs:0.0204,2.59999344847354e-11)
--(axis cs:0.0206,1.49260926007679e-11)
--(axis cs:0.0208,8.55259986988054e-12)
--(axis cs:0.021,4.89150808995468e-12)
--(axis cs:0.0212,2.7925055208629e-12)
--(axis cs:0.0214,1.59134957922416e-12)
--(axis cs:0.0216,9.05254517671245e-13)
--(axis cs:0.0218,5.14069999253187e-13)
--(axis cs:0.022,2.91429031467523e-13)
--(axis cs:0.0222,1.64935660252564e-13)
--(axis cs:0.0224,9.31921612189804e-14)
--(axis cs:0.0226,5.25701031751156e-14)
--(axis cs:0.0228,2.96076644544015e-14)
--(axis cs:0.023,1.66489289581266e-14)
--(axis cs:0.0232,9.34751020436902e-15)
--(axis cs:0.0234,5.24014720070934e-15)
--(axis cs:0.0236,2.93318222365883e-15)
--(axis cs:0.0238,1.63942853674557e-15)
--(axis cs:0.024,9.14983888264259e-16)
--cycle;
\path [draw=none, fill=blue, fill opacity=0.5]
(axis cs:0,0)
--(axis cs:0.0002,4.77632767978611e-17)
--(axis cs:0.0004,1.10350468092225e-11)
--(axis cs:0.0006,1.12309897522798e-08)
--(axis cs:0.0008,1.23997965836469e-06)
--(axis cs:0.001,4.05331104838323e-05)
--(axis cs:0.0012,0.000613522820835944)
--(axis cs:0.0014,0.00545862533187201)
--(axis cs:0.0016,0.032916327874361)
--(axis cs:0.0018,0.147470611530503)
--(axis cs:0.002,0.522639852402313)
--(axis cs:0.0022,1.53224950990862)
--(axis cs:0.0024,3.84087832479369)
--(axis cs:0.0026,8.44109535473997)
--(axis cs:0.0028,16.5844717211536)
--(axis cs:0.003,29.5836281445369)
--(axis cs:0.0032,48.5131391024827)
--(axis cs:0.0034,73.884182008032)
--(axis cs:0.0036,105.388702355709)
--(axis cs:0.0038,141.794147211553)
--(axis cs:0.004,181.026617480686)
--(axis cs:0.0042,220.426658502137)
--(axis cs:0.0044,257.117315904375)
--(axis cs:0.0046,288.401464242004)
--(axis cs:0.0048,312.107862563584)
--(axis cs:0.005,326.827472666344)
--(axis cs:0.0052,332.013328813822)
--(axis cs:0.0054,325.534799085384)
--(axis cs:0.0056,309.024310466083)
--(axis cs:0.0058,286.470438616187)
--(axis cs:0.006,259.748868967789)
--(axis cs:0.0062,230.696275128522)
--(axis cs:0.0064,200.958909646343)
--(axis cs:0.0066,171.896916227713)
--(axis cs:0.0068,144.541092554638)
--(axis cs:0.007,119.592670514931)
--(axis cs:0.0072,97.4538923817262)
--(axis cs:0.0074,78.2770485071028)
--(axis cs:0.0076,62.0212911535369)
--(axis cs:0.0078,48.5090753606369)
--(axis cs:0.008,37.4768080842784)
--(axis cs:0.0082,28.6167498138539)
--(axis cs:0.0084,21.6091651725825)
--(axis cs:0.0086,16.1450881904592)
--(axis cs:0.0088,11.9408949354873)
--(axis cs:0.009,8.74626266792109)
--(axis cs:0.0092,6.34716264087156)
--(axis cs:0.0094,4.56539895083454)
--(axis cs:0.0096,3.25596397229389)
--(axis cs:0.0098,2.30320217419976)
--(axis cs:0.01,1.61650454197908)
--(axis cs:0.0102,1.12602130592532)
--(axis cs:0.0104,0.77869215182818)
--(axis cs:0.0106,0.534751570883728)
--(axis cs:0.0108,0.364767719860701)
--(axis cs:0.011,0.247208643245249)
--(axis cs:0.0112,0.166491866350165)
--(axis cs:0.0114,0.111454712568036)
--(axis cs:0.0116,0.0741769407678494)
--(axis cs:0.0118,0.0490895356025638)
--(axis cs:0.012,0.0323101258225112)
--(axis cs:0.0122,0.0211540883083413)
--(axis cs:0.0124,0.0137793399025455)
--(axis cs:0.0126,0.00893120312896225)
--(axis cs:0.0128,0.00576109091603547)
--(axis cs:0.013,0.00369892387946485)
--(axis cs:0.0132,0.00236418677869029)
--(axis cs:0.0134,0.00150446130315117)
--(axis cs:0.0136,0.000953295817434555)
--(axis cs:0.0138,0.000601553104271344)
--(axis cs:0.014,0.000378067905717709)
--(axis cs:0.0142,0.000236680792640458)
--(axis cs:0.0144,0.0001476044276804)
--(axis cs:0.0146,9.17112954174405e-05)
--(axis cs:0.0148,5.67773940803756e-05)
--(axis cs:0.015,3.50265759457039e-05)
--(axis cs:0.0152,2.15341792001447e-05)
--(axis cs:0.0154,1.31948560323714e-05)
--(axis cs:0.0156,8.05865429334375e-06)
--(axis cs:0.0158,4.9061000592821e-06)
--(axis cs:0.016,2.97755266794665e-06)
--(axis cs:0.0162,1.8016217212e-06)
--(axis cs:0.0164,1.08687507048972e-06)
--(axis cs:0.0166,6.53788297169587e-07)
--(axis cs:0.0168,3.92161291128713e-07)
--(axis cs:0.017,2.34579429601707e-07)
--(axis cs:0.0172,1.39939183488036e-07)
--(axis cs:0.0174,8.32604236358792e-08)
--(axis cs:0.0176,4.94097379728479e-08)
--(axis cs:0.0178,2.92472529178579e-08)
--(axis cs:0.018,1.7269483325279e-08)
--(axis cs:0.0182,1.01722660889628e-08)
--(axis cs:0.0184,5.97753076019741e-09)
--(axis cs:0.0186,3.50439135698977e-09)
--(axis cs:0.0188,2.04979446219223e-09)
--(axis cs:0.019,1.19628479054174e-09)
--(axis cs:0.0192,6.96633567508887e-10)
--(axis cs:0.0194,4.04797826285059e-10)
--(axis cs:0.0196,2.34721972140453e-10)
--(axis cs:0.0198,1.35821466411226e-10)
--(axis cs:0.02,7.84330231220916e-11)
--(axis cs:0.0202,4.52024149086559e-11)
--(axis cs:0.0204,2.59999344847354e-11)
--(axis cs:0.0206,1.49260926007679e-11)
--(axis cs:0.0208,8.55259986988054e-12)
--(axis cs:0.021,4.89150808995468e-12)
--(axis cs:0.0212,2.7925055208629e-12)
--(axis cs:0.0214,1.59134957922416e-12)
--(axis cs:0.0216,9.05254517671245e-13)
--(axis cs:0.0218,5.14069999253187e-13)
--(axis cs:0.022,2.91429031467523e-13)
--(axis cs:0.0222,1.64935660252564e-13)
--(axis cs:0.0224,9.31921612189804e-14)
--(axis cs:0.0226,5.25701031751156e-14)
--(axis cs:0.0228,2.96076644544015e-14)
--(axis cs:0.023,1.66489289581266e-14)
--(axis cs:0.0232,9.34751020436902e-15)
--(axis cs:0.0234,5.24014720070934e-15)
--(axis cs:0.0236,2.93318222365883e-15)
--(axis cs:0.0238,1.63942853674557e-15)
--(axis cs:0.024,9.14983888264259e-16)
--cycle;
\path [draw=none, fill=blue, fill opacity=0.5]
(axis cs:0,0)
--(axis cs:0.0002,2.50033281924293e-17)
--(axis cs:0.0004,6.61920264526063e-12)
--(axis cs:0.0006,7.29531684613714e-09)
--(axis cs:0.0008,8.52296949347681e-07)
--(axis cs:0.001,2.91093689969302e-05)
--(axis cs:0.0012,0.000456687181093467)
--(axis cs:0.0014,0.00418826597701668)
--(axis cs:0.0016,0.025927869467974)
--(axis cs:0.0018,0.118883729553988)
--(axis cs:0.002,0.43015163399355)
--(axis cs:0.0022,1.28497176113775)
--(axis cs:0.0024,3.27666446986953)
--(axis cs:0.0026,7.31548767742108)
--(axis cs:0.0028,14.5842153965101)
--(axis cs:0.003,26.3714255968948)
--(axis cs:0.0032,43.7988227032249)
--(axis cs:0.0034,67.5058869162191)
--(axis cs:0.0036,97.3813437364043)
--(axis cs:0.0038,132.424374144004)
--(axis cs:0.004,170.782523087049)
--(axis cs:0.0042,209.96311643966)
--(axis cs:0.0044,247.169302110898)
--(axis cs:0.0046,279.684632551789)
--(axis cs:0.0048,305.226638812366)
--(axis cs:0.005,322.206786149517)
--(axis cs:0.0052,329.86291869807)
--(axis cs:0.0054,325.534799085384)
--(axis cs:0.0056,309.024310466083)
--(axis cs:0.0058,286.470438616187)
--(axis cs:0.006,259.748868967789)
--(axis cs:0.0062,230.696275128522)
--(axis cs:0.0064,200.958909646343)
--(axis cs:0.0066,171.896916227713)
--(axis cs:0.0068,144.541092554638)
--(axis cs:0.007,119.592670514931)
--(axis cs:0.0072,97.4538923817262)
--(axis cs:0.0074,78.2770485071028)
--(axis cs:0.0076,62.0212911535369)
--(axis cs:0.0078,48.5090753606369)
--(axis cs:0.008,37.4768080842784)
--(axis cs:0.0082,28.6167498138539)
--(axis cs:0.0084,21.6091651725825)
--(axis cs:0.0086,16.1450881904592)
--(axis cs:0.0088,11.9408949354873)
--(axis cs:0.009,8.74626266792109)
--(axis cs:0.0092,6.34716264087156)
--(axis cs:0.0094,4.56539895083454)
--(axis cs:0.0096,3.25596397229389)
--(axis cs:0.0098,2.30320217419976)
--(axis cs:0.01,1.61650454197908)
--(axis cs:0.0102,1.12602130592532)
--(axis cs:0.0104,0.77869215182818)
--(axis cs:0.0106,0.534751570883728)
--(axis cs:0.0108,0.364767719860701)
--(axis cs:0.011,0.247208643245249)
--(axis cs:0.0112,0.166491866350165)
--(axis cs:0.0114,0.111454712568036)
--(axis cs:0.0116,0.0741769407678494)
--(axis cs:0.0118,0.0490895356025638)
--(axis cs:0.012,0.0323101258225112)
--(axis cs:0.0122,0.0211540883083413)
--(axis cs:0.0124,0.0137793399025455)
--(axis cs:0.0126,0.00893120312896225)
--(axis cs:0.0128,0.00576109091603547)
--(axis cs:0.013,0.00369892387946485)
--(axis cs:0.0132,0.00236418677869029)
--(axis cs:0.0134,0.00150446130315117)
--(axis cs:0.0136,0.000953295817434555)
--(axis cs:0.0138,0.000601553104271344)
--(axis cs:0.014,0.000378067905717709)
--(axis cs:0.0142,0.000236680792640458)
--(axis cs:0.0144,0.0001476044276804)
--(axis cs:0.0146,9.17112954174405e-05)
--(axis cs:0.0148,5.67773940803756e-05)
--(axis cs:0.015,3.50265759457039e-05)
--(axis cs:0.0152,2.15341792001447e-05)
--(axis cs:0.0154,1.31948560323714e-05)
--(axis cs:0.0156,8.05865429334375e-06)
--(axis cs:0.0158,4.9061000592821e-06)
--(axis cs:0.016,2.97755266794665e-06)
--(axis cs:0.0162,1.8016217212e-06)
--(axis cs:0.0164,1.08687507048972e-06)
--(axis cs:0.0166,6.53788297169587e-07)
--(axis cs:0.0168,3.92161291128713e-07)
--(axis cs:0.017,2.34579429601707e-07)
--(axis cs:0.0172,1.39939183488036e-07)
--(axis cs:0.0174,8.32604236358792e-08)
--(axis cs:0.0176,4.94097379728479e-08)
--(axis cs:0.0178,2.92472529178579e-08)
--(axis cs:0.018,1.7269483325279e-08)
--(axis cs:0.0182,1.01722660889628e-08)
--(axis cs:0.0184,5.97753076019741e-09)
--(axis cs:0.0186,3.50439135698977e-09)
--(axis cs:0.0188,2.04979446219223e-09)
--(axis cs:0.019,1.19628479054174e-09)
--(axis cs:0.0192,6.96633567508887e-10)
--(axis cs:0.0194,4.04797826285059e-10)
--(axis cs:0.0196,2.34721972140453e-10)
--(axis cs:0.0198,1.35821466411226e-10)
--(axis cs:0.02,7.84330231220916e-11)
--(axis cs:0.0202,4.52024149086559e-11)
--(axis cs:0.0204,2.59999344847354e-11)
--(axis cs:0.0206,1.49260926007679e-11)
--(axis cs:0.0208,8.55259986988054e-12)
--(axis cs:0.021,4.89150808995468e-12)
--(axis cs:0.0212,2.7925055208629e-12)
--(axis cs:0.0214,1.59134957922416e-12)
--(axis cs:0.0216,9.05254517671245e-13)
--(axis cs:0.0218,5.14069999253187e-13)
--(axis cs:0.022,2.91429031467523e-13)
--(axis cs:0.0222,1.64935660252564e-13)
--(axis cs:0.0224,9.31921612189804e-14)
--(axis cs:0.0226,5.25701031751156e-14)
--(axis cs:0.0228,2.96076644544015e-14)
--(axis cs:0.023,1.66489289581266e-14)
--(axis cs:0.0232,9.34751020436902e-15)
--(axis cs:0.0234,5.24014720070934e-15)
--(axis cs:0.0236,2.93318222365883e-15)
--(axis cs:0.0238,1.63942853674557e-15)
--(axis cs:0.024,9.14983888264259e-16)
--cycle;
\path [draw=none, fill=blue, fill opacity=0.5]
(axis cs:0,0)
--(axis cs:0.0002,4.38630231113233e-32)
--(axis cs:0.0004,9.42164546775413e-24)
--(axis cs:0.0006,5.22996289695702e-19)
--(axis cs:0.0008,9.86216555458081e-16)
--(axis cs:0.001,2.9143627354464e-13)
--(axis cs:0.0012,2.66669889722348e-11)
--(axis cs:0.0014,1.08650456690335e-09)
--(axis cs:0.0016,2.44844210199571e-08)
--(axis cs:0.0018,3.50988320156314e-07)
--(axis cs:0.002,3.52140105748224e-06)
--(axis cs:0.0022,2.64692151139118e-05)
--(axis cs:0.0024,0.000156751672867648)
--(axis cs:0.0026,0.000759830128964595)
--(axis cs:0.0028,0.00310561691820713)
--(axis cs:0.003,0.0109579090659142)
--(axis cs:0.0032,0.034016993111393)
--(axis cs:0.0034,0.0943609589409247)
--(axis cs:0.0036,0.236919036551997)
--(axis cs:0.0038,0.544246235965627)
--(axis cs:0.004,1.15434405660579)
--(axis cs:0.0042,2.27822422319553)
--(axis cs:0.0044,4.21196189035521)
--(axis cs:0.0046,7.33697307600828)
--(axis cs:0.0048,12.1029173241866)
--(axis cs:0.005,18.9902597620454)
--(axis cs:0.0052,28.4537592035236)
--(axis cs:0.0054,40.8529986597383)
--(axis cs:0.0056,56.3802313963309)
--(axis cs:0.0058,74.9980614577203)
--(axis cs:0.006,96.399078434761)
--(axis cs:0.0062,119.996496092707)
--(axis cs:0.0064,144.949774971286)
--(axis cs:0.0066,170.223290391324)
--(axis cs:0.0068,144.541092554638)
--(axis cs:0.007,119.592670514931)
--(axis cs:0.0072,97.4538923817262)
--(axis cs:0.0074,78.2770485071028)
--(axis cs:0.0076,62.0212911535369)
--(axis cs:0.0078,48.5090753606369)
--(axis cs:0.008,37.4768080842784)
--(axis cs:0.0082,28.6167498138539)
--(axis cs:0.0084,21.6091651725825)
--(axis cs:0.0086,16.1450881904592)
--(axis cs:0.0088,11.9408949354873)
--(axis cs:0.009,8.74626266792109)
--(axis cs:0.0092,6.34716264087156)
--(axis cs:0.0094,4.56539895083454)
--(axis cs:0.0096,3.25596397229389)
--(axis cs:0.0098,2.30320217419976)
--(axis cs:0.01,1.61650454197908)
--(axis cs:0.0102,1.12602130592532)
--(axis cs:0.0104,0.77869215182818)
--(axis cs:0.0106,0.534751570883728)
--(axis cs:0.0108,0.364767719860701)
--(axis cs:0.011,0.247208643245249)
--(axis cs:0.0112,0.166491866350165)
--(axis cs:0.0114,0.111454712568036)
--(axis cs:0.0116,0.0741769407678494)
--(axis cs:0.0118,0.0490895356025638)
--(axis cs:0.012,0.0323101258225112)
--(axis cs:0.0122,0.0211540883083413)
--(axis cs:0.0124,0.0137793399025455)
--(axis cs:0.0126,0.00893120312896225)
--(axis cs:0.0128,0.00576109091603547)
--(axis cs:0.013,0.00369892387946485)
--(axis cs:0.0132,0.00236418677869029)
--(axis cs:0.0134,0.00150446130315117)
--(axis cs:0.0136,0.000953295817434555)
--(axis cs:0.0138,0.000601553104271344)
--(axis cs:0.014,0.000378067905717709)
--(axis cs:0.0142,0.000236680792640458)
--(axis cs:0.0144,0.0001476044276804)
--(axis cs:0.0146,9.17112954174405e-05)
--(axis cs:0.0148,5.67773940803756e-05)
--(axis cs:0.015,3.50265759457039e-05)
--(axis cs:0.0152,2.15341792001447e-05)
--(axis cs:0.0154,1.31948560323714e-05)
--(axis cs:0.0156,8.05865429334375e-06)
--(axis cs:0.0158,4.9061000592821e-06)
--(axis cs:0.016,2.97755266794665e-06)
--(axis cs:0.0162,1.8016217212e-06)
--(axis cs:0.0164,1.08687507048972e-06)
--(axis cs:0.0166,6.53788297169587e-07)
--(axis cs:0.0168,3.92161291128713e-07)
--(axis cs:0.017,2.34579429601707e-07)
--(axis cs:0.0172,1.39939183488036e-07)
--(axis cs:0.0174,8.32604236358792e-08)
--(axis cs:0.0176,4.94097379728479e-08)
--(axis cs:0.0178,2.92472529178579e-08)
--(axis cs:0.018,1.7269483325279e-08)
--(axis cs:0.0182,1.01722660889628e-08)
--(axis cs:0.0184,5.97753076019741e-09)
--(axis cs:0.0186,3.50439135698977e-09)
--(axis cs:0.0188,2.04979446219223e-09)
--(axis cs:0.019,1.19628479054174e-09)
--(axis cs:0.0192,6.96633567508887e-10)
--(axis cs:0.0194,4.04797826285059e-10)
--(axis cs:0.0196,2.34721972140453e-10)
--(axis cs:0.0198,1.35821466411226e-10)
--(axis cs:0.02,7.84330231220916e-11)
--(axis cs:0.0202,4.52024149086559e-11)
--(axis cs:0.0204,2.59999344847354e-11)
--(axis cs:0.0206,1.49260926007679e-11)
--(axis cs:0.0208,8.55259986988054e-12)
--(axis cs:0.021,4.89150808995468e-12)
--(axis cs:0.0212,2.7925055208629e-12)
--(axis cs:0.0214,1.59134957922416e-12)
--(axis cs:0.0216,9.05254517671245e-13)
--(axis cs:0.0218,5.14069999253187e-13)
--(axis cs:0.022,2.91429031467523e-13)
--(axis cs:0.0222,1.64935660252564e-13)
--(axis cs:0.0224,9.31921612189804e-14)
--(axis cs:0.0226,5.25701031751156e-14)
--(axis cs:0.0228,2.96076644544015e-14)
--(axis cs:0.023,1.66489289581266e-14)
--(axis cs:0.0232,9.34751020436902e-15)
--(axis cs:0.0234,5.24014720070934e-15)
--(axis cs:0.0236,2.93318222365883e-15)
--(axis cs:0.0238,1.63942853674557e-15)
--(axis cs:0.024,9.14983888264259e-16)
--cycle;
\path [draw=none, fill=blue, fill opacity=0.5]
(axis cs:0,0)
--(axis cs:0.0002,4.28964223063707e-25)
--(axis cs:0.0004,4.23925669893049e-18)
--(axis cs:0.0006,3.8843035174309e-14)
--(axis cs:0.0008,2.03980960361959e-11)
--(axis cs:0.001,2.23578413477403e-09)
--(axis cs:0.0012,9.09612932925116e-08)
--(axis cs:0.0014,1.86740983355346e-06)
--(axis cs:0.0016,2.32374652780952e-05)
--(axis cs:0.0018,0.00019726676995222)
--(axis cs:0.002,0.00123850159771812)
--(axis cs:0.0022,0.00609150384571705)
--(axis cs:0.0024,0.0244907822733581)
--(axis cs:0.0026,0.0831291604329364)
--(axis cs:0.0028,0.244273624310176)
--(axis cs:0.003,0.633888506344497)
--(axis cs:0.0032,1.47614150045633)
--(axis cs:0.0034,3.12549923236712)
--(axis cs:0.0036,6.08284419664121)
--(axis cs:0.0038,10.9810360144135)
--(axis cs:0.004,18.5299350809066)
--(axis cs:0.0042,29.4207126030825)
--(axis cs:0.0044,44.2015281358291)
--(axis cs:0.0046,63.1472686753469)
--(axis cs:0.0048,86.1512421387252)
--(axis cs:0.005,112.664536610035)
--(axis cs:0.0052,141.699695822165)
--(axis cs:0.0054,171.902056209116)
--(axis cs:0.0056,201.678253673426)
--(axis cs:0.0058,229.360553795801)
--(axis cs:0.006,253.380099640291)
--(axis cs:0.0062,230.696275128522)
--(axis cs:0.0064,200.958909646343)
--(axis cs:0.0066,171.896916227713)
--(axis cs:0.0068,144.541092554638)
--(axis cs:0.007,119.592670514931)
--(axis cs:0.0072,97.4538923817262)
--(axis cs:0.0074,78.2770485071028)
--(axis cs:0.0076,62.0212911535369)
--(axis cs:0.0078,48.5090753606369)
--(axis cs:0.008,37.4768080842784)
--(axis cs:0.0082,28.6167498138539)
--(axis cs:0.0084,21.6091651725825)
--(axis cs:0.0086,16.1450881904592)
--(axis cs:0.0088,11.9408949354873)
--(axis cs:0.009,8.74626266792109)
--(axis cs:0.0092,6.34716264087156)
--(axis cs:0.0094,4.56539895083454)
--(axis cs:0.0096,3.25596397229389)
--(axis cs:0.0098,2.30320217419976)
--(axis cs:0.01,1.61650454197908)
--(axis cs:0.0102,1.12602130592532)
--(axis cs:0.0104,0.77869215182818)
--(axis cs:0.0106,0.534751570883728)
--(axis cs:0.0108,0.364767719860701)
--(axis cs:0.011,0.247208643245249)
--(axis cs:0.0112,0.166491866350165)
--(axis cs:0.0114,0.111454712568036)
--(axis cs:0.0116,0.0741769407678494)
--(axis cs:0.0118,0.0490895356025638)
--(axis cs:0.012,0.0323101258225112)
--(axis cs:0.0122,0.0211540883083413)
--(axis cs:0.0124,0.0137793399025455)
--(axis cs:0.0126,0.00893120312896225)
--(axis cs:0.0128,0.00576109091603547)
--(axis cs:0.013,0.00369892387946485)
--(axis cs:0.0132,0.00236418677869029)
--(axis cs:0.0134,0.00150446130315117)
--(axis cs:0.0136,0.000953295817434555)
--(axis cs:0.0138,0.000601553104271344)
--(axis cs:0.014,0.000378067905717709)
--(axis cs:0.0142,0.000236680792640458)
--(axis cs:0.0144,0.0001476044276804)
--(axis cs:0.0146,9.17112954174405e-05)
--(axis cs:0.0148,5.67773940803756e-05)
--(axis cs:0.015,3.50265759457039e-05)
--(axis cs:0.0152,2.15341792001447e-05)
--(axis cs:0.0154,1.31948560323714e-05)
--(axis cs:0.0156,8.05865429334375e-06)
--(axis cs:0.0158,4.9061000592821e-06)
--(axis cs:0.016,2.97755266794665e-06)
--(axis cs:0.0162,1.8016217212e-06)
--(axis cs:0.0164,1.08687507048972e-06)
--(axis cs:0.0166,6.53788297169587e-07)
--(axis cs:0.0168,3.92161291128713e-07)
--(axis cs:0.017,2.34579429601707e-07)
--(axis cs:0.0172,1.39939183488036e-07)
--(axis cs:0.0174,8.32604236358792e-08)
--(axis cs:0.0176,4.94097379728479e-08)
--(axis cs:0.0178,2.92472529178579e-08)
--(axis cs:0.018,1.7269483325279e-08)
--(axis cs:0.0182,1.01722660889628e-08)
--(axis cs:0.0184,5.97753076019741e-09)
--(axis cs:0.0186,3.50439135698977e-09)
--(axis cs:0.0188,2.04979446219223e-09)
--(axis cs:0.019,1.19628479054174e-09)
--(axis cs:0.0192,6.96633567508887e-10)
--(axis cs:0.0194,4.04797826285059e-10)
--(axis cs:0.0196,2.34721972140453e-10)
--(axis cs:0.0198,1.35821466411226e-10)
--(axis cs:0.02,7.84330231220916e-11)
--(axis cs:0.0202,4.52024149086559e-11)
--(axis cs:0.0204,2.59999344847354e-11)
--(axis cs:0.0206,1.49260926007679e-11)
--(axis cs:0.0208,8.55259986988054e-12)
--(axis cs:0.021,4.89150808995468e-12)
--(axis cs:0.0212,2.7925055208629e-12)
--(axis cs:0.0214,1.59134957922416e-12)
--(axis cs:0.0216,9.05254517671245e-13)
--(axis cs:0.0218,5.14069999253187e-13)
--(axis cs:0.022,2.91429031467523e-13)
--(axis cs:0.0222,1.64935660252564e-13)
--(axis cs:0.0224,9.31921612189804e-14)
--(axis cs:0.0226,5.25701031751156e-14)
--(axis cs:0.0228,2.96076644544015e-14)
--(axis cs:0.023,1.66489289581266e-14)
--(axis cs:0.0232,9.34751020436902e-15)
--(axis cs:0.0234,5.24014720070934e-15)
--(axis cs:0.0236,2.93318222365883e-15)
--(axis cs:0.0238,1.63942853674557e-15)
--(axis cs:0.024,9.14983888264259e-16)
--cycle;
\path [draw=none, fill=blue, fill opacity=0.5]
(axis cs:0,0)
--(axis cs:0.0002,3.00097034650853e-39)
--(axis cs:0.0004,1.33180595158039e-29)
--(axis cs:0.0006,4.34794883594433e-24)
--(axis cs:0.0008,2.88281813447308e-20)
--(axis cs:0.001,2.25959278886682e-17)
--(axis cs:0.0012,4.58849753621502e-15)
--(axis cs:0.0014,3.66862376248184e-13)
--(axis cs:0.0016,1.4826057132876e-11)
--(axis cs:0.0018,3.55812394866283e-10)
--(axis cs:0.002,5.6607416830461e-09)
--(axis cs:0.0022,6.45750577460221e-08)
--(axis cs:0.0024,5.59706827273519e-07)
--(axis cs:0.0026,3.85186816583851e-06)
--(axis cs:0.0028,2.17797182193631e-05)
--(axis cs:0.003,0.000103963098101098)
--(axis cs:0.0032,0.000428198943529838)
--(axis cs:0.0034,0.00154923143069868)
--(axis cs:0.0036,0.00499716303584501)
--(axis cs:0.0038,0.0145497409226953)
--(axis cs:0.004,0.0386429091930658)
--(axis cs:0.0042,0.0944621505149839)
--(axis cs:0.0044,0.214173948050834)
--(axis cs:0.0046,0.453416129407434)
--(axis cs:0.0048,0.901527460031158)
--(axis cs:0.005,1.69212633933122)
--(axis cs:0.0052,3.01173851254532)
--(axis cs:0.0054,5.10351436468107)
--(axis cs:0.0056,8.26297420559593)
--(axis cs:0.0058,12.8233899369747)
--(axis cs:0.006,19.1298809719443)
--(axis cs:0.0062,27.5033666828899)
--(axis cs:0.0064,38.197769617239)
--(axis cs:0.0066,51.3557906719054)
--(axis cs:0.0068,66.9696976206799)
--(axis cs:0.007,84.8535623562632)
--(axis cs:0.0072,97.4538923817262)
--(axis cs:0.0074,78.2770485071028)
--(axis cs:0.0076,62.0212911535369)
--(axis cs:0.0078,48.5090753606369)
--(axis cs:0.008,37.4768080842784)
--(axis cs:0.0082,28.6167498138539)
--(axis cs:0.0084,21.6091651725825)
--(axis cs:0.0086,16.1450881904592)
--(axis cs:0.0088,11.9408949354873)
--(axis cs:0.009,8.74626266792109)
--(axis cs:0.0092,6.34716264087156)
--(axis cs:0.0094,4.56539895083454)
--(axis cs:0.0096,3.25596397229389)
--(axis cs:0.0098,2.30320217419976)
--(axis cs:0.01,1.61650454197908)
--(axis cs:0.0102,1.12602130592532)
--(axis cs:0.0104,0.77869215182818)
--(axis cs:0.0106,0.534751570883728)
--(axis cs:0.0108,0.364767719860701)
--(axis cs:0.011,0.247208643245249)
--(axis cs:0.0112,0.166491866350165)
--(axis cs:0.0114,0.111454712568036)
--(axis cs:0.0116,0.0741769407678494)
--(axis cs:0.0118,0.0490895356025638)
--(axis cs:0.012,0.0323101258225112)
--(axis cs:0.0122,0.0211540883083413)
--(axis cs:0.0124,0.0137793399025455)
--(axis cs:0.0126,0.00893120312896225)
--(axis cs:0.0128,0.00576109091603547)
--(axis cs:0.013,0.00369892387946485)
--(axis cs:0.0132,0.00236418677869029)
--(axis cs:0.0134,0.00150446130315117)
--(axis cs:0.0136,0.000953295817434555)
--(axis cs:0.0138,0.000601553104271344)
--(axis cs:0.014,0.000378067905717709)
--(axis cs:0.0142,0.000236680792640458)
--(axis cs:0.0144,0.0001476044276804)
--(axis cs:0.0146,9.17112954174405e-05)
--(axis cs:0.0148,5.67773940803756e-05)
--(axis cs:0.015,3.50265759457039e-05)
--(axis cs:0.0152,2.15341792001447e-05)
--(axis cs:0.0154,1.31948560323714e-05)
--(axis cs:0.0156,8.05865429334375e-06)
--(axis cs:0.0158,4.9061000592821e-06)
--(axis cs:0.016,2.97755266794665e-06)
--(axis cs:0.0162,1.8016217212e-06)
--(axis cs:0.0164,1.08687507048972e-06)
--(axis cs:0.0166,6.53788297169587e-07)
--(axis cs:0.0168,3.92161291128713e-07)
--(axis cs:0.017,2.34579429601707e-07)
--(axis cs:0.0172,1.39939183488036e-07)
--(axis cs:0.0174,8.32604236358792e-08)
--(axis cs:0.0176,4.94097379728479e-08)
--(axis cs:0.0178,2.92472529178579e-08)
--(axis cs:0.018,1.7269483325279e-08)
--(axis cs:0.0182,1.01722660889628e-08)
--(axis cs:0.0184,5.97753076019741e-09)
--(axis cs:0.0186,3.50439135698977e-09)
--(axis cs:0.0188,2.04979446219223e-09)
--(axis cs:0.019,1.19628479054174e-09)
--(axis cs:0.0192,6.96633567508887e-10)
--(axis cs:0.0194,4.04797826285059e-10)
--(axis cs:0.0196,2.34721972140453e-10)
--(axis cs:0.0198,1.35821466411226e-10)
--(axis cs:0.02,7.84330231220916e-11)
--(axis cs:0.0202,4.52024149086559e-11)
--(axis cs:0.0204,2.59999344847354e-11)
--(axis cs:0.0206,1.49260926007679e-11)
--(axis cs:0.0208,8.55259986988054e-12)
--(axis cs:0.021,4.89150808995468e-12)
--(axis cs:0.0212,2.7925055208629e-12)
--(axis cs:0.0214,1.59134957922416e-12)
--(axis cs:0.0216,9.05254517671245e-13)
--(axis cs:0.0218,5.14069999253187e-13)
--(axis cs:0.022,2.91429031467523e-13)
--(axis cs:0.0222,1.64935660252564e-13)
--(axis cs:0.0224,9.31921612189804e-14)
--(axis cs:0.0226,5.25701031751156e-14)
--(axis cs:0.0228,2.96076644544015e-14)
--(axis cs:0.023,1.66489289581266e-14)
--(axis cs:0.0232,9.34751020436902e-15)
--(axis cs:0.0234,5.24014720070934e-15)
--(axis cs:0.0236,2.93318222365883e-15)
--(axis cs:0.0238,1.63942853674557e-15)
--(axis cs:0.024,9.14983888264259e-16)
--cycle;
\path [draw=none, fill=blue, fill opacity=0.5]
(axis cs:0,0)
--(axis cs:0.0002,2.87061962533969e-17)
--(axis cs:0.0004,7.38215500264892e-12)
--(axis cs:0.0006,7.99925233557908e-09)
--(axis cs:0.0008,9.23345890210269e-07)
--(axis cs:0.001,3.12426163997331e-05)
--(axis cs:0.0012,0.000486425585682615)
--(axis cs:0.0014,0.00443227584242255)
--(axis cs:0.0016,0.0272853176447845)
--(axis cs:0.0018,0.124491716808887)
--(axis cs:0.002,0.448457426929982)
--(axis cs:0.0022,1.33431227744341)
--(axis cs:0.0024,3.39008775333182)
--(axis cs:0.0026,7.54334486722043)
--(axis cs:0.0028,14.9917839314935)
--(axis cs:0.003,27.0300203986886)
--(axis cs:0.0032,44.7711936223266)
--(axis cs:0.0034,68.829161876082)
--(axis cs:0.0036,99.0522136164206)
--(axis cs:0.0038,134.390978721141)
--(axis cs:0.004,172.945707968733)
--(axis cs:0.0042,212.187129650739)
--(axis cs:0.0044,249.299540863043)
--(axis cs:0.0046,281.568493665345)
--(axis cs:0.0048,306.733195142688)
--(axis cs:0.005,323.241784801446)
--(axis cs:0.0052,330.377055934632)
--(axis cs:0.0054,327.948504053069)
--(axis cs:0.0056,315.610146296069)
--(axis cs:0.0058,296.46985213264)
--(axis cs:0.006,272.272150752996)
--(axis cs:0.0062,244.826311575023)
--(axis cs:0.0064,215.835724402564)
--(axis cs:0.0066,186.77714737272)
--(axis cs:0.0068,158.83166939296)
--(axis cs:0.007,132.861649361001)
--(axis cs:0.0072,109.423525201146)
--(axis cs:0.0074,88.8048737038958)
--(axis cs:0.0076,71.0747137642668)
--(axis cs:0.0078,56.1379510814392)
--(axis cs:0.008,43.7873232069078)
--(axis cs:0.0082,33.7486675534372)
--(axis cs:0.0084,25.717456860267)
--(axis cs:0.0086,19.3861648116425)
--(axis cs:0.0088,14.4631132938788)
--(axis cs:0.009,10.6840718447148)
--(axis cs:0.0092,7.81813057951323)
--(axis cs:0.0094,5.66936123970293)
--(axis cs:0.0096,4.0756175265045)
--(axis cs:0.0098,2.90558575756373)
--(axis cs:0.01,2.05493739907559)
--(axis cs:0.0102,1.44219244318659)
--(axis cs:0.0104,1.00469625570293)
--(axis cs:0.0106,0.694949514002269)
--(axis cs:0.0108,0.477410379936765)
--(axis cs:0.011,0.325804876526534)
--(axis cs:0.0112,0.220928442520729)
--(axis cs:0.0114,0.148891477378268)
--(axis cs:0.0116,0.0997477329541815)
--(axis cs:0.0118,0.0664411871810735)
--(axis cs:0.012,0.0440103387810881)
--(axis cs:0.0122,0.0289956454569549)
--(axis cs:0.0124,0.0190040235452072)
--(axis cs:0.0126,0.0123926294820924)
--(axis cs:0.0128,0.00804180587557924)
--(axis cs:0.013,0.00519373618909181)
--(axis cs:0.0132,0.00333889993268949)
--(axis cs:0.0134,0.00213689123097192)
--(axis cs:0.0136,0.00136167233949938)
--(axis cs:0.0138,0.000864026867838253)
--(axis cs:0.014,0.000546005807160571)
--(axis cs:0.0142,0.000343661940044616)
--(axis cs:0.0144,0.000215464887035409)
--(axis cs:0.0146,0.000134578989767629)
--(axis cs:0.0148,8.37484307189143e-05)
--(axis cs:0.015,5.19297203253406e-05)
--(axis cs:0.0152,3.20874183650162e-05)
--(axis cs:0.0154,1.9759279261365e-05)
--(axis cs:0.0156,1.21272273451925e-05)
--(axis cs:0.0158,7.41892885450926e-06)
--(axis cs:0.016,4.52422298617316e-06)
--(axis cs:0.0162,2.75044196165776e-06)
--(axis cs:0.0164,1.66704773122399e-06)
--(axis cs:0.0166,1.00742074044527e-06)
--(axis cs:0.0168,6.07044031286042e-07)
--(axis cs:0.017,3.64757361091102e-07)
--(axis cs:0.0172,2.18569535391021e-07)
--(axis cs:0.0174,1.30618012066543e-07)
--(axis cs:0.0176,7.78519818383923e-08)
--(axis cs:0.0178,4.62821862125025e-08)
--(axis cs:0.018,2.74447485205084e-08)
--(axis cs:0.0182,1.6234116858454e-08)
--(axis cs:0.0184,9.57952945182513e-09)
--(axis cs:0.0186,5.63932522263679e-09)
--(axis cs:0.0188,3.31206028275151e-09)
--(axis cs:0.019,1.94078517597756e-09)
--(axis cs:0.0192,1.13470828859746e-09)
--(axis cs:0.0194,6.61968459383362e-10)
--(axis cs:0.0196,3.85349570575785e-10)
--(axis cs:0.0198,2.2384872308735e-10)
--(axis cs:0.02,1.29763824843081e-10)
--(axis cs:0.0202,7.50703658895768e-11)
--(axis cs:0.0204,4.3342645575275e-11)
--(axis cs:0.0206,2.49752548808235e-11)
--(axis cs:0.0208,1.43637307368144e-11)
--(axis cs:0.021,8.24521801416301e-12)
--(axis cs:0.0212,4.72420703343228e-12)
--(axis cs:0.0214,2.70185174122966e-12)
--(axis cs:0.0216,1.54245838297033e-12)
--(axis cs:0.0218,8.79018274445295e-13)
--(axis cs:0.022,5.00066381644288e-13)
--(axis cs:0.0222,2.83997926881723e-13)
--(axis cs:0.0224,1.61017304107681e-13)
--(axis cs:0.0226,9.11405390698865e-14)
--(axis cs:0.0228,5.15043337669998e-14)
--(axis cs:0.023,2.90589735679794e-14)
--(axis cs:0.0232,1.63693700702481e-14)
--(axis cs:0.0234,9.20681612959743e-15)
--(axis cs:0.0236,5.17038771890748e-15)
--(axis cs:0.0238,2.89923195175591e-15)
--(axis cs:0.024,1.6233001918256e-15)
--cycle;
\path [draw=none, fill=blue, fill opacity=0.5]
(axis cs:0,0)
--(axis cs:0.0002,2.50033281924293e-17)
--(axis cs:0.0004,6.61920264526063e-12)
--(axis cs:0.0006,7.29531684613714e-09)
--(axis cs:0.0008,8.52296949347681e-07)
--(axis cs:0.001,2.91093689969302e-05)
--(axis cs:0.0012,0.000456687181093467)
--(axis cs:0.0014,0.00418826597701668)
--(axis cs:0.0016,0.025927869467974)
--(axis cs:0.0018,0.118883729553988)
--(axis cs:0.002,0.43015163399355)
--(axis cs:0.0022,1.28497176113775)
--(axis cs:0.0024,3.27666446986953)
--(axis cs:0.0026,7.31548767742108)
--(axis cs:0.0028,14.5842153965101)
--(axis cs:0.003,26.3714255968948)
--(axis cs:0.0032,43.7988227032249)
--(axis cs:0.0034,67.5058869162191)
--(axis cs:0.0036,97.3813437364043)
--(axis cs:0.0038,132.424374144004)
--(axis cs:0.004,170.782523087049)
--(axis cs:0.0042,209.96311643966)
--(axis cs:0.0044,247.169302110898)
--(axis cs:0.0046,279.684632551789)
--(axis cs:0.0048,305.226638812366)
--(axis cs:0.005,322.206786149517)
--(axis cs:0.0052,329.86291869807)
--(axis cs:0.0054,328.251027276394)
--(axis cs:0.0056,317.691331028443)
--(axis cs:0.0058,300.056657202816)
--(axis cs:0.006,277.022112528999)
--(axis cs:0.0062,250.370518701325)
--(axis cs:0.0064,221.815786056421)
--(axis cs:0.0066,192.872948348342)
--(axis cs:0.0068,164.778860334457)
--(axis cs:0.007,138.459505004863)
--(axis cs:0.0072,114.534883469691)
--(axis cs:0.0074,93.3503654120481)
--(axis cs:0.0076,75.0235227223848)
--(axis cs:0.0078,59.4970615764508)
--(axis cs:0.008,46.5907661257491)
--(axis cs:0.0082,36.0477888103356)
--(axis cs:0.0084,27.5727849718714)
--(axis cs:0.0086,20.8610939827748)
--(axis cs:0.0088,15.6193599142421)
--(axis cs:0.009,11.5787010474745)
--(axis cs:0.0092,8.50186909911387)
--(axis cs:0.0094,6.18589236804231)
--(axis cs:0.0096,4.4615729931021)
--(axis cs:0.0098,3.19099043689396)
--(axis cs:0.01,2.26391267995024)
--(axis cs:0.0102,1.59377392616427)
--(axis cs:0.0104,1.11366586893935)
--(axis cs:0.0106,0.772618649198693)
--(axis cs:0.0108,0.532318675188314)
--(axis cs:0.011,0.36431942334206)
--(axis cs:0.0112,0.247741687742021)
--(axis cs:0.0114,0.167424348101066)
--(axis cs:0.0116,0.112468906597157)
--(axis cs:0.0118,0.0751151495753036)
--(axis cs:0.012,0.0498868750040704)
--(axis cs:0.0122,0.0329524086247424)
--(axis cs:0.0124,0.0216523333629739)
--(axis cs:0.0126,0.0141550013499145)
--(axis cs:0.0128,0.00920810860869809)
--(axis cs:0.013,0.0059614341297812)
--(axis cs:0.0132,0.0038416022772897)
--(axis cs:0.0134,0.00246441669806522)
--(axis cs:0.0136,0.00157402680036694)
--(axis cs:0.0138,0.00100105980744145)
--(axis cs:0.014,0.000634029135601277)
--(axis cs:0.0142,0.000399952922025719)
--(axis cs:0.0144,0.000251307884880475)
--(axis cs:0.0146,0.000157306294866217)
--(axis cs:0.0148,9.81007108197785e-05)
--(axis cs:0.015,6.0957317239249e-05)
--(axis cs:0.0152,3.77439280400285e-05)
--(axis cs:0.0154,2.32902558254876e-05)
--(axis cs:0.0156,1.43233367813761e-05)
--(axis cs:0.0158,8.77995696507616e-06)
--(axis cs:0.016,5.36479250008616e-06)
--(axis cs:0.0162,3.26782412750546e-06)
--(axis cs:0.0164,1.98445502917811e-06)
--(axis cs:0.0166,1.20152046791494e-06)
--(axis cs:0.0168,7.2536700869825e-07)
--(axis cs:0.017,4.36666105161689e-07)
--(axis cs:0.0172,2.62140167381038e-07)
--(axis cs:0.0174,1.56941018917358e-07)
--(axis cs:0.0176,9.37095089143312e-08)
--(axis cs:0.0178,5.58084270995014e-08)
--(axis cs:0.018,3.31519158355106e-08)
--(axis cs:0.0182,1.96441489875536e-08)
--(axis cs:0.0184,1.16116999001352e-08)
--(axis cs:0.0186,6.84727729805362e-09)
--(axis cs:0.0188,4.02828695370443e-09)
--(axis cs:0.019,2.36441351651664e-09)
--(axis cs:0.0192,1.38467088704529e-09)
--(axis cs:0.0194,8.09112143045524e-10)
--(axis cs:0.0196,4.71767761043607e-10)
--(axis cs:0.0198,2.74487754760908e-10)
--(axis cs:0.02,1.59371301958963e-10)
--(axis cs:0.0202,9.23435223039749e-11)
--(axis cs:0.0204,5.3398390260612e-11)
--(axis cs:0.0206,3.08170508635796e-11)
--(axis cs:0.0208,1.77504953099345e-11)
--(axis cs:0.021,1.02047233929746e-11)
--(axis cs:0.0212,5.85568632532008e-12)
--(axis cs:0.0214,3.35393171322925e-12)
--(axis cs:0.0216,1.91753860960166e-12)
--(axis cs:0.0218,1.09436142350987e-12)
--(axis cs:0.022,6.23472201634812e-13)
--(axis cs:0.0222,3.54589300004174e-13)
--(axis cs:0.0224,2.01325406384701e-13)
--(axis cs:0.0226,1.14116345766126e-13)
--(axis cs:0.0228,6.45780629660316e-14)
--(axis cs:0.023,3.64855860479582e-14)
--(axis cs:0.0232,2.058106140172e-14)
--(axis cs:0.0234,1.1591377006404e-14)
--(axis cs:0.0236,6.51828782611222e-15)
--(axis cs:0.0238,3.65993547211678e-15)
--(axis cs:0.024,2.05193979417027e-15)
--cycle;
\path [draw=none, fill=blue, fill opacity=0.5]
(axis cs:0,0)
--(axis cs:0.0002,4.38630231113233e-32)
--(axis cs:0.0004,9.42164546775413e-24)
--(axis cs:0.0006,5.22996289695702e-19)
--(axis cs:0.0008,9.86216555458081e-16)
--(axis cs:0.001,2.9143627354464e-13)
--(axis cs:0.0012,2.66669889722348e-11)
--(axis cs:0.0014,1.08650456690335e-09)
--(axis cs:0.0016,2.44844210199571e-08)
--(axis cs:0.0018,3.50988320156314e-07)
--(axis cs:0.002,3.52140105748224e-06)
--(axis cs:0.0022,2.64692151139118e-05)
--(axis cs:0.0024,0.000156751672867648)
--(axis cs:0.0026,0.000759830128964595)
--(axis cs:0.0028,0.00310561691820713)
--(axis cs:0.003,0.0109579090659142)
--(axis cs:0.0032,0.034016993111393)
--(axis cs:0.0034,0.0943609589409247)
--(axis cs:0.0036,0.236919036551997)
--(axis cs:0.0038,0.544246235965627)
--(axis cs:0.004,1.15434405660579)
--(axis cs:0.0042,2.27822422319553)
--(axis cs:0.0044,4.21196189035521)
--(axis cs:0.0046,7.33697307600828)
--(axis cs:0.0048,12.1029173241866)
--(axis cs:0.005,18.9902597620454)
--(axis cs:0.0052,28.4537592035236)
--(axis cs:0.0054,40.8529986597383)
--(axis cs:0.0056,56.3802313963309)
--(axis cs:0.0058,74.9980614577203)
--(axis cs:0.006,96.399078434761)
--(axis cs:0.0062,119.996496092707)
--(axis cs:0.0064,144.949774971286)
--(axis cs:0.0066,170.223290391324)
--(axis cs:0.0068,164.778860334457)
--(axis cs:0.007,138.459505004863)
--(axis cs:0.0072,114.534883469691)
--(axis cs:0.0074,93.3503654120481)
--(axis cs:0.0076,75.0235227223848)
--(axis cs:0.0078,59.4970615764508)
--(axis cs:0.008,46.5907661257491)
--(axis cs:0.0082,36.0477888103356)
--(axis cs:0.0084,27.5727849718714)
--(axis cs:0.0086,20.8610939827748)
--(axis cs:0.0088,15.6193599142421)
--(axis cs:0.009,11.5787010474745)
--(axis cs:0.0092,8.50186909911387)
--(axis cs:0.0094,6.18589236804231)
--(axis cs:0.0096,4.4615729931021)
--(axis cs:0.0098,3.19099043689396)
--(axis cs:0.01,2.26391267995024)
--(axis cs:0.0102,1.59377392616427)
--(axis cs:0.0104,1.11366586893935)
--(axis cs:0.0106,0.772618649198693)
--(axis cs:0.0108,0.532318675188314)
--(axis cs:0.011,0.36431942334206)
--(axis cs:0.0112,0.247741687742021)
--(axis cs:0.0114,0.167424348101066)
--(axis cs:0.0116,0.112468906597157)
--(axis cs:0.0118,0.0751151495753036)
--(axis cs:0.012,0.0498868750040704)
--(axis cs:0.0122,0.0329524086247424)
--(axis cs:0.0124,0.0216523333629739)
--(axis cs:0.0126,0.0141550013499145)
--(axis cs:0.0128,0.00920810860869809)
--(axis cs:0.013,0.0059614341297812)
--(axis cs:0.0132,0.0038416022772897)
--(axis cs:0.0134,0.00246441669806522)
--(axis cs:0.0136,0.00157402680036694)
--(axis cs:0.0138,0.00100105980744145)
--(axis cs:0.014,0.000634029135601277)
--(axis cs:0.0142,0.000399952922025719)
--(axis cs:0.0144,0.000251307884880475)
--(axis cs:0.0146,0.000157306294866217)
--(axis cs:0.0148,9.81007108197785e-05)
--(axis cs:0.015,6.0957317239249e-05)
--(axis cs:0.0152,3.77439280400285e-05)
--(axis cs:0.0154,2.32902558254876e-05)
--(axis cs:0.0156,1.43233367813761e-05)
--(axis cs:0.0158,8.77995696507616e-06)
--(axis cs:0.016,5.36479250008616e-06)
--(axis cs:0.0162,3.26782412750546e-06)
--(axis cs:0.0164,1.98445502917811e-06)
--(axis cs:0.0166,1.20152046791494e-06)
--(axis cs:0.0168,7.2536700869825e-07)
--(axis cs:0.017,4.36666105161689e-07)
--(axis cs:0.0172,2.62140167381038e-07)
--(axis cs:0.0174,1.56941018917358e-07)
--(axis cs:0.0176,9.37095089143312e-08)
--(axis cs:0.0178,5.58084270995014e-08)
--(axis cs:0.018,3.31519158355106e-08)
--(axis cs:0.0182,1.96441489875536e-08)
--(axis cs:0.0184,1.16116999001352e-08)
--(axis cs:0.0186,6.84727729805362e-09)
--(axis cs:0.0188,4.02828695370443e-09)
--(axis cs:0.019,2.36441351651664e-09)
--(axis cs:0.0192,1.38467088704529e-09)
--(axis cs:0.0194,8.09112143045524e-10)
--(axis cs:0.0196,4.71767761043607e-10)
--(axis cs:0.0198,2.74487754760908e-10)
--(axis cs:0.02,1.59371301958963e-10)
--(axis cs:0.0202,9.23435223039749e-11)
--(axis cs:0.0204,5.3398390260612e-11)
--(axis cs:0.0206,3.08170508635796e-11)
--(axis cs:0.0208,1.77504953099345e-11)
--(axis cs:0.021,1.02047233929746e-11)
--(axis cs:0.0212,5.85568632532008e-12)
--(axis cs:0.0214,3.35393171322925e-12)
--(axis cs:0.0216,1.91753860960166e-12)
--(axis cs:0.0218,1.09436142350987e-12)
--(axis cs:0.022,6.23472201634812e-13)
--(axis cs:0.0222,3.54589300004174e-13)
--(axis cs:0.0224,2.01325406384701e-13)
--(axis cs:0.0226,1.14116345766126e-13)
--(axis cs:0.0228,6.45780629660316e-14)
--(axis cs:0.023,3.64855860479582e-14)
--(axis cs:0.0232,2.058106140172e-14)
--(axis cs:0.0234,1.1591377006404e-14)
--(axis cs:0.0236,6.51828782611222e-15)
--(axis cs:0.0238,3.65993547211678e-15)
--(axis cs:0.024,2.05193979417027e-15)
--cycle;
\path [draw=none, fill=blue, fill opacity=0.5]
(axis cs:0,0)
--(axis cs:0.0002,4.28964223063707e-25)
--(axis cs:0.0004,4.23925669893049e-18)
--(axis cs:0.0006,3.8843035174309e-14)
--(axis cs:0.0008,2.03980960361959e-11)
--(axis cs:0.001,2.23578413477403e-09)
--(axis cs:0.0012,9.09612932925116e-08)
--(axis cs:0.0014,1.86740983355346e-06)
--(axis cs:0.0016,2.32374652780952e-05)
--(axis cs:0.0018,0.00019726676995222)
--(axis cs:0.002,0.00123850159771812)
--(axis cs:0.0022,0.00609150384571705)
--(axis cs:0.0024,0.0244907822733581)
--(axis cs:0.0026,0.0831291604329364)
--(axis cs:0.0028,0.244273624310176)
--(axis cs:0.003,0.633888506344497)
--(axis cs:0.0032,1.47614150045633)
--(axis cs:0.0034,3.12549923236712)
--(axis cs:0.0036,6.08284419664121)
--(axis cs:0.0038,10.9810360144135)
--(axis cs:0.004,18.5299350809066)
--(axis cs:0.0042,29.4207126030825)
--(axis cs:0.0044,44.2015281358291)
--(axis cs:0.0046,63.1472686753469)
--(axis cs:0.0048,86.1512421387252)
--(axis cs:0.005,112.664536610035)
--(axis cs:0.0052,141.699695822165)
--(axis cs:0.0054,171.902056209116)
--(axis cs:0.0056,201.678253673426)
--(axis cs:0.0058,229.360553795801)
--(axis cs:0.006,253.380099640291)
--(axis cs:0.0062,250.370518701325)
--(axis cs:0.0064,221.815786056421)
--(axis cs:0.0066,192.872948348342)
--(axis cs:0.0068,164.778860334457)
--(axis cs:0.007,138.459505004863)
--(axis cs:0.0072,114.534883469691)
--(axis cs:0.0074,93.3503654120481)
--(axis cs:0.0076,75.0235227223848)
--(axis cs:0.0078,59.4970615764508)
--(axis cs:0.008,46.5907661257491)
--(axis cs:0.0082,36.0477888103356)
--(axis cs:0.0084,27.5727849718714)
--(axis cs:0.0086,20.8610939827748)
--(axis cs:0.0088,15.6193599142421)
--(axis cs:0.009,11.5787010474745)
--(axis cs:0.0092,8.50186909911387)
--(axis cs:0.0094,6.18589236804231)
--(axis cs:0.0096,4.4615729931021)
--(axis cs:0.0098,3.19099043689396)
--(axis cs:0.01,2.26391267995024)
--(axis cs:0.0102,1.59377392616427)
--(axis cs:0.0104,1.11366586893935)
--(axis cs:0.0106,0.772618649198693)
--(axis cs:0.0108,0.532318675188314)
--(axis cs:0.011,0.36431942334206)
--(axis cs:0.0112,0.247741687742021)
--(axis cs:0.0114,0.167424348101066)
--(axis cs:0.0116,0.112468906597157)
--(axis cs:0.0118,0.0751151495753036)
--(axis cs:0.012,0.0498868750040704)
--(axis cs:0.0122,0.0329524086247424)
--(axis cs:0.0124,0.0216523333629739)
--(axis cs:0.0126,0.0141550013499145)
--(axis cs:0.0128,0.00920810860869809)
--(axis cs:0.013,0.0059614341297812)
--(axis cs:0.0132,0.0038416022772897)
--(axis cs:0.0134,0.00246441669806522)
--(axis cs:0.0136,0.00157402680036694)
--(axis cs:0.0138,0.00100105980744145)
--(axis cs:0.014,0.000634029135601277)
--(axis cs:0.0142,0.000399952922025719)
--(axis cs:0.0144,0.000251307884880475)
--(axis cs:0.0146,0.000157306294866217)
--(axis cs:0.0148,9.81007108197785e-05)
--(axis cs:0.015,6.0957317239249e-05)
--(axis cs:0.0152,3.77439280400285e-05)
--(axis cs:0.0154,2.32902558254876e-05)
--(axis cs:0.0156,1.43233367813761e-05)
--(axis cs:0.0158,8.77995696507616e-06)
--(axis cs:0.016,5.36479250008616e-06)
--(axis cs:0.0162,3.26782412750546e-06)
--(axis cs:0.0164,1.98445502917811e-06)
--(axis cs:0.0166,1.20152046791494e-06)
--(axis cs:0.0168,7.2536700869825e-07)
--(axis cs:0.017,4.36666105161689e-07)
--(axis cs:0.0172,2.62140167381038e-07)
--(axis cs:0.0174,1.56941018917358e-07)
--(axis cs:0.0176,9.37095089143312e-08)
--(axis cs:0.0178,5.58084270995014e-08)
--(axis cs:0.018,3.31519158355106e-08)
--(axis cs:0.0182,1.96441489875536e-08)
--(axis cs:0.0184,1.16116999001352e-08)
--(axis cs:0.0186,6.84727729805362e-09)
--(axis cs:0.0188,4.02828695370443e-09)
--(axis cs:0.019,2.36441351651664e-09)
--(axis cs:0.0192,1.38467088704529e-09)
--(axis cs:0.0194,8.09112143045524e-10)
--(axis cs:0.0196,4.71767761043607e-10)
--(axis cs:0.0198,2.74487754760908e-10)
--(axis cs:0.02,1.59371301958963e-10)
--(axis cs:0.0202,9.23435223039749e-11)
--(axis cs:0.0204,5.3398390260612e-11)
--(axis cs:0.0206,3.08170508635796e-11)
--(axis cs:0.0208,1.77504953099345e-11)
--(axis cs:0.021,1.02047233929746e-11)
--(axis cs:0.0212,5.85568632532008e-12)
--(axis cs:0.0214,3.35393171322925e-12)
--(axis cs:0.0216,1.91753860960166e-12)
--(axis cs:0.0218,1.09436142350987e-12)
--(axis cs:0.022,6.23472201634812e-13)
--(axis cs:0.0222,3.54589300004174e-13)
--(axis cs:0.0224,2.01325406384701e-13)
--(axis cs:0.0226,1.14116345766126e-13)
--(axis cs:0.0228,6.45780629660316e-14)
--(axis cs:0.023,3.64855860479582e-14)
--(axis cs:0.0232,2.058106140172e-14)
--(axis cs:0.0234,1.1591377006404e-14)
--(axis cs:0.0236,6.51828782611222e-15)
--(axis cs:0.0238,3.65993547211678e-15)
--(axis cs:0.024,2.05193979417027e-15)
--cycle;
\path [draw=none, fill=blue, fill opacity=0.5]
(axis cs:0,0)
--(axis cs:0.0002,3.00097034650853e-39)
--(axis cs:0.0004,1.33180595158039e-29)
--(axis cs:0.0006,4.34794883594433e-24)
--(axis cs:0.0008,2.88281813447308e-20)
--(axis cs:0.001,2.25959278886682e-17)
--(axis cs:0.0012,4.58849753621502e-15)
--(axis cs:0.0014,3.66862376248184e-13)
--(axis cs:0.0016,1.4826057132876e-11)
--(axis cs:0.0018,3.55812394866283e-10)
--(axis cs:0.002,5.6607416830461e-09)
--(axis cs:0.0022,6.45750577460221e-08)
--(axis cs:0.0024,5.59706827273519e-07)
--(axis cs:0.0026,3.85186816583851e-06)
--(axis cs:0.0028,2.17797182193631e-05)
--(axis cs:0.003,0.000103963098101098)
--(axis cs:0.0032,0.000428198943529838)
--(axis cs:0.0034,0.00154923143069868)
--(axis cs:0.0036,0.00499716303584501)
--(axis cs:0.0038,0.0145497409226953)
--(axis cs:0.004,0.0386429091930658)
--(axis cs:0.0042,0.0944621505149839)
--(axis cs:0.0044,0.214173948050834)
--(axis cs:0.0046,0.453416129407434)
--(axis cs:0.0048,0.901527460031158)
--(axis cs:0.005,1.69212633933122)
--(axis cs:0.0052,3.01173851254532)
--(axis cs:0.0054,5.10351436468107)
--(axis cs:0.0056,8.26297420559593)
--(axis cs:0.0058,12.8233899369747)
--(axis cs:0.006,19.1298809719443)
--(axis cs:0.0062,27.5033666828899)
--(axis cs:0.0064,38.197769617239)
--(axis cs:0.0066,51.3557906719054)
--(axis cs:0.0068,66.9696976206799)
--(axis cs:0.007,84.8535623562632)
--(axis cs:0.0072,104.632179248452)
--(axis cs:0.0074,93.3503654120481)
--(axis cs:0.0076,75.0235227223848)
--(axis cs:0.0078,59.4970615764508)
--(axis cs:0.008,46.5907661257491)
--(axis cs:0.0082,36.0477888103356)
--(axis cs:0.0084,27.5727849718714)
--(axis cs:0.0086,20.8610939827748)
--(axis cs:0.0088,15.6193599142421)
--(axis cs:0.009,11.5787010474745)
--(axis cs:0.0092,8.50186909911387)
--(axis cs:0.0094,6.18589236804231)
--(axis cs:0.0096,4.4615729931021)
--(axis cs:0.0098,3.19099043689396)
--(axis cs:0.01,2.26391267995024)
--(axis cs:0.0102,1.59377392616427)
--(axis cs:0.0104,1.11366586893935)
--(axis cs:0.0106,0.772618649198693)
--(axis cs:0.0108,0.532318675188314)
--(axis cs:0.011,0.36431942334206)
--(axis cs:0.0112,0.247741687742021)
--(axis cs:0.0114,0.167424348101066)
--(axis cs:0.0116,0.112468906597157)
--(axis cs:0.0118,0.0751151495753036)
--(axis cs:0.012,0.0498868750040704)
--(axis cs:0.0122,0.0329524086247424)
--(axis cs:0.0124,0.0216523333629739)
--(axis cs:0.0126,0.0141550013499145)
--(axis cs:0.0128,0.00920810860869809)
--(axis cs:0.013,0.0059614341297812)
--(axis cs:0.0132,0.0038416022772897)
--(axis cs:0.0134,0.00246441669806522)
--(axis cs:0.0136,0.00157402680036694)
--(axis cs:0.0138,0.00100105980744145)
--(axis cs:0.014,0.000634029135601277)
--(axis cs:0.0142,0.000399952922025719)
--(axis cs:0.0144,0.000251307884880475)
--(axis cs:0.0146,0.000157306294866217)
--(axis cs:0.0148,9.81007108197785e-05)
--(axis cs:0.015,6.0957317239249e-05)
--(axis cs:0.0152,3.77439280400285e-05)
--(axis cs:0.0154,2.32902558254876e-05)
--(axis cs:0.0156,1.43233367813761e-05)
--(axis cs:0.0158,8.77995696507616e-06)
--(axis cs:0.016,5.36479250008616e-06)
--(axis cs:0.0162,3.26782412750546e-06)
--(axis cs:0.0164,1.98445502917811e-06)
--(axis cs:0.0166,1.20152046791494e-06)
--(axis cs:0.0168,7.2536700869825e-07)
--(axis cs:0.017,4.36666105161689e-07)
--(axis cs:0.0172,2.62140167381038e-07)
--(axis cs:0.0174,1.56941018917358e-07)
--(axis cs:0.0176,9.37095089143312e-08)
--(axis cs:0.0178,5.58084270995014e-08)
--(axis cs:0.018,3.31519158355106e-08)
--(axis cs:0.0182,1.96441489875536e-08)
--(axis cs:0.0184,1.16116999001352e-08)
--(axis cs:0.0186,6.84727729805362e-09)
--(axis cs:0.0188,4.02828695370443e-09)
--(axis cs:0.019,2.36441351651664e-09)
--(axis cs:0.0192,1.38467088704529e-09)
--(axis cs:0.0194,8.09112143045524e-10)
--(axis cs:0.0196,4.71767761043607e-10)
--(axis cs:0.0198,2.74487754760908e-10)
--(axis cs:0.02,1.59371301958963e-10)
--(axis cs:0.0202,9.23435223039749e-11)
--(axis cs:0.0204,5.3398390260612e-11)
--(axis cs:0.0206,3.08170508635796e-11)
--(axis cs:0.0208,1.77504953099345e-11)
--(axis cs:0.021,1.02047233929746e-11)
--(axis cs:0.0212,5.85568632532008e-12)
--(axis cs:0.0214,3.35393171322925e-12)
--(axis cs:0.0216,1.91753860960166e-12)
--(axis cs:0.0218,1.09436142350987e-12)
--(axis cs:0.022,6.23472201634812e-13)
--(axis cs:0.0222,3.54589300004174e-13)
--(axis cs:0.0224,2.01325406384701e-13)
--(axis cs:0.0226,1.14116345766126e-13)
--(axis cs:0.0228,6.45780629660316e-14)
--(axis cs:0.023,3.64855860479582e-14)
--(axis cs:0.0232,2.058106140172e-14)
--(axis cs:0.0234,1.1591377006404e-14)
--(axis cs:0.0236,6.51828782611222e-15)
--(axis cs:0.0238,3.65993547211678e-15)
--(axis cs:0.024,2.05193979417027e-15)
--cycle;
\path [draw=none, fill=blue, fill opacity=0.5]
(axis cs:0,0)
--(axis cs:0.0002,2.50033281924293e-17)
--(axis cs:0.0004,6.61920264526063e-12)
--(axis cs:0.0006,7.29531684613714e-09)
--(axis cs:0.0008,8.52296949347681e-07)
--(axis cs:0.001,2.91093689969302e-05)
--(axis cs:0.0012,0.000456687181093467)
--(axis cs:0.0014,0.00418826597701668)
--(axis cs:0.0016,0.025927869467974)
--(axis cs:0.0018,0.118883729553988)
--(axis cs:0.002,0.43015163399355)
--(axis cs:0.0022,1.28497176113775)
--(axis cs:0.0024,3.27666446986953)
--(axis cs:0.0026,7.31548767742108)
--(axis cs:0.0028,14.5842153965101)
--(axis cs:0.003,26.3714255968948)
--(axis cs:0.0032,43.7988227032249)
--(axis cs:0.0034,67.5058869162191)
--(axis cs:0.0036,97.3813437364043)
--(axis cs:0.0038,132.424374144004)
--(axis cs:0.004,170.782523087049)
--(axis cs:0.0042,209.96311643966)
--(axis cs:0.0044,247.169302110898)
--(axis cs:0.0046,279.684632551789)
--(axis cs:0.0048,305.226638812366)
--(axis cs:0.005,322.206786149517)
--(axis cs:0.0052,329.86291869807)
--(axis cs:0.0054,327.948504053069)
--(axis cs:0.0056,315.610146296069)
--(axis cs:0.0058,296.46985213264)
--(axis cs:0.006,272.272150752996)
--(axis cs:0.0062,244.826311575023)
--(axis cs:0.0064,215.835724402564)
--(axis cs:0.0066,186.77714737272)
--(axis cs:0.0068,158.83166939296)
--(axis cs:0.007,132.861649361001)
--(axis cs:0.0072,109.423525201146)
--(axis cs:0.0074,88.8048737038958)
--(axis cs:0.0076,71.0747137642668)
--(axis cs:0.0078,56.1379510814392)
--(axis cs:0.008,43.7873232069078)
--(axis cs:0.0082,33.7486675534372)
--(axis cs:0.0084,25.717456860267)
--(axis cs:0.0086,19.3861648116425)
--(axis cs:0.0088,14.4631132938788)
--(axis cs:0.009,10.6840718447148)
--(axis cs:0.0092,7.81813057951323)
--(axis cs:0.0094,5.66936123970293)
--(axis cs:0.0096,4.0756175265045)
--(axis cs:0.0098,2.90558575756373)
--(axis cs:0.01,2.05493739907559)
--(axis cs:0.0102,1.44219244318659)
--(axis cs:0.0104,1.00469625570293)
--(axis cs:0.0106,0.694949514002269)
--(axis cs:0.0108,0.477410379936765)
--(axis cs:0.011,0.325804876526534)
--(axis cs:0.0112,0.220928442520729)
--(axis cs:0.0114,0.148891477378268)
--(axis cs:0.0116,0.0997477329541815)
--(axis cs:0.0118,0.0664411871810735)
--(axis cs:0.012,0.0440103387810881)
--(axis cs:0.0122,0.0289956454569549)
--(axis cs:0.0124,0.0190040235452072)
--(axis cs:0.0126,0.0123926294820924)
--(axis cs:0.0128,0.00804180587557924)
--(axis cs:0.013,0.00519373618909181)
--(axis cs:0.0132,0.00333889993268949)
--(axis cs:0.0134,0.00213689123097192)
--(axis cs:0.0136,0.00136167233949938)
--(axis cs:0.0138,0.000864026867838253)
--(axis cs:0.014,0.000546005807160571)
--(axis cs:0.0142,0.000343661940044616)
--(axis cs:0.0144,0.000215464887035409)
--(axis cs:0.0146,0.000134578989767629)
--(axis cs:0.0148,8.37484307189143e-05)
--(axis cs:0.015,5.19297203253406e-05)
--(axis cs:0.0152,3.20874183650162e-05)
--(axis cs:0.0154,1.9759279261365e-05)
--(axis cs:0.0156,1.21272273451925e-05)
--(axis cs:0.0158,7.41892885450926e-06)
--(axis cs:0.016,4.52422298617316e-06)
--(axis cs:0.0162,2.75044196165776e-06)
--(axis cs:0.0164,1.66704773122399e-06)
--(axis cs:0.0166,1.00742074044527e-06)
--(axis cs:0.0168,6.07044031286042e-07)
--(axis cs:0.017,3.64757361091102e-07)
--(axis cs:0.0172,2.18569535391021e-07)
--(axis cs:0.0174,1.30618012066543e-07)
--(axis cs:0.0176,7.78519818383923e-08)
--(axis cs:0.0178,4.62821862125025e-08)
--(axis cs:0.018,2.74447485205084e-08)
--(axis cs:0.0182,1.6234116858454e-08)
--(axis cs:0.0184,9.57952945182513e-09)
--(axis cs:0.0186,5.63932522263679e-09)
--(axis cs:0.0188,3.31206028275151e-09)
--(axis cs:0.019,1.94078517597756e-09)
--(axis cs:0.0192,1.13470828859746e-09)
--(axis cs:0.0194,6.61968459383362e-10)
--(axis cs:0.0196,3.85349570575785e-10)
--(axis cs:0.0198,2.2384872308735e-10)
--(axis cs:0.02,1.29763824843081e-10)
--(axis cs:0.0202,7.50703658895768e-11)
--(axis cs:0.0204,4.3342645575275e-11)
--(axis cs:0.0206,2.49752548808235e-11)
--(axis cs:0.0208,1.43637307368144e-11)
--(axis cs:0.021,8.24521801416301e-12)
--(axis cs:0.0212,4.72420703343228e-12)
--(axis cs:0.0214,2.70185174122966e-12)
--(axis cs:0.0216,1.54245838297033e-12)
--(axis cs:0.0218,8.79018274445295e-13)
--(axis cs:0.022,5.00066381644288e-13)
--(axis cs:0.0222,2.83997926881723e-13)
--(axis cs:0.0224,1.61017304107681e-13)
--(axis cs:0.0226,9.11405390698865e-14)
--(axis cs:0.0228,5.15043337669998e-14)
--(axis cs:0.023,2.90589735679794e-14)
--(axis cs:0.0232,1.63693700702481e-14)
--(axis cs:0.0234,9.20681612959743e-15)
--(axis cs:0.0236,5.17038771890748e-15)
--(axis cs:0.0238,2.89923195175591e-15)
--(axis cs:0.024,1.6233001918256e-15)
--cycle;
\path [draw=none, fill=blue, fill opacity=0.5]
(axis cs:0,0)
--(axis cs:0.0002,4.38630231113233e-32)
--(axis cs:0.0004,9.42164546775413e-24)
--(axis cs:0.0006,5.22996289695702e-19)
--(axis cs:0.0008,9.86216555458081e-16)
--(axis cs:0.001,2.9143627354464e-13)
--(axis cs:0.0012,2.66669889722348e-11)
--(axis cs:0.0014,1.08650456690335e-09)
--(axis cs:0.0016,2.44844210199571e-08)
--(axis cs:0.0018,3.50988320156314e-07)
--(axis cs:0.002,3.52140105748224e-06)
--(axis cs:0.0022,2.64692151139118e-05)
--(axis cs:0.0024,0.000156751672867648)
--(axis cs:0.0026,0.000759830128964595)
--(axis cs:0.0028,0.00310561691820713)
--(axis cs:0.003,0.0109579090659142)
--(axis cs:0.0032,0.034016993111393)
--(axis cs:0.0034,0.0943609589409247)
--(axis cs:0.0036,0.236919036551997)
--(axis cs:0.0038,0.544246235965627)
--(axis cs:0.004,1.15434405660579)
--(axis cs:0.0042,2.27822422319553)
--(axis cs:0.0044,4.21196189035521)
--(axis cs:0.0046,7.33697307600828)
--(axis cs:0.0048,12.1029173241866)
--(axis cs:0.005,18.9902597620454)
--(axis cs:0.0052,28.4537592035236)
--(axis cs:0.0054,40.8529986597383)
--(axis cs:0.0056,56.3802313963309)
--(axis cs:0.0058,74.9980614577203)
--(axis cs:0.006,96.399078434761)
--(axis cs:0.0062,119.996496092707)
--(axis cs:0.0064,144.949774971286)
--(axis cs:0.0066,170.223290391324)
--(axis cs:0.0068,158.83166939296)
--(axis cs:0.007,132.861649361001)
--(axis cs:0.0072,109.423525201146)
--(axis cs:0.0074,88.8048737038958)
--(axis cs:0.0076,71.0747137642668)
--(axis cs:0.0078,56.1379510814392)
--(axis cs:0.008,43.7873232069078)
--(axis cs:0.0082,33.7486675534372)
--(axis cs:0.0084,25.717456860267)
--(axis cs:0.0086,19.3861648116425)
--(axis cs:0.0088,14.4631132938788)
--(axis cs:0.009,10.6840718447148)
--(axis cs:0.0092,7.81813057951323)
--(axis cs:0.0094,5.66936123970293)
--(axis cs:0.0096,4.0756175265045)
--(axis cs:0.0098,2.90558575756373)
--(axis cs:0.01,2.05493739907559)
--(axis cs:0.0102,1.44219244318659)
--(axis cs:0.0104,1.00469625570293)
--(axis cs:0.0106,0.694949514002269)
--(axis cs:0.0108,0.477410379936765)
--(axis cs:0.011,0.325804876526534)
--(axis cs:0.0112,0.220928442520729)
--(axis cs:0.0114,0.148891477378268)
--(axis cs:0.0116,0.0997477329541815)
--(axis cs:0.0118,0.0664411871810735)
--(axis cs:0.012,0.0440103387810881)
--(axis cs:0.0122,0.0289956454569549)
--(axis cs:0.0124,0.0190040235452072)
--(axis cs:0.0126,0.0123926294820924)
--(axis cs:0.0128,0.00804180587557924)
--(axis cs:0.013,0.00519373618909181)
--(axis cs:0.0132,0.00333889993268949)
--(axis cs:0.0134,0.00213689123097192)
--(axis cs:0.0136,0.00136167233949938)
--(axis cs:0.0138,0.000864026867838253)
--(axis cs:0.014,0.000546005807160571)
--(axis cs:0.0142,0.000343661940044616)
--(axis cs:0.0144,0.000215464887035409)
--(axis cs:0.0146,0.000134578989767629)
--(axis cs:0.0148,8.37484307189143e-05)
--(axis cs:0.015,5.19297203253406e-05)
--(axis cs:0.0152,3.20874183650162e-05)
--(axis cs:0.0154,1.9759279261365e-05)
--(axis cs:0.0156,1.21272273451925e-05)
--(axis cs:0.0158,7.41892885450926e-06)
--(axis cs:0.016,4.52422298617316e-06)
--(axis cs:0.0162,2.75044196165776e-06)
--(axis cs:0.0164,1.66704773122399e-06)
--(axis cs:0.0166,1.00742074044527e-06)
--(axis cs:0.0168,6.07044031286042e-07)
--(axis cs:0.017,3.64757361091102e-07)
--(axis cs:0.0172,2.18569535391021e-07)
--(axis cs:0.0174,1.30618012066543e-07)
--(axis cs:0.0176,7.78519818383923e-08)
--(axis cs:0.0178,4.62821862125025e-08)
--(axis cs:0.018,2.74447485205084e-08)
--(axis cs:0.0182,1.6234116858454e-08)
--(axis cs:0.0184,9.57952945182513e-09)
--(axis cs:0.0186,5.63932522263679e-09)
--(axis cs:0.0188,3.31206028275151e-09)
--(axis cs:0.019,1.94078517597756e-09)
--(axis cs:0.0192,1.13470828859746e-09)
--(axis cs:0.0194,6.61968459383362e-10)
--(axis cs:0.0196,3.85349570575785e-10)
--(axis cs:0.0198,2.2384872308735e-10)
--(axis cs:0.02,1.29763824843081e-10)
--(axis cs:0.0202,7.50703658895768e-11)
--(axis cs:0.0204,4.3342645575275e-11)
--(axis cs:0.0206,2.49752548808235e-11)
--(axis cs:0.0208,1.43637307368144e-11)
--(axis cs:0.021,8.24521801416301e-12)
--(axis cs:0.0212,4.72420703343228e-12)
--(axis cs:0.0214,2.70185174122966e-12)
--(axis cs:0.0216,1.54245838297033e-12)
--(axis cs:0.0218,8.79018274445295e-13)
--(axis cs:0.022,5.00066381644288e-13)
--(axis cs:0.0222,2.83997926881723e-13)
--(axis cs:0.0224,1.61017304107681e-13)
--(axis cs:0.0226,9.11405390698865e-14)
--(axis cs:0.0228,5.15043337669998e-14)
--(axis cs:0.023,2.90589735679794e-14)
--(axis cs:0.0232,1.63693700702481e-14)
--(axis cs:0.0234,9.20681612959743e-15)
--(axis cs:0.0236,5.17038771890748e-15)
--(axis cs:0.0238,2.89923195175591e-15)
--(axis cs:0.024,1.6233001918256e-15)
--cycle;
\path [draw=none, fill=blue, fill opacity=0.5]
(axis cs:0,0)
--(axis cs:0.0002,4.28964223063707e-25)
--(axis cs:0.0004,4.23925669893049e-18)
--(axis cs:0.0006,3.8843035174309e-14)
--(axis cs:0.0008,2.03980960361959e-11)
--(axis cs:0.001,2.23578413477403e-09)
--(axis cs:0.0012,9.09612932925116e-08)
--(axis cs:0.0014,1.86740983355346e-06)
--(axis cs:0.0016,2.32374652780952e-05)
--(axis cs:0.0018,0.00019726676995222)
--(axis cs:0.002,0.00123850159771812)
--(axis cs:0.0022,0.00609150384571705)
--(axis cs:0.0024,0.0244907822733581)
--(axis cs:0.0026,0.0831291604329364)
--(axis cs:0.0028,0.244273624310176)
--(axis cs:0.003,0.633888506344497)
--(axis cs:0.0032,1.47614150045633)
--(axis cs:0.0034,3.12549923236712)
--(axis cs:0.0036,6.08284419664121)
--(axis cs:0.0038,10.9810360144135)
--(axis cs:0.004,18.5299350809066)
--(axis cs:0.0042,29.4207126030825)
--(axis cs:0.0044,44.2015281358291)
--(axis cs:0.0046,63.1472686753469)
--(axis cs:0.0048,86.1512421387252)
--(axis cs:0.005,112.664536610035)
--(axis cs:0.0052,141.699695822165)
--(axis cs:0.0054,171.902056209116)
--(axis cs:0.0056,201.678253673426)
--(axis cs:0.0058,229.360553795801)
--(axis cs:0.006,253.380099640291)
--(axis cs:0.0062,244.826311575023)
--(axis cs:0.0064,215.835724402564)
--(axis cs:0.0066,186.77714737272)
--(axis cs:0.0068,158.83166939296)
--(axis cs:0.007,132.861649361001)
--(axis cs:0.0072,109.423525201146)
--(axis cs:0.0074,88.8048737038958)
--(axis cs:0.0076,71.0747137642668)
--(axis cs:0.0078,56.1379510814392)
--(axis cs:0.008,43.7873232069078)
--(axis cs:0.0082,33.7486675534372)
--(axis cs:0.0084,25.717456860267)
--(axis cs:0.0086,19.3861648116425)
--(axis cs:0.0088,14.4631132938788)
--(axis cs:0.009,10.6840718447148)
--(axis cs:0.0092,7.81813057951323)
--(axis cs:0.0094,5.66936123970293)
--(axis cs:0.0096,4.0756175265045)
--(axis cs:0.0098,2.90558575756373)
--(axis cs:0.01,2.05493739907559)
--(axis cs:0.0102,1.44219244318659)
--(axis cs:0.0104,1.00469625570293)
--(axis cs:0.0106,0.694949514002269)
--(axis cs:0.0108,0.477410379936765)
--(axis cs:0.011,0.325804876526534)
--(axis cs:0.0112,0.220928442520729)
--(axis cs:0.0114,0.148891477378268)
--(axis cs:0.0116,0.0997477329541815)
--(axis cs:0.0118,0.0664411871810735)
--(axis cs:0.012,0.0440103387810881)
--(axis cs:0.0122,0.0289956454569549)
--(axis cs:0.0124,0.0190040235452072)
--(axis cs:0.0126,0.0123926294820924)
--(axis cs:0.0128,0.00804180587557924)
--(axis cs:0.013,0.00519373618909181)
--(axis cs:0.0132,0.00333889993268949)
--(axis cs:0.0134,0.00213689123097192)
--(axis cs:0.0136,0.00136167233949938)
--(axis cs:0.0138,0.000864026867838253)
--(axis cs:0.014,0.000546005807160571)
--(axis cs:0.0142,0.000343661940044616)
--(axis cs:0.0144,0.000215464887035409)
--(axis cs:0.0146,0.000134578989767629)
--(axis cs:0.0148,8.37484307189143e-05)
--(axis cs:0.015,5.19297203253406e-05)
--(axis cs:0.0152,3.20874183650162e-05)
--(axis cs:0.0154,1.9759279261365e-05)
--(axis cs:0.0156,1.21272273451925e-05)
--(axis cs:0.0158,7.41892885450926e-06)
--(axis cs:0.016,4.52422298617316e-06)
--(axis cs:0.0162,2.75044196165776e-06)
--(axis cs:0.0164,1.66704773122399e-06)
--(axis cs:0.0166,1.00742074044527e-06)
--(axis cs:0.0168,6.07044031286042e-07)
--(axis cs:0.017,3.64757361091102e-07)
--(axis cs:0.0172,2.18569535391021e-07)
--(axis cs:0.0174,1.30618012066543e-07)
--(axis cs:0.0176,7.78519818383923e-08)
--(axis cs:0.0178,4.62821862125025e-08)
--(axis cs:0.018,2.74447485205084e-08)
--(axis cs:0.0182,1.6234116858454e-08)
--(axis cs:0.0184,9.57952945182513e-09)
--(axis cs:0.0186,5.63932522263679e-09)
--(axis cs:0.0188,3.31206028275151e-09)
--(axis cs:0.019,1.94078517597756e-09)
--(axis cs:0.0192,1.13470828859746e-09)
--(axis cs:0.0194,6.61968459383362e-10)
--(axis cs:0.0196,3.85349570575785e-10)
--(axis cs:0.0198,2.2384872308735e-10)
--(axis cs:0.02,1.29763824843081e-10)
--(axis cs:0.0202,7.50703658895768e-11)
--(axis cs:0.0204,4.3342645575275e-11)
--(axis cs:0.0206,2.49752548808235e-11)
--(axis cs:0.0208,1.43637307368144e-11)
--(axis cs:0.021,8.24521801416301e-12)
--(axis cs:0.0212,4.72420703343228e-12)
--(axis cs:0.0214,2.70185174122966e-12)
--(axis cs:0.0216,1.54245838297033e-12)
--(axis cs:0.0218,8.79018274445295e-13)
--(axis cs:0.022,5.00066381644288e-13)
--(axis cs:0.0222,2.83997926881723e-13)
--(axis cs:0.0224,1.61017304107681e-13)
--(axis cs:0.0226,9.11405390698865e-14)
--(axis cs:0.0228,5.15043337669998e-14)
--(axis cs:0.023,2.90589735679794e-14)
--(axis cs:0.0232,1.63693700702481e-14)
--(axis cs:0.0234,9.20681612959743e-15)
--(axis cs:0.0236,5.17038771890748e-15)
--(axis cs:0.0238,2.89923195175591e-15)
--(axis cs:0.024,1.6233001918256e-15)
--cycle;
\path [draw=none, fill=blue, fill opacity=0.5]
(axis cs:0,0)
--(axis cs:0.0002,3.00097034650853e-39)
--(axis cs:0.0004,1.33180595158039e-29)
--(axis cs:0.0006,4.34794883594433e-24)
--(axis cs:0.0008,2.88281813447308e-20)
--(axis cs:0.001,2.25959278886682e-17)
--(axis cs:0.0012,4.58849753621502e-15)
--(axis cs:0.0014,3.66862376248184e-13)
--(axis cs:0.0016,1.4826057132876e-11)
--(axis cs:0.0018,3.55812394866283e-10)
--(axis cs:0.002,5.6607416830461e-09)
--(axis cs:0.0022,6.45750577460221e-08)
--(axis cs:0.0024,5.59706827273519e-07)
--(axis cs:0.0026,3.85186816583851e-06)
--(axis cs:0.0028,2.17797182193631e-05)
--(axis cs:0.003,0.000103963098101098)
--(axis cs:0.0032,0.000428198943529838)
--(axis cs:0.0034,0.00154923143069868)
--(axis cs:0.0036,0.00499716303584501)
--(axis cs:0.0038,0.0145497409226953)
--(axis cs:0.004,0.0386429091930658)
--(axis cs:0.0042,0.0944621505149839)
--(axis cs:0.0044,0.214173948050834)
--(axis cs:0.0046,0.453416129407434)
--(axis cs:0.0048,0.901527460031158)
--(axis cs:0.005,1.69212633933122)
--(axis cs:0.0052,3.01173851254532)
--(axis cs:0.0054,5.10351436468107)
--(axis cs:0.0056,8.26297420559593)
--(axis cs:0.0058,12.8233899369747)
--(axis cs:0.006,19.1298809719443)
--(axis cs:0.0062,27.5033666828899)
--(axis cs:0.0064,38.197769617239)
--(axis cs:0.0066,51.3557906719054)
--(axis cs:0.0068,66.9696976206799)
--(axis cs:0.007,84.8535623562632)
--(axis cs:0.0072,104.632179248452)
--(axis cs:0.0074,88.8048737038958)
--(axis cs:0.0076,71.0747137642668)
--(axis cs:0.0078,56.1379510814392)
--(axis cs:0.008,43.7873232069078)
--(axis cs:0.0082,33.7486675534372)
--(axis cs:0.0084,25.717456860267)
--(axis cs:0.0086,19.3861648116425)
--(axis cs:0.0088,14.4631132938788)
--(axis cs:0.009,10.6840718447148)
--(axis cs:0.0092,7.81813057951323)
--(axis cs:0.0094,5.66936123970293)
--(axis cs:0.0096,4.0756175265045)
--(axis cs:0.0098,2.90558575756373)
--(axis cs:0.01,2.05493739907559)
--(axis cs:0.0102,1.44219244318659)
--(axis cs:0.0104,1.00469625570293)
--(axis cs:0.0106,0.694949514002269)
--(axis cs:0.0108,0.477410379936765)
--(axis cs:0.011,0.325804876526534)
--(axis cs:0.0112,0.220928442520729)
--(axis cs:0.0114,0.148891477378268)
--(axis cs:0.0116,0.0997477329541815)
--(axis cs:0.0118,0.0664411871810735)
--(axis cs:0.012,0.0440103387810881)
--(axis cs:0.0122,0.0289956454569549)
--(axis cs:0.0124,0.0190040235452072)
--(axis cs:0.0126,0.0123926294820924)
--(axis cs:0.0128,0.00804180587557924)
--(axis cs:0.013,0.00519373618909181)
--(axis cs:0.0132,0.00333889993268949)
--(axis cs:0.0134,0.00213689123097192)
--(axis cs:0.0136,0.00136167233949938)
--(axis cs:0.0138,0.000864026867838253)
--(axis cs:0.014,0.000546005807160571)
--(axis cs:0.0142,0.000343661940044616)
--(axis cs:0.0144,0.000215464887035409)
--(axis cs:0.0146,0.000134578989767629)
--(axis cs:0.0148,8.37484307189143e-05)
--(axis cs:0.015,5.19297203253406e-05)
--(axis cs:0.0152,3.20874183650162e-05)
--(axis cs:0.0154,1.9759279261365e-05)
--(axis cs:0.0156,1.21272273451925e-05)
--(axis cs:0.0158,7.41892885450926e-06)
--(axis cs:0.016,4.52422298617316e-06)
--(axis cs:0.0162,2.75044196165776e-06)
--(axis cs:0.0164,1.66704773122399e-06)
--(axis cs:0.0166,1.00742074044527e-06)
--(axis cs:0.0168,6.07044031286042e-07)
--(axis cs:0.017,3.64757361091102e-07)
--(axis cs:0.0172,2.18569535391021e-07)
--(axis cs:0.0174,1.30618012066543e-07)
--(axis cs:0.0176,7.78519818383923e-08)
--(axis cs:0.0178,4.62821862125025e-08)
--(axis cs:0.018,2.74447485205084e-08)
--(axis cs:0.0182,1.6234116858454e-08)
--(axis cs:0.0184,9.57952945182513e-09)
--(axis cs:0.0186,5.63932522263679e-09)
--(axis cs:0.0188,3.31206028275151e-09)
--(axis cs:0.019,1.94078517597756e-09)
--(axis cs:0.0192,1.13470828859746e-09)
--(axis cs:0.0194,6.61968459383362e-10)
--(axis cs:0.0196,3.85349570575785e-10)
--(axis cs:0.0198,2.2384872308735e-10)
--(axis cs:0.02,1.29763824843081e-10)
--(axis cs:0.0202,7.50703658895768e-11)
--(axis cs:0.0204,4.3342645575275e-11)
--(axis cs:0.0206,2.49752548808235e-11)
--(axis cs:0.0208,1.43637307368144e-11)
--(axis cs:0.021,8.24521801416301e-12)
--(axis cs:0.0212,4.72420703343228e-12)
--(axis cs:0.0214,2.70185174122966e-12)
--(axis cs:0.0216,1.54245838297033e-12)
--(axis cs:0.0218,8.79018274445295e-13)
--(axis cs:0.022,5.00066381644288e-13)
--(axis cs:0.0222,2.83997926881723e-13)
--(axis cs:0.0224,1.61017304107681e-13)
--(axis cs:0.0226,9.11405390698865e-14)
--(axis cs:0.0228,5.15043337669998e-14)
--(axis cs:0.023,2.90589735679794e-14)
--(axis cs:0.0232,1.63693700702481e-14)
--(axis cs:0.0234,9.20681612959743e-15)
--(axis cs:0.0236,5.17038771890748e-15)
--(axis cs:0.0238,2.89923195175591e-15)
--(axis cs:0.024,1.6233001918256e-15)
--cycle;
\path [draw=none, fill=blue, fill opacity=0.5]
(axis cs:0,0)
--(axis cs:0.0002,4.38630231113233e-32)
--(axis cs:0.0004,9.42164546775413e-24)
--(axis cs:0.0006,5.22996289695702e-19)
--(axis cs:0.0008,9.86216555458081e-16)
--(axis cs:0.001,2.9143627354464e-13)
--(axis cs:0.0012,2.66669889722348e-11)
--(axis cs:0.0014,1.08650456690335e-09)
--(axis cs:0.0016,2.44844210199571e-08)
--(axis cs:0.0018,3.50988320156314e-07)
--(axis cs:0.002,3.52140105748224e-06)
--(axis cs:0.0022,2.64692151139118e-05)
--(axis cs:0.0024,0.000156751672867648)
--(axis cs:0.0026,0.000759830128964595)
--(axis cs:0.0028,0.00310561691820713)
--(axis cs:0.003,0.0109579090659142)
--(axis cs:0.0032,0.034016993111393)
--(axis cs:0.0034,0.0943609589409247)
--(axis cs:0.0036,0.236919036551997)
--(axis cs:0.0038,0.544246235965627)
--(axis cs:0.004,1.15434405660579)
--(axis cs:0.0042,2.27822422319553)
--(axis cs:0.0044,4.21196189035521)
--(axis cs:0.0046,7.33697307600828)
--(axis cs:0.0048,12.1029173241866)
--(axis cs:0.005,18.9902597620454)
--(axis cs:0.0052,28.4537592035236)
--(axis cs:0.0054,40.8529986597383)
--(axis cs:0.0056,56.3802313963309)
--(axis cs:0.0058,74.9980614577203)
--(axis cs:0.006,96.399078434761)
--(axis cs:0.0062,119.996496092707)
--(axis cs:0.0064,144.949774971286)
--(axis cs:0.0066,170.223290391324)
--(axis cs:0.0068,166.39094249668)
--(axis cs:0.007,139.984977047662)
--(axis cs:0.0072,115.934331544271)
--(axis cs:0.0074,94.6001679291746)
--(axis cs:0.0076,76.1134941728492)
--(axis cs:0.0078,60.4276155973675)
--(axis cs:0.008,47.3700206604959)
--(axis cs:0.0082,36.6889069284383)
--(axis cs:0.0084,28.0917234489209)
--(axis cs:0.0086,21.2748316880067)
--(axis cs:0.0088,15.9446047501456)
--(axis cs:0.009,11.8310271298241)
--(axis cs:0.0092,8.6952111181369)
--(axis cs:0.0094,6.33231622002179)
--(axis cs:0.0096,4.57124530993385)
--(axis cs:0.0098,3.27227978320237)
--(axis cs:0.01,2.32356823135405)
--(axis cs:0.0102,1.6371408200762)
--(axis cs:0.0104,1.14490856272043)
--(axis cs:0.0106,0.794933744641845)
--(axis cs:0.0108,0.548126540735307)
--(axis cs:0.011,0.375429681218069)
--(axis cs:0.0112,0.255491533381875)
--(axis cs:0.0114,0.172791105980003)
--(axis cs:0.0116,0.116159569591466)
--(axis cs:0.0118,0.0776362247336516)
--(axis cs:0.012,0.0515979327701638)
--(axis cs:0.0122,0.0341065074305667)
--(axis cs:0.0124,0.0224261120651527)
--(axis cs:0.0126,0.0146707956910648)
--(axis cs:0.0128,0.00955001556189569)
--(axis cs:0.013,0.00618685377119825)
--(axis cs:0.0132,0.00398944660171198)
--(axis cs:0.0134,0.00256089269833464)
--(axis cs:0.0136,0.00163667442640372)
--(axis cs:0.0138,0.00104154779409292)
--(axis cs:0.014,0.000660075452761531)
--(axis cs:0.0142,0.000416634011714716)
--(axis cs:0.0144,0.000261944833104547)
--(axis cs:0.0146,0.000164060561318553)
--(axis cs:0.0148,0.000102372004077384)
--(axis cs:0.015,6.36476741482536e-05)
--(axis cs:0.0152,3.94319450280711e-05)
--(axis cs:0.0154,2.43453839750448e-05)
--(axis cs:0.0156,1.49804470564146e-05)
--(axis cs:0.0158,9.18772890022433e-06)
--(axis cs:0.016,5.61695593416862e-06)
--(axis cs:0.0162,3.42323116077753e-06)
--(axis cs:0.0164,2.07991462131355e-06)
--(axis cs:0.0166,1.25996770085529e-06)
--(axis cs:0.0168,7.6103984052187e-07)
--(axis cs:0.017,4.58371763162202e-07)
--(axis cs:0.0172,2.7530760687414e-07)
--(axis cs:0.0174,1.64905414482916e-07)
--(axis cs:0.0176,9.85129938774576e-08)
--(axis cs:0.0178,5.86973778185031e-08)
--(axis cs:0.018,3.48846457806647e-08)
--(axis cs:0.0182,2.06806141146334e-08)
--(axis cs:0.0184,1.2230053205374e-08)
--(axis cs:0.0186,7.21523772012225e-09)
--(axis cs:0.0188,4.24669638531252e-09)
--(axis cs:0.019,2.49373484918187e-09)
--(axis cs:0.0192,1.46105786890699e-09)
--(axis cs:0.0194,8.54125365511676e-10)
--(axis cs:0.0196,4.98231604938351e-10)
--(axis cs:0.0198,2.90010826031021e-10)
--(axis cs:0.02,1.68456481004786e-10)
--(axis cs:0.0202,9.76491778580082e-11)
--(axis cs:0.0204,5.64901972032708e-11)
--(axis cs:0.0206,3.26149731643592e-11)
--(axis cs:0.0208,1.87938492401451e-11)
--(axis cs:0.021,1.08089655494924e-11)
--(axis cs:0.0212,6.20492712167372e-12)
--(axis cs:0.0214,3.55539178611517e-12)
--(axis cs:0.0216,2.03352795779895e-12)
--(axis cs:0.0218,1.16101553384927e-12)
--(axis cs:0.022,6.61704434051589e-13)
--(axis cs:0.0222,3.76479012548669e-13)
--(axis cs:0.0224,2.13835816271175e-13)
--(axis cs:0.0226,1.21253697975333e-13)
--(axis cs:0.0228,6.86429595305323e-14)
--(axis cs:0.023,3.87966979691754e-14)
--(axis cs:0.0232,2.18928491194375e-14)
--(axis cs:0.0234,1.23347194344956e-14)
--(axis cs:0.0236,6.93882890962642e-15)
--(axis cs:0.0238,3.89747350253865e-15)
--(axis cs:0.024,2.18589940234101e-15)
--cycle;
\path [draw=none, fill=blue, fill opacity=0.5]
(axis cs:0,0)
--(axis cs:0.0002,4.28964223063707e-25)
--(axis cs:0.0004,4.23925669893049e-18)
--(axis cs:0.0006,3.8843035174309e-14)
--(axis cs:0.0008,2.03980960361959e-11)
--(axis cs:0.001,2.23578413477403e-09)
--(axis cs:0.0012,9.09612932925116e-08)
--(axis cs:0.0014,1.86740983355346e-06)
--(axis cs:0.0016,2.32374652780952e-05)
--(axis cs:0.0018,0.00019726676995222)
--(axis cs:0.002,0.00123850159771812)
--(axis cs:0.0022,0.00609150384571705)
--(axis cs:0.0024,0.0244907822733581)
--(axis cs:0.0026,0.0831291604329364)
--(axis cs:0.0028,0.244273624310176)
--(axis cs:0.003,0.633888506344497)
--(axis cs:0.0032,1.47614150045633)
--(axis cs:0.0034,3.12549923236712)
--(axis cs:0.0036,6.08284419664121)
--(axis cs:0.0038,10.9810360144135)
--(axis cs:0.004,18.5299350809066)
--(axis cs:0.0042,29.4207126030825)
--(axis cs:0.0044,44.2015281358291)
--(axis cs:0.0046,63.1472686753469)
--(axis cs:0.0048,86.1512421387252)
--(axis cs:0.005,112.664536610035)
--(axis cs:0.0052,141.699695822165)
--(axis cs:0.0054,171.902056209116)
--(axis cs:0.0056,201.678253673426)
--(axis cs:0.0058,229.360553795801)
--(axis cs:0.006,253.380099640291)
--(axis cs:0.0062,251.83822176869)
--(axis cs:0.0064,223.41460037644)
--(axis cs:0.0066,194.515095466985)
--(axis cs:0.0068,166.39094249668)
--(axis cs:0.007,139.984977047662)
--(axis cs:0.0072,115.934331544271)
--(axis cs:0.0074,94.6001679291746)
--(axis cs:0.0076,76.1134941728492)
--(axis cs:0.0078,60.4276155973675)
--(axis cs:0.008,47.3700206604959)
--(axis cs:0.0082,36.6889069284383)
--(axis cs:0.0084,28.0917234489209)
--(axis cs:0.0086,21.2748316880067)
--(axis cs:0.0088,15.9446047501456)
--(axis cs:0.009,11.8310271298241)
--(axis cs:0.0092,8.6952111181369)
--(axis cs:0.0094,6.33231622002179)
--(axis cs:0.0096,4.57124530993385)
--(axis cs:0.0098,3.27227978320237)
--(axis cs:0.01,2.32356823135405)
--(axis cs:0.0102,1.6371408200762)
--(axis cs:0.0104,1.14490856272043)
--(axis cs:0.0106,0.794933744641845)
--(axis cs:0.0108,0.548126540735307)
--(axis cs:0.011,0.375429681218069)
--(axis cs:0.0112,0.255491533381875)
--(axis cs:0.0114,0.172791105980003)
--(axis cs:0.0116,0.116159569591466)
--(axis cs:0.0118,0.0776362247336516)
--(axis cs:0.012,0.0515979327701638)
--(axis cs:0.0122,0.0341065074305667)
--(axis cs:0.0124,0.0224261120651527)
--(axis cs:0.0126,0.0146707956910648)
--(axis cs:0.0128,0.00955001556189569)
--(axis cs:0.013,0.00618685377119825)
--(axis cs:0.0132,0.00398944660171198)
--(axis cs:0.0134,0.00256089269833464)
--(axis cs:0.0136,0.00163667442640372)
--(axis cs:0.0138,0.00104154779409292)
--(axis cs:0.014,0.000660075452761531)
--(axis cs:0.0142,0.000416634011714716)
--(axis cs:0.0144,0.000261944833104547)
--(axis cs:0.0146,0.000164060561318553)
--(axis cs:0.0148,0.000102372004077384)
--(axis cs:0.015,6.36476741482536e-05)
--(axis cs:0.0152,3.94319450280711e-05)
--(axis cs:0.0154,2.43453839750448e-05)
--(axis cs:0.0156,1.49804470564146e-05)
--(axis cs:0.0158,9.18772890022433e-06)
--(axis cs:0.016,5.61695593416862e-06)
--(axis cs:0.0162,3.42323116077753e-06)
--(axis cs:0.0164,2.07991462131355e-06)
--(axis cs:0.0166,1.25996770085529e-06)
--(axis cs:0.0168,7.6103984052187e-07)
--(axis cs:0.017,4.58371763162202e-07)
--(axis cs:0.0172,2.7530760687414e-07)
--(axis cs:0.0174,1.64905414482916e-07)
--(axis cs:0.0176,9.85129938774576e-08)
--(axis cs:0.0178,5.86973778185031e-08)
--(axis cs:0.018,3.48846457806647e-08)
--(axis cs:0.0182,2.06806141146334e-08)
--(axis cs:0.0184,1.2230053205374e-08)
--(axis cs:0.0186,7.21523772012225e-09)
--(axis cs:0.0188,4.24669638531252e-09)
--(axis cs:0.019,2.49373484918187e-09)
--(axis cs:0.0192,1.46105786890699e-09)
--(axis cs:0.0194,8.54125365511676e-10)
--(axis cs:0.0196,4.98231604938351e-10)
--(axis cs:0.0198,2.90010826031021e-10)
--(axis cs:0.02,1.68456481004786e-10)
--(axis cs:0.0202,9.76491778580082e-11)
--(axis cs:0.0204,5.64901972032708e-11)
--(axis cs:0.0206,3.26149731643592e-11)
--(axis cs:0.0208,1.87938492401451e-11)
--(axis cs:0.021,1.08089655494924e-11)
--(axis cs:0.0212,6.20492712167372e-12)
--(axis cs:0.0214,3.55539178611517e-12)
--(axis cs:0.0216,2.03352795779895e-12)
--(axis cs:0.0218,1.16101553384927e-12)
--(axis cs:0.022,6.61704434051589e-13)
--(axis cs:0.0222,3.76479012548669e-13)
--(axis cs:0.0224,2.13835816271175e-13)
--(axis cs:0.0226,1.21253697975333e-13)
--(axis cs:0.0228,6.86429595305323e-14)
--(axis cs:0.023,3.87966979691754e-14)
--(axis cs:0.0232,2.18928491194375e-14)
--(axis cs:0.0234,1.23347194344956e-14)
--(axis cs:0.0236,6.93882890962642e-15)
--(axis cs:0.0238,3.89747350253865e-15)
--(axis cs:0.024,2.18589940234101e-15)
--cycle;
\path [draw=none, fill=blue, fill opacity=0.5]
(axis cs:0,0)
--(axis cs:0.0002,3.00097034650853e-39)
--(axis cs:0.0004,1.33180595158039e-29)
--(axis cs:0.0006,4.34794883594433e-24)
--(axis cs:0.0008,2.88281813447308e-20)
--(axis cs:0.001,2.25959278886682e-17)
--(axis cs:0.0012,4.58849753621502e-15)
--(axis cs:0.0014,3.66862376248184e-13)
--(axis cs:0.0016,1.4826057132876e-11)
--(axis cs:0.0018,3.55812394866283e-10)
--(axis cs:0.002,5.6607416830461e-09)
--(axis cs:0.0022,6.45750577460221e-08)
--(axis cs:0.0024,5.59706827273519e-07)
--(axis cs:0.0026,3.85186816583851e-06)
--(axis cs:0.0028,2.17797182193631e-05)
--(axis cs:0.003,0.000103963098101098)
--(axis cs:0.0032,0.000428198943529838)
--(axis cs:0.0034,0.00154923143069868)
--(axis cs:0.0036,0.00499716303584501)
--(axis cs:0.0038,0.0145497409226953)
--(axis cs:0.004,0.0386429091930658)
--(axis cs:0.0042,0.0944621505149839)
--(axis cs:0.0044,0.214173948050834)
--(axis cs:0.0046,0.453416129407434)
--(axis cs:0.0048,0.901527460031158)
--(axis cs:0.005,1.69212633933122)
--(axis cs:0.0052,3.01173851254532)
--(axis cs:0.0054,5.10351436468107)
--(axis cs:0.0056,8.26297420559593)
--(axis cs:0.0058,12.8233899369747)
--(axis cs:0.006,19.1298809719443)
--(axis cs:0.0062,27.5033666828899)
--(axis cs:0.0064,38.197769617239)
--(axis cs:0.0066,51.3557906719054)
--(axis cs:0.0068,66.9696976206799)
--(axis cs:0.007,84.8535623562632)
--(axis cs:0.0072,104.632179248452)
--(axis cs:0.0074,94.6001679291746)
--(axis cs:0.0076,76.1134941728492)
--(axis cs:0.0078,60.4276155973675)
--(axis cs:0.008,47.3700206604959)
--(axis cs:0.0082,36.6889069284383)
--(axis cs:0.0084,28.0917234489209)
--(axis cs:0.0086,21.2748316880067)
--(axis cs:0.0088,15.9446047501456)
--(axis cs:0.009,11.8310271298241)
--(axis cs:0.0092,8.6952111181369)
--(axis cs:0.0094,6.33231622002179)
--(axis cs:0.0096,4.57124530993385)
--(axis cs:0.0098,3.27227978320237)
--(axis cs:0.01,2.32356823135405)
--(axis cs:0.0102,1.6371408200762)
--(axis cs:0.0104,1.14490856272043)
--(axis cs:0.0106,0.794933744641845)
--(axis cs:0.0108,0.548126540735307)
--(axis cs:0.011,0.375429681218069)
--(axis cs:0.0112,0.255491533381875)
--(axis cs:0.0114,0.172791105980003)
--(axis cs:0.0116,0.116159569591466)
--(axis cs:0.0118,0.0776362247336516)
--(axis cs:0.012,0.0515979327701638)
--(axis cs:0.0122,0.0341065074305667)
--(axis cs:0.0124,0.0224261120651527)
--(axis cs:0.0126,0.0146707956910648)
--(axis cs:0.0128,0.00955001556189569)
--(axis cs:0.013,0.00618685377119825)
--(axis cs:0.0132,0.00398944660171198)
--(axis cs:0.0134,0.00256089269833464)
--(axis cs:0.0136,0.00163667442640372)
--(axis cs:0.0138,0.00104154779409292)
--(axis cs:0.014,0.000660075452761531)
--(axis cs:0.0142,0.000416634011714716)
--(axis cs:0.0144,0.000261944833104547)
--(axis cs:0.0146,0.000164060561318553)
--(axis cs:0.0148,0.000102372004077384)
--(axis cs:0.015,6.36476741482536e-05)
--(axis cs:0.0152,3.94319450280711e-05)
--(axis cs:0.0154,2.43453839750448e-05)
--(axis cs:0.0156,1.49804470564146e-05)
--(axis cs:0.0158,9.18772890022433e-06)
--(axis cs:0.016,5.61695593416862e-06)
--(axis cs:0.0162,3.42323116077753e-06)
--(axis cs:0.0164,2.07991462131355e-06)
--(axis cs:0.0166,1.25996770085529e-06)
--(axis cs:0.0168,7.6103984052187e-07)
--(axis cs:0.017,4.58371763162202e-07)
--(axis cs:0.0172,2.7530760687414e-07)
--(axis cs:0.0174,1.64905414482916e-07)
--(axis cs:0.0176,9.85129938774576e-08)
--(axis cs:0.0178,5.86973778185031e-08)
--(axis cs:0.018,3.48846457806647e-08)
--(axis cs:0.0182,2.06806141146334e-08)
--(axis cs:0.0184,1.2230053205374e-08)
--(axis cs:0.0186,7.21523772012225e-09)
--(axis cs:0.0188,4.24669638531252e-09)
--(axis cs:0.019,2.49373484918187e-09)
--(axis cs:0.0192,1.46105786890699e-09)
--(axis cs:0.0194,8.54125365511676e-10)
--(axis cs:0.0196,4.98231604938351e-10)
--(axis cs:0.0198,2.90010826031021e-10)
--(axis cs:0.02,1.68456481004786e-10)
--(axis cs:0.0202,9.76491778580082e-11)
--(axis cs:0.0204,5.64901972032708e-11)
--(axis cs:0.0206,3.26149731643592e-11)
--(axis cs:0.0208,1.87938492401451e-11)
--(axis cs:0.021,1.08089655494924e-11)
--(axis cs:0.0212,6.20492712167372e-12)
--(axis cs:0.0214,3.55539178611517e-12)
--(axis cs:0.0216,2.03352795779895e-12)
--(axis cs:0.0218,1.16101553384927e-12)
--(axis cs:0.022,6.61704434051589e-13)
--(axis cs:0.0222,3.76479012548669e-13)
--(axis cs:0.0224,2.13835816271175e-13)
--(axis cs:0.0226,1.21253697975333e-13)
--(axis cs:0.0228,6.86429595305323e-14)
--(axis cs:0.023,3.87966979691754e-14)
--(axis cs:0.0232,2.18928491194375e-14)
--(axis cs:0.0234,1.23347194344956e-14)
--(axis cs:0.0236,6.93882890962642e-15)
--(axis cs:0.0238,3.89747350253865e-15)
--(axis cs:0.024,2.18589940234101e-15)
--cycle;
\path [draw=none, fill=blue, fill opacity=0.5]
(axis cs:0,0)
--(axis cs:0.0002,4.38630231113233e-32)
--(axis cs:0.0004,9.42164546775413e-24)
--(axis cs:0.0006,5.22996289695702e-19)
--(axis cs:0.0008,9.86216555458081e-16)
--(axis cs:0.001,2.9143627354464e-13)
--(axis cs:0.0012,2.66669889722348e-11)
--(axis cs:0.0014,1.08650456690335e-09)
--(axis cs:0.0016,2.44844210199571e-08)
--(axis cs:0.0018,3.50988320156314e-07)
--(axis cs:0.002,3.52140105748224e-06)
--(axis cs:0.0022,2.64692151139118e-05)
--(axis cs:0.0024,0.000156751672867648)
--(axis cs:0.0026,0.000759830128964595)
--(axis cs:0.0028,0.00310561691820713)
--(axis cs:0.003,0.0109579090659142)
--(axis cs:0.0032,0.034016993111393)
--(axis cs:0.0034,0.0943609589409247)
--(axis cs:0.0036,0.236919036551997)
--(axis cs:0.0038,0.544246235965627)
--(axis cs:0.004,1.15434405660579)
--(axis cs:0.0042,2.27822422319553)
--(axis cs:0.0044,4.21196189035521)
--(axis cs:0.0046,7.33697307600828)
--(axis cs:0.0048,12.1029173241866)
--(axis cs:0.005,18.9902597620454)
--(axis cs:0.0052,28.4537592035236)
--(axis cs:0.0054,40.8529986597383)
--(axis cs:0.0056,56.3802313963309)
--(axis cs:0.0058,74.9980614577203)
--(axis cs:0.006,96.399078434761)
--(axis cs:0.0062,119.996496092707)
--(axis cs:0.0064,144.949774971286)
--(axis cs:0.0066,170.223290391324)
--(axis cs:0.0068,194.670631562012)
--(axis cs:0.007,217.133180785212)
--(axis cs:0.0072,236.539881854436)
--(axis cs:0.0074,251.995687216163)
--(axis cs:0.0076,240.109565154236)
--(axis cs:0.0078,218.542425587442)
--(axis cs:0.008,195.733553164238)
--(axis cs:0.0082,172.639607097943)
--(axis cs:0.0084,150.06481979406)
--(axis cs:0.0086,128.63994996229)
--(axis cs:0.0088,108.819371575935)
--(axis cs:0.009,90.8921276480365)
--(axis cs:0.0092,75.002450205011)
--(axis cs:0.0094,61.1755282281401)
--(axis cs:0.0096,49.344973528739)
--(axis cs:0.0098,39.3792886394928)
--(axis cs:0.01,31.1055204646969)
--(axis cs:0.0102,24.3290782826384)
--(axis cs:0.0104,18.8493422807296)
--(axis cs:0.0106,14.4711661403143)
--(axis cs:0.0108,11.012689711825)
--(axis cs:0.011,8.31004844305997)
--(axis cs:0.0112,6.21962606128075)
--(axis cs:0.0114,4.61847851618221)
--(axis cs:0.0116,3.40348972257237)
--(axis cs:0.0118,2.48972713377504)
--(axis cs:0.012,1.80836540381978)
--(axis cs:0.0122,1.3044512893037)
--(axis cs:0.0124,0.934699493011194)
--(axis cs:0.0126,0.665440554604601)
--(axis cs:0.0128,0.470788676221954)
--(axis cs:0.013,0.331058380959296)
--(axis cs:0.0132,0.231432089769658)
--(axis cs:0.0134,0.160863683425722)
--(axis cs:0.0136,0.111193545097079)
--(axis cs:0.0138,0.0764463638792013)
--(axis cs:0.014,0.0522823826082634)
--(axis cs:0.0142,0.035574433229481)
--(axis cs:0.0144,0.0240860094782703)
--(axis cs:0.0146,0.016229070006668)
--(axis cs:0.0148,0.0108837773668214)
--(axis cs:0.015,0.00726567536281934)
--(axis cs:0.0152,0.00482873788155366)
--(axis cs:0.0154,0.00319522579235741)
--(axis cs:0.0156,0.00210536227202998)
--(axis cs:0.0158,0.00138151231424654)
--(axis cs:0.016,0.000902877637792488)
--(axis cs:0.0162,0.00058774804174528)
--(axis cs:0.0164,0.000381137818224309)
--(axis cs:0.0166,0.000246229671424615)
--(axis cs:0.0168,0.000158490686887278)
--(axis cs:0.017,0.000101650220991295)
--(axis cs:0.0172,6.49664372059235e-05)
--(axis cs:0.0174,4.13789657654764e-05)
--(axis cs:0.0176,2.62671302556463e-05)
--(axis cs:0.0178,1.66195588353419e-05)
--(axis cs:0.018,1.04816733125501e-05)
--(axis cs:0.0182,6.58984806578497e-06)
--(axis cs:0.0184,4.13030631697093e-06)
--(axis cs:0.0186,2.58094470824131e-06)
--(axis cs:0.0188,1.60801959781534e-06)
--(axis cs:0.019,9.98955043649507e-07)
--(axis cs:0.0192,6.18824475032918e-07)
--(axis cs:0.0194,3.8227873195778e-07)
--(axis cs:0.0196,2.35508912963666e-07)
--(axis cs:0.0198,1.44701102682449e-07)
--(axis cs:0.02,8.8673868227445e-08)
--(axis cs:0.0202,5.42001101106119e-08)
--(axis cs:0.0204,3.3045028622791e-08)
--(axis cs:0.0206,2.00971134614997e-08)
--(axis cs:0.0208,1.21927699557356e-08)
--(axis cs:0.021,7.37957263668854e-09)
--(axis cs:0.0212,4.45593279488571e-09)
--(axis cs:0.0214,2.68437120910418e-09)
--(axis cs:0.0216,1.61346839341639e-09)
--(axis cs:0.0218,9.67629833967732e-10)
--(axis cs:0.022,5.79035802943952e-10)
--(axis cs:0.0222,3.45752238012749e-10)
--(axis cs:0.0224,2.06017255977852e-10)
--(axis cs:0.0226,1.22500087346044e-10)
--(axis cs:0.0228,7.26905882086746e-11)
--(axis cs:0.023,4.30470453490778e-11)
--(axis cs:0.0232,2.54416874400353e-11)
--(axis cs:0.0234,1.50071953893039e-11)
--(axis cs:0.0236,8.83522328295906e-12)
--(axis cs:0.0238,5.19173912534171e-12)
--(axis cs:0.024,3.04507630349425e-12)
--cycle;
\path [draw=none, fill=blue, fill opacity=0.5]
(axis cs:0,0)
--(axis cs:0.0002,3.00097034650853e-39)
--(axis cs:0.0004,1.33180595158039e-29)
--(axis cs:0.0006,4.34794883594433e-24)
--(axis cs:0.0008,2.88281813447308e-20)
--(axis cs:0.001,2.25959278886682e-17)
--(axis cs:0.0012,4.58849753621502e-15)
--(axis cs:0.0014,3.66862376248184e-13)
--(axis cs:0.0016,1.4826057132876e-11)
--(axis cs:0.0018,3.55812394866283e-10)
--(axis cs:0.002,5.6607416830461e-09)
--(axis cs:0.0022,6.45750577460221e-08)
--(axis cs:0.0024,5.59706827273519e-07)
--(axis cs:0.0026,3.85186816583851e-06)
--(axis cs:0.0028,2.17797182193631e-05)
--(axis cs:0.003,0.000103963098101098)
--(axis cs:0.0032,0.000428198943529838)
--(axis cs:0.0034,0.00154923143069868)
--(axis cs:0.0036,0.00499716303584501)
--(axis cs:0.0038,0.0145497409226953)
--(axis cs:0.004,0.0386429091930658)
--(axis cs:0.0042,0.0944621505149839)
--(axis cs:0.0044,0.214173948050834)
--(axis cs:0.0046,0.453416129407434)
--(axis cs:0.0048,0.901527460031158)
--(axis cs:0.005,1.69212633933122)
--(axis cs:0.0052,3.01173851254532)
--(axis cs:0.0054,5.10351436468107)
--(axis cs:0.0056,8.26297420559593)
--(axis cs:0.0058,12.8233899369747)
--(axis cs:0.006,19.1298809719443)
--(axis cs:0.0062,27.5033666828899)
--(axis cs:0.0064,38.197769617239)
--(axis cs:0.0066,51.3557906719054)
--(axis cs:0.0068,66.9696976206799)
--(axis cs:0.007,84.8535623562632)
--(axis cs:0.0072,104.632179248452)
--(axis cs:0.0074,125.749693106881)
--(axis cs:0.0076,147.49817102892)
--(axis cs:0.0078,169.06349078164)
--(axis cs:0.008,189.583502870042)
--(axis cs:0.0082,208.211850369454)
--(axis cs:0.0084,224.180309039073)
--(axis cs:0.0086,236.853041252113)
--(axis cs:0.0088,229.651425311885)
--(axis cs:0.009,212.138042254452)
--(axis cs:0.0092,193.172172002542)
--(axis cs:0.0094,173.505046651131)
--(axis cs:0.0096,153.804630673516)
--(axis cs:0.0098,134.632315345131)
--(axis cs:0.01,116.431517634651)
--(axis cs:0.0102,99.5265342599624)
--(axis cs:0.0104,84.1294470340541)
--(axis cs:0.0106,70.3526878790396)
--(axis cs:0.0108,58.2249725528034)
--(axis cs:0.011,47.708610089379)
--(axis cs:0.0112,38.7166022410736)
--(axis cs:0.0114,31.128389387621)
--(axis cs:0.0116,24.8035206519357)
--(axis cs:0.0118,19.5928901237509)
--(axis cs:0.012,15.3474695532124)
--(axis cs:0.0122,11.9246760970423)
--(axis cs:0.0124,9.19264705566421)
--(axis cs:0.0126,7.03276338370732)
--(axis cs:0.0128,5.34078412532097)
--(axis cs:0.013,4.02693908502129)
--(axis cs:0.0132,3.01528987771138)
--(axis cs:0.0134,2.24262063508061)
--(axis cs:0.0136,1.65706714665881)
--(axis cs:0.0138,1.21664268008281)
--(axis cs:0.014,0.887773612098774)
--(axis cs:0.0142,0.643920058287472)
--(axis cs:0.0144,0.464326410982204)
--(axis cs:0.0146,0.332923736774929)
--(axis cs:0.0148,0.237389505687213)
--(axis cs:0.015,0.16835904727833)
--(axis cs:0.0152,0.118776329266274)
--(axis cs:0.0154,0.0833680759840717)
--(axis cs:0.0156,0.0582239674953547)
--(axis cs:0.0158,0.040465929398118)
--(axis cs:0.016,0.0279907474620875)
--(axis cs:0.0162,0.0192719805847814)
--(axis cs:0.0164,0.0132090893039687)
--(axis cs:0.0166,0.00901363845163145)
--(axis cs:0.0168,0.00612424454255283)
--(axis cs:0.017,0.00414355273124991)
--(axis cs:0.0172,0.00279191693605117)
--(axis cs:0.0174,0.00187361899036578)
--(axis cs:0.0176,0.00125241352004595)
--(axis cs:0.0178,0.000833948261027627)
--(axis cs:0.018,0.000553211657396421)
--(axis cs:0.0182,0.000365627762028711)
--(axis cs:0.0184,0.000240777698977602)
--(axis cs:0.0186,0.00015799929429328)
--(axis cs:0.0188,0.00010332069503795)
--(axis cs:0.019,6.73353725225924e-05)
--(axis cs:0.0192,4.37373559046175e-05)
--(axis cs:0.0194,2.83167695105991e-05)
--(axis cs:0.0196,1.82744596079048e-05)
--(axis cs:0.0198,1.17566049371012e-05)
--(axis cs:0.02,7.540183680422e-06)
--(axis cs:0.0202,4.82136313147342e-06)
--(axis cs:0.0204,3.07376190545244e-06)
--(axis cs:0.0206,1.95392053995944e-06)
--(axis cs:0.0208,1.23851994013429e-06)
--(axis cs:0.021,7.82854424753148e-07)
--(axis cs:0.0212,4.93472270100794e-07)
--(axis cs:0.0214,3.10219752930684e-07)
--(axis cs:0.0216,1.94500945874803e-07)
--(axis cs:0.0218,1.21629694569092e-07)
--(axis cs:0.022,7.58652067521529e-08)
--(axis cs:0.0222,4.72008455684243e-08)
--(axis cs:0.0224,2.92940539123387e-08)
--(axis cs:0.0226,1.81363424296575e-08)
--(axis cs:0.0228,1.12015459029968e-08)
--(axis cs:0.023,6.90210048232848e-09)
--(axis cs:0.0232,4.24303135636104e-09)
--(axis cs:0.0234,2.6024297581957e-09)
--(axis cs:0.0236,1.59259501455864e-09)
--(axis cs:0.0238,9.72457426693062e-10)
--(axis cs:0.024,5.92501857818914e-10)
--cycle;
\path [draw=none, fill=blue, fill opacity=0.5]
(axis cs:0,0)
--(axis cs:0.0002,3.00097034650853e-39)
--(axis cs:0.0004,1.33180595158039e-29)
--(axis cs:0.0006,4.34794883594433e-24)
--(axis cs:0.0008,2.88281813447308e-20)
--(axis cs:0.001,2.25959278886682e-17)
--(axis cs:0.0012,4.58849753621502e-15)
--(axis cs:0.0014,3.66862376248184e-13)
--(axis cs:0.0016,1.4826057132876e-11)
--(axis cs:0.0018,3.55812394866283e-10)
--(axis cs:0.002,5.6607416830461e-09)
--(axis cs:0.0022,6.45750577460221e-08)
--(axis cs:0.0024,5.59706827273519e-07)
--(axis cs:0.0026,3.85186816583851e-06)
--(axis cs:0.0028,2.17797182193631e-05)
--(axis cs:0.003,0.000103963098101098)
--(axis cs:0.0032,0.000428198943529838)
--(axis cs:0.0034,0.00154923143069868)
--(axis cs:0.0036,0.00499716303584501)
--(axis cs:0.0038,0.0145497409226953)
--(axis cs:0.004,0.0386429091930658)
--(axis cs:0.0042,0.0944621505149839)
--(axis cs:0.0044,0.214173948050834)
--(axis cs:0.0046,0.453416129407434)
--(axis cs:0.0048,0.901527460031158)
--(axis cs:0.005,1.69212633933122)
--(axis cs:0.0052,3.01173851254532)
--(axis cs:0.0054,5.10351436468107)
--(axis cs:0.0056,8.26297420559593)
--(axis cs:0.0058,12.8233899369747)
--(axis cs:0.006,19.1298809719443)
--(axis cs:0.0062,27.5033666828899)
--(axis cs:0.0064,38.197769617239)
--(axis cs:0.0066,51.3557906719054)
--(axis cs:0.0068,66.9696976206799)
--(axis cs:0.007,84.8535623562632)
--(axis cs:0.0072,104.632179248452)
--(axis cs:0.0074,125.749693106881)
--(axis cs:0.0076,147.49817102892)
--(axis cs:0.0078,169.06349078164)
--(axis cs:0.008,189.583502870042)
--(axis cs:0.0082,172.639607097943)
--(axis cs:0.0084,150.06481979406)
--(axis cs:0.0086,128.63994996229)
--(axis cs:0.0088,108.819371575935)
--(axis cs:0.009,90.8921276480365)
--(axis cs:0.0092,75.002450205011)
--(axis cs:0.0094,61.1755282281401)
--(axis cs:0.0096,49.344973528739)
--(axis cs:0.0098,39.3792886394928)
--(axis cs:0.01,31.1055204646969)
--(axis cs:0.0102,24.3290782826384)
--(axis cs:0.0104,18.8493422807296)
--(axis cs:0.0106,14.4711661403143)
--(axis cs:0.0108,11.012689711825)
--(axis cs:0.011,8.31004844305997)
--(axis cs:0.0112,6.21962606128075)
--(axis cs:0.0114,4.61847851618221)
--(axis cs:0.0116,3.40348972257237)
--(axis cs:0.0118,2.48972713377504)
--(axis cs:0.012,1.80836540381978)
--(axis cs:0.0122,1.3044512893037)
--(axis cs:0.0124,0.934699493011194)
--(axis cs:0.0126,0.665440554604601)
--(axis cs:0.0128,0.470788676221954)
--(axis cs:0.013,0.331058380959296)
--(axis cs:0.0132,0.231432089769658)
--(axis cs:0.0134,0.160863683425722)
--(axis cs:0.0136,0.111193545097079)
--(axis cs:0.0138,0.0764463638792013)
--(axis cs:0.014,0.0522823826082634)
--(axis cs:0.0142,0.035574433229481)
--(axis cs:0.0144,0.0240860094782703)
--(axis cs:0.0146,0.016229070006668)
--(axis cs:0.0148,0.0108837773668214)
--(axis cs:0.015,0.00726567536281934)
--(axis cs:0.0152,0.00482873788155366)
--(axis cs:0.0154,0.00319522579235741)
--(axis cs:0.0156,0.00210536227202998)
--(axis cs:0.0158,0.00138151231424654)
--(axis cs:0.016,0.000902877637792488)
--(axis cs:0.0162,0.00058774804174528)
--(axis cs:0.0164,0.000381137818224309)
--(axis cs:0.0166,0.000246229671424615)
--(axis cs:0.0168,0.000158490686887278)
--(axis cs:0.017,0.000101650220991295)
--(axis cs:0.0172,6.49664372059235e-05)
--(axis cs:0.0174,4.13789657654764e-05)
--(axis cs:0.0176,2.62671302556463e-05)
--(axis cs:0.0178,1.66195588353419e-05)
--(axis cs:0.018,1.04816733125501e-05)
--(axis cs:0.0182,6.58984806578497e-06)
--(axis cs:0.0184,4.13030631697093e-06)
--(axis cs:0.0186,2.58094470824131e-06)
--(axis cs:0.0188,1.60801959781534e-06)
--(axis cs:0.019,9.98955043649507e-07)
--(axis cs:0.0192,6.18824475032918e-07)
--(axis cs:0.0194,3.8227873195778e-07)
--(axis cs:0.0196,2.35508912963666e-07)
--(axis cs:0.0198,1.44701102682449e-07)
--(axis cs:0.02,8.8673868227445e-08)
--(axis cs:0.0202,5.42001101106119e-08)
--(axis cs:0.0204,3.3045028622791e-08)
--(axis cs:0.0206,2.00971134614997e-08)
--(axis cs:0.0208,1.21927699557356e-08)
--(axis cs:0.021,7.37957263668854e-09)
--(axis cs:0.0212,4.45593279488571e-09)
--(axis cs:0.0214,2.68437120910418e-09)
--(axis cs:0.0216,1.61346839341639e-09)
--(axis cs:0.0218,9.67629833967732e-10)
--(axis cs:0.022,5.79035802943952e-10)
--(axis cs:0.0222,3.45752238012749e-10)
--(axis cs:0.0224,2.06017255977852e-10)
--(axis cs:0.0226,1.22500087346044e-10)
--(axis cs:0.0228,7.26905882086746e-11)
--(axis cs:0.023,4.30470453490778e-11)
--(axis cs:0.0232,2.54416874400353e-11)
--(axis cs:0.0234,1.50071953893039e-11)
--(axis cs:0.0236,8.83522328295906e-12)
--(axis cs:0.0238,5.19173912534171e-12)
--(axis cs:0.024,3.04507630349425e-12)
--cycle;
\addplot [semithick, color0, opacity=0.8]
table {%
0 0
0.0002 1.35779868892061e-09
0.0004 7.03264588182167e-06
0.0006 0.00077576425362708
0.0008 0.0176964530460067
0.001 0.170204669765065
0.0012 0.94796666203739
0.0014 3.62127703759399
0.0016 10.4967990499091
0.0018 24.6423166977686
0.002 48.984743375605
0.0022 85.1096124950481
0.0024 132.315987069706
0.0026 187.366018608405
0.0028 245.031490169539
0.003 299.204482964383
0.0032 344.166625554478
0.0034 375.633399784193
0.0036 391.341382229453
0.0038 391.131253002505
0.004 376.6239364113
0.0042 350.659364364021
0.0044 316.671390945087
0.0046 278.132151246613
0.0048 238.141886324419
0.005 199.186771160928
0.0052 163.048755076709
0.0054 130.83056941505
0.0056 103.053206760269
0.0058 79.7873803369428
0.006 60.7897332615976
0.0062 45.6250166719352
0.0064 33.764694951832
0.0066 24.6593694570613
0.0068 17.7868661496403
0.007 12.6801601962024
0.0072 8.94007607834724
0.0074 6.23747618667221
0.0076 4.30890608686193
0.0078 2.94873796599512
0.008 1.99995481107366
0.0082 1.34495499920915
0.0084 0.897168396988933
0.0086 0.59385652774837
0.0088 0.390195217766181
0.009 0.254575414817491
0.0092 0.164974583259305
0.0094 0.106220200791715
0.0096 0.0679676620931031
0.0098 0.0432325808603513
0.01 0.0273423296596845
0.0102 0.0171977344356047
0.0104 0.0107599089132325
0.0106 0.00669780790315936
0.0108 0.00414881285588331
0.011 0.00255774928923221
0.0112 0.00156966462258963
0.0114 0.000959043980204529
0.0116 0.000583468440526314
0.0118 0.000353512309656768
0.012 0.000213333020725685
0.0122 0.000128242850643948
0.0124 7.68037201326851e-05
0.0126 4.58305761506692e-05
0.0128 2.72520977192749e-05
0.013 1.61495898117959e-05
0.0132 9.53857042332152e-06
0.0134 5.6157357557516e-06
0.0136 3.29587795741631e-06
0.0138 1.92847688063577e-06
0.014 1.12505239925148e-06
0.0142 6.54456317675556e-07
0.0144 3.79639444120029e-07
0.0146 2.19622342593566e-07
0.0148 1.26714616937668e-07
0.015 7.29207563822728e-08
0.0152 4.18579500369891e-08
0.0154 2.39681255123372e-08
0.0156 1.36913240004885e-08
0.0158 7.8025585154859e-09
0.016 4.43642182094435e-09
0.0162 2.51684199065801e-09
0.0164 1.4247171282791e-09
0.0166 8.04771081382379e-10
0.0168 4.53636274462889e-10
0.017 2.55184812810522e-10
0.0172 1.43262573263436e-10
0.0174 8.02712667488462e-11
0.0176 4.48905435239373e-11
0.0178 2.50573023493042e-11
0.018 1.3960957590044e-11
0.0182 7.76450428095243e-12
0.0184 4.31067763326576e-12
0.0186 2.3890530370788e-12
0.0188 1.32181029046682e-12
0.019 7.30112445413e-13
0.0192 4.02625619692298e-13
0.0194 2.21675354345489e-13
0.0196 1.21857053967287e-13
0.0198 6.68826864553411e-14
0.02 3.66537700284644e-14
0.0202 2.00575217613878e-14
0.0204 1.0959758809364e-14
0.0206 5.97999627347004e-15
0.0208 3.25827538993974e-15
0.021 1.77285035089687e-15
0.0212 9.63305278865802e-16
0.0214 5.22724971007102e-16
0.0216 2.83275877689995e-16
0.0218 1.53314174574747e-16
0.022 8.28705967052991e-17
0.0222 4.47376214030539e-17
0.0224 2.41217142550633e-17
0.0226 1.29901647606133e-17
0.0228 6.98715695218015e-18
0.023 3.75382142728391e-18
0.0232 2.01438141289907e-18
0.0234 1.07972288043767e-18
0.0236 5.78086370487335e-19
0.0238 3.09164855713053e-19
0.024 1.65162493477036e-19
};
\addlegendentry{hair spray}
\addplot [semithick, color1, opacity=0.8]
table {%
0 0
0.0002 4.22795300245434e-13
0.0004 1.37496114174896e-08
0.0006 4.44349918357242e-06
0.0008 0.000217352763044482
0.001 0.00377803209216558
0.0012 0.0341292040287655
0.0014 0.196252804450894
0.0016 0.81078219972159
0.0018 2.60195337937075
0.002 6.84154460221035
0.0022 15.3102330091327
0.0024 29.989931796765
0.0026 52.5284673552818
0.0028 83.6444501303574
0.003 122.689795384513
0.0032 167.536087950985
0.0034 214.832099146605
0.0036 260.553774989619
0.0038 300.686244763745
0.004 331.861083501153
0.0042 351.812088721991
0.0044 359.582629080389
0.0046 355.488541792698
0.0048 340.892444776181
0.005 317.870126226892
0.0052 288.849553341082
0.0054 256.286017343453
0.0056 222.412439864701
0.0058 189.079722158915
0.006 157.683074976465
0.0062 129.15840845009
0.0064 104.027759779785
0.0066 82.4728224869533
0.0068 64.418951169431
0.007 49.6167625224316
0.0072 37.7133085450788
0.0074 28.3089543469283
0.0076 20.9991936459158
0.0078 15.4026440241404
0.008 11.1775340891288
0.0082 8.02935931856564
0.0084 5.71228280177854
0.0086 4.02649685568888
0.0088 2.81329566877787
0.009 1.94913981490007
0.0092 1.33957890477063
0.0094 0.913564437606188
0.0096 0.618435189640704
0.0098 0.415683987125745
0.01 0.277503574526014
0.0102 0.184045366929219
0.0104 0.121294291427453
0.0106 0.0794542187269003
0.0108 0.0517429197120304
0.011 0.0335068200368587
0.0112 0.0215799201029034
0.0114 0.0138255445023038
0.0116 0.00881267294525788
0.0118 0.00558983901571188
0.012 0.00352879748888552
0.0122 0.00221746133601891
0.0124 0.00138723530792914
0.0126 0.000864109583526985
0.0128 0.000536005372954958
0.013 0.00033113529045748
0.0132 0.000203765039925337
0.0134 0.000124908300098191
0.0136 7.62847405416888e-05
0.0138 4.6420953790526e-05
0.014 2.81490511547575e-05
0.0142 1.70109369999016e-05
0.0144 1.02458314352702e-05
0.0146 6.15118439560684e-06
0.0148 3.68128443355421e-06
0.015 2.1963631010631e-06
0.0152 1.30649181636633e-06
0.0154 7.74888778617601e-07
0.0156 4.5828247881716e-07
0.0158 2.70282688558907e-07
0.016 1.58972888238179e-07
0.0162 9.3255712797952e-08
0.0164 5.45634777566251e-08
0.0166 3.18440689554753e-08
0.0168 1.85387204875035e-08
0.017 1.07666238287212e-08
0.0172 6.238080275349e-09
0.0174 3.60592150732689e-09
0.0176 2.07968212190716e-09
0.0178 1.19677852625666e-09
0.018 6.87205808105705e-10
0.0182 3.93763499329262e-10
0.0184 2.25153370260175e-10
0.0186 1.28479536596914e-10
0.0188 7.3167725019549e-11
0.019 4.15864867650185e-11
0.0192 2.3591115560581e-11
0.0194 1.33574712820726e-11
0.0196 7.54909716595647e-12
0.0198 4.2586875071307e-12
0.02 2.39817608909909e-12
0.0202 1.34810900327012e-12
0.0204 7.56521520562321e-13
0.0206 4.23821702237897e-13
0.0208 2.37041085325845e-13
0.021 1.32359502443409e-13
0.0212 7.37887008057146e-14
0.0214 4.10714156450754e-14
0.0216 2.28252767651452e-14
0.0218 1.26657288252302e-14
0.022 7.0176700806092e-15
0.0222 3.88253085379576e-15
0.0224 2.14489671943879e-15
0.0226 1.18325234325703e-15
0.0228 6.51834935557492e-16
0.023 3.58588750638783e-16
0.0232 1.96998857004142e-16
0.0234 1.08080608000636e-16
0.0236 5.92185349401817e-17
0.0238 3.2404244611104e-17
0.024 1.77087884307283e-17
};
\addlegendentry{groom}
\addplot [semithick, color2, opacity=0.8]
table {%
0 0
0.0002 7.61192645735247e-14
0.0004 3.61092349697281e-09
0.0006 1.45540906761335e-06
0.0008 8.32729233244913e-05
0.001 0.00163463517706023
0.0012 0.016309637005717
0.0014 0.102008198661971
0.0016 0.453261075131758
0.0018 1.5511170267273
0.002 4.31978738912353
0.0022 10.1830168367104
0.0024 20.9167445288748
0.0026 38.2727217377937
0.0028 63.4606200393569
0.003 96.6581704221682
0.0032 136.725048683837
0.0034 181.227009291322
0.0036 226.769934038948
0.0038 269.547320550091
0.004 305.953317774446
0.0042 333.115658709357
0.0044 349.246132216938
0.0046 353.767577642377
0.0048 347.233700435724
0.005 331.097097920626
0.0052 307.397169591144
0.0054 278.436077650686
0.0056 246.494649171655
0.0058 213.618758823852
0.006 181.486437029819
0.0062 151.350391217959
0.0064 124.041120785831
0.0066 100.011933662309
0.0068 79.4075693181014
0.007 62.1411541566471
0.0072 47.9683798653516
0.0074 36.5520078018336
0.0076 27.5133978528566
0.0078 20.4704445030645
0.008 15.063048139348
0.0082 10.9681854093018
0.0084 7.90696667177293
0.0086 5.64598607193122
0.0088 3.99495713920016
0.009 2.80221773170423
0.0092 1.94927234418957
0.0094 1.34516936751816
0.0096 0.9212096338105
0.0098 0.626255300784991
0.01 0.42274846574095
0.0102 0.283445185521903
0.0104 0.188809728248726
0.0106 0.124983681895028
0.0108 0.0822347618707651
0.011 0.0537928772674796
0.0112 0.0349905096527558
0.0114 0.0226368578650084
0.0116 0.0145680770183744
0.0118 0.0093279080114947
0.012 0.00594338868209651
0.0122 0.00376894854181817
0.0124 0.00237907146773072
0.0126 0.00149505738056734
0.0128 0.000935470911759743
0.013 0.000582882458416365
0.0132 0.000361712565739222
0.0134 0.0002235781404359
0.0136 0.000137666147229334
0.0138 8.44508158591391e-05
0.014 5.16183585156872e-05
0.0142 3.14391682796165e-05
0.0144 1.90829492356098e-05
0.0146 1.15442812485094e-05
0.0148 6.9610222268174e-06
0.015 4.18408115287677e-06
0.0152 2.50716509816976e-06
0.0154 1.49780389284948e-06
0.0156 8.92171408243447e-07
0.0158 5.29899951305346e-07
0.016 3.13849492572507e-07
0.0162 1.85378594792661e-07
0.0164 1.0920333976009e-07
0.0166 6.41618691818942e-08
0.0168 3.7601785096784e-08
0.017 2.19813819838646e-08
0.0172 1.2818595305834e-08
0.0174 7.45740627181704e-09
0.0176 4.32831953568279e-09
0.0178 2.5064349436483e-09
0.018 1.4481712293854e-09
0.0182 8.3489105423697e-10
0.0184 4.80292177084666e-10
0.0186 2.75718431925808e-10
0.0188 1.57953347408636e-10
0.019 9.03050845606961e-11
0.0192 5.15267728595561e-11
0.0194 2.93432056497533e-11
0.0196 1.66783099860174e-11
0.0198 9.46197732965061e-12
0.02 5.3581168652589e-12
0.0202 3.02870843139119e-12
0.0204 1.70896006831727e-12
0.0206 9.62607563091312e-13
0.0208 5.4128111253654e-13
0.021 3.03854508696917e-13
0.0212 1.70290450773551e-13
0.0214 9.52815773485272e-14
0.0216 5.32271859618645e-14
0.0218 2.96876323072844e-14
0.022 1.65327977068176e-14
0.0222 9.19299054581887e-15
0.0224 5.10408139634969e-15
0.0226 2.82969120950356e-15
0.0228 1.56650310711317e-15
0.023 8.65972411911608e-16
0.0232 4.78043037803117e-16
0.0234 2.63529598823739e-16
0.0236 1.45077594230184e-16
0.0238 7.97606489054128e-17
0.024 4.37928315113879e-17
};
\addlegendentry{yellow lady's slipper}
\addplot [semithick, color3, opacity=0.8]
table {%
0 0
0.0002 1.63304653410572e-16
0.0004 2.90980802338031e-11
0.0006 2.54391471865381e-08
0.0008 2.52149359912999e-06
0.001 7.58076561312621e-05
0.0012 0.00107160766687709
0.0014 0.0089985803413974
0.0016 0.0516110463947514
0.0018 0.221227821059011
0.002 0.753634925751084
0.0022 2.13180511178777
0.0024 5.17199814367324
0.0026 11.0298671973361
0.0028 21.0757198851318
0.003 36.6331090584455
0.0032 58.6338009969209
0.0034 87.2857386482926
0.0036 121.857708878618
0.0038 160.652447408685
0.004 201.183753056193
0.0042 240.515622550635
0.0044 275.68210296399
0.0046 304.095014317866
0.0048 323.861262245946
0.005 333.96274892689
0.0052 334.287936806464
0.0054 325.534799085384
0.0056 309.024310466083
0.0058 286.470438616187
0.006 259.748868967789
0.0062 230.696275128522
0.0064 200.958909646343
0.0066 171.896916227713
0.0068 144.541092554638
0.007 119.592670514931
0.0072 97.4538923817262
0.0074 78.2770485071028
0.0076 62.0212911535369
0.0078 48.5090753606369
0.008 37.4768080842784
0.0082 28.6167498138539
0.0084 21.6091651725825
0.0086 16.1450881904592
0.0088 11.9408949354873
0.009 8.74626266792109
0.0092 6.34716264087156
0.0094 4.56539895083454
0.0096 3.25596397229389
0.0098 2.30320217419976
0.01 1.61650454197908
0.0102 1.12602130592532
0.0104 0.77869215182818
0.0106 0.534751570883728
0.0108 0.364767719860701
0.011 0.247208643245249
0.0112 0.166491866350165
0.0114 0.111454712568036
0.0116 0.0741769407678494
0.0118 0.0490895356025638
0.012 0.0323101258225112
0.0122 0.0211540883083413
0.0124 0.0137793399025455
0.0126 0.00893120312896225
0.0128 0.00576109091603547
0.013 0.00369892387946485
0.0132 0.00236418677869029
0.0134 0.00150446130315117
0.0136 0.000953295817434555
0.0138 0.000601553104271344
0.014 0.000378067905717709
0.0142 0.000236680792640458
0.0144 0.0001476044276804
0.0146 9.17112954174405e-05
0.0148 5.67773940803756e-05
0.015 3.50265759457039e-05
0.0152 2.15341792001447e-05
0.0154 1.31948560323714e-05
0.0156 8.05865429334375e-06
0.0158 4.9061000592821e-06
0.016 2.97755266794665e-06
0.0162 1.8016217212e-06
0.0164 1.08687507048972e-06
0.0166 6.53788297169587e-07
0.0168 3.92161291128713e-07
0.017 2.34579429601707e-07
0.0172 1.39939183488036e-07
0.0174 8.32604236358792e-08
0.0176 4.94097379728479e-08
0.0178 2.92472529178579e-08
0.018 1.7269483325279e-08
0.0182 1.01722660889628e-08
0.0184 5.97753076019741e-09
0.0186 3.50439135698977e-09
0.0188 2.04979446219223e-09
0.019 1.19628479054174e-09
0.0192 6.96633567508887e-10
0.0194 4.04797826285059e-10
0.0196 2.34721972140453e-10
0.0198 1.35821466411226e-10
0.02 7.84330231220916e-11
0.0202 4.52024149086559e-11
0.0204 2.59999344847354e-11
0.0206 1.49260926007679e-11
0.0208 8.55259986988054e-12
0.021 4.89150808995468e-12
0.0212 2.7925055208629e-12
0.0214 1.59134957922416e-12
0.0216 9.05254517671245e-13
0.0218 5.14069999253187e-13
0.022 2.91429031467523e-13
0.0222 1.64935660252564e-13
0.0224 9.31921612189804e-14
0.0226 5.25701031751156e-14
0.0228 2.96076644544015e-14
0.023 1.66489289581266e-14
0.0232 9.34751020436902e-15
0.0234 5.24014720070934e-15
0.0236 2.93318222365883e-15
0.0238 1.63942853674557e-15
0.024 9.14983888264259e-16
};
\addlegendentry{monarch}
\addplot [semithick, color4, opacity=0.8]
table {%
0 0
0.0002 2.87061962533969e-17
0.0004 7.38215500264892e-12
0.0006 7.99925233557908e-09
0.0008 9.23345890210269e-07
0.001 3.12426163997331e-05
0.0012 0.000486425585682615
0.0014 0.00443227584242255
0.0016 0.0272853176447845
0.0018 0.124491716808887
0.002 0.448457426929982
0.0022 1.33431227744341
0.0024 3.39008775333182
0.0026 7.54334486722043
0.0028 14.9917839314935
0.003 27.0300203986886
0.0032 44.7711936223266
0.0034 68.829161876082
0.0036 99.0522136164206
0.0038 134.390978721141
0.004 172.945707968733
0.0042 212.187129650739
0.0044 249.299540863043
0.0046 281.568493665345
0.0048 306.733195142688
0.005 323.241784801446
0.0052 330.377055934632
0.0054 328.251027276394
0.0056 317.691331028443
0.0058 300.056657202816
0.006 277.022112528999
0.0062 250.370518701325
0.0064 221.815786056421
0.0066 192.872948348342
0.0068 164.778860334457
0.007 138.459505004863
0.0072 114.534883469691
0.0074 93.3503654120481
0.0076 75.0235227223848
0.0078 59.4970615764508
0.008 46.5907661257491
0.0082 36.0477888103356
0.0084 27.5727849718714
0.0086 20.8610939827748
0.0088 15.6193599142421
0.009 11.5787010474745
0.0092 8.50186909911387
0.0094 6.18589236804231
0.0096 4.4615729931021
0.0098 3.19099043689396
0.01 2.26391267995024
0.0102 1.59377392616427
0.0104 1.11366586893935
0.0106 0.772618649198693
0.0108 0.532318675188314
0.011 0.36431942334206
0.0112 0.247741687742021
0.0114 0.167424348101066
0.0116 0.112468906597157
0.0118 0.0751151495753036
0.012 0.0498868750040704
0.0122 0.0329524086247424
0.0124 0.0216523333629739
0.0126 0.0141550013499145
0.0128 0.00920810860869809
0.013 0.0059614341297812
0.0132 0.0038416022772897
0.0134 0.00246441669806522
0.0136 0.00157402680036694
0.0138 0.00100105980744145
0.014 0.000634029135601277
0.0142 0.000399952922025719
0.0144 0.000251307884880475
0.0146 0.000157306294866217
0.0148 9.81007108197785e-05
0.015 6.0957317239249e-05
0.0152 3.77439280400285e-05
0.0154 2.32902558254876e-05
0.0156 1.43233367813761e-05
0.0158 8.77995696507616e-06
0.016 5.36479250008616e-06
0.0162 3.26782412750546e-06
0.0164 1.98445502917811e-06
0.0166 1.20152046791494e-06
0.0168 7.2536700869825e-07
0.017 4.36666105161689e-07
0.0172 2.62140167381038e-07
0.0174 1.56941018917358e-07
0.0176 9.37095089143312e-08
0.0178 5.58084270995014e-08
0.018 3.31519158355106e-08
0.0182 1.96441489875536e-08
0.0184 1.16116999001352e-08
0.0186 6.84727729805362e-09
0.0188 4.02828695370443e-09
0.019 2.36441351651664e-09
0.0192 1.38467088704529e-09
0.0194 8.09112143045524e-10
0.0196 4.71767761043607e-10
0.0198 2.74487754760908e-10
0.02 1.59371301958963e-10
0.0202 9.23435223039749e-11
0.0204 5.3398390260612e-11
0.0206 3.08170508635796e-11
0.0208 1.77504953099345e-11
0.021 1.02047233929746e-11
0.0212 5.85568632532008e-12
0.0214 3.35393171322925e-12
0.0216 1.91753860960166e-12
0.0218 1.09436142350987e-12
0.022 6.23472201634812e-13
0.0222 3.54589300004174e-13
0.0224 2.01325406384701e-13
0.0226 1.14116345766126e-13
0.0228 6.45780629660316e-14
0.023 3.64855860479582e-14
0.0232 2.058106140172e-14
0.0234 1.1591377006404e-14
0.0236 6.51828782611222e-15
0.0238 3.65993547211678e-15
0.024 2.05193979417027e-15
};
\addlegendentry{zebra}
\addplot [semithick, color5, opacity=0.8]
table {%
0 0
0.0002 4.77632767978611e-17
0.0004 1.10350468092225e-11
0.0006 1.12309897522798e-08
0.0008 1.23997965836469e-06
0.001 4.05331104838323e-05
0.0012 0.000613522820835944
0.0014 0.00545862533187201
0.0016 0.032916327874361
0.0018 0.147470611530503
0.002 0.522639852402313
0.0022 1.53224950990862
0.0024 3.84087832479369
0.0026 8.44109535473997
0.0028 16.5844717211536
0.003 29.5836281445369
0.0032 48.5131391024827
0.0034 73.884182008032
0.0036 105.388702355709
0.0038 141.794147211553
0.004 181.026617480686
0.0042 220.426658502137
0.0044 257.117315904375
0.0046 288.401464242004
0.0048 312.107862563584
0.005 326.827472666344
0.0052 332.013328813822
0.0054 327.948504053069
0.0056 315.610146296069
0.0058 296.46985213264
0.006 272.272150752996
0.0062 244.826311575023
0.0064 215.835724402564
0.0066 186.77714737272
0.0068 158.83166939296
0.007 132.861649361001
0.0072 109.423525201146
0.0074 88.8048737038958
0.0076 71.0747137642668
0.0078 56.1379510814392
0.008 43.7873232069078
0.0082 33.7486675534372
0.0084 25.717456860267
0.0086 19.3861648116425
0.0088 14.4631132938788
0.009 10.6840718447148
0.0092 7.81813057951323
0.0094 5.66936123970293
0.0096 4.0756175265045
0.0098 2.90558575756373
0.01 2.05493739907559
0.0102 1.44219244318659
0.0104 1.00469625570293
0.0106 0.694949514002269
0.0108 0.477410379936765
0.011 0.325804876526534
0.0112 0.220928442520729
0.0114 0.148891477378268
0.0116 0.0997477329541815
0.0118 0.0664411871810735
0.012 0.0440103387810881
0.0122 0.0289956454569549
0.0124 0.0190040235452072
0.0126 0.0123926294820924
0.0128 0.00804180587557924
0.013 0.00519373618909181
0.0132 0.00333889993268949
0.0134 0.00213689123097192
0.0136 0.00136167233949938
0.0138 0.000864026867838253
0.014 0.000546005807160571
0.0142 0.000343661940044616
0.0144 0.000215464887035409
0.0146 0.000134578989767629
0.0148 8.37484307189143e-05
0.015 5.19297203253406e-05
0.0152 3.20874183650162e-05
0.0154 1.9759279261365e-05
0.0156 1.21272273451925e-05
0.0158 7.41892885450926e-06
0.016 4.52422298617316e-06
0.0162 2.75044196165776e-06
0.0164 1.66704773122399e-06
0.0166 1.00742074044527e-06
0.0168 6.07044031286042e-07
0.017 3.64757361091102e-07
0.0172 2.18569535391021e-07
0.0174 1.30618012066543e-07
0.0176 7.78519818383923e-08
0.0178 4.62821862125025e-08
0.018 2.74447485205084e-08
0.0182 1.6234116858454e-08
0.0184 9.57952945182513e-09
0.0186 5.63932522263679e-09
0.0188 3.31206028275151e-09
0.019 1.94078517597756e-09
0.0192 1.13470828859746e-09
0.0194 6.61968459383362e-10
0.0196 3.85349570575785e-10
0.0198 2.2384872308735e-10
0.02 1.29763824843081e-10
0.0202 7.50703658895768e-11
0.0204 4.3342645575275e-11
0.0206 2.49752548808235e-11
0.0208 1.43637307368144e-11
0.021 8.24521801416301e-12
0.0212 4.72420703343228e-12
0.0214 2.70185174122966e-12
0.0216 1.54245838297033e-12
0.0218 8.79018274445295e-13
0.022 5.00066381644288e-13
0.0222 2.83997926881723e-13
0.0224 1.61017304107681e-13
0.0226 9.11405390698865e-14
0.0228 5.15043337669998e-14
0.023 2.90589735679794e-14
0.0232 1.63693700702481e-14
0.0234 9.20681612959743e-15
0.0236 5.17038771890748e-15
0.0238 2.89923195175591e-15
0.024 1.6233001918256e-15
};
\addlegendentry{lab coat}
\addplot [semithick, color6, opacity=0.8]
table {%
0 0
0.0002 2.50033281924293e-17
0.0004 6.61920264526063e-12
0.0006 7.29531684613714e-09
0.0008 8.52296949347681e-07
0.001 2.91093689969302e-05
0.0012 0.000456687181093467
0.0014 0.00418826597701668
0.0016 0.025927869467974
0.0018 0.118883729553988
0.002 0.43015163399355
0.0022 1.28497176113775
0.0024 3.27666446986953
0.0026 7.31548767742108
0.0028 14.5842153965101
0.003 26.3714255968948
0.0032 43.7988227032249
0.0034 67.5058869162191
0.0036 97.3813437364043
0.0038 132.424374144004
0.004 170.782523087049
0.0042 209.96311643966
0.0044 247.169302110898
0.0046 279.684632551789
0.0048 305.226638812366
0.005 322.206786149517
0.0052 329.86291869807
0.0054 328.26096479538
0.0056 318.187486292588
0.0058 300.969431361826
0.006 278.261630974412
0.0062 251.83822176869
0.0064 223.41460037644
0.0066 194.515095466985
0.0068 166.39094249668
0.007 139.984977047662
0.0072 115.934331544271
0.0074 94.6001679291746
0.0076 76.1134941728492
0.0078 60.4276155973675
0.008 47.3700206604959
0.0082 36.6889069284383
0.0084 28.0917234489209
0.0086 21.2748316880067
0.0088 15.9446047501456
0.009 11.8310271298241
0.0092 8.6952111181369
0.0094 6.33231622002179
0.0096 4.57124530993385
0.0098 3.27227978320237
0.01 2.32356823135405
0.0102 1.6371408200762
0.0104 1.14490856272043
0.0106 0.794933744641845
0.0108 0.548126540735307
0.011 0.375429681218069
0.0112 0.255491533381875
0.0114 0.172791105980003
0.0116 0.116159569591466
0.0118 0.0776362247336516
0.012 0.0515979327701638
0.0122 0.0341065074305667
0.0124 0.0224261120651527
0.0126 0.0146707956910648
0.0128 0.00955001556189569
0.013 0.00618685377119825
0.0132 0.00398944660171198
0.0134 0.00256089269833464
0.0136 0.00163667442640372
0.0138 0.00104154779409292
0.014 0.000660075452761531
0.0142 0.000416634011714716
0.0144 0.000261944833104547
0.0146 0.000164060561318553
0.0148 0.000102372004077384
0.015 6.36476741482536e-05
0.0152 3.94319450280711e-05
0.0154 2.43453839750448e-05
0.0156 1.49804470564146e-05
0.0158 9.18772890022433e-06
0.016 5.61695593416862e-06
0.0162 3.42323116077753e-06
0.0164 2.07991462131355e-06
0.0166 1.25996770085529e-06
0.0168 7.6103984052187e-07
0.017 4.58371763162202e-07
0.0172 2.7530760687414e-07
0.0174 1.64905414482916e-07
0.0176 9.85129938774576e-08
0.0178 5.86973778185031e-08
0.018 3.48846457806647e-08
0.0182 2.06806141146334e-08
0.0184 1.2230053205374e-08
0.0186 7.21523772012225e-09
0.0188 4.24669638531252e-09
0.019 2.49373484918187e-09
0.0192 1.46105786890699e-09
0.0194 8.54125365511676e-10
0.0196 4.98231604938351e-10
0.0198 2.90010826031021e-10
0.02 1.68456481004786e-10
0.0202 9.76491778580082e-11
0.0204 5.64901972032708e-11
0.0206 3.26149731643592e-11
0.0208 1.87938492401451e-11
0.021 1.08089655494924e-11
0.0212 6.20492712167372e-12
0.0214 3.55539178611517e-12
0.0216 2.03352795779895e-12
0.0218 1.16101553384927e-12
0.022 6.61704434051589e-13
0.0222 3.76479012548669e-13
0.0224 2.13835816271175e-13
0.0226 1.21253697975333e-13
0.0228 6.86429595305323e-14
0.023 3.87966979691754e-14
0.0232 2.18928491194375e-14
0.0234 1.23347194344956e-14
0.0236 6.93882890962642e-15
0.0238 3.89747350253865e-15
0.024 2.18589940234101e-15
};
\addlegendentry{rapeseed}
\addplot [semithick, white!49.8039215686275!black, opacity=0.8]
table {%
0 0
0.0002 4.38630231113233e-32
0.0004 9.42164546775413e-24
0.0006 5.22996289695702e-19
0.0008 9.86216555458081e-16
0.001 2.9143627354464e-13
0.0012 2.66669889722348e-11
0.0014 1.08650456690335e-09
0.0016 2.44844210199571e-08
0.0018 3.50988320156314e-07
0.002 3.52140105748224e-06
0.0022 2.64692151139118e-05
0.0024 0.000156751672867648
0.0026 0.000759830128964595
0.0028 0.00310561691820713
0.003 0.0109579090659142
0.0032 0.034016993111393
0.0034 0.0943609589409247
0.0036 0.236919036551997
0.0038 0.544246235965627
0.004 1.15434405660579
0.0042 2.27822422319553
0.0044 4.21196189035521
0.0046 7.33697307600828
0.0048 12.1029173241866
0.005 18.9902597620454
0.0052 28.4537592035236
0.0054 40.8529986597383
0.0056 56.3802313963309
0.0058 74.9980614577203
0.006 96.399078434761
0.0062 119.996496092707
0.0064 144.949774971286
0.0066 170.223290391324
0.0068 194.670631562012
0.007 217.133180785212
0.0072 236.539881854436
0.0074 251.995687216163
0.0076 262.848713411787
0.0078 268.729942990138
0.008 269.563567784846
0.0082 265.550010653712
0.0084 257.126721925673
0.0086 244.913714220057
0.0088 229.651425311885
0.009 212.138042254452
0.0092 193.172172002542
0.0094 173.505046651131
0.0096 153.804630673516
0.0098 134.632315345131
0.01 116.431517634651
0.0102 99.5265342599624
0.0104 84.1294470340541
0.0106 70.3526878790396
0.0108 58.2249725528034
0.011 47.708610089379
0.0112 38.7166022410736
0.0114 31.128389387621
0.0116 24.8035206519357
0.0118 19.5928901237509
0.012 15.3474695532124
0.0122 11.9246760970423
0.0124 9.19264705566421
0.0126 7.03276338370732
0.0128 5.34078412532097
0.013 4.02693908502129
0.0132 3.01528987771138
0.0134 2.24262063508061
0.0136 1.65706714665881
0.0138 1.21664268008281
0.014 0.887773612098774
0.0142 0.643920058287472
0.0144 0.464326410982204
0.0146 0.332923736774929
0.0148 0.237389505687213
0.015 0.16835904727833
0.0152 0.118776329266274
0.0154 0.0833680759840717
0.0156 0.0582239674953547
0.0158 0.040465929398118
0.016 0.0279907474620875
0.0162 0.0192719805847814
0.0164 0.0132090893039687
0.0166 0.00901363845163145
0.0168 0.00612424454255283
0.017 0.00414355273124991
0.0172 0.00279191693605117
0.0174 0.00187361899036578
0.0176 0.00125241352004595
0.0178 0.000833948261027627
0.018 0.000553211657396421
0.0182 0.000365627762028711
0.0184 0.000240777698977602
0.0186 0.00015799929429328
0.0188 0.00010332069503795
0.019 6.73353725225924e-05
0.0192 4.37373559046175e-05
0.0194 2.83167695105991e-05
0.0196 1.82744596079048e-05
0.0198 1.17566049371012e-05
0.02 7.540183680422e-06
0.0202 4.82136313147342e-06
0.0204 3.07376190545244e-06
0.0206 1.95392053995944e-06
0.0208 1.23851994013429e-06
0.021 7.82854424753148e-07
0.0212 4.93472270100794e-07
0.0214 3.10219752930684e-07
0.0216 1.94500945874803e-07
0.0218 1.21629694569092e-07
0.022 7.58652067521529e-08
0.0222 4.72008455684243e-08
0.0224 2.92940539123387e-08
0.0226 1.81363424296575e-08
0.0228 1.12015459029968e-08
0.023 6.90210048232848e-09
0.0232 4.24303135636104e-09
0.0234 2.6024297581957e-09
0.0236 1.59259501455864e-09
0.0238 9.72457426693062e-10
0.024 5.92501857818914e-10
};
\addlegendentry{face powder}
\addplot [semithick, color7, opacity=0.8]
table {%
0 0
0.0002 4.28964223063707e-25
0.0004 4.23925669893049e-18
0.0006 3.8843035174309e-14
0.0008 2.03980960361959e-11
0.001 2.23578413477403e-09
0.0012 9.09612932925116e-08
0.0014 1.86740983355346e-06
0.0016 2.32374652780952e-05
0.0018 0.00019726676995222
0.002 0.00123850159771812
0.0022 0.00609150384571705
0.0024 0.0244907822733581
0.0026 0.0831291604329364
0.0028 0.244273624310176
0.003 0.633888506344497
0.0032 1.47614150045633
0.0034 3.12549923236712
0.0036 6.08284419664121
0.0038 10.9810360144135
0.004 18.5299350809066
0.0042 29.4207126030825
0.0044 44.2015281358291
0.0046 63.1472686753469
0.0048 86.1512421387252
0.005 112.664536610035
0.0052 141.699695822165
0.0054 171.902056209116
0.0056 201.678253673426
0.0058 229.360553795801
0.006 253.380099640291
0.0062 272.422559738058
0.0064 285.545107710075
0.0066 292.24226921071
0.0068 292.457685433134
0.007 286.54728048786
0.0072 275.205345355062
0.0074 259.368105858065
0.0076 240.109565154236
0.0078 218.542425587442
0.008 195.733553164238
0.0082 172.639607097943
0.0084 150.06481979406
0.0086 128.63994996229
0.0088 108.819371575935
0.009 90.8921276480365
0.0092 75.002450205011
0.0094 61.1755282281401
0.0096 49.344973528739
0.0098 39.3792886394928
0.01 31.1055204646969
0.0102 24.3290782826384
0.0104 18.8493422807296
0.0106 14.4711661403143
0.0108 11.012689711825
0.011 8.31004844305997
0.0112 6.21962606128075
0.0114 4.61847851618221
0.0116 3.40348972257237
0.0118 2.48972713377504
0.012 1.80836540381978
0.0122 1.3044512893037
0.0124 0.934699493011194
0.0126 0.665440554604601
0.0128 0.470788676221954
0.013 0.331058380959296
0.0132 0.231432089769658
0.0134 0.160863683425722
0.0136 0.111193545097079
0.0138 0.0764463638792013
0.014 0.0522823826082634
0.0142 0.035574433229481
0.0144 0.0240860094782703
0.0146 0.016229070006668
0.0148 0.0108837773668214
0.015 0.00726567536281934
0.0152 0.00482873788155366
0.0154 0.00319522579235741
0.0156 0.00210536227202998
0.0158 0.00138151231424654
0.016 0.000902877637792488
0.0162 0.00058774804174528
0.0164 0.000381137818224309
0.0166 0.000246229671424615
0.0168 0.000158490686887278
0.017 0.000101650220991295
0.0172 6.49664372059235e-05
0.0174 4.13789657654764e-05
0.0176 2.62671302556463e-05
0.0178 1.66195588353419e-05
0.018 1.04816733125501e-05
0.0182 6.58984806578497e-06
0.0184 4.13030631697093e-06
0.0186 2.58094470824131e-06
0.0188 1.60801959781534e-06
0.019 9.98955043649507e-07
0.0192 6.18824475032918e-07
0.0194 3.8227873195778e-07
0.0196 2.35508912963666e-07
0.0198 1.44701102682449e-07
0.02 8.8673868227445e-08
0.0202 5.42001101106119e-08
0.0204 3.3045028622791e-08
0.0206 2.00971134614997e-08
0.0208 1.21927699557356e-08
0.021 7.37957263668854e-09
0.0212 4.45593279488571e-09
0.0214 2.68437120910418e-09
0.0216 1.61346839341639e-09
0.0218 9.67629833967732e-10
0.022 5.79035802943952e-10
0.0222 3.45752238012749e-10
0.0224 2.06017255977852e-10
0.0226 1.22500087346044e-10
0.0228 7.26905882086746e-11
0.023 4.30470453490778e-11
0.0232 2.54416874400353e-11
0.0234 1.50071953893039e-11
0.0236 8.83522328295906e-12
0.0238 5.19173912534171e-12
0.024 3.04507630349425e-12
};
\addlegendentry{stethoscope}
\addplot [semithick, color8, opacity=0.8]
table {%
0 0
0.0002 3.00097034650853e-39
0.0004 1.33180595158039e-29
0.0006 4.34794883594433e-24
0.0008 2.88281813447308e-20
0.001 2.25959278886682e-17
0.0012 4.58849753621502e-15
0.0014 3.66862376248184e-13
0.0016 1.4826057132876e-11
0.0018 3.55812394866283e-10
0.002 5.6607416830461e-09
0.0022 6.45750577460221e-08
0.0024 5.59706827273519e-07
0.0026 3.85186816583851e-06
0.0028 2.17797182193631e-05
0.003 0.000103963098101098
0.0032 0.000428198943529838
0.0034 0.00154923143069868
0.0036 0.00499716303584501
0.0038 0.0145497409226953
0.004 0.0386429091930658
0.0042 0.0944621505149839
0.0044 0.214173948050834
0.0046 0.453416129407434
0.0048 0.901527460031158
0.005 1.69212633933122
0.0052 3.01173851254532
0.0054 5.10351436468107
0.0056 8.26297420559593
0.0058 12.8233899369747
0.006 19.1298809719443
0.0062 27.5033666828899
0.0064 38.197769617239
0.0066 51.3557906719054
0.0068 66.9696976206799
0.007 84.8535623562632
0.0072 104.632179248452
0.0074 125.749693106881
0.0076 147.49817102892
0.0078 169.06349078164
0.008 189.583502870042
0.0082 208.211850369454
0.0084 224.180309039073
0.0086 236.853041252113
0.0088 245.767559381408
0.009 250.659161932366
0.0092 251.46778261353
0.0094 248.328241752336
0.0096 241.546547045343
0.0098 231.565995646678
0.01 218.927330084995
0.0102 204.227139631271
0.0104 188.078188340518
0.0106 171.074540058163
0.0108 153.763395548578
0.011 136.624597564587
0.0112 120.05790451958
0.0114 104.37745251686
0.0116 89.8123511282774
0.0118 76.5120906958682
0.012 64.5553546633596
0.0122 53.9608918195513
0.0124 44.6992668179575
0.0126 36.7045303356969
0.0128 29.8850957453023
0.013 24.1333483356877
0.0132 19.3337260066615
0.0134 15.3691854660599
0.0136 12.1261008516253
0.0138 9.49773331026087
0.014 7.38646490920381
0.0142 5.70501483272526
0.0144 4.37685734578378
0.0146 3.33604646533651
0.0148 2.52662782719297
0.015 1.90178889191935
0.0152 1.42286819328888
0.0154 1.05831543316462
0.0156 0.782668499914721
0.0158 0.575591761439819
0.016 0.42100248768295
0.0162 0.306298810261757
0.0164 0.221692797162713
0.0166 0.159645451042525
0.0168 0.114396138404948
0.017 0.0815765573886167
0.0172 0.0578983497313499
0.0174 0.0409034327024832
0.0176 0.0287667280593347
0.0178 0.0201419352110787
0.018 0.0140421424372554
0.0182 0.00974825920262719
0.0184 0.00673939677729878
0.0186 0.00464037135271352
0.0188 0.00318242727831303
0.019 0.00217406945196777
0.0192 0.0014795564597677
0.0194 0.00100314992469322
0.0196 0.000677654421267221
0.0198 0.000456131239621635
0.02 0.00030594299538263
0.0202 0.000204498212267579
0.0204 0.000136227600576992
0.0206 9.04471207709321e-05
0.0208 5.98556420372414e-05
0.021 3.94840818321226e-05
0.0212 2.59639493292087e-05
0.0214 1.70206315355404e-05
0.0216 1.11239892354256e-05
0.0218 7.24850644094875e-06
0.022 4.70936173768033e-06
0.0222 3.05086497834185e-06
0.0224 1.97084564601369e-06
0.0226 1.26961378811133e-06
0.0228 8.15643160454774e-07
0.023 5.22586210984816e-07
0.0232 3.33936576376143e-07
0.0234 2.12832014598657e-07
0.0236 1.35299126133894e-07
0.0238 8.57937766006982e-08
0.024 5.42670960075561e-08
};
\addlegendentry{notebook}

\nextgroupplot[
height=\figheight,
legend cell align={left},
legend style={fill opacity=0.8, draw opacity=1, text opacity=1, draw=white!80!black},
tick align=outside,
tick pos=both,
width=\figwidth,
x grid style={white!69.0196078431373!black},
xmin=-0.00395, xmax=0.08295,
xtick style={color=black},
xtick={-0.01,0,0.01,0.02,0.03,0.04,0.05,0.06,0.07,0.08,0.09},
xticklabels={−0.01,0.00,0.01,0.02,0.03,0.04,0.05,0.06,0.07,0.08,0.09},
y grid style={white!69.0196078431373!black},
ymin=-19.2883988333006, ymax=405.056375499312,
ytick style={color=black}
]
\path [draw=none, fill=blue, fill opacity=0.5]
(axis cs:0,0)
--(axis cs:0.001,7.5306614288899e-12)
--(axis cs:0.002,0.000274798181256152)
--(axis cs:0.003,0.726067519098887)
--(axis cs:0.004,38.5850050992666)
--(axis cs:0.005,240.526523527352)
--(axis cs:0.006,327.748216254779)
--(axis cs:0.007,120.464572363519)
--(axis cs:0.008,23.8796063318801)
--(axis cs:0.009,2.94990766857048)
--(axis cs:0.01,0.250643700619543)
--(axis cs:0.011,0.0157158047851612)
--(axis cs:0.012,0.000765974288866585)
--(axis cs:0.013,3.01873758278666e-05)
--(axis cs:0.014,9.91948194179301e-07)
--(axis cs:0.015,2.78459696414973e-08)
--(axis cs:0.016,6.81012680057153e-10)
--(axis cs:0.017,1.47443986465567e-11)
--(axis cs:0.018,2.86379607366953e-13)
--(axis cs:0.019,5.04577475923118e-15)
--(axis cs:0.02,8.14081250518803e-17)
--(axis cs:0.021,1.21240104315078e-18)
--(axis cs:0.022,1.67826570738711e-20)
--(axis cs:0.023,2.17223903510431e-22)
--(axis cs:0.024,2.64272039884326e-24)
--(axis cs:0.025,3.03583177227021e-26)
--(axis cs:0.026,3.30627384009937e-28)
--(axis cs:0.027,3.4259996322124e-30)
--(axis cs:0.028,3.38848729028021e-32)
--(axis cs:0.029,3.20797503156245e-34)
--(axis cs:0.03,2.91455887514182e-36)
--(axis cs:0.031,2.54701683678907e-38)
--(axis cs:0.032,2.14542085003383e-40)
--(axis cs:0.033,1.74516065791736e-42)
--(axis cs:0.034,1.37324210175997e-44)
--(axis cs:0.035,1.04695668328753e-46)
--(axis cs:0.036,7.74467701834709e-49)
--(axis cs:0.037,5.56598850592264e-51)
--(axis cs:0.038,3.89108690866414e-53)
--(axis cs:0.039,2.64894566260278e-55)
--(axis cs:0.04,1.7579047473431e-57)
--(axis cs:0.041,1.1382848720731e-59)
--(axis cs:0.042,7.19818173995335e-62)
--(axis cs:0.043,4.44904007966932e-64)
--(axis cs:0.044,2.68975072054131e-66)
--(axis cs:0.045,1.59172417834704e-68)
--(axis cs:0.046,9.22615701160169e-71)
--(axis cs:0.047,5.24130452219474e-73)
--(axis cs:0.048,2.91994339665418e-75)
--(axis cs:0.049,1.59610345182032e-77)
--(axis cs:0.05,8.56484836365445e-80)
--(axis cs:0.051,4.51395662597842e-82)
--(axis cs:0.052,2.33759228681477e-84)
--(axis cs:0.053,1.1899726713028e-86)
--(axis cs:0.054,5.95709600611141e-89)
--(axis cs:0.055,2.93375825045685e-91)
--(axis cs:0.056,1.42186957992117e-93)
--(axis cs:0.057,6.78400314278218e-96)
--(axis cs:0.058,3.18742806762942e-98)
--(axis cs:0.059,1.47520611752496e-100)
--(axis cs:0.06,6.72738069349805e-103)
--(axis cs:0.061,3.02368756955887e-105)
--(axis cs:0.062,1.33978794899461e-107)
--(axis cs:0.063,5.85394296544167e-110)
--(axis cs:0.064,2.5227484169707e-112)
--(axis cs:0.065,1.07252462238867e-114)
--(axis cs:0.066,4.49923502697876e-117)
--(axis cs:0.067,1.86274763805703e-119)
--(axis cs:0.068,7.61263576576289e-122)
--(axis cs:0.069,3.07156468604694e-124)
--(axis cs:0.07,1.223776543627e-126)
--(axis cs:0.071,4.81540981868275e-129)
--(axis cs:0.072,1.87163275890967e-131)
--(axis cs:0.073,7.18668810287969e-134)
--(axis cs:0.074,2.72658696141498e-136)
--(axis cs:0.075,1.02223603237452e-138)
--(axis cs:0.076,3.78774357598603e-141)
--(axis cs:0.077,1.38727104736618e-143)
--(axis cs:0.078,5.02278879483122e-146)
--(axis cs:0.079,1.79796240717077e-148)
--cycle;
\path [draw=none, fill=blue, fill opacity=0.5]
(axis cs:0,0)
--(axis cs:0.001,1.54507617310345e-12)
--(axis cs:0.002,0.000103297845546622)
--(axis cs:0.003,0.389075904777595)
--(axis cs:0.004,26.5972695349157)
--(axis cs:0.005,201.601714961611)
--(axis cs:0.006,327.748216254779)
--(axis cs:0.007,120.464572363519)
--(axis cs:0.008,23.8796063318801)
--(axis cs:0.009,2.94990766857048)
--(axis cs:0.01,0.250643700619543)
--(axis cs:0.011,0.0157158047851612)
--(axis cs:0.012,0.000765974288866585)
--(axis cs:0.013,3.01873758278666e-05)
--(axis cs:0.014,9.91948194179301e-07)
--(axis cs:0.015,2.78459696414973e-08)
--(axis cs:0.016,6.81012680057153e-10)
--(axis cs:0.017,1.47443986465567e-11)
--(axis cs:0.018,2.86379607366953e-13)
--(axis cs:0.019,5.04577475923118e-15)
--(axis cs:0.02,8.14081250518803e-17)
--(axis cs:0.021,1.21240104315078e-18)
--(axis cs:0.022,1.67826570738711e-20)
--(axis cs:0.023,2.17223903510431e-22)
--(axis cs:0.024,2.64272039884326e-24)
--(axis cs:0.025,3.03583177227021e-26)
--(axis cs:0.026,3.30627384009937e-28)
--(axis cs:0.027,3.4259996322124e-30)
--(axis cs:0.028,3.38848729028021e-32)
--(axis cs:0.029,3.20797503156245e-34)
--(axis cs:0.03,2.91455887514182e-36)
--(axis cs:0.031,2.54701683678907e-38)
--(axis cs:0.032,2.14542085003383e-40)
--(axis cs:0.033,1.74516065791736e-42)
--(axis cs:0.034,1.37324210175997e-44)
--(axis cs:0.035,1.04695668328753e-46)
--(axis cs:0.036,7.74467701834709e-49)
--(axis cs:0.037,5.56598850592264e-51)
--(axis cs:0.038,3.89108690866414e-53)
--(axis cs:0.039,2.64894566260278e-55)
--(axis cs:0.04,1.7579047473431e-57)
--(axis cs:0.041,1.1382848720731e-59)
--(axis cs:0.042,7.19818173995335e-62)
--(axis cs:0.043,4.44904007966932e-64)
--(axis cs:0.044,2.68975072054131e-66)
--(axis cs:0.045,1.59172417834704e-68)
--(axis cs:0.046,9.22615701160169e-71)
--(axis cs:0.047,5.24130452219474e-73)
--(axis cs:0.048,2.91994339665418e-75)
--(axis cs:0.049,1.59610345182032e-77)
--(axis cs:0.05,8.56484836365445e-80)
--(axis cs:0.051,4.51395662597842e-82)
--(axis cs:0.052,2.33759228681477e-84)
--(axis cs:0.053,1.1899726713028e-86)
--(axis cs:0.054,5.95709600611141e-89)
--(axis cs:0.055,2.93375825045685e-91)
--(axis cs:0.056,1.42186957992117e-93)
--(axis cs:0.057,6.78400314278218e-96)
--(axis cs:0.058,3.18742806762942e-98)
--(axis cs:0.059,1.47520611752496e-100)
--(axis cs:0.06,6.72738069349805e-103)
--(axis cs:0.061,3.02368756955887e-105)
--(axis cs:0.062,1.33978794899461e-107)
--(axis cs:0.063,5.85394296544167e-110)
--(axis cs:0.064,2.5227484169707e-112)
--(axis cs:0.065,1.07252462238867e-114)
--(axis cs:0.066,4.49923502697876e-117)
--(axis cs:0.067,1.86274763805703e-119)
--(axis cs:0.068,7.61263576576289e-122)
--(axis cs:0.069,3.07156468604694e-124)
--(axis cs:0.07,1.223776543627e-126)
--(axis cs:0.071,4.81540981868275e-129)
--(axis cs:0.072,1.87163275890967e-131)
--(axis cs:0.073,7.18668810287969e-134)
--(axis cs:0.074,2.72658696141498e-136)
--(axis cs:0.075,1.02223603237452e-138)
--(axis cs:0.076,3.78774357598603e-141)
--(axis cs:0.077,1.38727104736618e-143)
--(axis cs:0.078,5.02278879483122e-146)
--(axis cs:0.079,1.79796240717077e-148)
--cycle;
\path [draw=none, fill=blue, fill opacity=0.5]
(axis cs:0,0)
--(axis cs:0.001,0)
--(axis cs:0.002,2.03014981812474e-298)
--(axis cs:0.003,9.31724947218955e-249)
--(axis cs:0.004,3.43265608779348e-214)
--(axis cs:0.005,6.7527672970312e-188)
--(axis cs:0.006,7.76162738981217e-167)
--(axis cs:0.007,2.17868917797679e-149)
--(axis cs:0.008,1.38640655839917e-134)
--(axis cs:0.009,8.41804844498113e-122)
--(axis cs:0.01,1.30117864576101e-110)
--(axis cs:0.011,1.03096053371918e-100)
--(axis cs:0.012,7.02052109385449e-92)
--(axis cs:0.013,6.08361715932729e-84)
--(axis cs:0.014,9.10136664373718e-77)
--(axis cs:0.015,2.9938465544224e-70)
--(axis cs:0.016,2.63145915791406e-64)
--(axis cs:0.017,7.24866962376275e-59)
--(axis cs:0.018,7.14131860903066e-54)
--(axis cs:0.019,2.81065697786411e-49)
--(axis cs:0.02,4.8528543899075e-45)
--(axis cs:0.021,3.9814621023952e-41)
--(axis cs:0.022,1.66262383997763e-37)
--(axis cs:0.023,3.75075058539897e-34)
--(axis cs:0.024,4.81478504563591e-31)
--(axis cs:0.025,3.68101006289339e-28)
--(axis cs:0.026,3.30627384009937e-28)
--(axis cs:0.027,3.4259996322124e-30)
--(axis cs:0.028,3.38848729028021e-32)
--(axis cs:0.029,3.20797503156245e-34)
--(axis cs:0.03,2.91455887514182e-36)
--(axis cs:0.031,2.54701683678907e-38)
--(axis cs:0.032,2.14542085003383e-40)
--(axis cs:0.033,1.74516065791736e-42)
--(axis cs:0.034,1.37324210175997e-44)
--(axis cs:0.035,1.04695668328753e-46)
--(axis cs:0.036,7.74467701834709e-49)
--(axis cs:0.037,5.56598850592264e-51)
--(axis cs:0.038,3.89108690866414e-53)
--(axis cs:0.039,2.64894566260278e-55)
--(axis cs:0.04,1.7579047473431e-57)
--(axis cs:0.041,1.1382848720731e-59)
--(axis cs:0.042,7.19818173995335e-62)
--(axis cs:0.043,4.44904007966932e-64)
--(axis cs:0.044,2.68975072054131e-66)
--(axis cs:0.045,1.59172417834704e-68)
--(axis cs:0.046,9.22615701160169e-71)
--(axis cs:0.047,5.24130452219474e-73)
--(axis cs:0.048,2.91994339665418e-75)
--(axis cs:0.049,1.59610345182032e-77)
--(axis cs:0.05,8.56484836365445e-80)
--(axis cs:0.051,4.51395662597842e-82)
--(axis cs:0.052,2.33759228681477e-84)
--(axis cs:0.053,1.1899726713028e-86)
--(axis cs:0.054,5.95709600611141e-89)
--(axis cs:0.055,2.93375825045685e-91)
--(axis cs:0.056,1.42186957992117e-93)
--(axis cs:0.057,6.78400314278218e-96)
--(axis cs:0.058,3.18742806762942e-98)
--(axis cs:0.059,1.47520611752496e-100)
--(axis cs:0.06,6.72738069349805e-103)
--(axis cs:0.061,3.02368756955887e-105)
--(axis cs:0.062,1.33978794899461e-107)
--(axis cs:0.063,5.85394296544167e-110)
--(axis cs:0.064,2.5227484169707e-112)
--(axis cs:0.065,1.07252462238867e-114)
--(axis cs:0.066,4.49923502697876e-117)
--(axis cs:0.067,1.86274763805703e-119)
--(axis cs:0.068,7.61263576576289e-122)
--(axis cs:0.069,3.07156468604694e-124)
--(axis cs:0.07,1.223776543627e-126)
--(axis cs:0.071,4.81540981868275e-129)
--(axis cs:0.072,1.87163275890967e-131)
--(axis cs:0.073,7.18668810287969e-134)
--(axis cs:0.074,2.72658696141498e-136)
--(axis cs:0.075,1.02223603237452e-138)
--(axis cs:0.076,3.78774357598603e-141)
--(axis cs:0.077,1.38727104736618e-143)
--(axis cs:0.078,5.02278879483122e-146)
--(axis cs:0.079,1.79796240717077e-148)
--cycle;
\path [draw=none, fill=blue, fill opacity=0.5]
(axis cs:0,0)
--(axis cs:0.001,0)
--(axis cs:0.002,1.80855680220225e-315)
--(axis cs:0.003,9.70145177563557e-264)
--(axis cs:0.004,1.05136674560834e-227)
--(axis cs:0.005,2.85674390501788e-200)
--(axis cs:0.006,2.81155839455437e-178)
--(axis cs:0.007,4.8583770711663e-160)
--(axis cs:0.008,1.4948608620893e-144)
--(axis cs:0.009,3.64944965351347e-131)
--(axis cs:0.01,1.96096183052097e-119)
--(axis cs:0.011,4.80140730189426e-109)
--(axis cs:0.012,9.16808095919789e-100)
--(axis cs:0.013,2.05309186557918e-91)
--(axis cs:0.014,7.40465975766897e-84)
--(axis cs:0.015,5.53049730591008e-77)
--(axis cs:0.016,1.04764202141619e-70)
--(axis cs:0.017,5.94088345168626e-65)
--(axis cs:0.018,1.15696407390828e-59)
--(axis cs:0.019,8.68116657991054e-55)
--(axis cs:0.02,2.76624283557957e-50)
--(axis cs:0.021,4.067523087894e-46)
--(axis cs:0.022,2.96437169477005e-42)
--(axis cs:0.023,1.1391789303744e-38)
--(axis cs:0.024,2.4364950354929e-35)
--(axis cs:0.025,3.04113977224369e-32)
--(axis cs:0.026,2.30974489923338e-29)
--(axis cs:0.027,3.4259996322124e-30)
--(axis cs:0.028,3.38848729028021e-32)
--(axis cs:0.029,3.20797503156245e-34)
--(axis cs:0.03,2.91455887514182e-36)
--(axis cs:0.031,2.54701683678907e-38)
--(axis cs:0.032,2.14542085003383e-40)
--(axis cs:0.033,1.74516065791736e-42)
--(axis cs:0.034,1.37324210175997e-44)
--(axis cs:0.035,1.04695668328753e-46)
--(axis cs:0.036,7.74467701834709e-49)
--(axis cs:0.037,5.56598850592264e-51)
--(axis cs:0.038,3.89108690866414e-53)
--(axis cs:0.039,2.64894566260278e-55)
--(axis cs:0.04,1.7579047473431e-57)
--(axis cs:0.041,1.1382848720731e-59)
--(axis cs:0.042,7.19818173995335e-62)
--(axis cs:0.043,4.44904007966932e-64)
--(axis cs:0.044,2.68975072054131e-66)
--(axis cs:0.045,1.59172417834704e-68)
--(axis cs:0.046,9.22615701160169e-71)
--(axis cs:0.047,5.24130452219474e-73)
--(axis cs:0.048,2.91994339665418e-75)
--(axis cs:0.049,1.59610345182032e-77)
--(axis cs:0.05,8.56484836365445e-80)
--(axis cs:0.051,4.51395662597842e-82)
--(axis cs:0.052,2.33759228681477e-84)
--(axis cs:0.053,1.1899726713028e-86)
--(axis cs:0.054,5.95709600611141e-89)
--(axis cs:0.055,2.93375825045685e-91)
--(axis cs:0.056,1.42186957992117e-93)
--(axis cs:0.057,6.78400314278218e-96)
--(axis cs:0.058,3.18742806762942e-98)
--(axis cs:0.059,1.47520611752496e-100)
--(axis cs:0.06,6.72738069349805e-103)
--(axis cs:0.061,3.02368756955887e-105)
--(axis cs:0.062,1.33978794899461e-107)
--(axis cs:0.063,5.85394296544167e-110)
--(axis cs:0.064,2.5227484169707e-112)
--(axis cs:0.065,1.07252462238867e-114)
--(axis cs:0.066,4.49923502697876e-117)
--(axis cs:0.067,1.86274763805703e-119)
--(axis cs:0.068,7.61263576576289e-122)
--(axis cs:0.069,3.07156468604694e-124)
--(axis cs:0.07,1.223776543627e-126)
--(axis cs:0.071,4.81540981868275e-129)
--(axis cs:0.072,1.87163275890967e-131)
--(axis cs:0.073,7.18668810287969e-134)
--(axis cs:0.074,2.72658696141498e-136)
--(axis cs:0.075,1.02223603237452e-138)
--(axis cs:0.076,3.78774357598603e-141)
--(axis cs:0.077,1.38727104736618e-143)
--(axis cs:0.078,5.02278879483122e-146)
--(axis cs:0.079,1.79796240717077e-148)
--cycle;
\path [draw=none, fill=blue, fill opacity=0.5]
(axis cs:0,0)
--(axis cs:0.001,1.54507617310345e-12)
--(axis cs:0.002,0.000103297845546622)
--(axis cs:0.003,0.389075904777595)
--(axis cs:0.004,26.5972695349157)
--(axis cs:0.005,201.601714961611)
--(axis cs:0.006,379.407730891438)
--(axis cs:0.007,242.029049094857)
--(axis cs:0.008,76.2778530918751)
--(axis cs:0.009,14.1897249207388)
--(axis cs:0.01,1.73950412373907)
--(axis cs:0.011,0.152008438140823)
--(axis cs:0.012,0.0100343318310296)
--(axis cs:0.013,0.000522894914383514)
--(axis cs:0.014,2.22593210062889e-05)
--(axis cs:0.015,7.95362205700697e-07)
--(axis cs:0.016,2.43821639111533e-08)
--(axis cs:0.017,6.52833161576269e-10)
--(axis cs:0.018,1.54949091871247e-11)
--(axis cs:0.019,3.30081818819095e-13)
--(axis cs:0.02,6.37761267195656e-15)
--(axis cs:0.021,1.12768675745017e-16)
--(axis cs:0.022,1.83890182477974e-18)
--(axis cs:0.023,2.78398641404362e-20)
--(axis cs:0.024,3.93588135036249e-22)
--(axis cs:0.025,5.22276708280347e-24)
--(axis cs:0.026,6.53429752148489e-26)
--(axis cs:0.027,7.73874298767626e-28)
--(axis cs:0.028,8.70681062016143e-30)
--(axis cs:0.029,9.33567371416515e-32)
--(axis cs:0.03,9.56686925278931e-34)
--(axis cs:0.031,9.39396244637523e-36)
--(axis cs:0.032,8.85916625253614e-38)
--(axis cs:0.033,8.04115421740054e-40)
--(axis cs:0.034,7.03816279242595e-42)
--(axis cs:0.035,5.9508151634981e-44)
--(axis cs:0.036,4.86817215486112e-46)
--(axis cs:0.037,3.85892291489979e-48)
--(axis cs:0.038,2.96799969450108e-50)
--(axis cs:0.039,2.21768501820965e-52)
--(axis cs:0.04,1.61165961489089e-54)
--(axis cs:0.041,1.14037062157614e-56)
--(axis cs:0.042,7.86404411613883e-59)
--(axis cs:0.043,5.29018007480495e-61)
--(axis cs:0.044,3.47447450623553e-63)
--(axis cs:0.045,2.22969892206919e-65)
--(axis cs:0.046,1.39914562167862e-67)
--(axis cs:0.047,8.59091243668763e-70)
--(axis cs:0.048,5.16483090466877e-72)
--(axis cs:0.049,3.04212035615481e-74)
--(axis cs:0.05,1.75649483966228e-76)
--(axis cs:0.051,9.94716423556478e-79)
--(axis cs:0.052,5.52778956810444e-81)
--(axis cs:0.053,3.01584215061926e-83)
--(axis cs:0.054,1.61608372204751e-85)
--(axis cs:0.055,8.50941165787103e-88)
--(axis cs:0.056,4.40441006928744e-90)
--(axis cs:0.057,2.24177045413869e-92)
--(axis cs:0.058,1.12244109840084e-94)
--(axis cs:0.059,5.53032489304425e-97)
--(axis cs:0.06,2.68219524276422e-99)
--(axis cs:0.061,1.28089529088182e-101)
--(axis cs:0.062,6.02482626864346e-104)
--(axis cs:0.063,2.79190725558905e-106)
--(axis cs:0.064,1.27495555216651e-108)
--(axis cs:0.065,5.73896050562254e-111)
--(axis cs:0.066,2.54692954451068e-113)
--(axis cs:0.067,1.11466057994533e-115)
--(axis cs:0.068,4.81174176262973e-118)
--(axis cs:0.069,2.04919635644202e-120)
--(axis cs:0.07,8.61132562339268e-123)
--(axis cs:0.071,3.57142365684475e-125)
--(axis cs:0.072,1.46208892184532e-127)
--(axis cs:0.073,5.90935354694027e-130)
--(axis cs:0.074,2.3583562855075e-132)
--(axis cs:0.075,9.29498163467781e-135)
--(axis cs:0.076,3.61843567662549e-137)
--(axis cs:0.077,1.39151263131687e-139)
--(axis cs:0.078,5.28695472209426e-142)
--(axis cs:0.079,1.98486905737179e-144)
--cycle;
\path [draw=none, fill=blue, fill opacity=0.5]
(axis cs:0,0)
--(axis cs:0.001,0)
--(axis cs:0.002,2.03014981812474e-298)
--(axis cs:0.003,9.31724947218955e-249)
--(axis cs:0.004,3.43265608779348e-214)
--(axis cs:0.005,6.7527672970312e-188)
--(axis cs:0.006,7.76162738981217e-167)
--(axis cs:0.007,2.17868917797679e-149)
--(axis cs:0.008,1.38640655839917e-134)
--(axis cs:0.009,8.41804844498113e-122)
--(axis cs:0.01,1.30117864576101e-110)
--(axis cs:0.011,1.03096053371918e-100)
--(axis cs:0.012,7.02052109385449e-92)
--(axis cs:0.013,6.08361715932729e-84)
--(axis cs:0.014,9.10136664373718e-77)
--(axis cs:0.015,2.9938465544224e-70)
--(axis cs:0.016,2.63145915791406e-64)
--(axis cs:0.017,7.24866962376275e-59)
--(axis cs:0.018,7.14131860903066e-54)
--(axis cs:0.019,2.81065697786411e-49)
--(axis cs:0.02,4.8528543899075e-45)
--(axis cs:0.021,3.9814621023952e-41)
--(axis cs:0.022,1.66262383997763e-37)
--(axis cs:0.023,3.75075058539897e-34)
--(axis cs:0.024,4.81478504563591e-31)
--(axis cs:0.025,3.68101006289339e-28)
--(axis cs:0.026,6.53429752148489e-26)
--(axis cs:0.027,7.73874298767626e-28)
--(axis cs:0.028,8.70681062016143e-30)
--(axis cs:0.029,9.33567371416515e-32)
--(axis cs:0.03,9.56686925278931e-34)
--(axis cs:0.031,9.39396244637523e-36)
--(axis cs:0.032,8.85916625253614e-38)
--(axis cs:0.033,8.04115421740054e-40)
--(axis cs:0.034,7.03816279242595e-42)
--(axis cs:0.035,5.9508151634981e-44)
--(axis cs:0.036,4.86817215486112e-46)
--(axis cs:0.037,3.85892291489979e-48)
--(axis cs:0.038,2.96799969450108e-50)
--(axis cs:0.039,2.21768501820965e-52)
--(axis cs:0.04,1.61165961489089e-54)
--(axis cs:0.041,1.14037062157614e-56)
--(axis cs:0.042,7.86404411613883e-59)
--(axis cs:0.043,5.29018007480495e-61)
--(axis cs:0.044,3.47447450623553e-63)
--(axis cs:0.045,2.22969892206919e-65)
--(axis cs:0.046,1.39914562167862e-67)
--(axis cs:0.047,8.59091243668763e-70)
--(axis cs:0.048,5.16483090466877e-72)
--(axis cs:0.049,3.04212035615481e-74)
--(axis cs:0.05,1.75649483966228e-76)
--(axis cs:0.051,9.94716423556478e-79)
--(axis cs:0.052,5.52778956810444e-81)
--(axis cs:0.053,3.01584215061926e-83)
--(axis cs:0.054,1.61608372204751e-85)
--(axis cs:0.055,8.50941165787103e-88)
--(axis cs:0.056,4.40441006928744e-90)
--(axis cs:0.057,2.24177045413869e-92)
--(axis cs:0.058,1.12244109840084e-94)
--(axis cs:0.059,5.53032489304425e-97)
--(axis cs:0.06,2.68219524276422e-99)
--(axis cs:0.061,1.28089529088182e-101)
--(axis cs:0.062,6.02482626864346e-104)
--(axis cs:0.063,2.79190725558905e-106)
--(axis cs:0.064,1.27495555216651e-108)
--(axis cs:0.065,5.73896050562254e-111)
--(axis cs:0.066,2.54692954451068e-113)
--(axis cs:0.067,1.11466057994533e-115)
--(axis cs:0.068,4.81174176262973e-118)
--(axis cs:0.069,2.04919635644202e-120)
--(axis cs:0.07,8.61132562339268e-123)
--(axis cs:0.071,3.57142365684475e-125)
--(axis cs:0.072,1.46208892184532e-127)
--(axis cs:0.073,5.90935354694027e-130)
--(axis cs:0.074,2.3583562855075e-132)
--(axis cs:0.075,9.29498163467781e-135)
--(axis cs:0.076,3.61843567662549e-137)
--(axis cs:0.077,1.39151263131687e-139)
--(axis cs:0.078,5.28695472209426e-142)
--(axis cs:0.079,1.98486905737179e-144)
--cycle;
\path [draw=none, fill=blue, fill opacity=0.5]
(axis cs:0,0)
--(axis cs:0.001,0)
--(axis cs:0.002,1.80855680220225e-315)
--(axis cs:0.003,9.70145177563557e-264)
--(axis cs:0.004,1.05136674560834e-227)
--(axis cs:0.005,2.85674390501788e-200)
--(axis cs:0.006,2.81155839455437e-178)
--(axis cs:0.007,4.8583770711663e-160)
--(axis cs:0.008,1.4948608620893e-144)
--(axis cs:0.009,3.64944965351347e-131)
--(axis cs:0.01,1.96096183052097e-119)
--(axis cs:0.011,4.80140730189426e-109)
--(axis cs:0.012,9.16808095919789e-100)
--(axis cs:0.013,2.05309186557918e-91)
--(axis cs:0.014,7.40465975766897e-84)
--(axis cs:0.015,5.53049730591008e-77)
--(axis cs:0.016,1.04764202141619e-70)
--(axis cs:0.017,5.94088345168626e-65)
--(axis cs:0.018,1.15696407390828e-59)
--(axis cs:0.019,8.68116657991054e-55)
--(axis cs:0.02,2.76624283557957e-50)
--(axis cs:0.021,4.067523087894e-46)
--(axis cs:0.022,2.96437169477005e-42)
--(axis cs:0.023,1.1391789303744e-38)
--(axis cs:0.024,2.4364950354929e-35)
--(axis cs:0.025,3.04113977224369e-32)
--(axis cs:0.026,2.30974489923338e-29)
--(axis cs:0.027,7.73874298767626e-28)
--(axis cs:0.028,8.70681062016143e-30)
--(axis cs:0.029,9.33567371416515e-32)
--(axis cs:0.03,9.56686925278931e-34)
--(axis cs:0.031,9.39396244637523e-36)
--(axis cs:0.032,8.85916625253614e-38)
--(axis cs:0.033,8.04115421740054e-40)
--(axis cs:0.034,7.03816279242595e-42)
--(axis cs:0.035,5.9508151634981e-44)
--(axis cs:0.036,4.86817215486112e-46)
--(axis cs:0.037,3.85892291489979e-48)
--(axis cs:0.038,2.96799969450108e-50)
--(axis cs:0.039,2.21768501820965e-52)
--(axis cs:0.04,1.61165961489089e-54)
--(axis cs:0.041,1.14037062157614e-56)
--(axis cs:0.042,7.86404411613883e-59)
--(axis cs:0.043,5.29018007480495e-61)
--(axis cs:0.044,3.47447450623553e-63)
--(axis cs:0.045,2.22969892206919e-65)
--(axis cs:0.046,1.39914562167862e-67)
--(axis cs:0.047,8.59091243668763e-70)
--(axis cs:0.048,5.16483090466877e-72)
--(axis cs:0.049,3.04212035615481e-74)
--(axis cs:0.05,1.75649483966228e-76)
--(axis cs:0.051,9.94716423556478e-79)
--(axis cs:0.052,5.52778956810444e-81)
--(axis cs:0.053,3.01584215061926e-83)
--(axis cs:0.054,1.61608372204751e-85)
--(axis cs:0.055,8.50941165787103e-88)
--(axis cs:0.056,4.40441006928744e-90)
--(axis cs:0.057,2.24177045413869e-92)
--(axis cs:0.058,1.12244109840084e-94)
--(axis cs:0.059,5.53032489304425e-97)
--(axis cs:0.06,2.68219524276422e-99)
--(axis cs:0.061,1.28089529088182e-101)
--(axis cs:0.062,6.02482626864346e-104)
--(axis cs:0.063,2.79190725558905e-106)
--(axis cs:0.064,1.27495555216651e-108)
--(axis cs:0.065,5.73896050562254e-111)
--(axis cs:0.066,2.54692954451068e-113)
--(axis cs:0.067,1.11466057994533e-115)
--(axis cs:0.068,4.81174176262973e-118)
--(axis cs:0.069,2.04919635644202e-120)
--(axis cs:0.07,8.61132562339268e-123)
--(axis cs:0.071,3.57142365684475e-125)
--(axis cs:0.072,1.46208892184532e-127)
--(axis cs:0.073,5.90935354694027e-130)
--(axis cs:0.074,2.3583562855075e-132)
--(axis cs:0.075,9.29498163467781e-135)
--(axis cs:0.076,3.61843567662549e-137)
--(axis cs:0.077,1.39151263131687e-139)
--(axis cs:0.078,5.28695472209426e-142)
--(axis cs:0.079,1.98486905737179e-144)
--cycle;
\path [draw=none, fill=blue, fill opacity=0.5]
(axis cs:0,0)
--(axis cs:0.001,0)
--(axis cs:0.002,2.03014981812474e-298)
--(axis cs:0.003,9.31724947218955e-249)
--(axis cs:0.004,3.43265608779348e-214)
--(axis cs:0.005,6.7527672970312e-188)
--(axis cs:0.006,7.76162738981217e-167)
--(axis cs:0.007,2.17868917797679e-149)
--(axis cs:0.008,1.38640655839917e-134)
--(axis cs:0.009,8.41804844498113e-122)
--(axis cs:0.01,1.30117864576101e-110)
--(axis cs:0.011,1.03096053371918e-100)
--(axis cs:0.012,7.02052109385449e-92)
--(axis cs:0.013,6.08361715932729e-84)
--(axis cs:0.014,9.10136664373718e-77)
--(axis cs:0.015,2.9938465544224e-70)
--(axis cs:0.016,2.63145915791406e-64)
--(axis cs:0.017,7.24866962376275e-59)
--(axis cs:0.018,7.14131860903066e-54)
--(axis cs:0.019,2.81065697786411e-49)
--(axis cs:0.02,4.8528543899075e-45)
--(axis cs:0.021,3.9814621023952e-41)
--(axis cs:0.022,1.66262383997763e-37)
--(axis cs:0.023,3.75075058539897e-34)
--(axis cs:0.024,4.81478504563591e-31)
--(axis cs:0.025,3.68101006289339e-28)
--(axis cs:0.026,1.7448352772851e-25)
--(axis cs:0.027,2.87951541234212e-27)
--(axis cs:0.028,3.347144735942e-29)
--(axis cs:0.029,3.70377304132389e-31)
--(axis cs:0.03,3.91292511145641e-33)
--(axis cs:0.031,3.95724341416798e-35)
--(axis cs:0.032,3.84021062163493e-37)
--(axis cs:0.033,3.58367905773059e-39)
--(axis cs:0.034,3.22233679509323e-41)
--(axis cs:0.035,2.79680089338449e-43)
--(axis cs:0.036,2.34701312578845e-45)
--(axis cs:0.037,1.90716464500048e-47)
--(axis cs:0.038,1.50273367988648e-49)
--(axis cs:0.039,1.14961412015033e-51)
--(axis cs:0.04,8.54891254145121e-54)
--(axis cs:0.041,6.18632121539339e-56)
--(axis cs:0.042,4.36070060708094e-58)
--(axis cs:0.043,2.99702431427062e-60)
--(axis cs:0.044,2.01008257706246e-62)
--(axis cs:0.045,1.3166812239921e-64)
--(axis cs:0.046,8.42985841306704e-67)
--(axis cs:0.047,5.27886972115446e-69)
--(axis cs:0.048,3.23542200270096e-71)
--(axis cs:0.049,1.9420497751163e-73)
--(axis cs:0.05,1.14230634263601e-75)
--(axis cs:0.051,6.58772715040932e-78)
--(axis cs:0.052,3.72686076805855e-80)
--(axis cs:0.053,2.0692640951467e-82)
--(axis cs:0.054,1.12811236001118e-84)
--(axis cs:0.055,6.04142947874886e-87)
--(axis cs:0.056,3.17947028224844e-89)
--(axis cs:0.057,1.64499513577052e-91)
--(axis cs:0.058,8.37003553177858e-94)
--(axis cs:0.059,4.18979187704877e-96)
--(axis cs:0.06,2.06396268510268e-98)
--(axis cs:0.061,1.00089821811303e-100)
--(axis cs:0.062,4.77951971652832e-103)
--(axis cs:0.063,2.24805031380068e-105)
--(axis cs:0.064,1.04176645239681e-107)
--(axis cs:0.065,4.75758742514123e-110)
--(axis cs:0.066,2.14170401581855e-112)
--(axis cs:0.067,9.50577648796099e-115)
--(axis cs:0.068,4.16069364536263e-117)
--(axis cs:0.069,1.79632112254605e-119)
--(axis cs:0.07,7.65117697083413e-122)
--(axis cs:0.071,3.21573727167356e-124)
--(axis cs:0.072,1.3338875227021e-126)
--(axis cs:0.073,5.46158868186337e-129)
--(axis cs:0.074,2.20775751184168e-131)
--(axis cs:0.075,8.81220691817332e-134)
--(axis cs:0.076,3.47363588128382e-136)
--(axis cs:0.077,1.35242355307076e-138)
--(axis cs:0.078,5.20151489599907e-141)
--(axis cs:0.079,1.9764825045569e-143)
--cycle;
\path [draw=none, fill=blue, fill opacity=0.5]
(axis cs:0,0)
--(axis cs:0.001,0)
--(axis cs:0.002,1.80855680220225e-315)
--(axis cs:0.003,9.70145177563557e-264)
--(axis cs:0.004,1.05136674560834e-227)
--(axis cs:0.005,2.85674390501788e-200)
--(axis cs:0.006,2.81155839455437e-178)
--(axis cs:0.007,4.8583770711663e-160)
--(axis cs:0.008,1.4948608620893e-144)
--(axis cs:0.009,3.64944965351347e-131)
--(axis cs:0.01,1.96096183052097e-119)
--(axis cs:0.011,4.80140730189426e-109)
--(axis cs:0.012,9.16808095919789e-100)
--(axis cs:0.013,2.05309186557918e-91)
--(axis cs:0.014,7.40465975766897e-84)
--(axis cs:0.015,5.53049730591008e-77)
--(axis cs:0.016,1.04764202141619e-70)
--(axis cs:0.017,5.94088345168626e-65)
--(axis cs:0.018,1.15696407390828e-59)
--(axis cs:0.019,8.68116657991054e-55)
--(axis cs:0.02,2.76624283557957e-50)
--(axis cs:0.021,4.067523087894e-46)
--(axis cs:0.022,2.96437169477005e-42)
--(axis cs:0.023,1.1391789303744e-38)
--(axis cs:0.024,2.4364950354929e-35)
--(axis cs:0.025,3.04113977224369e-32)
--(axis cs:0.026,2.30974489923338e-29)
--(axis cs:0.027,2.87951541234212e-27)
--(axis cs:0.028,3.347144735942e-29)
--(axis cs:0.029,3.70377304132389e-31)
--(axis cs:0.03,3.91292511145641e-33)
--(axis cs:0.031,3.95724341416798e-35)
--(axis cs:0.032,3.84021062163493e-37)
--(axis cs:0.033,3.58367905773059e-39)
--(axis cs:0.034,3.22233679509323e-41)
--(axis cs:0.035,2.79680089338449e-43)
--(axis cs:0.036,2.34701312578845e-45)
--(axis cs:0.037,1.90716464500048e-47)
--(axis cs:0.038,1.50273367988648e-49)
--(axis cs:0.039,1.14961412015033e-51)
--(axis cs:0.04,8.54891254145121e-54)
--(axis cs:0.041,6.18632121539339e-56)
--(axis cs:0.042,4.36070060708094e-58)
--(axis cs:0.043,2.99702431427062e-60)
--(axis cs:0.044,2.01008257706246e-62)
--(axis cs:0.045,1.3166812239921e-64)
--(axis cs:0.046,8.42985841306704e-67)
--(axis cs:0.047,5.27886972115446e-69)
--(axis cs:0.048,3.23542200270096e-71)
--(axis cs:0.049,1.9420497751163e-73)
--(axis cs:0.05,1.14230634263601e-75)
--(axis cs:0.051,6.58772715040932e-78)
--(axis cs:0.052,3.72686076805855e-80)
--(axis cs:0.053,2.0692640951467e-82)
--(axis cs:0.054,1.12811236001118e-84)
--(axis cs:0.055,6.04142947874886e-87)
--(axis cs:0.056,3.17947028224844e-89)
--(axis cs:0.057,1.64499513577052e-91)
--(axis cs:0.058,8.37003553177858e-94)
--(axis cs:0.059,4.18979187704877e-96)
--(axis cs:0.06,2.06396268510268e-98)
--(axis cs:0.061,1.00089821811303e-100)
--(axis cs:0.062,4.77951971652832e-103)
--(axis cs:0.063,2.24805031380068e-105)
--(axis cs:0.064,1.04176645239681e-107)
--(axis cs:0.065,4.75758742514123e-110)
--(axis cs:0.066,2.14170401581855e-112)
--(axis cs:0.067,9.50577648796099e-115)
--(axis cs:0.068,4.16069364536263e-117)
--(axis cs:0.069,1.79632112254605e-119)
--(axis cs:0.07,7.65117697083413e-122)
--(axis cs:0.071,3.21573727167356e-124)
--(axis cs:0.072,1.3338875227021e-126)
--(axis cs:0.073,5.46158868186337e-129)
--(axis cs:0.074,2.20775751184168e-131)
--(axis cs:0.075,8.81220691817332e-134)
--(axis cs:0.076,3.47363588128382e-136)
--(axis cs:0.077,1.35242355307076e-138)
--(axis cs:0.078,5.20151489599907e-141)
--(axis cs:0.079,1.9764825045569e-143)
--cycle;
\path [draw=none, fill=blue, fill opacity=0.5]
(axis cs:0,0)
--(axis cs:0.001,0)
--(axis cs:0.002,1.80855680220225e-315)
--(axis cs:0.003,9.70145177563557e-264)
--(axis cs:0.004,1.05136674560834e-227)
--(axis cs:0.005,2.85674390501788e-200)
--(axis cs:0.006,2.81155839455437e-178)
--(axis cs:0.007,4.8583770711663e-160)
--(axis cs:0.008,1.4948608620893e-144)
--(axis cs:0.009,3.64944965351347e-131)
--(axis cs:0.01,1.96096183052097e-119)
--(axis cs:0.011,4.80140730189426e-109)
--(axis cs:0.012,9.16808095919789e-100)
--(axis cs:0.013,2.05309186557918e-91)
--(axis cs:0.014,7.40465975766897e-84)
--(axis cs:0.015,5.53049730591008e-77)
--(axis cs:0.016,1.04764202141619e-70)
--(axis cs:0.017,5.94088345168626e-65)
--(axis cs:0.018,1.15696407390828e-59)
--(axis cs:0.019,8.68116657991054e-55)
--(axis cs:0.02,2.76624283557957e-50)
--(axis cs:0.021,4.067523087894e-46)
--(axis cs:0.022,2.96437169477005e-42)
--(axis cs:0.023,1.1391789303744e-38)
--(axis cs:0.024,2.4364950354929e-35)
--(axis cs:0.025,3.04113977224369e-32)
--(axis cs:0.026,2.30974489923338e-29)
--(axis cs:0.027,1.10777243852506e-26)
--(axis cs:0.028,3.46768904113277e-24)
--(axis cs:0.029,7.29736102881152e-22)
--(axis cs:0.03,1.06011885358369e-19)
--(axis cs:0.031,1.08892641163811e-17)
--(axis cs:0.032,8.08163789135568e-16)
--(axis cs:0.033,4.41967312327648e-14)
--(axis cs:0.034,1.81317194065596e-12)
--(axis cs:0.035,5.67202228756521e-11)
--(axis cs:0.036,1.37334311873352e-09)
--(axis cs:0.037,2.6092776248558e-08)
--(axis cs:0.038,3.93951153191311e-07)
--(axis cs:0.039,4.78187779279172e-06)
--(axis cs:0.04,4.71690864312789e-05)
--(axis cs:0.041,0.000381893197883262)
--(axis cs:0.042,0.00256130614332519)
--(axis cs:0.043,0.014352988741732)
--(axis cs:0.044,0.0677412580805793)
--(axis cs:0.045,0.271286854721883)
--(axis cs:0.046,0.928302837236422)
--(axis cs:0.047,2.73187999040491)
--(axis cs:0.048,6.95650837216909)
--(axis cs:0.049,15.4156351666923)
--(axis cs:0.05,29.888477272067)
--(axis cs:0.051,50.9578824651769)
--(axis cs:0.052,76.7624887430961)
--(axis cs:0.053,102.627406880749)
--(axis cs:0.054,122.290117184095)
--(axis cs:0.055,106.720404349739)
--(axis cs:0.056,81.7461846877089)
--(axis cs:0.057,56.6564076062584)
--(axis cs:0.058,35.6457911566639)
--(axis cs:0.059,20.4215361638852)
--(axis cs:0.06,10.6847582024863)
--(axis cs:0.061,5.11975073391132)
--(axis cs:0.062,2.25264653553158)
--(axis cs:0.063,0.912418523163483)
--(axis cs:0.064,0.341033103361625)
--(axis cs:0.065,0.117895196684117)
--(axis cs:0.066,0.0377783218158894)
--(axis cs:0.067,0.0112445715052932)
--(axis cs:0.068,0.00311503556845258)
--(axis cs:0.069,0.000804694274913388)
--(axis cs:0.07,0.000194195232215713)
--(axis cs:0.071,4.38575079758384e-05)
--(axis cs:0.072,9.28480952591536e-06)
--(axis cs:0.073,1.84553096076627e-06)
--(axis cs:0.074,3.4494983594163e-07)
--(axis cs:0.075,6.07178930992988e-08)
--(axis cs:0.076,1.00790078761584e-08)
--(axis cs:0.077,1.57997087654177e-09)
--(axis cs:0.078,2.34194866640972e-10)
--(axis cs:0.079,3.28660519679006e-11)
--cycle;
\addplot [semithick, color0, opacity=0.8]
table {%
0 0
0.001 3.12799477908694e-09
0.002 0.0104356901882469
0.003 6.79404167174568
0.004 133.504789945242
0.005 384.380205282111
0.006 327.748216254779
0.007 120.464572363519
0.008 23.8796063318801
0.009 2.94990766857048
0.01 0.250643700619543
0.011 0.0157158047851612
0.012 0.000765974288866585
0.013 3.01873758278666e-05
0.014 9.91948194179301e-07
0.015 2.78459696414973e-08
0.016 6.81012680057153e-10
0.017 1.47443986465567e-11
0.018 2.86379607366953e-13
0.019 5.04577475923118e-15
0.02 8.14081250518803e-17
0.021 1.21240104315078e-18
0.022 1.67826570738711e-20
0.023 2.17223903510431e-22
0.024 2.64272039884326e-24
0.025 3.03583177227021e-26
0.026 3.30627384009937e-28
0.027 3.4259996322124e-30
0.028 3.38848729028021e-32
0.029 3.20797503156245e-34
0.03 2.91455887514182e-36
0.031 2.54701683678907e-38
0.032 2.14542085003383e-40
0.033 1.74516065791736e-42
0.034 1.37324210175997e-44
0.035 1.04695668328753e-46
0.036 7.74467701834709e-49
0.037 5.56598850592264e-51
0.038 3.89108690866414e-53
0.039 2.64894566260278e-55
0.04 1.7579047473431e-57
0.041 1.1382848720731e-59
0.042 7.19818173995335e-62
0.043 4.44904007966932e-64
0.044 2.68975072054131e-66
0.045 1.59172417834704e-68
0.046 9.22615701160169e-71
0.047 5.24130452219474e-73
0.048 2.91994339665418e-75
0.049 1.59610345182032e-77
0.05 8.56484836365445e-80
0.051 4.51395662597842e-82
0.052 2.33759228681477e-84
0.053 1.1899726713028e-86
0.054 5.95709600611141e-89
0.055 2.93375825045685e-91
0.056 1.42186957992117e-93
0.057 6.78400314278218e-96
0.058 3.18742806762942e-98
0.059 1.47520611752496e-100
0.06 6.72738069349805e-103
0.061 3.02368756955887e-105
0.062 1.33978794899461e-107
0.063 5.85394296544167e-110
0.064 2.5227484169707e-112
0.065 1.07252462238867e-114
0.066 4.49923502697876e-117
0.067 1.86274763805703e-119
0.068 7.61263576576289e-122
0.069 3.07156468604694e-124
0.07 1.223776543627e-126
0.071 4.81540981868275e-129
0.072 1.87163275890967e-131
0.073 7.18668810287969e-134
0.074 2.72658696141498e-136
0.075 1.02223603237452e-138
0.076 3.78774357598603e-141
0.077 1.38727104736618e-143
0.078 5.02278879483122e-146
0.079 1.79796240717077e-148
};
\addlegendentry{hand-held computer}
\addplot [semithick, color1, opacity=0.8]
table {%
0 0
0.001 7.5306614288899e-12
0.002 0.000274798181256152
0.003 0.726067519098887
0.004 38.5850050992666
0.005 240.526523527352
0.006 385.767976666012
0.007 242.029049094857
0.008 76.2778530918751
0.009 14.1897249207388
0.01 1.73950412373907
0.011 0.152008438140823
0.012 0.0100343318310296
0.013 0.000522894914383514
0.014 2.22593210062889e-05
0.015 7.95362205700697e-07
0.016 2.43821639111533e-08
0.017 6.52833161576269e-10
0.018 1.54949091871247e-11
0.019 3.30081818819095e-13
0.02 6.37761267195656e-15
0.021 1.12768675745017e-16
0.022 1.83890182477974e-18
0.023 2.78398641404362e-20
0.024 3.93588135036249e-22
0.025 5.22276708280347e-24
0.026 6.53429752148489e-26
0.027 7.73874298767626e-28
0.028 8.70681062016143e-30
0.029 9.33567371416515e-32
0.03 9.56686925278931e-34
0.031 9.39396244637523e-36
0.032 8.85916625253614e-38
0.033 8.04115421740054e-40
0.034 7.03816279242595e-42
0.035 5.9508151634981e-44
0.036 4.86817215486112e-46
0.037 3.85892291489979e-48
0.038 2.96799969450108e-50
0.039 2.21768501820965e-52
0.04 1.61165961489089e-54
0.041 1.14037062157614e-56
0.042 7.86404411613883e-59
0.043 5.29018007480495e-61
0.044 3.47447450623553e-63
0.045 2.22969892206919e-65
0.046 1.39914562167862e-67
0.047 8.59091243668763e-70
0.048 5.16483090466877e-72
0.049 3.04212035615481e-74
0.05 1.75649483966228e-76
0.051 9.94716423556478e-79
0.052 5.52778956810444e-81
0.053 3.01584215061926e-83
0.054 1.61608372204751e-85
0.055 8.50941165787103e-88
0.056 4.40441006928744e-90
0.057 2.24177045413869e-92
0.058 1.12244109840084e-94
0.059 5.53032489304425e-97
0.06 2.68219524276422e-99
0.061 1.28089529088182e-101
0.062 6.02482626864346e-104
0.063 2.79190725558905e-106
0.064 1.27495555216651e-108
0.065 5.73896050562254e-111
0.066 2.54692954451068e-113
0.067 1.11466057994533e-115
0.068 4.81174176262973e-118
0.069 2.04919635644202e-120
0.07 8.61132562339268e-123
0.071 3.57142365684475e-125
0.072 1.46208892184532e-127
0.073 5.90935354694027e-130
0.074 2.3583562855075e-132
0.075 9.29498163467781e-135
0.076 3.61843567662549e-137
0.077 1.39151263131687e-139
0.078 5.28695472209426e-142
0.079 1.98486905737179e-144
};
\addlegendentry{modem}
\addplot [semithick, color2, opacity=0.8]
table {%
0 0
0.001 1.54507617310345e-12
0.002 0.000103297845546622
0.003 0.389075904777595
0.004 26.5972695349157
0.005 201.601714961611
0.006 379.407730891438
0.007 272.536584451434
0.008 96.5878221831444
0.009 19.9296193965953
0.01 2.68068654080692
0.011 0.25478677443149
0.012 0.0181611293086123
0.013 0.00101572116203083
0.014 4.61668082374535e-05
0.015 1.75349045304705e-06
0.016 5.69172870947981e-08
0.017 1.60814554079926e-09
0.018 4.01560220815062e-11
0.019 8.97534456856203e-13
0.02 1.81511901111211e-14
0.021 3.35200788459136e-16
0.022 5.6975098784988e-18
0.023 8.97471233639538e-20
0.024 1.31797116930899e-21
0.025 1.81391612404523e-23
0.026 2.35050874966248e-25
0.027 2.87951541234212e-27
0.028 3.347144735942e-29
0.029 3.70377304132389e-31
0.03 3.91292511145641e-33
0.031 3.95724341416798e-35
0.032 3.84021062163493e-37
0.033 3.58367905773059e-39
0.034 3.22233679509323e-41
0.035 2.79680089338449e-43
0.036 2.34701312578845e-45
0.037 1.90716464500048e-47
0.038 1.50273367988648e-49
0.039 1.14961412015033e-51
0.04 8.54891254145121e-54
0.041 6.18632121539339e-56
0.042 4.36070060708094e-58
0.043 2.99702431427062e-60
0.044 2.01008257706246e-62
0.045 1.3166812239921e-64
0.046 8.42985841306704e-67
0.047 5.27886972115446e-69
0.048 3.23542200270096e-71
0.049 1.9420497751163e-73
0.05 1.14230634263601e-75
0.051 6.58772715040932e-78
0.052 3.72686076805855e-80
0.053 2.0692640951467e-82
0.054 1.12811236001118e-84
0.055 6.04142947874886e-87
0.056 3.17947028224844e-89
0.057 1.64499513577052e-91
0.058 8.37003553177858e-94
0.059 4.18979187704877e-96
0.06 2.06396268510268e-98
0.061 1.00089821811303e-100
0.062 4.77951971652832e-103
0.063 2.24805031380068e-105
0.064 1.04176645239681e-107
0.065 4.75758742514123e-110
0.066 2.14170401581855e-112
0.067 9.50577648796099e-115
0.068 4.16069364536263e-117
0.069 1.79632112254605e-119
0.07 7.65117697083413e-122
0.071 3.21573727167356e-124
0.072 1.3338875227021e-126
0.073 5.46158868186337e-129
0.074 2.20775751184168e-131
0.075 8.81220691817332e-134
0.076 3.47363588128382e-136
0.077 1.35242355307076e-138
0.078 5.20151489599907e-141
0.079 1.9764825045569e-143
};
\addlegendentry{monarch}
\addplot [semithick, color3, opacity=0.8]
table {%
0 0
0.001 0
0.002 2.03014981812474e-298
0.003 9.31724947218955e-249
0.004 3.43265608779348e-214
0.005 6.7527672970312e-188
0.006 7.76162738981217e-167
0.007 2.17868917797679e-149
0.008 1.38640655839917e-134
0.009 8.41804844498113e-122
0.01 1.30117864576101e-110
0.011 1.03096053371918e-100
0.012 7.02052109385449e-92
0.013 6.08361715932729e-84
0.014 9.10136664373718e-77
0.015 2.9938465544224e-70
0.016 2.63145915791406e-64
0.017 7.24866962376275e-59
0.018 7.14131860903066e-54
0.019 2.81065697786411e-49
0.02 4.8528543899075e-45
0.021 3.9814621023952e-41
0.022 1.66262383997763e-37
0.023 3.75075058539897e-34
0.024 4.81478504563591e-31
0.025 3.68101006289339e-28
0.026 1.7448352772851e-25
0.027 5.31405791598588e-23
0.028 1.07344254292672e-20
0.029 1.47963582086616e-18
0.03 1.42771325901343e-16
0.031 9.86807578423092e-15
0.032 4.9885202295173e-13
0.033 1.87960150719886e-11
0.034 5.37014449796131e-10
0.035 1.18182541398533e-08
0.036 2.03241242377467e-07
0.037 2.76752621777603e-06
0.038 3.02039888371721e-05
0.039 0.000267171126317532
0.04 0.00193534239420308
0.041 0.0115911488263814
0.042 0.0579097743935045
0.043 0.24334233564868
0.044 0.866682440700314
0.045 2.63505481152107
0.046 6.88512674359099
0.047 15.5576619911148
0.048 30.5796352672969
0.049 52.5730895589514
0.05 79.4658468174158
0.051 106.118807913106
0.052 125.772257455593
0.053 132.870937441889
0.054 125.630327628736
0.055 106.720404349739
0.056 81.7461846877089
0.057 56.6564076062584
0.058 35.6457911566639
0.059 20.4215361638852
0.06 10.6847582024863
0.061 5.11975073391132
0.062 2.25264653553158
0.063 0.912418523163483
0.064 0.341033103361625
0.065 0.117895196684117
0.066 0.0377783218158894
0.067 0.0112445715052932
0.068 0.00311503556845258
0.069 0.000804694274913388
0.07 0.000194195232215713
0.071 4.38575079758384e-05
0.072 9.28480952591536e-06
0.073 1.84553096076627e-06
0.074 3.4494983594163e-07
0.075 6.07178930992988e-08
0.076 1.00790078761584e-08
0.077 1.57997087654177e-09
0.078 2.34194866640972e-10
0.079 3.28660519679006e-11
};
\addlegendentry{notebook}
\addplot [semithick, color4, opacity=0.8]
table {%
0 0
0.001 0
0.002 1.80855680220225e-315
0.003 9.70145177563557e-264
0.004 1.05136674560834e-227
0.005 2.85674390501788e-200
0.006 2.81155839455437e-178
0.007 4.8583770711663e-160
0.008 1.4948608620893e-144
0.009 3.64944965351347e-131
0.01 1.96096183052097e-119
0.011 4.80140730189426e-109
0.012 9.16808095919789e-100
0.013 2.05309186557918e-91
0.014 7.40465975766897e-84
0.015 5.53049730591008e-77
0.016 1.04764202141619e-70
0.017 5.94088345168626e-65
0.018 1.15696407390828e-59
0.019 8.68116657991054e-55
0.02 2.76624283557957e-50
0.021 4.067523087894e-46
0.022 2.96437169477005e-42
0.023 1.1391789303744e-38
0.024 2.4364950354929e-35
0.025 3.04113977224369e-32
0.026 2.30974489923338e-29
0.027 1.10777243852506e-26
0.028 3.46768904113277e-24
0.029 7.29736102881152e-22
0.03 1.06011885358369e-19
0.031 1.08892641163811e-17
0.032 8.08163789135568e-16
0.033 4.41967312327648e-14
0.034 1.81317194065596e-12
0.035 5.67202228756521e-11
0.036 1.37334311873352e-09
0.037 2.6092776248558e-08
0.038 3.93951153191311e-07
0.039 4.78187779279172e-06
0.04 4.71690864312789e-05
0.041 0.000381893197883262
0.042 0.00256130614332519
0.043 0.014352988741732
0.044 0.0677412580805793
0.045 0.271286854721883
0.046 0.928302837236422
0.047 2.73187999040491
0.048 6.95650837216909
0.049 15.4156351666923
0.05 29.888477272067
0.051 50.9578824651769
0.052 76.7624887430961
0.053 102.627406880749
0.054 122.290117184095
0.055 130.397323714525
0.056 124.89262241123
0.057 107.832503506074
0.058 84.2131684940615
0.059 59.67941266359
0.06 38.4954460427062
0.061 22.667022450672
0.062 12.2173822067682
0.063 6.04367182065603
0.064 2.75073599622089
0.065 1.15466968341511
0.066 0.448038380940999
0.067 0.161051298616836
0.068 0.0537409788583914
0.069 0.0166801633893602
0.07 0.00482471916574644
0.071 0.00130289406164708
0.072 0.000329053471811256
0.073 7.78516177636076e-05
0.074 1.72824893734016e-05
0.075 3.60535441701532e-06
0.076 7.07834065268705e-07
0.077 1.3096954280877e-07
0.078 2.28693620624404e-08
0.079 3.77354414828933e-09
};
\addlegendentry{laptop}

\nextgroupplot[
height=\figheight,
legend cell align={left},
legend style={fill opacity=0.8, draw opacity=1, text opacity=1, draw=white!80!black},
tick align=outside,
tick pos=both,
width=\figwidth,
x grid style={white!69.0196078431373!black},
xmin=-0.00197, xmax=0.04137,
xtick style={color=black},
xtick={-0.005,0,0.005,0.01,0.015,0.02,0.025,0.03,0.035,0.04,0.045},
xticklabels={−0.005,0.000,0.005,0.010,0.015,0.020,0.025,0.030,0.035,0.040,0.045},
y grid style={white!69.0196078431373!black},
ymin=-23.7094907923839, ymax=497.899306640063,
ytick style={color=black}
]
\path [draw=none, fill=blue, fill opacity=0.5]
(axis cs:0,0)
--(axis cs:0.0002,4.04368514862087e-35)
--(axis cs:0.0004,1.25272008938949e-24)
--(axis cs:0.0006,9.65346259577257e-19)
--(axis cs:0.0008,9.69094321532076e-15)
--(axis cs:0.001,9.01653559458301e-12)
--(axis cs:0.0012,1.86323886206438e-09)
--(axis cs:0.0014,1.36308759054038e-07)
--(axis cs:0.0016,4.66297027012948e-06)
--(axis cs:0.0018,8.92566476238349e-05)
--(axis cs:0.002,0.0010806496464931)
--(axis cs:0.0022,0.00903151459729266)
--(axis cs:0.0024,0.0555775579214406)
--(axis cs:0.0026,0.264477804740357)
--(axis cs:0.0028,1.01106053297254)
--(axis cs:0.003,3.20022716358246)
--(axis cs:0.0032,8.59359582485187)
--(axis cs:0.0034,19.9714205468096)
--(axis cs:0.0036,40.8364337262247)
--(axis cs:0.0038,74.489190934246)
--(axis cs:0.004,122.63737500972)
--(axis cs:0.0042,184.064660928411)
--(axis cs:0.0044,254.018066296762)
--(axis cs:0.0046,324.74056978027)
--(axis cs:0.0048,387.083461543562)
--(axis cs:0.005,432.655136750648)
--(axis cs:0.0052,444.894558756359)
--(axis cs:0.0054,398.581697445687)
--(axis cs:0.0056,340.698284065176)
--(axis cs:0.0058,278.765390132612)
--(axis cs:0.006,218.979357069213)
--(axis cs:0.0062,165.58315457169)
--(axis cs:0.0064,120.815079026538)
--(axis cs:0.0066,85.2442631844142)
--(axis cs:0.0068,58.2787711177048)
--(axis cs:0.007,38.6762174669473)
--(axis cs:0.0072,24.9566689864978)
--(axis cs:0.0074,15.6819348518085)
--(axis cs:0.0076,9.6092950684639)
--(axis cs:0.0078,5.74941026071925)
--(axis cs:0.008,3.36289542641486)
--(axis cs:0.0082,1.92504865939838)
--(axis cs:0.0084,1.07957702041241)
--(axis cs:0.0086,0.593694418792337)
--(axis cs:0.0088,0.320446113786265)
--(axis cs:0.009,0.169898314826288)
--(axis cs:0.0092,0.0885524876070683)
--(axis cs:0.0094,0.0454050383416075)
--(axis cs:0.0096,0.0229188611207476)
--(axis cs:0.0098,0.0113957692002469)
--(axis cs:0.01,0.00558489580279418)
--(axis cs:0.0102,0.00269929974117084)
--(axis cs:0.0104,0.00128730466490826)
--(axis cs:0.0106,0.000606069215412055)
--(axis cs:0.0108,0.000281824066500681)
--(axis cs:0.011,0.000129491710086973)
--(axis cs:0.0112,5.88161400591037e-05)
--(axis cs:0.0114,2.64188958878439e-05)
--(axis cs:0.0116,1.17397891884797e-05)
--(axis cs:0.0118,5.16284297470597e-06)
--(axis cs:0.012,2.24775082783557e-06)
--(axis cs:0.0122,9.69121820503786e-07)
--(axis cs:0.0124,4.13916758159382e-07)
--(axis cs:0.0126,1.75177831434875e-07)
--(axis cs:0.0128,7.34848547104344e-08)
--(axis cs:0.013,3.05621933299358e-08)
--(axis cs:0.0132,1.26051877673596e-08)
--(axis cs:0.0134,5.1570032813465e-09)
--(axis cs:0.0136,2.0932893944091e-09)
--(axis cs:0.0138,8.43220198538763e-10)
--(axis cs:0.014,3.37151058369145e-10)
--(axis cs:0.0142,1.33834435916904e-10)
--(axis cs:0.0144,5.27539961091201e-11)
--(axis cs:0.0146,2.06522717779017e-11)
--(axis cs:0.0148,8.03124382031559e-12)
--(axis cs:0.015,3.10294950655172e-12)
--(axis cs:0.0152,1.19128308015436e-12)
--(axis cs:0.0154,4.54540149913224e-13)
--(axis cs:0.0156,1.72390181140141e-13)
--(axis cs:0.0158,6.49978903194613e-14)
--(axis cs:0.016,2.43665206508037e-14)
--(axis cs:0.0162,9.08351819191971e-15)
--(axis cs:0.0164,3.36773254243571e-15)
--(axis cs:0.0166,1.24193420842092e-15)
--(axis cs:0.0168,4.55606253832252e-16)
--(axis cs:0.017,1.66288257913556e-16)
--(axis cs:0.0172,6.03897535029679e-17)
--(axis cs:0.0174,2.1824381482602e-17)
--(axis cs:0.0176,7.84952309420417e-18)
--(axis cs:0.0178,2.81003232437693e-18)
--(axis cs:0.018,1.00135639304338e-18)
--(axis cs:0.0182,3.55235641677016e-19)
--(axis cs:0.0184,1.25468504813723e-19)
--(axis cs:0.0186,4.4124701636192e-20)
--(axis cs:0.0188,1.54523668801194e-20)
--(axis cs:0.019,5.38902904610328e-21)
--(axis cs:0.0192,1.87181662615934e-21)
--(axis cs:0.0194,6.47570756759223e-22)
--(axis cs:0.0196,2.23159383585354e-22)
--(axis cs:0.0198,7.66088454499077e-23)
--(axis cs:0.02,2.62004893068498e-23)
--(axis cs:0.0202,8.92763785971532e-24)
--(axis cs:0.0204,3.03102471455776e-24)
--(axis cs:0.0206,1.02540718757071e-24)
--(axis cs:0.0208,3.45688238253561e-25)
--(axis cs:0.021,1.1613973381258e-25)
--(axis cs:0.0212,3.88875765368101e-26)
--(axis cs:0.0214,1.29777643437928e-26)
--(axis cs:0.0216,4.31690271137589e-27)
--(axis cs:0.0218,1.43136948614258e-27)
--(axis cs:0.022,4.7310926805422e-28)
--(axis cs:0.0222,1.55891999605845e-28)
--(axis cs:0.0224,5.12107018312939e-29)
--(axis cs:0.0226,1.67723311744901e-29)
--(axis cs:0.0228,5.47699873987991e-30)
--(axis cs:0.023,1.78331691502149e-30)
--(axis cs:0.0232,5.78989560321019e-31)
--(axis cs:0.0234,1.87451269749009e-31)
--(axis cs:0.0236,6.05201690680936e-32)
--(axis cs:0.0238,1.94860647371264e-32)
--(axis cs:0.024,6.25717244805115e-33)
--(axis cs:0.0242,2.0039152602773e-33)
--(axis cs:0.0244,6.40095343780388e-34)
--(axis cs:0.0246,2.0393440465356e-34)
--(axis cs:0.0248,6.48086350573449e-35)
--(axis cs:0.025,2.05441148839129e-35)
--(axis cs:0.0252,6.4963504541668e-36)
--(axis cs:0.0254,2.04924464279847e-36)
--(axis cs:0.0256,6.44874307040056e-37)
--(axis cs:0.0258,2.02454508309857e-37)
--(axis cs:0.026,6.34110320078355e-38)
--(axis cs:0.0262,1.98153040354014e-38)
--(axis cs:0.0264,6.17800786984118e-39)
--(axis cs:0.0266,1.92185633197062e-39)
--(axis cs:0.0268,5.96527926936136e-40)
--(axis cs:0.027,1.84752537803359e-40)
--(axis cs:0.0272,5.70968188505042e-41)
--(axis cs:0.0274,1.76078815528641e-41)
--(axis cs:0.0276,5.4186059585716e-42)
--(axis cs:0.0278,1.66404327671261e-42)
--(axis cs:0.028,5.09975510408039e-43)
--(axis cs:0.0282,1.5597411127119e-43)
--(axis cs:0.0284,4.76085349466722e-44)
--(axis cs:0.0286,1.45029581675223e-44)
--(axis cs:0.0288,4.40938493130639e-45)
--(axis cs:0.029,1.3380089717475e-45)
--(axis cs:0.0292,4.05237263667289e-46)
--(axis cs:0.0294,1.22500709199272e-46)
--(axis cs:0.0296,3.69620508597466e-47)
--(axis cs:0.0298,1.11319412170508e-47)
--(axis cs:0.03,3.34650985758505e-48)
--(axis cs:0.0302,1.00421907518375e-48)
--(axis cs:0.0304,3.00807455797737e-49)
--(axis cs:0.0306,8.99458118226471e-50)
--(axis cs:0.0308,2.68481148237362e-50)
--(axis cs:0.031,8.00009727731514e-51)
--(axis cs:0.0312,2.37976088390367e-51)
--(axis cs:0.0314,7.06701098749733e-52)
--(axis cs:0.0316,2.0951265513022e-52)
--(axis cs:0.0318,6.2010369128817e-53)
--(axis cs:0.032,1.83233680229848e-53)
--(axis cs:0.0322,5.40555705491807e-54)
--(axis cs:0.0324,1.59212391595571e-54)
--(axis cs:0.0326,4.68189317030407e-55)
--(axis cs:0.0328,1.37461535925932e-55)
--(axis cs:0.033,4.0296066309506e-56)
--(axis cs:0.0332,1.17943080786081e-56)
--(axis cs:0.0334,3.44680810294073e-57)
--(axis cs:0.0336,1.00577981451736e-57)
--(axis cs:0.0338,2.93046227125017e-58)
--(axis cs:0.034,8.52555945485244e-59)
--(axis cs:0.0342,2.4766759715603e-59)
--(axis cs:0.0344,7.184242020946e-60)
--(axis cs:0.0346,2.08096120856607e-60)
--(axis cs:0.0348,6.01899505301586e-61)
--(axis cs:0.035,1.73846745830563e-61)
--(axis cs:0.0352,5.01414895848345e-62)
--(axis cs:0.0354,1.4441802412468e-62)
--(axis cs:0.0356,4.15378859254108e-63)
--(axis cs:0.0358,1.1930851351353e-63)
--(axis cs:0.036,3.42221805823752e-64)
--(axis cs:0.0362,9.80298138120522e-65)
--(axis cs:0.0364,2.80432216538558e-65)
--(axis cs:0.0366,8.01164595555668e-66)
--(axis cs:0.0368,2.28583311371841e-66)
--(axis cs:0.037,6.51329797203411e-67)
--(axis cs:0.0372,1.85351319674645e-67)
--(axis cs:0.0374,5.26784882758811e-68)
--(axis cs:0.0376,1.4952654973008e-68)
--(axis cs:0.0378,4.23891920517638e-69)
--(axis cs:0.038,1.2001848555619e-69)
--(axis cs:0.0382,3.39392008576943e-70)
--(axis cs:0.0384,9.58561031136903e-71)
--(axis cs:0.0386,2.70400050940451e-71)
--(axis cs:0.0388,7.61845138342213e-72)
--(axis cs:0.039,2.14389538326699e-72)
--(axis cs:0.0392,6.02589274852925e-73)
--(axis cs:0.0394,1.69170264255059e-73)
--cycle;
\path [draw=none, fill=blue, fill opacity=0.5]
(axis cs:0,0)
--(axis cs:0.0002,6.88149847089383e-283)
--(axis cs:0.0004,5.38329324547414e-232)
--(axis cs:0.0006,1.80526708928995e-202)
--(axis cs:0.0008,1.0802222727238e-181)
--(axis cs:0.001,1.03927455538975e-165)
--(axis cs:0.0012,9.284355067112e-153)
--(axis cs:0.0014,6.70264999764463e-142)
--(axis cs:0.0016,1.42270014450123e-132)
--(axis cs:0.0018,2.04241209795704e-124)
--(axis cs:0.002,3.50239373677996e-117)
--(axis cs:0.0022,1.0763994483646e-110)
--(axis cs:0.0024,7.99953899528515e-105)
--(axis cs:0.0026,1.80482808273425e-99)
--(axis cs:0.0028,1.47530594980937e-94)
--(axis cs:0.003,5.0266949465783e-90)
--(axis cs:0.0032,7.99309022920627e-86)
--(axis cs:0.0034,6.50610273957532e-82)
--(axis cs:0.0036,2.92653432604146e-78)
--(axis cs:0.0038,7.75652633853781e-75)
--(axis cs:0.004,1.27887150345601e-71)
--(axis cs:0.0042,1.37387136164765e-68)
--(axis cs:0.0044,1.00075911725486e-65)
--(axis cs:0.0046,5.11650284411508e-63)
--(axis cs:0.0048,1.89215762100807e-60)
--(axis cs:0.005,5.19708066546697e-58)
--(axis cs:0.0052,1.08519112913327e-55)
--(axis cs:0.0054,1.75864114867945e-53)
--(axis cs:0.0056,2.25305595468255e-51)
--(axis cs:0.0058,2.31978418006334e-49)
--(axis cs:0.006,1.94819763121604e-47)
--(axis cs:0.0062,1.35246061951633e-45)
--(axis cs:0.0064,7.85536854187431e-44)
--(axis cs:0.0066,3.85939041326377e-42)
--(axis cs:0.0068,1.61999834203236e-40)
--(axis cs:0.007,5.8628757428418e-39)
--(axis cs:0.0072,1.84471455956781e-37)
--(axis cs:0.0074,5.08503961140203e-36)
--(axis cs:0.0076,1.23670034865055e-34)
--(axis cs:0.0078,2.67091172323767e-33)
--(axis cs:0.008,5.15329732842381e-32)
--(axis cs:0.0082,8.93211136472859e-31)
--(axis cs:0.0084,1.39799146700368e-29)
--(axis cs:0.0086,1.98525615382389e-28)
--(axis cs:0.0088,2.56937152193057e-27)
--(axis cs:0.009,3.04327839977601e-26)
--(axis cs:0.0092,3.31167624895744e-25)
--(axis cs:0.0094,3.32294847816438e-24)
--(axis cs:0.0096,3.08494431084593e-23)
--(axis cs:0.0098,2.65831236795176e-22)
--(axis cs:0.01,2.1325663849262e-21)
--(axis cs:0.0102,1.59721161269042e-20)
--(axis cs:0.0104,1.11979829006939e-19)
--(axis cs:0.0106,7.36754354703253e-19)
--(axis cs:0.0108,4.55972127990545e-18)
--(axis cs:0.011,2.66046231163419e-17)
--(axis cs:0.0112,1.46655123482879e-16)
--(axis cs:0.0114,7.65293478264192e-16)
--(axis cs:0.0116,3.7876809871124e-15)
--(axis cs:0.0118,1.78120985335917e-14)
--(axis cs:0.012,7.97251566953794e-14)
--(axis cs:0.0122,3.4018858950948e-13)
--(axis cs:0.0124,1.38598721522646e-12)
--(axis cs:0.0126,5.39948902691054e-12)
--(axis cs:0.0128,2.01422837127883e-11)
--(axis cs:0.013,7.20456391440064e-11)
--(axis cs:0.0132,2.47402100952021e-10)
--(axis cs:0.0134,8.16628494582459e-10)
--(axis cs:0.0136,2.0932893944091e-09)
--(axis cs:0.0138,8.43220198538763e-10)
--(axis cs:0.014,3.37151058369145e-10)
--(axis cs:0.0142,1.33834435916904e-10)
--(axis cs:0.0144,5.27539961091201e-11)
--(axis cs:0.0146,2.06522717779017e-11)
--(axis cs:0.0148,8.03124382031559e-12)
--(axis cs:0.015,3.10294950655172e-12)
--(axis cs:0.0152,1.19128308015436e-12)
--(axis cs:0.0154,4.54540149913224e-13)
--(axis cs:0.0156,1.72390181140141e-13)
--(axis cs:0.0158,6.49978903194613e-14)
--(axis cs:0.016,2.43665206508037e-14)
--(axis cs:0.0162,9.08351819191971e-15)
--(axis cs:0.0164,3.36773254243571e-15)
--(axis cs:0.0166,1.24193420842092e-15)
--(axis cs:0.0168,4.55606253832252e-16)
--(axis cs:0.017,1.66288257913556e-16)
--(axis cs:0.0172,6.03897535029679e-17)
--(axis cs:0.0174,2.1824381482602e-17)
--(axis cs:0.0176,7.84952309420417e-18)
--(axis cs:0.0178,2.81003232437693e-18)
--(axis cs:0.018,1.00135639304338e-18)
--(axis cs:0.0182,3.55235641677016e-19)
--(axis cs:0.0184,1.25468504813723e-19)
--(axis cs:0.0186,4.4124701636192e-20)
--(axis cs:0.0188,1.54523668801194e-20)
--(axis cs:0.019,5.38902904610328e-21)
--(axis cs:0.0192,1.87181662615934e-21)
--(axis cs:0.0194,6.47570756759223e-22)
--(axis cs:0.0196,2.23159383585354e-22)
--(axis cs:0.0198,7.66088454499077e-23)
--(axis cs:0.02,2.62004893068498e-23)
--(axis cs:0.0202,8.92763785971532e-24)
--(axis cs:0.0204,3.03102471455776e-24)
--(axis cs:0.0206,1.02540718757071e-24)
--(axis cs:0.0208,3.45688238253561e-25)
--(axis cs:0.021,1.1613973381258e-25)
--(axis cs:0.0212,3.88875765368101e-26)
--(axis cs:0.0214,1.29777643437928e-26)
--(axis cs:0.0216,4.31690271137589e-27)
--(axis cs:0.0218,1.43136948614258e-27)
--(axis cs:0.022,4.7310926805422e-28)
--(axis cs:0.0222,1.55891999605845e-28)
--(axis cs:0.0224,5.12107018312939e-29)
--(axis cs:0.0226,1.67723311744901e-29)
--(axis cs:0.0228,5.47699873987991e-30)
--(axis cs:0.023,1.78331691502149e-30)
--(axis cs:0.0232,5.78989560321019e-31)
--(axis cs:0.0234,1.87451269749009e-31)
--(axis cs:0.0236,6.05201690680936e-32)
--(axis cs:0.0238,1.94860647371264e-32)
--(axis cs:0.024,6.25717244805115e-33)
--(axis cs:0.0242,2.0039152602773e-33)
--(axis cs:0.0244,6.40095343780388e-34)
--(axis cs:0.0246,2.0393440465356e-34)
--(axis cs:0.0248,6.48086350573449e-35)
--(axis cs:0.025,2.05441148839129e-35)
--(axis cs:0.0252,6.4963504541668e-36)
--(axis cs:0.0254,2.04924464279847e-36)
--(axis cs:0.0256,6.44874307040056e-37)
--(axis cs:0.0258,2.02454508309857e-37)
--(axis cs:0.026,6.34110320078355e-38)
--(axis cs:0.0262,1.98153040354014e-38)
--(axis cs:0.0264,6.17800786984118e-39)
--(axis cs:0.0266,1.92185633197062e-39)
--(axis cs:0.0268,5.96527926936136e-40)
--(axis cs:0.027,1.84752537803359e-40)
--(axis cs:0.0272,5.70968188505042e-41)
--(axis cs:0.0274,1.76078815528641e-41)
--(axis cs:0.0276,5.4186059585716e-42)
--(axis cs:0.0278,1.66404327671261e-42)
--(axis cs:0.028,5.09975510408039e-43)
--(axis cs:0.0282,1.5597411127119e-43)
--(axis cs:0.0284,4.76085349466722e-44)
--(axis cs:0.0286,1.45029581675223e-44)
--(axis cs:0.0288,4.40938493130639e-45)
--(axis cs:0.029,1.3380089717475e-45)
--(axis cs:0.0292,4.05237263667289e-46)
--(axis cs:0.0294,1.22500709199272e-46)
--(axis cs:0.0296,3.69620508597466e-47)
--(axis cs:0.0298,1.11319412170508e-47)
--(axis cs:0.03,3.34650985758505e-48)
--(axis cs:0.0302,1.00421907518375e-48)
--(axis cs:0.0304,3.00807455797737e-49)
--(axis cs:0.0306,8.99458118226471e-50)
--(axis cs:0.0308,2.68481148237362e-50)
--(axis cs:0.031,8.00009727731514e-51)
--(axis cs:0.0312,2.37976088390367e-51)
--(axis cs:0.0314,7.06701098749733e-52)
--(axis cs:0.0316,2.0951265513022e-52)
--(axis cs:0.0318,6.2010369128817e-53)
--(axis cs:0.032,1.83233680229848e-53)
--(axis cs:0.0322,5.40555705491807e-54)
--(axis cs:0.0324,1.59212391595571e-54)
--(axis cs:0.0326,4.68189317030407e-55)
--(axis cs:0.0328,1.37461535925932e-55)
--(axis cs:0.033,4.0296066309506e-56)
--(axis cs:0.0332,1.17943080786081e-56)
--(axis cs:0.0334,3.44680810294073e-57)
--(axis cs:0.0336,1.00577981451736e-57)
--(axis cs:0.0338,2.93046227125017e-58)
--(axis cs:0.034,8.52555945485244e-59)
--(axis cs:0.0342,2.4766759715603e-59)
--(axis cs:0.0344,7.184242020946e-60)
--(axis cs:0.0346,2.08096120856607e-60)
--(axis cs:0.0348,6.01899505301586e-61)
--(axis cs:0.035,1.73846745830563e-61)
--(axis cs:0.0352,5.01414895848345e-62)
--(axis cs:0.0354,1.4441802412468e-62)
--(axis cs:0.0356,4.15378859254108e-63)
--(axis cs:0.0358,1.1930851351353e-63)
--(axis cs:0.036,3.42221805823752e-64)
--(axis cs:0.0362,9.80298138120522e-65)
--(axis cs:0.0364,2.80432216538558e-65)
--(axis cs:0.0366,8.01164595555668e-66)
--(axis cs:0.0368,2.28583311371841e-66)
--(axis cs:0.037,6.51329797203411e-67)
--(axis cs:0.0372,1.85351319674645e-67)
--(axis cs:0.0374,5.26784882758811e-68)
--(axis cs:0.0376,1.4952654973008e-68)
--(axis cs:0.0378,4.23891920517638e-69)
--(axis cs:0.038,1.2001848555619e-69)
--(axis cs:0.0382,3.39392008576943e-70)
--(axis cs:0.0384,9.58561031136903e-71)
--(axis cs:0.0386,2.70400050940451e-71)
--(axis cs:0.0388,7.61845138342213e-72)
--(axis cs:0.039,2.14389538326699e-72)
--(axis cs:0.0392,6.02589274852925e-73)
--(axis cs:0.0394,1.69170264255059e-73)
--cycle;
\path [draw=none, fill=blue, fill opacity=0.5]
(axis cs:0,0)
--(axis cs:0.0002,1.24681940844559e-185)
--(axis cs:0.0004,3.78429989649157e-149)
--(axis cs:0.0006,4.69624390399748e-128)
--(axis cs:0.0008,2.91814488510357e-113)
--(axis cs:0.001,6.39937004086229e-102)
--(axis cs:0.0012,9.19292845684038e-93)
--(axis cs:0.0014,4.10935002282113e-85)
--(axis cs:0.0016,1.44888628628164e-78)
--(axis cs:0.0018,7.34152876082861e-73)
--(axis cs:0.002,8.0525019711728e-68)
--(axis cs:0.0022,2.56068941187556e-63)
--(axis cs:0.0024,2.9292421581097e-59)
--(axis cs:0.0026,1.41995830907816e-55)
--(axis cs:0.0028,3.31301642990214e-52)
--(axis cs:0.003,4.11568387030809e-49)
--(axis cs:0.0032,2.95307692152295e-46)
--(axis cs:0.0034,1.30806963691571e-43)
--(axis cs:0.0036,3.77970210701459e-41)
--(axis cs:0.0038,7.46130251996167e-39)
--(axis cs:0.004,1.04633990725249e-36)
--(axis cs:0.0042,1.07774752268699e-34)
--(axis cs:0.0044,8.3909217479038e-33)
--(axis cs:0.0046,5.06232673594148e-31)
--(axis cs:0.0048,2.41856883933146e-29)
--(axis cs:0.005,9.32610456511297e-28)
--(axis cs:0.0052,2.95167434916092e-26)
--(axis cs:0.0054,7.78268945730413e-25)
--(axis cs:0.0056,1.73238744737497e-23)
--(axis cs:0.0058,3.29434213604689e-22)
--(axis cs:0.006,5.40916986257712e-21)
--(axis cs:0.0062,7.74292596344275e-20)
--(axis cs:0.0064,9.74695105038983e-19)
--(axis cs:0.0066,1.08755078580298e-17)
--(axis cs:0.0068,1.08334855272526e-16)
--(axis cs:0.007,9.69782013606882e-16)
--(axis cs:0.0072,7.84830360507471e-15)
--(axis cs:0.0074,5.7738733862825e-14)
--(axis cs:0.0076,3.88106550443288e-13)
--(axis cs:0.0078,2.39473656171717e-12)
--(axis cs:0.008,1.36227283385691e-11)
--(axis cs:0.0082,7.17311040949162e-11)
--(axis cs:0.0084,3.50913326863578e-10)
--(axis cs:0.0086,1.6004392380367e-09)
--(axis cs:0.0088,6.82684778432269e-09)
--(axis cs:0.009,2.73176680030113e-08)
--(axis cs:0.0092,1.02831056135345e-07)
--(axis cs:0.0094,3.65088994324908e-07)
--(axis cs:0.0096,1.22555178042531e-06)
--(axis cs:0.0098,3.89871873397428e-06)
--(axis cs:0.01,1.17789817815586e-05)
--(axis cs:0.0102,3.3866611040304e-05)
--(axis cs:0.0104,9.28421083067902e-05)
--(axis cs:0.0106,0.000243114650517988)
--(axis cs:0.0108,0.000281824066500681)
--(axis cs:0.011,0.000129491710086973)
--(axis cs:0.0112,5.88161400591037e-05)
--(axis cs:0.0114,2.64188958878439e-05)
--(axis cs:0.0116,1.17397891884797e-05)
--(axis cs:0.0118,5.16284297470597e-06)
--(axis cs:0.012,2.24775082783557e-06)
--(axis cs:0.0122,9.69121820503786e-07)
--(axis cs:0.0124,4.13916758159382e-07)
--(axis cs:0.0126,1.75177831434875e-07)
--(axis cs:0.0128,7.34848547104344e-08)
--(axis cs:0.013,3.05621933299358e-08)
--(axis cs:0.0132,1.26051877673596e-08)
--(axis cs:0.0134,5.1570032813465e-09)
--(axis cs:0.0136,2.0932893944091e-09)
--(axis cs:0.0138,8.43220198538763e-10)
--(axis cs:0.014,3.37151058369145e-10)
--(axis cs:0.0142,1.33834435916904e-10)
--(axis cs:0.0144,5.27539961091201e-11)
--(axis cs:0.0146,2.06522717779017e-11)
--(axis cs:0.0148,8.03124382031559e-12)
--(axis cs:0.015,3.10294950655172e-12)
--(axis cs:0.0152,1.19128308015436e-12)
--(axis cs:0.0154,4.54540149913224e-13)
--(axis cs:0.0156,1.72390181140141e-13)
--(axis cs:0.0158,6.49978903194613e-14)
--(axis cs:0.016,2.43665206508037e-14)
--(axis cs:0.0162,9.08351819191971e-15)
--(axis cs:0.0164,3.36773254243571e-15)
--(axis cs:0.0166,1.24193420842092e-15)
--(axis cs:0.0168,4.55606253832252e-16)
--(axis cs:0.017,1.66288257913556e-16)
--(axis cs:0.0172,6.03897535029679e-17)
--(axis cs:0.0174,2.1824381482602e-17)
--(axis cs:0.0176,7.84952309420417e-18)
--(axis cs:0.0178,2.81003232437693e-18)
--(axis cs:0.018,1.00135639304338e-18)
--(axis cs:0.0182,3.55235641677016e-19)
--(axis cs:0.0184,1.25468504813723e-19)
--(axis cs:0.0186,4.4124701636192e-20)
--(axis cs:0.0188,1.54523668801194e-20)
--(axis cs:0.019,5.38902904610328e-21)
--(axis cs:0.0192,1.87181662615934e-21)
--(axis cs:0.0194,6.47570756759223e-22)
--(axis cs:0.0196,2.23159383585354e-22)
--(axis cs:0.0198,7.66088454499077e-23)
--(axis cs:0.02,2.62004893068498e-23)
--(axis cs:0.0202,8.92763785971532e-24)
--(axis cs:0.0204,3.03102471455776e-24)
--(axis cs:0.0206,1.02540718757071e-24)
--(axis cs:0.0208,3.45688238253561e-25)
--(axis cs:0.021,1.1613973381258e-25)
--(axis cs:0.0212,3.88875765368101e-26)
--(axis cs:0.0214,1.29777643437928e-26)
--(axis cs:0.0216,4.31690271137589e-27)
--(axis cs:0.0218,1.43136948614258e-27)
--(axis cs:0.022,4.7310926805422e-28)
--(axis cs:0.0222,1.55891999605845e-28)
--(axis cs:0.0224,5.12107018312939e-29)
--(axis cs:0.0226,1.67723311744901e-29)
--(axis cs:0.0228,5.47699873987991e-30)
--(axis cs:0.023,1.78331691502149e-30)
--(axis cs:0.0232,5.78989560321019e-31)
--(axis cs:0.0234,1.87451269749009e-31)
--(axis cs:0.0236,6.05201690680936e-32)
--(axis cs:0.0238,1.94860647371264e-32)
--(axis cs:0.024,6.25717244805115e-33)
--(axis cs:0.0242,2.0039152602773e-33)
--(axis cs:0.0244,6.40095343780388e-34)
--(axis cs:0.0246,2.0393440465356e-34)
--(axis cs:0.0248,6.48086350573449e-35)
--(axis cs:0.025,2.05441148839129e-35)
--(axis cs:0.0252,6.4963504541668e-36)
--(axis cs:0.0254,2.04924464279847e-36)
--(axis cs:0.0256,6.44874307040056e-37)
--(axis cs:0.0258,2.02454508309857e-37)
--(axis cs:0.026,6.34110320078355e-38)
--(axis cs:0.0262,1.98153040354014e-38)
--(axis cs:0.0264,6.17800786984118e-39)
--(axis cs:0.0266,1.92185633197062e-39)
--(axis cs:0.0268,5.96527926936136e-40)
--(axis cs:0.027,1.84752537803359e-40)
--(axis cs:0.0272,5.70968188505042e-41)
--(axis cs:0.0274,1.76078815528641e-41)
--(axis cs:0.0276,5.4186059585716e-42)
--(axis cs:0.0278,1.66404327671261e-42)
--(axis cs:0.028,5.09975510408039e-43)
--(axis cs:0.0282,1.5597411127119e-43)
--(axis cs:0.0284,4.76085349466722e-44)
--(axis cs:0.0286,1.45029581675223e-44)
--(axis cs:0.0288,4.40938493130639e-45)
--(axis cs:0.029,1.3380089717475e-45)
--(axis cs:0.0292,4.05237263667289e-46)
--(axis cs:0.0294,1.22500709199272e-46)
--(axis cs:0.0296,3.69620508597466e-47)
--(axis cs:0.0298,1.11319412170508e-47)
--(axis cs:0.03,3.34650985758505e-48)
--(axis cs:0.0302,1.00421907518375e-48)
--(axis cs:0.0304,3.00807455797737e-49)
--(axis cs:0.0306,8.99458118226471e-50)
--(axis cs:0.0308,2.68481148237362e-50)
--(axis cs:0.031,8.00009727731514e-51)
--(axis cs:0.0312,2.37976088390367e-51)
--(axis cs:0.0314,7.06701098749733e-52)
--(axis cs:0.0316,2.0951265513022e-52)
--(axis cs:0.0318,6.2010369128817e-53)
--(axis cs:0.032,1.83233680229848e-53)
--(axis cs:0.0322,5.40555705491807e-54)
--(axis cs:0.0324,1.59212391595571e-54)
--(axis cs:0.0326,4.68189317030407e-55)
--(axis cs:0.0328,1.37461535925932e-55)
--(axis cs:0.033,4.0296066309506e-56)
--(axis cs:0.0332,1.17943080786081e-56)
--(axis cs:0.0334,3.44680810294073e-57)
--(axis cs:0.0336,1.00577981451736e-57)
--(axis cs:0.0338,2.93046227125017e-58)
--(axis cs:0.034,8.52555945485244e-59)
--(axis cs:0.0342,2.4766759715603e-59)
--(axis cs:0.0344,7.184242020946e-60)
--(axis cs:0.0346,2.08096120856607e-60)
--(axis cs:0.0348,6.01899505301586e-61)
--(axis cs:0.035,1.73846745830563e-61)
--(axis cs:0.0352,5.01414895848345e-62)
--(axis cs:0.0354,1.4441802412468e-62)
--(axis cs:0.0356,4.15378859254108e-63)
--(axis cs:0.0358,1.1930851351353e-63)
--(axis cs:0.036,3.42221805823752e-64)
--(axis cs:0.0362,9.80298138120522e-65)
--(axis cs:0.0364,2.80432216538558e-65)
--(axis cs:0.0366,8.01164595555668e-66)
--(axis cs:0.0368,2.28583311371841e-66)
--(axis cs:0.037,6.51329797203411e-67)
--(axis cs:0.0372,1.85351319674645e-67)
--(axis cs:0.0374,5.26784882758811e-68)
--(axis cs:0.0376,1.4952654973008e-68)
--(axis cs:0.0378,4.23891920517638e-69)
--(axis cs:0.038,1.2001848555619e-69)
--(axis cs:0.0382,3.39392008576943e-70)
--(axis cs:0.0384,9.58561031136903e-71)
--(axis cs:0.0386,2.70400050940451e-71)
--(axis cs:0.0388,7.61845138342213e-72)
--(axis cs:0.039,2.14389538326699e-72)
--(axis cs:0.0392,6.02589274852925e-73)
--(axis cs:0.0394,1.69170264255059e-73)
--cycle;
\path [draw=none, fill=blue, fill opacity=0.5]
(axis cs:0,0)
--(axis cs:0.0002,7.07051949791944e-220)
--(axis cs:0.0004,3.35713224621954e-178)
--(axis cs:0.0006,4.55876472150082e-154)
--(axis cs:0.0008,4.06372766744917e-137)
--(axis cs:0.001,4.19927757126269e-124)
--(axis cs:0.0012,1.40567805458719e-113)
--(axis cs:0.0014,9.00643402552461e-105)
--(axis cs:0.0016,3.18925325198261e-97)
--(axis cs:0.0018,1.23690071259004e-90)
--(axis cs:0.002,8.38122637353824e-85)
--(axis cs:0.0022,1.38442183846587e-79)
--(axis cs:0.0024,7.12902189670533e-75)
--(axis cs:0.0026,1.3794522412844e-70)
--(axis cs:0.0028,1.15971222412979e-66)
--(axis cs:0.003,4.75273753659039e-63)
--(axis cs:0.0032,1.04179315001234e-59)
--(axis cs:0.0034,1.31771394306738e-56)
--(axis cs:0.0036,1.02415094765992e-53)
--(axis cs:0.0038,5.15560039198329e-51)
--(axis cs:0.004,1.75757764872752e-48)
--(axis cs:0.0042,4.2148638496568e-46)
--(axis cs:0.0044,7.3467295284001e-44)
--(axis cs:0.0046,9.57542435968325e-42)
--(axis cs:0.0048,9.56565522459126e-40)
--(axis cs:0.005,7.48492710605918e-38)
--(axis cs:0.0052,4.67617522469868e-36)
--(axis cs:0.0054,2.37244858785325e-34)
--(axis cs:0.0056,9.92369722206861e-33)
--(axis cs:0.0058,3.46894198997681e-31)
--(axis cs:0.006,1.02575999424703e-29)
--(axis cs:0.0062,2.59405173111155e-28)
--(axis cs:0.0064,5.66632176036028e-27)
--(axis cs:0.0066,1.07875184310474e-25)
--(axis cs:0.0068,1.80466875933379e-24)
--(axis cs:0.007,2.67286801654633e-23)
--(axis cs:0.0072,3.52886813743904e-22)
--(axis cs:0.0074,4.1792715817461e-21)
--(axis cs:0.0076,4.46564241233065e-20)
--(axis cs:0.0078,4.32813414557636e-19)
--(axis cs:0.008,3.82375668734256e-18)
--(axis cs:0.0082,3.09338087199957e-17)
--(axis cs:0.0084,2.3012710139591e-16)
--(axis cs:0.0086,1.58052451958739e-15)
--(axis cs:0.0088,1.00582834740469e-14)
--(axis cs:0.009,5.95139142432837e-14)
--(axis cs:0.0092,3.28450725704436e-13)
--(axis cs:0.0094,1.69580021234189e-12)
--(axis cs:0.0096,8.21383031334312e-12)
--(axis cs:0.0098,3.74215771035445e-11)
--(axis cs:0.01,1.60758325803006e-10)
--(axis cs:0.0102,6.52688622108691e-10)
--(axis cs:0.0104,2.50996299416417e-09)
--(axis cs:0.0106,9.16116446427591e-09)
--(axis cs:0.0108,3.17979701351146e-08)
--(axis cs:0.011,1.05150251268355e-07)
--(axis cs:0.0112,3.31846466176376e-07)
--(axis cs:0.0114,1.00113761423932e-06)
--(axis cs:0.0116,2.89172206747296e-06)
--(axis cs:0.0118,5.16284297470597e-06)
--(axis cs:0.012,2.24775082783557e-06)
--(axis cs:0.0122,9.69121820503786e-07)
--(axis cs:0.0124,4.13916758159382e-07)
--(axis cs:0.0126,1.75177831434875e-07)
--(axis cs:0.0128,7.34848547104344e-08)
--(axis cs:0.013,3.05621933299358e-08)
--(axis cs:0.0132,1.26051877673596e-08)
--(axis cs:0.0134,5.1570032813465e-09)
--(axis cs:0.0136,2.0932893944091e-09)
--(axis cs:0.0138,8.43220198538763e-10)
--(axis cs:0.014,3.37151058369145e-10)
--(axis cs:0.0142,1.33834435916904e-10)
--(axis cs:0.0144,5.27539961091201e-11)
--(axis cs:0.0146,2.06522717779017e-11)
--(axis cs:0.0148,8.03124382031559e-12)
--(axis cs:0.015,3.10294950655172e-12)
--(axis cs:0.0152,1.19128308015436e-12)
--(axis cs:0.0154,4.54540149913224e-13)
--(axis cs:0.0156,1.72390181140141e-13)
--(axis cs:0.0158,6.49978903194613e-14)
--(axis cs:0.016,2.43665206508037e-14)
--(axis cs:0.0162,9.08351819191971e-15)
--(axis cs:0.0164,3.36773254243571e-15)
--(axis cs:0.0166,1.24193420842092e-15)
--(axis cs:0.0168,4.55606253832252e-16)
--(axis cs:0.017,1.66288257913556e-16)
--(axis cs:0.0172,6.03897535029679e-17)
--(axis cs:0.0174,2.1824381482602e-17)
--(axis cs:0.0176,7.84952309420417e-18)
--(axis cs:0.0178,2.81003232437693e-18)
--(axis cs:0.018,1.00135639304338e-18)
--(axis cs:0.0182,3.55235641677016e-19)
--(axis cs:0.0184,1.25468504813723e-19)
--(axis cs:0.0186,4.4124701636192e-20)
--(axis cs:0.0188,1.54523668801194e-20)
--(axis cs:0.019,5.38902904610328e-21)
--(axis cs:0.0192,1.87181662615934e-21)
--(axis cs:0.0194,6.47570756759223e-22)
--(axis cs:0.0196,2.23159383585354e-22)
--(axis cs:0.0198,7.66088454499077e-23)
--(axis cs:0.02,2.62004893068498e-23)
--(axis cs:0.0202,8.92763785971532e-24)
--(axis cs:0.0204,3.03102471455776e-24)
--(axis cs:0.0206,1.02540718757071e-24)
--(axis cs:0.0208,3.45688238253561e-25)
--(axis cs:0.021,1.1613973381258e-25)
--(axis cs:0.0212,3.88875765368101e-26)
--(axis cs:0.0214,1.29777643437928e-26)
--(axis cs:0.0216,4.31690271137589e-27)
--(axis cs:0.0218,1.43136948614258e-27)
--(axis cs:0.022,4.7310926805422e-28)
--(axis cs:0.0222,1.55891999605845e-28)
--(axis cs:0.0224,5.12107018312939e-29)
--(axis cs:0.0226,1.67723311744901e-29)
--(axis cs:0.0228,5.47699873987991e-30)
--(axis cs:0.023,1.78331691502149e-30)
--(axis cs:0.0232,5.78989560321019e-31)
--(axis cs:0.0234,1.87451269749009e-31)
--(axis cs:0.0236,6.05201690680936e-32)
--(axis cs:0.0238,1.94860647371264e-32)
--(axis cs:0.024,6.25717244805115e-33)
--(axis cs:0.0242,2.0039152602773e-33)
--(axis cs:0.0244,6.40095343780388e-34)
--(axis cs:0.0246,2.0393440465356e-34)
--(axis cs:0.0248,6.48086350573449e-35)
--(axis cs:0.025,2.05441148839129e-35)
--(axis cs:0.0252,6.4963504541668e-36)
--(axis cs:0.0254,2.04924464279847e-36)
--(axis cs:0.0256,6.44874307040056e-37)
--(axis cs:0.0258,2.02454508309857e-37)
--(axis cs:0.026,6.34110320078355e-38)
--(axis cs:0.0262,1.98153040354014e-38)
--(axis cs:0.0264,6.17800786984118e-39)
--(axis cs:0.0266,1.92185633197062e-39)
--(axis cs:0.0268,5.96527926936136e-40)
--(axis cs:0.027,1.84752537803359e-40)
--(axis cs:0.0272,5.70968188505042e-41)
--(axis cs:0.0274,1.76078815528641e-41)
--(axis cs:0.0276,5.4186059585716e-42)
--(axis cs:0.0278,1.66404327671261e-42)
--(axis cs:0.028,5.09975510408039e-43)
--(axis cs:0.0282,1.5597411127119e-43)
--(axis cs:0.0284,4.76085349466722e-44)
--(axis cs:0.0286,1.45029581675223e-44)
--(axis cs:0.0288,4.40938493130639e-45)
--(axis cs:0.029,1.3380089717475e-45)
--(axis cs:0.0292,4.05237263667289e-46)
--(axis cs:0.0294,1.22500709199272e-46)
--(axis cs:0.0296,3.69620508597466e-47)
--(axis cs:0.0298,1.11319412170508e-47)
--(axis cs:0.03,3.34650985758505e-48)
--(axis cs:0.0302,1.00421907518375e-48)
--(axis cs:0.0304,3.00807455797737e-49)
--(axis cs:0.0306,8.99458118226471e-50)
--(axis cs:0.0308,2.68481148237362e-50)
--(axis cs:0.031,8.00009727731514e-51)
--(axis cs:0.0312,2.37976088390367e-51)
--(axis cs:0.0314,7.06701098749733e-52)
--(axis cs:0.0316,2.0951265513022e-52)
--(axis cs:0.0318,6.2010369128817e-53)
--(axis cs:0.032,1.83233680229848e-53)
--(axis cs:0.0322,5.40555705491807e-54)
--(axis cs:0.0324,1.59212391595571e-54)
--(axis cs:0.0326,4.68189317030407e-55)
--(axis cs:0.0328,1.37461535925932e-55)
--(axis cs:0.033,4.0296066309506e-56)
--(axis cs:0.0332,1.17943080786081e-56)
--(axis cs:0.0334,3.44680810294073e-57)
--(axis cs:0.0336,1.00577981451736e-57)
--(axis cs:0.0338,2.93046227125017e-58)
--(axis cs:0.034,8.52555945485244e-59)
--(axis cs:0.0342,2.4766759715603e-59)
--(axis cs:0.0344,7.184242020946e-60)
--(axis cs:0.0346,2.08096120856607e-60)
--(axis cs:0.0348,6.01899505301586e-61)
--(axis cs:0.035,1.73846745830563e-61)
--(axis cs:0.0352,5.01414895848345e-62)
--(axis cs:0.0354,1.4441802412468e-62)
--(axis cs:0.0356,4.15378859254108e-63)
--(axis cs:0.0358,1.1930851351353e-63)
--(axis cs:0.036,3.42221805823752e-64)
--(axis cs:0.0362,9.80298138120522e-65)
--(axis cs:0.0364,2.80432216538558e-65)
--(axis cs:0.0366,8.01164595555668e-66)
--(axis cs:0.0368,2.28583311371841e-66)
--(axis cs:0.037,6.51329797203411e-67)
--(axis cs:0.0372,1.85351319674645e-67)
--(axis cs:0.0374,5.26784882758811e-68)
--(axis cs:0.0376,1.4952654973008e-68)
--(axis cs:0.0378,4.23891920517638e-69)
--(axis cs:0.038,1.2001848555619e-69)
--(axis cs:0.0382,3.39392008576943e-70)
--(axis cs:0.0384,9.58561031136903e-71)
--(axis cs:0.0386,2.70400050940451e-71)
--(axis cs:0.0388,7.61845138342213e-72)
--(axis cs:0.039,2.14389538326699e-72)
--(axis cs:0.0392,6.02589274852925e-73)
--(axis cs:0.0394,1.69170264255059e-73)
--cycle;
\path [draw=none, fill=blue, fill opacity=0.5]
(axis cs:0,0)
--(axis cs:0.0002,1.59862054051401e-63)
--(axis cs:0.0004,2.01433889039574e-47)
--(axis cs:0.0006,2.97077629654634e-38)
--(axis cs:0.0008,6.36171904282589e-32)
--(axis cs:0.001,3.79438340819427e-27)
--(axis cs:0.0012,2.34966409367811e-23)
--(axis cs:0.0014,3.04782019614733e-20)
--(axis cs:0.0016,1.25904990534363e-17)
--(axis cs:0.0018,2.17034474693635e-15)
--(axis cs:0.002,1.87750330629904e-13)
--(axis cs:0.0022,9.2972654226903e-12)
--(axis cs:0.0024,2.90438635134726e-10)
--(axis cs:0.0026,6.16207408914511e-09)
--(axis cs:0.0028,9.40339740634871e-08)
--(axis cs:0.003,1.0801313742615e-06)
--(axis cs:0.0032,9.68774339936991e-06)
--(axis cs:0.0034,6.99108734785651e-05)
--(axis cs:0.0036,0.000416130640341037)
--(axis cs:0.0038,0.00208599494885405)
--(axis cs:0.004,0.00896271696907974)
--(axis cs:0.0042,0.0335068515028188)
--(axis cs:0.0044,0.110410042257394)
--(axis cs:0.0046,0.32428663564024)
--(axis cs:0.0048,0.857310496430777)
--(axis cs:0.005,2.0575910593128)
--(axis cs:0.0052,4.51727747541394)
--(axis cs:0.0054,9.13280377204685)
--(axis cs:0.0056,17.105512518158)
--(axis cs:0.0058,29.8396920784007)
--(axis cs:0.006,48.7151828613912)
--(axis cs:0.0062,74.7526085244504)
--(axis cs:0.0064,108.238434806541)
--(axis cs:0.0066,85.2442631844142)
--(axis cs:0.0068,58.2787711177048)
--(axis cs:0.007,38.6762174669473)
--(axis cs:0.0072,24.9566689864978)
--(axis cs:0.0074,15.6819348518085)
--(axis cs:0.0076,9.6092950684639)
--(axis cs:0.0078,5.74941026071925)
--(axis cs:0.008,3.36289542641486)
--(axis cs:0.0082,1.92504865939838)
--(axis cs:0.0084,1.07957702041241)
--(axis cs:0.0086,0.593694418792337)
--(axis cs:0.0088,0.320446113786265)
--(axis cs:0.009,0.169898314826288)
--(axis cs:0.0092,0.0885524876070683)
--(axis cs:0.0094,0.0454050383416075)
--(axis cs:0.0096,0.0229188611207476)
--(axis cs:0.0098,0.0113957692002469)
--(axis cs:0.01,0.00558489580279418)
--(axis cs:0.0102,0.00269929974117084)
--(axis cs:0.0104,0.00128730466490826)
--(axis cs:0.0106,0.000606069215412055)
--(axis cs:0.0108,0.000281824066500681)
--(axis cs:0.011,0.000129491710086973)
--(axis cs:0.0112,5.88161400591037e-05)
--(axis cs:0.0114,2.64188958878439e-05)
--(axis cs:0.0116,1.17397891884797e-05)
--(axis cs:0.0118,5.16284297470597e-06)
--(axis cs:0.012,2.24775082783557e-06)
--(axis cs:0.0122,9.69121820503786e-07)
--(axis cs:0.0124,4.13916758159382e-07)
--(axis cs:0.0126,1.75177831434875e-07)
--(axis cs:0.0128,7.34848547104344e-08)
--(axis cs:0.013,3.05621933299358e-08)
--(axis cs:0.0132,1.26051877673596e-08)
--(axis cs:0.0134,5.1570032813465e-09)
--(axis cs:0.0136,2.0932893944091e-09)
--(axis cs:0.0138,8.43220198538763e-10)
--(axis cs:0.014,3.37151058369145e-10)
--(axis cs:0.0142,1.33834435916904e-10)
--(axis cs:0.0144,5.27539961091201e-11)
--(axis cs:0.0146,2.06522717779017e-11)
--(axis cs:0.0148,8.03124382031559e-12)
--(axis cs:0.015,3.10294950655172e-12)
--(axis cs:0.0152,1.19128308015436e-12)
--(axis cs:0.0154,4.54540149913224e-13)
--(axis cs:0.0156,1.72390181140141e-13)
--(axis cs:0.0158,6.49978903194613e-14)
--(axis cs:0.016,2.43665206508037e-14)
--(axis cs:0.0162,9.08351819191971e-15)
--(axis cs:0.0164,3.36773254243571e-15)
--(axis cs:0.0166,1.24193420842092e-15)
--(axis cs:0.0168,4.55606253832252e-16)
--(axis cs:0.017,1.66288257913556e-16)
--(axis cs:0.0172,6.03897535029679e-17)
--(axis cs:0.0174,2.1824381482602e-17)
--(axis cs:0.0176,7.84952309420417e-18)
--(axis cs:0.0178,2.81003232437693e-18)
--(axis cs:0.018,1.00135639304338e-18)
--(axis cs:0.0182,3.55235641677016e-19)
--(axis cs:0.0184,1.25468504813723e-19)
--(axis cs:0.0186,4.4124701636192e-20)
--(axis cs:0.0188,1.54523668801194e-20)
--(axis cs:0.019,5.38902904610328e-21)
--(axis cs:0.0192,1.87181662615934e-21)
--(axis cs:0.0194,6.47570756759223e-22)
--(axis cs:0.0196,2.23159383585354e-22)
--(axis cs:0.0198,7.66088454499077e-23)
--(axis cs:0.02,2.62004893068498e-23)
--(axis cs:0.0202,8.92763785971532e-24)
--(axis cs:0.0204,3.03102471455776e-24)
--(axis cs:0.0206,1.02540718757071e-24)
--(axis cs:0.0208,3.45688238253561e-25)
--(axis cs:0.021,1.1613973381258e-25)
--(axis cs:0.0212,3.88875765368101e-26)
--(axis cs:0.0214,1.29777643437928e-26)
--(axis cs:0.0216,4.31690271137589e-27)
--(axis cs:0.0218,1.43136948614258e-27)
--(axis cs:0.022,4.7310926805422e-28)
--(axis cs:0.0222,1.55891999605845e-28)
--(axis cs:0.0224,5.12107018312939e-29)
--(axis cs:0.0226,1.67723311744901e-29)
--(axis cs:0.0228,5.47699873987991e-30)
--(axis cs:0.023,1.78331691502149e-30)
--(axis cs:0.0232,5.78989560321019e-31)
--(axis cs:0.0234,1.87451269749009e-31)
--(axis cs:0.0236,6.05201690680936e-32)
--(axis cs:0.0238,1.94860647371264e-32)
--(axis cs:0.024,6.25717244805115e-33)
--(axis cs:0.0242,2.0039152602773e-33)
--(axis cs:0.0244,6.40095343780388e-34)
--(axis cs:0.0246,2.0393440465356e-34)
--(axis cs:0.0248,6.48086350573449e-35)
--(axis cs:0.025,2.05441148839129e-35)
--(axis cs:0.0252,6.4963504541668e-36)
--(axis cs:0.0254,2.04924464279847e-36)
--(axis cs:0.0256,6.44874307040056e-37)
--(axis cs:0.0258,2.02454508309857e-37)
--(axis cs:0.026,6.34110320078355e-38)
--(axis cs:0.0262,1.98153040354014e-38)
--(axis cs:0.0264,6.17800786984118e-39)
--(axis cs:0.0266,1.92185633197062e-39)
--(axis cs:0.0268,5.96527926936136e-40)
--(axis cs:0.027,1.84752537803359e-40)
--(axis cs:0.0272,5.70968188505042e-41)
--(axis cs:0.0274,1.76078815528641e-41)
--(axis cs:0.0276,5.4186059585716e-42)
--(axis cs:0.0278,1.66404327671261e-42)
--(axis cs:0.028,5.09975510408039e-43)
--(axis cs:0.0282,1.5597411127119e-43)
--(axis cs:0.0284,4.76085349466722e-44)
--(axis cs:0.0286,1.45029581675223e-44)
--(axis cs:0.0288,4.40938493130639e-45)
--(axis cs:0.029,1.3380089717475e-45)
--(axis cs:0.0292,4.05237263667289e-46)
--(axis cs:0.0294,1.22500709199272e-46)
--(axis cs:0.0296,3.69620508597466e-47)
--(axis cs:0.0298,1.11319412170508e-47)
--(axis cs:0.03,3.34650985758505e-48)
--(axis cs:0.0302,1.00421907518375e-48)
--(axis cs:0.0304,3.00807455797737e-49)
--(axis cs:0.0306,8.99458118226471e-50)
--(axis cs:0.0308,2.68481148237362e-50)
--(axis cs:0.031,8.00009727731514e-51)
--(axis cs:0.0312,2.37976088390367e-51)
--(axis cs:0.0314,7.06701098749733e-52)
--(axis cs:0.0316,2.0951265513022e-52)
--(axis cs:0.0318,6.2010369128817e-53)
--(axis cs:0.032,1.83233680229848e-53)
--(axis cs:0.0322,5.40555705491807e-54)
--(axis cs:0.0324,1.59212391595571e-54)
--(axis cs:0.0326,4.68189317030407e-55)
--(axis cs:0.0328,1.37461535925932e-55)
--(axis cs:0.033,4.0296066309506e-56)
--(axis cs:0.0332,1.17943080786081e-56)
--(axis cs:0.0334,3.44680810294073e-57)
--(axis cs:0.0336,1.00577981451736e-57)
--(axis cs:0.0338,2.93046227125017e-58)
--(axis cs:0.034,8.52555945485244e-59)
--(axis cs:0.0342,2.4766759715603e-59)
--(axis cs:0.0344,7.184242020946e-60)
--(axis cs:0.0346,2.08096120856607e-60)
--(axis cs:0.0348,6.01899505301586e-61)
--(axis cs:0.035,1.73846745830563e-61)
--(axis cs:0.0352,5.01414895848345e-62)
--(axis cs:0.0354,1.4441802412468e-62)
--(axis cs:0.0356,4.15378859254108e-63)
--(axis cs:0.0358,1.1930851351353e-63)
--(axis cs:0.036,3.42221805823752e-64)
--(axis cs:0.0362,9.80298138120522e-65)
--(axis cs:0.0364,2.80432216538558e-65)
--(axis cs:0.0366,8.01164595555668e-66)
--(axis cs:0.0368,2.28583311371841e-66)
--(axis cs:0.037,6.51329797203411e-67)
--(axis cs:0.0372,1.85351319674645e-67)
--(axis cs:0.0374,5.26784882758811e-68)
--(axis cs:0.0376,1.4952654973008e-68)
--(axis cs:0.0378,4.23891920517638e-69)
--(axis cs:0.038,1.2001848555619e-69)
--(axis cs:0.0382,3.39392008576943e-70)
--(axis cs:0.0384,9.58561031136903e-71)
--(axis cs:0.0386,2.70400050940451e-71)
--(axis cs:0.0388,7.61845138342213e-72)
--(axis cs:0.039,2.14389538326699e-72)
--(axis cs:0.0392,6.02589274852925e-73)
--(axis cs:0.0394,1.69170264255059e-73)
--cycle;
\path [draw=none, fill=blue, fill opacity=0.5]
(axis cs:0,0)
--(axis cs:0.0002,2.55329303009598e-95)
--(axis cs:0.0004,1.68182412350504e-73)
--(axis cs:0.0006,5.49781289889967e-61)
--(axis cs:0.0008,2.78718791670709e-52)
--(axis cs:0.001,1.15539866519647e-45)
--(axis cs:0.0012,2.29045339557146e-40)
--(axis cs:0.0014,5.57050636268994e-36)
--(axis cs:0.0016,2.91664333325709e-32)
--(axis cs:0.0018,4.72521876543319e-29)
--(axis cs:0.002,3.03440752670528e-26)
--(axis cs:0.0022,9.2163699131261e-24)
--(axis cs:0.0024,1.50843826973494e-21)
--(axis cs:0.0026,1.46891166386691e-19)
--(axis cs:0.0028,9.19186255467381e-18)
--(axis cs:0.003,3.92879593928923e-16)
--(axis cs:0.0032,1.20485095111868e-14)
--(axis cs:0.0034,2.75996884779829e-13)
--(axis cs:0.0036,4.8825997596155e-12)
--(axis cs:0.0038,6.85966297166094e-11)
--(axis cs:0.004,7.83646193730588e-10)
--(axis cs:0.0042,7.42781149314522e-09)
--(axis cs:0.0044,5.94374689671225e-08)
--(axis cs:0.0046,4.07612920102254e-07)
--(axis cs:0.0048,2.42727662171799e-06)
--(axis cs:0.005,1.26961478962858e-05)
--(axis cs:0.0052,5.89271481250347e-05)
--(axis cs:0.0054,0.00024488352824238)
--(axis cs:0.0056,0.000918517912095324)
--(axis cs:0.0058,0.00313195835741374)
--(axis cs:0.006,0.00977111118024442)
--(axis cs:0.0062,0.0280540417034728)
--(axis cs:0.0064,0.0745167406233305)
--(axis cs:0.0066,0.183988958680021)
--(axis cs:0.0068,0.424127151084662)
--(axis cs:0.007,0.916409463495715)
--(axis cs:0.0072,1.86272770489421)
--(axis cs:0.0074,3.57374020060483)
--(axis cs:0.0076,6.49145456201812)
--(axis cs:0.0078,5.74941026071925)
--(axis cs:0.008,3.36289542641486)
--(axis cs:0.0082,1.92504865939838)
--(axis cs:0.0084,1.07957702041241)
--(axis cs:0.0086,0.593694418792337)
--(axis cs:0.0088,0.320446113786265)
--(axis cs:0.009,0.169898314826288)
--(axis cs:0.0092,0.0885524876070683)
--(axis cs:0.0094,0.0454050383416075)
--(axis cs:0.0096,0.0229188611207476)
--(axis cs:0.0098,0.0113957692002469)
--(axis cs:0.01,0.00558489580279418)
--(axis cs:0.0102,0.00269929974117084)
--(axis cs:0.0104,0.00128730466490826)
--(axis cs:0.0106,0.000606069215412055)
--(axis cs:0.0108,0.000281824066500681)
--(axis cs:0.011,0.000129491710086973)
--(axis cs:0.0112,5.88161400591037e-05)
--(axis cs:0.0114,2.64188958878439e-05)
--(axis cs:0.0116,1.17397891884797e-05)
--(axis cs:0.0118,5.16284297470597e-06)
--(axis cs:0.012,2.24775082783557e-06)
--(axis cs:0.0122,9.69121820503786e-07)
--(axis cs:0.0124,4.13916758159382e-07)
--(axis cs:0.0126,1.75177831434875e-07)
--(axis cs:0.0128,7.34848547104344e-08)
--(axis cs:0.013,3.05621933299358e-08)
--(axis cs:0.0132,1.26051877673596e-08)
--(axis cs:0.0134,5.1570032813465e-09)
--(axis cs:0.0136,2.0932893944091e-09)
--(axis cs:0.0138,8.43220198538763e-10)
--(axis cs:0.014,3.37151058369145e-10)
--(axis cs:0.0142,1.33834435916904e-10)
--(axis cs:0.0144,5.27539961091201e-11)
--(axis cs:0.0146,2.06522717779017e-11)
--(axis cs:0.0148,8.03124382031559e-12)
--(axis cs:0.015,3.10294950655172e-12)
--(axis cs:0.0152,1.19128308015436e-12)
--(axis cs:0.0154,4.54540149913224e-13)
--(axis cs:0.0156,1.72390181140141e-13)
--(axis cs:0.0158,6.49978903194613e-14)
--(axis cs:0.016,2.43665206508037e-14)
--(axis cs:0.0162,9.08351819191971e-15)
--(axis cs:0.0164,3.36773254243571e-15)
--(axis cs:0.0166,1.24193420842092e-15)
--(axis cs:0.0168,4.55606253832252e-16)
--(axis cs:0.017,1.66288257913556e-16)
--(axis cs:0.0172,6.03897535029679e-17)
--(axis cs:0.0174,2.1824381482602e-17)
--(axis cs:0.0176,7.84952309420417e-18)
--(axis cs:0.0178,2.81003232437693e-18)
--(axis cs:0.018,1.00135639304338e-18)
--(axis cs:0.0182,3.55235641677016e-19)
--(axis cs:0.0184,1.25468504813723e-19)
--(axis cs:0.0186,4.4124701636192e-20)
--(axis cs:0.0188,1.54523668801194e-20)
--(axis cs:0.019,5.38902904610328e-21)
--(axis cs:0.0192,1.87181662615934e-21)
--(axis cs:0.0194,6.47570756759223e-22)
--(axis cs:0.0196,2.23159383585354e-22)
--(axis cs:0.0198,7.66088454499077e-23)
--(axis cs:0.02,2.62004893068498e-23)
--(axis cs:0.0202,8.92763785971532e-24)
--(axis cs:0.0204,3.03102471455776e-24)
--(axis cs:0.0206,1.02540718757071e-24)
--(axis cs:0.0208,3.45688238253561e-25)
--(axis cs:0.021,1.1613973381258e-25)
--(axis cs:0.0212,3.88875765368101e-26)
--(axis cs:0.0214,1.29777643437928e-26)
--(axis cs:0.0216,4.31690271137589e-27)
--(axis cs:0.0218,1.43136948614258e-27)
--(axis cs:0.022,4.7310926805422e-28)
--(axis cs:0.0222,1.55891999605845e-28)
--(axis cs:0.0224,5.12107018312939e-29)
--(axis cs:0.0226,1.67723311744901e-29)
--(axis cs:0.0228,5.47699873987991e-30)
--(axis cs:0.023,1.78331691502149e-30)
--(axis cs:0.0232,5.78989560321019e-31)
--(axis cs:0.0234,1.87451269749009e-31)
--(axis cs:0.0236,6.05201690680936e-32)
--(axis cs:0.0238,1.94860647371264e-32)
--(axis cs:0.024,6.25717244805115e-33)
--(axis cs:0.0242,2.0039152602773e-33)
--(axis cs:0.0244,6.40095343780388e-34)
--(axis cs:0.0246,2.0393440465356e-34)
--(axis cs:0.0248,6.48086350573449e-35)
--(axis cs:0.025,2.05441148839129e-35)
--(axis cs:0.0252,6.4963504541668e-36)
--(axis cs:0.0254,2.04924464279847e-36)
--(axis cs:0.0256,6.44874307040056e-37)
--(axis cs:0.0258,2.02454508309857e-37)
--(axis cs:0.026,6.34110320078355e-38)
--(axis cs:0.0262,1.98153040354014e-38)
--(axis cs:0.0264,6.17800786984118e-39)
--(axis cs:0.0266,1.92185633197062e-39)
--(axis cs:0.0268,5.96527926936136e-40)
--(axis cs:0.027,1.84752537803359e-40)
--(axis cs:0.0272,5.70968188505042e-41)
--(axis cs:0.0274,1.76078815528641e-41)
--(axis cs:0.0276,5.4186059585716e-42)
--(axis cs:0.0278,1.66404327671261e-42)
--(axis cs:0.028,5.09975510408039e-43)
--(axis cs:0.0282,1.5597411127119e-43)
--(axis cs:0.0284,4.76085349466722e-44)
--(axis cs:0.0286,1.45029581675223e-44)
--(axis cs:0.0288,4.40938493130639e-45)
--(axis cs:0.029,1.3380089717475e-45)
--(axis cs:0.0292,4.05237263667289e-46)
--(axis cs:0.0294,1.22500709199272e-46)
--(axis cs:0.0296,3.69620508597466e-47)
--(axis cs:0.0298,1.11319412170508e-47)
--(axis cs:0.03,3.34650985758505e-48)
--(axis cs:0.0302,1.00421907518375e-48)
--(axis cs:0.0304,3.00807455797737e-49)
--(axis cs:0.0306,8.99458118226471e-50)
--(axis cs:0.0308,2.68481148237362e-50)
--(axis cs:0.031,8.00009727731514e-51)
--(axis cs:0.0312,2.37976088390367e-51)
--(axis cs:0.0314,7.06701098749733e-52)
--(axis cs:0.0316,2.0951265513022e-52)
--(axis cs:0.0318,6.2010369128817e-53)
--(axis cs:0.032,1.83233680229848e-53)
--(axis cs:0.0322,5.40555705491807e-54)
--(axis cs:0.0324,1.59212391595571e-54)
--(axis cs:0.0326,4.68189317030407e-55)
--(axis cs:0.0328,1.37461535925932e-55)
--(axis cs:0.033,4.0296066309506e-56)
--(axis cs:0.0332,1.17943080786081e-56)
--(axis cs:0.0334,3.44680810294073e-57)
--(axis cs:0.0336,1.00577981451736e-57)
--(axis cs:0.0338,2.93046227125017e-58)
--(axis cs:0.034,8.52555945485244e-59)
--(axis cs:0.0342,2.4766759715603e-59)
--(axis cs:0.0344,7.184242020946e-60)
--(axis cs:0.0346,2.08096120856607e-60)
--(axis cs:0.0348,6.01899505301586e-61)
--(axis cs:0.035,1.73846745830563e-61)
--(axis cs:0.0352,5.01414895848345e-62)
--(axis cs:0.0354,1.4441802412468e-62)
--(axis cs:0.0356,4.15378859254108e-63)
--(axis cs:0.0358,1.1930851351353e-63)
--(axis cs:0.036,3.42221805823752e-64)
--(axis cs:0.0362,9.80298138120522e-65)
--(axis cs:0.0364,2.80432216538558e-65)
--(axis cs:0.0366,8.01164595555668e-66)
--(axis cs:0.0368,2.28583311371841e-66)
--(axis cs:0.037,6.51329797203411e-67)
--(axis cs:0.0372,1.85351319674645e-67)
--(axis cs:0.0374,5.26784882758811e-68)
--(axis cs:0.0376,1.4952654973008e-68)
--(axis cs:0.0378,4.23891920517638e-69)
--(axis cs:0.038,1.2001848555619e-69)
--(axis cs:0.0382,3.39392008576943e-70)
--(axis cs:0.0384,9.58561031136903e-71)
--(axis cs:0.0386,2.70400050940451e-71)
--(axis cs:0.0388,7.61845138342213e-72)
--(axis cs:0.039,2.14389538326699e-72)
--(axis cs:0.0392,6.02589274852925e-73)
--(axis cs:0.0394,1.69170264255059e-73)
--cycle;
\path [draw=none, fill=blue, fill opacity=0.5]
(axis cs:0,0)
--(axis cs:0.0002,6.88149847089383e-283)
--(axis cs:0.0004,5.38329324547414e-232)
--(axis cs:0.0006,1.80526708928995e-202)
--(axis cs:0.0008,1.0802222727238e-181)
--(axis cs:0.001,1.03927455538975e-165)
--(axis cs:0.0012,9.284355067112e-153)
--(axis cs:0.0014,6.70264999764463e-142)
--(axis cs:0.0016,1.42270014450123e-132)
--(axis cs:0.0018,2.04241209795704e-124)
--(axis cs:0.002,3.50239373677996e-117)
--(axis cs:0.0022,1.0763994483646e-110)
--(axis cs:0.0024,7.99953899528515e-105)
--(axis cs:0.0026,1.80482808273425e-99)
--(axis cs:0.0028,1.47530594980937e-94)
--(axis cs:0.003,5.0266949465783e-90)
--(axis cs:0.0032,7.99309022920627e-86)
--(axis cs:0.0034,6.50610273957532e-82)
--(axis cs:0.0036,2.92653432604146e-78)
--(axis cs:0.0038,7.75652633853781e-75)
--(axis cs:0.004,1.27887150345601e-71)
--(axis cs:0.0042,1.37387136164765e-68)
--(axis cs:0.0044,1.00075911725486e-65)
--(axis cs:0.0046,5.11650284411508e-63)
--(axis cs:0.0048,1.89215762100807e-60)
--(axis cs:0.005,5.19708066546697e-58)
--(axis cs:0.0052,1.08519112913327e-55)
--(axis cs:0.0054,1.75864114867945e-53)
--(axis cs:0.0056,2.25305595468255e-51)
--(axis cs:0.0058,2.31978418006334e-49)
--(axis cs:0.006,1.94819763121604e-47)
--(axis cs:0.0062,1.35246061951633e-45)
--(axis cs:0.0064,7.85536854187431e-44)
--(axis cs:0.0066,3.85939041326377e-42)
--(axis cs:0.0068,1.61999834203236e-40)
--(axis cs:0.007,5.8628757428418e-39)
--(axis cs:0.0072,1.84471455956781e-37)
--(axis cs:0.0074,5.08503961140203e-36)
--(axis cs:0.0076,1.23670034865055e-34)
--(axis cs:0.0078,2.67091172323767e-33)
--(axis cs:0.008,5.15329732842381e-32)
--(axis cs:0.0082,8.93211136472859e-31)
--(axis cs:0.0084,1.39799146700368e-29)
--(axis cs:0.0086,1.98525615382389e-28)
--(axis cs:0.0088,2.56937152193057e-27)
--(axis cs:0.009,3.04327839977601e-26)
--(axis cs:0.0092,3.31167624895744e-25)
--(axis cs:0.0094,3.32294847816438e-24)
--(axis cs:0.0096,3.08494431084593e-23)
--(axis cs:0.0098,2.65831236795176e-22)
--(axis cs:0.01,2.1325663849262e-21)
--(axis cs:0.0102,1.59721161269042e-20)
--(axis cs:0.0104,1.11979829006939e-19)
--(axis cs:0.0106,7.36754354703253e-19)
--(axis cs:0.0108,4.55972127990545e-18)
--(axis cs:0.011,2.66046231163419e-17)
--(axis cs:0.0112,1.46655123482879e-16)
--(axis cs:0.0114,7.65293478264192e-16)
--(axis cs:0.0116,3.7876809871124e-15)
--(axis cs:0.0118,1.78120985335917e-14)
--(axis cs:0.012,7.97251566953794e-14)
--(axis cs:0.0122,3.4018858950948e-13)
--(axis cs:0.0124,1.38598721522646e-12)
--(axis cs:0.0126,5.39948902691054e-12)
--(axis cs:0.0128,2.01422837127883e-11)
--(axis cs:0.013,7.20456391440064e-11)
--(axis cs:0.0132,2.47402100952021e-10)
--(axis cs:0.0134,8.16628494582459e-10)
--(axis cs:0.0136,2.59403838921817e-09)
--(axis cs:0.0138,7.93856711154157e-09)
--(axis cs:0.014,5.79914781556753e-09)
--(axis cs:0.0142,2.39742918516806e-09)
--(axis cs:0.0144,9.83622316412938e-10)
--(axis cs:0.0146,4.00589423659466e-10)
--(axis cs:0.0148,1.61973166427244e-10)
--(axis cs:0.015,6.50338909349618e-11)
--(axis cs:0.0152,2.59338417965968e-11)
--(axis cs:0.0154,1.02730351130093e-11)
--(axis cs:0.0156,4.04302985151058e-12)
--(axis cs:0.0158,1.58110282220569e-12)
--(axis cs:0.016,6.14503562618277e-13)
--(axis cs:0.0162,2.37390584422077e-13)
--(axis cs:0.0164,9.1167215395404e-14)
--(axis cs:0.0166,3.48104066439517e-14)
--(axis cs:0.0168,1.32169630292309e-14)
--(axis cs:0.017,4.99069887932052e-15)
--(axis cs:0.0172,1.87435212674958e-15)
--(axis cs:0.0174,7.00248674727646e-16)
--(axis cs:0.0176,2.60264010969948e-16)
--(axis cs:0.0178,9.62463769271557e-17)
--(axis cs:0.018,3.54167940614942e-17)
--(axis cs:0.0182,1.29698003774648e-17)
--(axis cs:0.0184,4.7271530918389e-18)
--(axis cs:0.0186,1.71494467053479e-18)
--(axis cs:0.0188,6.19334149097625e-19)
--(axis cs:0.019,2.22670930626025e-19)
--(axis cs:0.0192,7.97082893690277e-20)
--(axis cs:0.0194,2.84106797975296e-20)
--(axis cs:0.0196,1.00840155791236e-20)
--(axis cs:0.0198,3.56445901111221e-21)
--(axis cs:0.02,1.25486042699487e-21)
--(axis cs:0.0202,4.40018920549441e-22)
--(axis cs:0.0204,1.53692577618315e-22)
--(axis cs:0.0206,5.34774408776781e-23)
--(axis cs:0.0208,1.85376292933918e-23)
--(axis cs:0.021,6.40224732564722e-24)
--(axis cs:0.0212,2.20309578999997e-24)
--(axis cs:0.0214,7.55412564088324e-25)
--(axis cs:0.0216,2.58113884798824e-25)
--(axis cs:0.0218,8.7890227501203e-26)
--(axis cs:0.022,2.98261281271697e-26)
--(axis cs:0.0222,1.00879882995762e-26)
--(axis cs:0.0224,3.40084889151921e-27)
--(axis cs:0.0226,1.14279470199847e-27)
--(axis cs:0.0228,3.82797991013378e-28)
--(axis cs:0.023,1.27824314593684e-28)
--(axis cs:0.0232,4.25520963387255e-29)
--(axis cs:0.0234,1.41225408013708e-29)
--(axis cs:0.0236,4.67314546043816e-30)
--(axis cs:0.0238,1.54180704611887e-30)
--(axis cs:0.024,5.07217470679387e-31)
--(axis cs:0.0242,1.66387440219997e-31)
--(axis cs:0.0244,5.4428614463191e-32)
--(axis cs:0.0246,1.77554717500868e-32)
--(axis cs:0.0248,5.77633952463157e-33)
--(axis cs:0.025,1.87415588932812e-33)
--(axis cs:0.0252,6.06467804373849e-34)
--(axis cs:0.0254,1.95737920655727e-34)
--(axis cs:0.0256,6.30119810772174e-35)
--(axis cs:0.0258,2.02333425052184e-35)
--(axis cs:0.026,6.48072365427423e-36)
--(axis cs:0.0262,2.07064396562142e-36)
--(axis cs:0.0264,6.59975431842602e-37)
--(axis cs:0.0266,2.09847858540786e-37)
--(axis cs:0.0268,6.65655398417971e-38)
--(axis cs:0.027,2.10657012662075e-38)
--(axis cs:0.0272,6.65115446267256e-39)
--(axis cs:0.0274,2.0952006066426e-39)
--(axis cs:0.0276,6.5852791193173e-40)
--(axis cs:0.0278,2.06516627344732e-40)
--(axis cs:0.028,6.46219586689164e-41)
--(axis cs:0.0282,2.0177214295937e-41)
--(axis cs:0.0284,6.28651246634623e-42)
--(axis cs:0.0286,1.95450635618193e-42)
--(axis cs:0.0288,6.06392937465401e-43)
--(axis cs:0.029,1.87746436861622e-43)
--(axis cs:0.0292,5.8009664131289e-44)
--(axis cs:0.0294,1.78875316998022e-44)
--(axis cs:0.0296,5.50467937292145e-45)
--(axis cs:0.0298,1.69065544747347e-45)
--(axis cs:0.03,5.18238148622274e-46)
--(axis cs:0.0302,1.58549314377919e-46)
--(axis cs:0.0304,4.84138269202927e-47)
--(axis cs:0.0306,1.47554910417118e-47)
--(axis cs:0.0308,4.48875706629181e-48)
--(axis cs:0.031,1.36299893327607e-48)
--(axis cs:0.0312,4.13114592599943e-49)
--(axis cs:0.0314,1.24985497245378e-49)
--(axis cs:0.0316,3.77460119646181e-50)
--(axis cs:0.0318,1.13792340123613e-50)
--(axis cs:0.032,3.4244708583103e-51)
--(axis cs:0.0322,1.02877463298859e-51)
--(axis cs:0.0324,3.08532582781497e-52)
--(axis cs:0.0326,9.2372645944798e-53)
--(axis cs:0.0328,2.7609255834434e-53)
--(axis cs:0.033,8.23838828545169e-54)
--(axis cs:0.0332,2.45421829531249e-54)
--(axis cs:0.0334,7.29918727184782e-55)
--(axis cs:0.0336,2.16737015903793e-55)
--(axis cs:0.0338,6.42533384294508e-56)
--(axis cs:0.034,1.90181805977008e-56)
--(axis cs:0.0342,5.62029897630528e-57)
--(axis cs:0.0344,1.65833954505777e-57)
--(axis cs:0.0346,4.88559400972066e-58)
--(axis cs:0.0348,1.43713429611765e-58)
--(axis cs:0.035,4.22104002158139e-59)
--(axis cs:0.0352,1.2379117751643e-59)
--(axis cs:0.0354,3.62504910675334e-60)
--(axis cs:0.0356,1.05998040843784e-60)
--(axis cs:0.0358,3.09490410293414e-61)
--(axis cs:0.036,9.02334461341669e-62)
--(axis cs:0.0362,2.62702593828628e-62)
--(axis cs:0.0364,7.63735603578234e-63)
--(axis cs:0.0366,2.21722043117976e-63)
--(axis cs:0.0368,6.42786990157014e-64)
--(axis cs:0.037,1.86089891413977e-64)
--(axis cs:0.0372,5.37998453912766e-65)
--(axis cs:0.0374,1.5532693565287e-65)
--(axis cs:0.0376,4.47842085405192e-66)
--(axis cs:0.0378,1.28949664552523e-66)
--(axis cs:0.038,3.70798048055983e-67)
--(axis cs:0.0382,1.06483231137532e-67)
--(axis cs:0.0384,3.05390932274284e-68)
--(axis cs:0.0386,8.74715240754835e-69)
--(axis cs:0.0388,2.50217293175214e-69)
--(axis cs:0.039,7.14845826172323e-70)
--(axis cs:0.0392,2.03965250032098e-70)
--(axis cs:0.0394,5.8123658209206e-71)
--cycle;
\path [draw=none, fill=blue, fill opacity=0.5]
(axis cs:0,0)
--(axis cs:0.0002,1.24681940844559e-185)
--(axis cs:0.0004,3.78429989649157e-149)
--(axis cs:0.0006,4.69624390399748e-128)
--(axis cs:0.0008,2.91814488510357e-113)
--(axis cs:0.001,6.39937004086229e-102)
--(axis cs:0.0012,9.19292845684038e-93)
--(axis cs:0.0014,4.10935002282113e-85)
--(axis cs:0.0016,1.44888628628164e-78)
--(axis cs:0.0018,7.34152876082861e-73)
--(axis cs:0.002,8.0525019711728e-68)
--(axis cs:0.0022,2.56068941187556e-63)
--(axis cs:0.0024,2.9292421581097e-59)
--(axis cs:0.0026,1.41995830907816e-55)
--(axis cs:0.0028,3.31301642990214e-52)
--(axis cs:0.003,4.11568387030809e-49)
--(axis cs:0.0032,2.95307692152295e-46)
--(axis cs:0.0034,1.30806963691571e-43)
--(axis cs:0.0036,3.77970210701459e-41)
--(axis cs:0.0038,7.46130251996167e-39)
--(axis cs:0.004,1.04633990725249e-36)
--(axis cs:0.0042,1.07774752268699e-34)
--(axis cs:0.0044,8.3909217479038e-33)
--(axis cs:0.0046,5.06232673594148e-31)
--(axis cs:0.0048,2.41856883933146e-29)
--(axis cs:0.005,9.32610456511297e-28)
--(axis cs:0.0052,2.95167434916092e-26)
--(axis cs:0.0054,7.78268945730413e-25)
--(axis cs:0.0056,1.73238744737497e-23)
--(axis cs:0.0058,3.29434213604689e-22)
--(axis cs:0.006,5.40916986257712e-21)
--(axis cs:0.0062,7.74292596344275e-20)
--(axis cs:0.0064,9.74695105038983e-19)
--(axis cs:0.0066,1.08755078580298e-17)
--(axis cs:0.0068,1.08334855272526e-16)
--(axis cs:0.007,9.69782013606882e-16)
--(axis cs:0.0072,7.84830360507471e-15)
--(axis cs:0.0074,5.7738733862825e-14)
--(axis cs:0.0076,3.88106550443288e-13)
--(axis cs:0.0078,2.39473656171717e-12)
--(axis cs:0.008,1.36227283385691e-11)
--(axis cs:0.0082,7.17311040949162e-11)
--(axis cs:0.0084,3.50913326863578e-10)
--(axis cs:0.0086,1.6004392380367e-09)
--(axis cs:0.0088,6.82684778432269e-09)
--(axis cs:0.009,2.73176680030113e-08)
--(axis cs:0.0092,1.02831056135345e-07)
--(axis cs:0.0094,3.65088994324908e-07)
--(axis cs:0.0096,1.22555178042531e-06)
--(axis cs:0.0098,3.89871873397428e-06)
--(axis cs:0.01,1.17789817815586e-05)
--(axis cs:0.0102,3.3866611040304e-05)
--(axis cs:0.0104,9.28421083067902e-05)
--(axis cs:0.0106,0.000243114650517988)
--(axis cs:0.0108,0.000609129762155081)
--(axis cs:0.011,0.00111790791988685)
--(axis cs:0.0112,0.000534562153913613)
--(axis cs:0.0114,0.00025255910834419)
--(axis cs:0.0116,0.000117944587665254)
--(axis cs:0.0118,5.44642522814749e-05)
--(axis cs:0.012,2.48784617872186e-05)
--(axis cs:0.0122,1.12451673411679e-05)
--(axis cs:0.0124,5.03133566191059e-06)
--(axis cs:0.0126,2.22901629785383e-06)
--(axis cs:0.0128,9.78108167672577e-07)
--(axis cs:0.013,4.25235215155964e-07)
--(axis cs:0.0132,1.8321436994974e-07)
--(axis cs:0.0134,7.82513271224556e-08)
--(axis cs:0.0136,3.31386328456154e-08)
--(axis cs:0.0138,1.39184931767362e-08)
--(axis cs:0.014,5.79914781556753e-09)
--(axis cs:0.0142,2.39742918516806e-09)
--(axis cs:0.0144,9.83622316412938e-10)
--(axis cs:0.0146,4.00589423659466e-10)
--(axis cs:0.0148,1.61973166427244e-10)
--(axis cs:0.015,6.50338909349618e-11)
--(axis cs:0.0152,2.59338417965968e-11)
--(axis cs:0.0154,1.02730351130093e-11)
--(axis cs:0.0156,4.04302985151058e-12)
--(axis cs:0.0158,1.58110282220569e-12)
--(axis cs:0.016,6.14503562618277e-13)
--(axis cs:0.0162,2.37390584422077e-13)
--(axis cs:0.0164,9.1167215395404e-14)
--(axis cs:0.0166,3.48104066439517e-14)
--(axis cs:0.0168,1.32169630292309e-14)
--(axis cs:0.017,4.99069887932052e-15)
--(axis cs:0.0172,1.87435212674958e-15)
--(axis cs:0.0174,7.00248674727646e-16)
--(axis cs:0.0176,2.60264010969948e-16)
--(axis cs:0.0178,9.62463769271557e-17)
--(axis cs:0.018,3.54167940614942e-17)
--(axis cs:0.0182,1.29698003774648e-17)
--(axis cs:0.0184,4.7271530918389e-18)
--(axis cs:0.0186,1.71494467053479e-18)
--(axis cs:0.0188,6.19334149097625e-19)
--(axis cs:0.019,2.22670930626025e-19)
--(axis cs:0.0192,7.97082893690277e-20)
--(axis cs:0.0194,2.84106797975296e-20)
--(axis cs:0.0196,1.00840155791236e-20)
--(axis cs:0.0198,3.56445901111221e-21)
--(axis cs:0.02,1.25486042699487e-21)
--(axis cs:0.0202,4.40018920549441e-22)
--(axis cs:0.0204,1.53692577618315e-22)
--(axis cs:0.0206,5.34774408776781e-23)
--(axis cs:0.0208,1.85376292933918e-23)
--(axis cs:0.021,6.40224732564722e-24)
--(axis cs:0.0212,2.20309578999997e-24)
--(axis cs:0.0214,7.55412564088324e-25)
--(axis cs:0.0216,2.58113884798824e-25)
--(axis cs:0.0218,8.7890227501203e-26)
--(axis cs:0.022,2.98261281271697e-26)
--(axis cs:0.0222,1.00879882995762e-26)
--(axis cs:0.0224,3.40084889151921e-27)
--(axis cs:0.0226,1.14279470199847e-27)
--(axis cs:0.0228,3.82797991013378e-28)
--(axis cs:0.023,1.27824314593684e-28)
--(axis cs:0.0232,4.25520963387255e-29)
--(axis cs:0.0234,1.41225408013708e-29)
--(axis cs:0.0236,4.67314546043816e-30)
--(axis cs:0.0238,1.54180704611887e-30)
--(axis cs:0.024,5.07217470679387e-31)
--(axis cs:0.0242,1.66387440219997e-31)
--(axis cs:0.0244,5.4428614463191e-32)
--(axis cs:0.0246,1.77554717500868e-32)
--(axis cs:0.0248,5.77633952463157e-33)
--(axis cs:0.025,1.87415588932812e-33)
--(axis cs:0.0252,6.06467804373849e-34)
--(axis cs:0.0254,1.95737920655727e-34)
--(axis cs:0.0256,6.30119810772174e-35)
--(axis cs:0.0258,2.02333425052184e-35)
--(axis cs:0.026,6.48072365427423e-36)
--(axis cs:0.0262,2.07064396562142e-36)
--(axis cs:0.0264,6.59975431842602e-37)
--(axis cs:0.0266,2.09847858540786e-37)
--(axis cs:0.0268,6.65655398417971e-38)
--(axis cs:0.027,2.10657012662075e-38)
--(axis cs:0.0272,6.65115446267256e-39)
--(axis cs:0.0274,2.0952006066426e-39)
--(axis cs:0.0276,6.5852791193173e-40)
--(axis cs:0.0278,2.06516627344732e-40)
--(axis cs:0.028,6.46219586689164e-41)
--(axis cs:0.0282,2.0177214295937e-41)
--(axis cs:0.0284,6.28651246634623e-42)
--(axis cs:0.0286,1.95450635618193e-42)
--(axis cs:0.0288,6.06392937465401e-43)
--(axis cs:0.029,1.87746436861622e-43)
--(axis cs:0.0292,5.8009664131289e-44)
--(axis cs:0.0294,1.78875316998022e-44)
--(axis cs:0.0296,5.50467937292145e-45)
--(axis cs:0.0298,1.69065544747347e-45)
--(axis cs:0.03,5.18238148622274e-46)
--(axis cs:0.0302,1.58549314377919e-46)
--(axis cs:0.0304,4.84138269202927e-47)
--(axis cs:0.0306,1.47554910417118e-47)
--(axis cs:0.0308,4.48875706629181e-48)
--(axis cs:0.031,1.36299893327607e-48)
--(axis cs:0.0312,4.13114592599943e-49)
--(axis cs:0.0314,1.24985497245378e-49)
--(axis cs:0.0316,3.77460119646181e-50)
--(axis cs:0.0318,1.13792340123613e-50)
--(axis cs:0.032,3.4244708583103e-51)
--(axis cs:0.0322,1.02877463298859e-51)
--(axis cs:0.0324,3.08532582781497e-52)
--(axis cs:0.0326,9.2372645944798e-53)
--(axis cs:0.0328,2.7609255834434e-53)
--(axis cs:0.033,8.23838828545169e-54)
--(axis cs:0.0332,2.45421829531249e-54)
--(axis cs:0.0334,7.29918727184782e-55)
--(axis cs:0.0336,2.16737015903793e-55)
--(axis cs:0.0338,6.42533384294508e-56)
--(axis cs:0.034,1.90181805977008e-56)
--(axis cs:0.0342,5.62029897630528e-57)
--(axis cs:0.0344,1.65833954505777e-57)
--(axis cs:0.0346,4.88559400972066e-58)
--(axis cs:0.0348,1.43713429611765e-58)
--(axis cs:0.035,4.22104002158139e-59)
--(axis cs:0.0352,1.2379117751643e-59)
--(axis cs:0.0354,3.62504910675334e-60)
--(axis cs:0.0356,1.05998040843784e-60)
--(axis cs:0.0358,3.09490410293414e-61)
--(axis cs:0.036,9.02334461341669e-62)
--(axis cs:0.0362,2.62702593828628e-62)
--(axis cs:0.0364,7.63735603578234e-63)
--(axis cs:0.0366,2.21722043117976e-63)
--(axis cs:0.0368,6.42786990157014e-64)
--(axis cs:0.037,1.86089891413977e-64)
--(axis cs:0.0372,5.37998453912766e-65)
--(axis cs:0.0374,1.5532693565287e-65)
--(axis cs:0.0376,4.47842085405192e-66)
--(axis cs:0.0378,1.28949664552523e-66)
--(axis cs:0.038,3.70798048055983e-67)
--(axis cs:0.0382,1.06483231137532e-67)
--(axis cs:0.0384,3.05390932274284e-68)
--(axis cs:0.0386,8.74715240754835e-69)
--(axis cs:0.0388,2.50217293175214e-69)
--(axis cs:0.039,7.14845826172323e-70)
--(axis cs:0.0392,2.03965250032098e-70)
--(axis cs:0.0394,5.8123658209206e-71)
--cycle;
\path [draw=none, fill=blue, fill opacity=0.5]
(axis cs:0,0)
--(axis cs:0.0002,7.07051949791944e-220)
--(axis cs:0.0004,3.35713224621954e-178)
--(axis cs:0.0006,4.55876472150082e-154)
--(axis cs:0.0008,4.06372766744917e-137)
--(axis cs:0.001,4.19927757126269e-124)
--(axis cs:0.0012,1.40567805458719e-113)
--(axis cs:0.0014,9.00643402552461e-105)
--(axis cs:0.0016,3.18925325198261e-97)
--(axis cs:0.0018,1.23690071259004e-90)
--(axis cs:0.002,8.38122637353824e-85)
--(axis cs:0.0022,1.38442183846587e-79)
--(axis cs:0.0024,7.12902189670533e-75)
--(axis cs:0.0026,1.3794522412844e-70)
--(axis cs:0.0028,1.15971222412979e-66)
--(axis cs:0.003,4.75273753659039e-63)
--(axis cs:0.0032,1.04179315001234e-59)
--(axis cs:0.0034,1.31771394306738e-56)
--(axis cs:0.0036,1.02415094765992e-53)
--(axis cs:0.0038,5.15560039198329e-51)
--(axis cs:0.004,1.75757764872752e-48)
--(axis cs:0.0042,4.2148638496568e-46)
--(axis cs:0.0044,7.3467295284001e-44)
--(axis cs:0.0046,9.57542435968325e-42)
--(axis cs:0.0048,9.56565522459126e-40)
--(axis cs:0.005,7.48492710605918e-38)
--(axis cs:0.0052,4.67617522469868e-36)
--(axis cs:0.0054,2.37244858785325e-34)
--(axis cs:0.0056,9.92369722206861e-33)
--(axis cs:0.0058,3.46894198997681e-31)
--(axis cs:0.006,1.02575999424703e-29)
--(axis cs:0.0062,2.59405173111155e-28)
--(axis cs:0.0064,5.66632176036028e-27)
--(axis cs:0.0066,1.07875184310474e-25)
--(axis cs:0.0068,1.80466875933379e-24)
--(axis cs:0.007,2.67286801654633e-23)
--(axis cs:0.0072,3.52886813743904e-22)
--(axis cs:0.0074,4.1792715817461e-21)
--(axis cs:0.0076,4.46564241233065e-20)
--(axis cs:0.0078,4.32813414557636e-19)
--(axis cs:0.008,3.82375668734256e-18)
--(axis cs:0.0082,3.09338087199957e-17)
--(axis cs:0.0084,2.3012710139591e-16)
--(axis cs:0.0086,1.58052451958739e-15)
--(axis cs:0.0088,1.00582834740469e-14)
--(axis cs:0.009,5.95139142432837e-14)
--(axis cs:0.0092,3.28450725704436e-13)
--(axis cs:0.0094,1.69580021234189e-12)
--(axis cs:0.0096,8.21383031334312e-12)
--(axis cs:0.0098,3.74215771035445e-11)
--(axis cs:0.01,1.60758325803006e-10)
--(axis cs:0.0102,6.52688622108691e-10)
--(axis cs:0.0104,2.50996299416417e-09)
--(axis cs:0.0106,9.16116446427591e-09)
--(axis cs:0.0108,3.17979701351146e-08)
--(axis cs:0.011,1.05150251268355e-07)
--(axis cs:0.0112,3.31846466176376e-07)
--(axis cs:0.0114,1.00113761423932e-06)
--(axis cs:0.0116,2.89172206747296e-06)
--(axis cs:0.0118,8.00879727229867e-06)
--(axis cs:0.012,2.12978350679836e-05)
--(axis cs:0.0122,1.12451673411679e-05)
--(axis cs:0.0124,5.03133566191059e-06)
--(axis cs:0.0126,2.22901629785383e-06)
--(axis cs:0.0128,9.78108167672577e-07)
--(axis cs:0.013,4.25235215155964e-07)
--(axis cs:0.0132,1.8321436994974e-07)
--(axis cs:0.0134,7.82513271224556e-08)
--(axis cs:0.0136,3.31386328456154e-08)
--(axis cs:0.0138,1.39184931767362e-08)
--(axis cs:0.014,5.79914781556753e-09)
--(axis cs:0.0142,2.39742918516806e-09)
--(axis cs:0.0144,9.83622316412938e-10)
--(axis cs:0.0146,4.00589423659466e-10)
--(axis cs:0.0148,1.61973166427244e-10)
--(axis cs:0.015,6.50338909349618e-11)
--(axis cs:0.0152,2.59338417965968e-11)
--(axis cs:0.0154,1.02730351130093e-11)
--(axis cs:0.0156,4.04302985151058e-12)
--(axis cs:0.0158,1.58110282220569e-12)
--(axis cs:0.016,6.14503562618277e-13)
--(axis cs:0.0162,2.37390584422077e-13)
--(axis cs:0.0164,9.1167215395404e-14)
--(axis cs:0.0166,3.48104066439517e-14)
--(axis cs:0.0168,1.32169630292309e-14)
--(axis cs:0.017,4.99069887932052e-15)
--(axis cs:0.0172,1.87435212674958e-15)
--(axis cs:0.0174,7.00248674727646e-16)
--(axis cs:0.0176,2.60264010969948e-16)
--(axis cs:0.0178,9.62463769271557e-17)
--(axis cs:0.018,3.54167940614942e-17)
--(axis cs:0.0182,1.29698003774648e-17)
--(axis cs:0.0184,4.7271530918389e-18)
--(axis cs:0.0186,1.71494467053479e-18)
--(axis cs:0.0188,6.19334149097625e-19)
--(axis cs:0.019,2.22670930626025e-19)
--(axis cs:0.0192,7.97082893690277e-20)
--(axis cs:0.0194,2.84106797975296e-20)
--(axis cs:0.0196,1.00840155791236e-20)
--(axis cs:0.0198,3.56445901111221e-21)
--(axis cs:0.02,1.25486042699487e-21)
--(axis cs:0.0202,4.40018920549441e-22)
--(axis cs:0.0204,1.53692577618315e-22)
--(axis cs:0.0206,5.34774408776781e-23)
--(axis cs:0.0208,1.85376292933918e-23)
--(axis cs:0.021,6.40224732564722e-24)
--(axis cs:0.0212,2.20309578999997e-24)
--(axis cs:0.0214,7.55412564088324e-25)
--(axis cs:0.0216,2.58113884798824e-25)
--(axis cs:0.0218,8.7890227501203e-26)
--(axis cs:0.022,2.98261281271697e-26)
--(axis cs:0.0222,1.00879882995762e-26)
--(axis cs:0.0224,3.40084889151921e-27)
--(axis cs:0.0226,1.14279470199847e-27)
--(axis cs:0.0228,3.82797991013378e-28)
--(axis cs:0.023,1.27824314593684e-28)
--(axis cs:0.0232,4.25520963387255e-29)
--(axis cs:0.0234,1.41225408013708e-29)
--(axis cs:0.0236,4.67314546043816e-30)
--(axis cs:0.0238,1.54180704611887e-30)
--(axis cs:0.024,5.07217470679387e-31)
--(axis cs:0.0242,1.66387440219997e-31)
--(axis cs:0.0244,5.4428614463191e-32)
--(axis cs:0.0246,1.77554717500868e-32)
--(axis cs:0.0248,5.77633952463157e-33)
--(axis cs:0.025,1.87415588932812e-33)
--(axis cs:0.0252,6.06467804373849e-34)
--(axis cs:0.0254,1.95737920655727e-34)
--(axis cs:0.0256,6.30119810772174e-35)
--(axis cs:0.0258,2.02333425052184e-35)
--(axis cs:0.026,6.48072365427423e-36)
--(axis cs:0.0262,2.07064396562142e-36)
--(axis cs:0.0264,6.59975431842602e-37)
--(axis cs:0.0266,2.09847858540786e-37)
--(axis cs:0.0268,6.65655398417971e-38)
--(axis cs:0.027,2.10657012662075e-38)
--(axis cs:0.0272,6.65115446267256e-39)
--(axis cs:0.0274,2.0952006066426e-39)
--(axis cs:0.0276,6.5852791193173e-40)
--(axis cs:0.0278,2.06516627344732e-40)
--(axis cs:0.028,6.46219586689164e-41)
--(axis cs:0.0282,2.0177214295937e-41)
--(axis cs:0.0284,6.28651246634623e-42)
--(axis cs:0.0286,1.95450635618193e-42)
--(axis cs:0.0288,6.06392937465401e-43)
--(axis cs:0.029,1.87746436861622e-43)
--(axis cs:0.0292,5.8009664131289e-44)
--(axis cs:0.0294,1.78875316998022e-44)
--(axis cs:0.0296,5.50467937292145e-45)
--(axis cs:0.0298,1.69065544747347e-45)
--(axis cs:0.03,5.18238148622274e-46)
--(axis cs:0.0302,1.58549314377919e-46)
--(axis cs:0.0304,4.84138269202927e-47)
--(axis cs:0.0306,1.47554910417118e-47)
--(axis cs:0.0308,4.48875706629181e-48)
--(axis cs:0.031,1.36299893327607e-48)
--(axis cs:0.0312,4.13114592599943e-49)
--(axis cs:0.0314,1.24985497245378e-49)
--(axis cs:0.0316,3.77460119646181e-50)
--(axis cs:0.0318,1.13792340123613e-50)
--(axis cs:0.032,3.4244708583103e-51)
--(axis cs:0.0322,1.02877463298859e-51)
--(axis cs:0.0324,3.08532582781497e-52)
--(axis cs:0.0326,9.2372645944798e-53)
--(axis cs:0.0328,2.7609255834434e-53)
--(axis cs:0.033,8.23838828545169e-54)
--(axis cs:0.0332,2.45421829531249e-54)
--(axis cs:0.0334,7.29918727184782e-55)
--(axis cs:0.0336,2.16737015903793e-55)
--(axis cs:0.0338,6.42533384294508e-56)
--(axis cs:0.034,1.90181805977008e-56)
--(axis cs:0.0342,5.62029897630528e-57)
--(axis cs:0.0344,1.65833954505777e-57)
--(axis cs:0.0346,4.88559400972066e-58)
--(axis cs:0.0348,1.43713429611765e-58)
--(axis cs:0.035,4.22104002158139e-59)
--(axis cs:0.0352,1.2379117751643e-59)
--(axis cs:0.0354,3.62504910675334e-60)
--(axis cs:0.0356,1.05998040843784e-60)
--(axis cs:0.0358,3.09490410293414e-61)
--(axis cs:0.036,9.02334461341669e-62)
--(axis cs:0.0362,2.62702593828628e-62)
--(axis cs:0.0364,7.63735603578234e-63)
--(axis cs:0.0366,2.21722043117976e-63)
--(axis cs:0.0368,6.42786990157014e-64)
--(axis cs:0.037,1.86089891413977e-64)
--(axis cs:0.0372,5.37998453912766e-65)
--(axis cs:0.0374,1.5532693565287e-65)
--(axis cs:0.0376,4.47842085405192e-66)
--(axis cs:0.0378,1.28949664552523e-66)
--(axis cs:0.038,3.70798048055983e-67)
--(axis cs:0.0382,1.06483231137532e-67)
--(axis cs:0.0384,3.05390932274284e-68)
--(axis cs:0.0386,8.74715240754835e-69)
--(axis cs:0.0388,2.50217293175214e-69)
--(axis cs:0.039,7.14845826172323e-70)
--(axis cs:0.0392,2.03965250032098e-70)
--(axis cs:0.0394,5.8123658209206e-71)
--cycle;
\path [draw=none, fill=blue, fill opacity=0.5]
(axis cs:0,0)
--(axis cs:0.0002,1.59862054051401e-63)
--(axis cs:0.0004,2.01433889039574e-47)
--(axis cs:0.0006,2.97077629654634e-38)
--(axis cs:0.0008,6.36171904282589e-32)
--(axis cs:0.001,3.79438340819427e-27)
--(axis cs:0.0012,2.34966409367811e-23)
--(axis cs:0.0014,3.04782019614733e-20)
--(axis cs:0.0016,1.25904990534363e-17)
--(axis cs:0.0018,2.17034474693635e-15)
--(axis cs:0.002,1.87750330629904e-13)
--(axis cs:0.0022,9.2972654226903e-12)
--(axis cs:0.0024,2.90438635134726e-10)
--(axis cs:0.0026,6.16207408914511e-09)
--(axis cs:0.0028,9.40339740634871e-08)
--(axis cs:0.003,1.0801313742615e-06)
--(axis cs:0.0032,9.68774339936991e-06)
--(axis cs:0.0034,6.99108734785651e-05)
--(axis cs:0.0036,0.000416130640341037)
--(axis cs:0.0038,0.00208599494885405)
--(axis cs:0.004,0.00896271696907974)
--(axis cs:0.0042,0.0335068515028188)
--(axis cs:0.0044,0.110410042257394)
--(axis cs:0.0046,0.32428663564024)
--(axis cs:0.0048,0.857310496430777)
--(axis cs:0.005,2.0575910593128)
--(axis cs:0.0052,4.51727747541394)
--(axis cs:0.0054,9.13280377204685)
--(axis cs:0.0056,17.105512518158)
--(axis cs:0.0058,29.8396920784007)
--(axis cs:0.006,48.7151828613912)
--(axis cs:0.0062,74.7526085244504)
--(axis cs:0.0064,108.238434806541)
--(axis cs:0.0066,148.413644172392)
--(axis cs:0.0068,127.889117288277)
--(axis cs:0.007,92.161852432548)
--(axis cs:0.0072,64.4283023771624)
--(axis cs:0.0074,43.7649384616828)
--(axis cs:0.0076,28.9307060251247)
--(axis cs:0.0078,18.6372741031152)
--(axis cs:0.008,11.7154187861733)
--(axis cs:0.0082,7.19457466602162)
--(axis cs:0.0084,4.32122161225997)
--(axis cs:0.0086,2.54103263182437)
--(axis cs:0.0088,1.4643115661178)
--(axis cs:0.009,0.82768562598747)
--(axis cs:0.0092,0.459271463321069)
--(axis cs:0.0094,0.250371931866057)
--(axis cs:0.0096,0.134193806359963)
--(axis cs:0.0098,0.0707634126579589)
--(axis cs:0.01,0.0367362846068369)
--(axis cs:0.0102,0.018786932501781)
--(axis cs:0.0104,0.00946976789378442)
--(axis cs:0.0106,0.00470738837305254)
--(axis cs:0.0108,0.00230886868815283)
--(axis cs:0.011,0.00111790791988685)
--(axis cs:0.0112,0.000534562153913613)
--(axis cs:0.0114,0.00025255910834419)
--(axis cs:0.0116,0.000117944587665254)
--(axis cs:0.0118,5.44642522814749e-05)
--(axis cs:0.012,2.48784617872186e-05)
--(axis cs:0.0122,1.12451673411679e-05)
--(axis cs:0.0124,5.03133566191059e-06)
--(axis cs:0.0126,2.22901629785383e-06)
--(axis cs:0.0128,9.78108167672577e-07)
--(axis cs:0.013,4.25235215155964e-07)
--(axis cs:0.0132,1.8321436994974e-07)
--(axis cs:0.0134,7.82513271224556e-08)
--(axis cs:0.0136,3.31386328456154e-08)
--(axis cs:0.0138,1.39184931767362e-08)
--(axis cs:0.014,5.79914781556753e-09)
--(axis cs:0.0142,2.39742918516806e-09)
--(axis cs:0.0144,9.83622316412938e-10)
--(axis cs:0.0146,4.00589423659466e-10)
--(axis cs:0.0148,1.61973166427244e-10)
--(axis cs:0.015,6.50338909349618e-11)
--(axis cs:0.0152,2.59338417965968e-11)
--(axis cs:0.0154,1.02730351130093e-11)
--(axis cs:0.0156,4.04302985151058e-12)
--(axis cs:0.0158,1.58110282220569e-12)
--(axis cs:0.016,6.14503562618277e-13)
--(axis cs:0.0162,2.37390584422077e-13)
--(axis cs:0.0164,9.1167215395404e-14)
--(axis cs:0.0166,3.48104066439517e-14)
--(axis cs:0.0168,1.32169630292309e-14)
--(axis cs:0.017,4.99069887932052e-15)
--(axis cs:0.0172,1.87435212674958e-15)
--(axis cs:0.0174,7.00248674727646e-16)
--(axis cs:0.0176,2.60264010969948e-16)
--(axis cs:0.0178,9.62463769271557e-17)
--(axis cs:0.018,3.54167940614942e-17)
--(axis cs:0.0182,1.29698003774648e-17)
--(axis cs:0.0184,4.7271530918389e-18)
--(axis cs:0.0186,1.71494467053479e-18)
--(axis cs:0.0188,6.19334149097625e-19)
--(axis cs:0.019,2.22670930626025e-19)
--(axis cs:0.0192,7.97082893690277e-20)
--(axis cs:0.0194,2.84106797975296e-20)
--(axis cs:0.0196,1.00840155791236e-20)
--(axis cs:0.0198,3.56445901111221e-21)
--(axis cs:0.02,1.25486042699487e-21)
--(axis cs:0.0202,4.40018920549441e-22)
--(axis cs:0.0204,1.53692577618315e-22)
--(axis cs:0.0206,5.34774408776781e-23)
--(axis cs:0.0208,1.85376292933918e-23)
--(axis cs:0.021,6.40224732564722e-24)
--(axis cs:0.0212,2.20309578999997e-24)
--(axis cs:0.0214,7.55412564088324e-25)
--(axis cs:0.0216,2.58113884798824e-25)
--(axis cs:0.0218,8.7890227501203e-26)
--(axis cs:0.022,2.98261281271697e-26)
--(axis cs:0.0222,1.00879882995762e-26)
--(axis cs:0.0224,3.40084889151921e-27)
--(axis cs:0.0226,1.14279470199847e-27)
--(axis cs:0.0228,3.82797991013378e-28)
--(axis cs:0.023,1.27824314593684e-28)
--(axis cs:0.0232,4.25520963387255e-29)
--(axis cs:0.0234,1.41225408013708e-29)
--(axis cs:0.0236,4.67314546043816e-30)
--(axis cs:0.0238,1.54180704611887e-30)
--(axis cs:0.024,5.07217470679387e-31)
--(axis cs:0.0242,1.66387440219997e-31)
--(axis cs:0.0244,5.4428614463191e-32)
--(axis cs:0.0246,1.77554717500868e-32)
--(axis cs:0.0248,5.77633952463157e-33)
--(axis cs:0.025,1.87415588932812e-33)
--(axis cs:0.0252,6.06467804373849e-34)
--(axis cs:0.0254,1.95737920655727e-34)
--(axis cs:0.0256,6.30119810772174e-35)
--(axis cs:0.0258,2.02333425052184e-35)
--(axis cs:0.026,6.48072365427423e-36)
--(axis cs:0.0262,2.07064396562142e-36)
--(axis cs:0.0264,6.59975431842602e-37)
--(axis cs:0.0266,2.09847858540786e-37)
--(axis cs:0.0268,6.65655398417971e-38)
--(axis cs:0.027,2.10657012662075e-38)
--(axis cs:0.0272,6.65115446267256e-39)
--(axis cs:0.0274,2.0952006066426e-39)
--(axis cs:0.0276,6.5852791193173e-40)
--(axis cs:0.0278,2.06516627344732e-40)
--(axis cs:0.028,6.46219586689164e-41)
--(axis cs:0.0282,2.0177214295937e-41)
--(axis cs:0.0284,6.28651246634623e-42)
--(axis cs:0.0286,1.95450635618193e-42)
--(axis cs:0.0288,6.06392937465401e-43)
--(axis cs:0.029,1.87746436861622e-43)
--(axis cs:0.0292,5.8009664131289e-44)
--(axis cs:0.0294,1.78875316998022e-44)
--(axis cs:0.0296,5.50467937292145e-45)
--(axis cs:0.0298,1.69065544747347e-45)
--(axis cs:0.03,5.18238148622274e-46)
--(axis cs:0.0302,1.58549314377919e-46)
--(axis cs:0.0304,4.84138269202927e-47)
--(axis cs:0.0306,1.47554910417118e-47)
--(axis cs:0.0308,4.48875706629181e-48)
--(axis cs:0.031,1.36299893327607e-48)
--(axis cs:0.0312,4.13114592599943e-49)
--(axis cs:0.0314,1.24985497245378e-49)
--(axis cs:0.0316,3.77460119646181e-50)
--(axis cs:0.0318,1.13792340123613e-50)
--(axis cs:0.032,3.4244708583103e-51)
--(axis cs:0.0322,1.02877463298859e-51)
--(axis cs:0.0324,3.08532582781497e-52)
--(axis cs:0.0326,9.2372645944798e-53)
--(axis cs:0.0328,2.7609255834434e-53)
--(axis cs:0.033,8.23838828545169e-54)
--(axis cs:0.0332,2.45421829531249e-54)
--(axis cs:0.0334,7.29918727184782e-55)
--(axis cs:0.0336,2.16737015903793e-55)
--(axis cs:0.0338,6.42533384294508e-56)
--(axis cs:0.034,1.90181805977008e-56)
--(axis cs:0.0342,5.62029897630528e-57)
--(axis cs:0.0344,1.65833954505777e-57)
--(axis cs:0.0346,4.88559400972066e-58)
--(axis cs:0.0348,1.43713429611765e-58)
--(axis cs:0.035,4.22104002158139e-59)
--(axis cs:0.0352,1.2379117751643e-59)
--(axis cs:0.0354,3.62504910675334e-60)
--(axis cs:0.0356,1.05998040843784e-60)
--(axis cs:0.0358,3.09490410293414e-61)
--(axis cs:0.036,9.02334461341669e-62)
--(axis cs:0.0362,2.62702593828628e-62)
--(axis cs:0.0364,7.63735603578234e-63)
--(axis cs:0.0366,2.21722043117976e-63)
--(axis cs:0.0368,6.42786990157014e-64)
--(axis cs:0.037,1.86089891413977e-64)
--(axis cs:0.0372,5.37998453912766e-65)
--(axis cs:0.0374,1.5532693565287e-65)
--(axis cs:0.0376,4.47842085405192e-66)
--(axis cs:0.0378,1.28949664552523e-66)
--(axis cs:0.038,3.70798048055983e-67)
--(axis cs:0.0382,1.06483231137532e-67)
--(axis cs:0.0384,3.05390932274284e-68)
--(axis cs:0.0386,8.74715240754835e-69)
--(axis cs:0.0388,2.50217293175214e-69)
--(axis cs:0.039,7.14845826172323e-70)
--(axis cs:0.0392,2.03965250032098e-70)
--(axis cs:0.0394,5.8123658209206e-71)
--cycle;
\path [draw=none, fill=blue, fill opacity=0.5]
(axis cs:0,0)
--(axis cs:0.0002,2.55329303009598e-95)
--(axis cs:0.0004,1.68182412350504e-73)
--(axis cs:0.0006,5.49781289889967e-61)
--(axis cs:0.0008,2.78718791670709e-52)
--(axis cs:0.001,1.15539866519647e-45)
--(axis cs:0.0012,2.29045339557146e-40)
--(axis cs:0.0014,5.57050636268994e-36)
--(axis cs:0.0016,2.91664333325709e-32)
--(axis cs:0.0018,4.72521876543319e-29)
--(axis cs:0.002,3.03440752670528e-26)
--(axis cs:0.0022,9.2163699131261e-24)
--(axis cs:0.0024,1.50843826973494e-21)
--(axis cs:0.0026,1.46891166386691e-19)
--(axis cs:0.0028,9.19186255467381e-18)
--(axis cs:0.003,3.92879593928923e-16)
--(axis cs:0.0032,1.20485095111868e-14)
--(axis cs:0.0034,2.75996884779829e-13)
--(axis cs:0.0036,4.8825997596155e-12)
--(axis cs:0.0038,6.85966297166094e-11)
--(axis cs:0.004,7.83646193730588e-10)
--(axis cs:0.0042,7.42781149314522e-09)
--(axis cs:0.0044,5.94374689671225e-08)
--(axis cs:0.0046,4.07612920102254e-07)
--(axis cs:0.0048,2.42727662171799e-06)
--(axis cs:0.005,1.26961478962858e-05)
--(axis cs:0.0052,5.89271481250347e-05)
--(axis cs:0.0054,0.00024488352824238)
--(axis cs:0.0056,0.000918517912095324)
--(axis cs:0.0058,0.00313195835741374)
--(axis cs:0.006,0.00977111118024442)
--(axis cs:0.0062,0.0280540417034728)
--(axis cs:0.0064,0.0745167406233305)
--(axis cs:0.0066,0.183988958680021)
--(axis cs:0.0068,0.424127151084662)
--(axis cs:0.007,0.916409463495715)
--(axis cs:0.0072,1.86272770489421)
--(axis cs:0.0074,3.57374020060483)
--(axis cs:0.0076,6.49145456201812)
--(axis cs:0.0078,11.1952645063309)
--(axis cs:0.008,11.7154187861733)
--(axis cs:0.0082,7.19457466602162)
--(axis cs:0.0084,4.32122161225997)
--(axis cs:0.0086,2.54103263182437)
--(axis cs:0.0088,1.4643115661178)
--(axis cs:0.009,0.82768562598747)
--(axis cs:0.0092,0.459271463321069)
--(axis cs:0.0094,0.250371931866057)
--(axis cs:0.0096,0.134193806359963)
--(axis cs:0.0098,0.0707634126579589)
--(axis cs:0.01,0.0367362846068369)
--(axis cs:0.0102,0.018786932501781)
--(axis cs:0.0104,0.00946976789378442)
--(axis cs:0.0106,0.00470738837305254)
--(axis cs:0.0108,0.00230886868815283)
--(axis cs:0.011,0.00111790791988685)
--(axis cs:0.0112,0.000534562153913613)
--(axis cs:0.0114,0.00025255910834419)
--(axis cs:0.0116,0.000117944587665254)
--(axis cs:0.0118,5.44642522814749e-05)
--(axis cs:0.012,2.48784617872186e-05)
--(axis cs:0.0122,1.12451673411679e-05)
--(axis cs:0.0124,5.03133566191059e-06)
--(axis cs:0.0126,2.22901629785383e-06)
--(axis cs:0.0128,9.78108167672577e-07)
--(axis cs:0.013,4.25235215155964e-07)
--(axis cs:0.0132,1.8321436994974e-07)
--(axis cs:0.0134,7.82513271224556e-08)
--(axis cs:0.0136,3.31386328456154e-08)
--(axis cs:0.0138,1.39184931767362e-08)
--(axis cs:0.014,5.79914781556753e-09)
--(axis cs:0.0142,2.39742918516806e-09)
--(axis cs:0.0144,9.83622316412938e-10)
--(axis cs:0.0146,4.00589423659466e-10)
--(axis cs:0.0148,1.61973166427244e-10)
--(axis cs:0.015,6.50338909349618e-11)
--(axis cs:0.0152,2.59338417965968e-11)
--(axis cs:0.0154,1.02730351130093e-11)
--(axis cs:0.0156,4.04302985151058e-12)
--(axis cs:0.0158,1.58110282220569e-12)
--(axis cs:0.016,6.14503562618277e-13)
--(axis cs:0.0162,2.37390584422077e-13)
--(axis cs:0.0164,9.1167215395404e-14)
--(axis cs:0.0166,3.48104066439517e-14)
--(axis cs:0.0168,1.32169630292309e-14)
--(axis cs:0.017,4.99069887932052e-15)
--(axis cs:0.0172,1.87435212674958e-15)
--(axis cs:0.0174,7.00248674727646e-16)
--(axis cs:0.0176,2.60264010969948e-16)
--(axis cs:0.0178,9.62463769271557e-17)
--(axis cs:0.018,3.54167940614942e-17)
--(axis cs:0.0182,1.29698003774648e-17)
--(axis cs:0.0184,4.7271530918389e-18)
--(axis cs:0.0186,1.71494467053479e-18)
--(axis cs:0.0188,6.19334149097625e-19)
--(axis cs:0.019,2.22670930626025e-19)
--(axis cs:0.0192,7.97082893690277e-20)
--(axis cs:0.0194,2.84106797975296e-20)
--(axis cs:0.0196,1.00840155791236e-20)
--(axis cs:0.0198,3.56445901111221e-21)
--(axis cs:0.02,1.25486042699487e-21)
--(axis cs:0.0202,4.40018920549441e-22)
--(axis cs:0.0204,1.53692577618315e-22)
--(axis cs:0.0206,5.34774408776781e-23)
--(axis cs:0.0208,1.85376292933918e-23)
--(axis cs:0.021,6.40224732564722e-24)
--(axis cs:0.0212,2.20309578999997e-24)
--(axis cs:0.0214,7.55412564088324e-25)
--(axis cs:0.0216,2.58113884798824e-25)
--(axis cs:0.0218,8.7890227501203e-26)
--(axis cs:0.022,2.98261281271697e-26)
--(axis cs:0.0222,1.00879882995762e-26)
--(axis cs:0.0224,3.40084889151921e-27)
--(axis cs:0.0226,1.14279470199847e-27)
--(axis cs:0.0228,3.82797991013378e-28)
--(axis cs:0.023,1.27824314593684e-28)
--(axis cs:0.0232,4.25520963387255e-29)
--(axis cs:0.0234,1.41225408013708e-29)
--(axis cs:0.0236,4.67314546043816e-30)
--(axis cs:0.0238,1.54180704611887e-30)
--(axis cs:0.024,5.07217470679387e-31)
--(axis cs:0.0242,1.66387440219997e-31)
--(axis cs:0.0244,5.4428614463191e-32)
--(axis cs:0.0246,1.77554717500868e-32)
--(axis cs:0.0248,5.77633952463157e-33)
--(axis cs:0.025,1.87415588932812e-33)
--(axis cs:0.0252,6.06467804373849e-34)
--(axis cs:0.0254,1.95737920655727e-34)
--(axis cs:0.0256,6.30119810772174e-35)
--(axis cs:0.0258,2.02333425052184e-35)
--(axis cs:0.026,6.48072365427423e-36)
--(axis cs:0.0262,2.07064396562142e-36)
--(axis cs:0.0264,6.59975431842602e-37)
--(axis cs:0.0266,2.09847858540786e-37)
--(axis cs:0.0268,6.65655398417971e-38)
--(axis cs:0.027,2.10657012662075e-38)
--(axis cs:0.0272,6.65115446267256e-39)
--(axis cs:0.0274,2.0952006066426e-39)
--(axis cs:0.0276,6.5852791193173e-40)
--(axis cs:0.0278,2.06516627344732e-40)
--(axis cs:0.028,6.46219586689164e-41)
--(axis cs:0.0282,2.0177214295937e-41)
--(axis cs:0.0284,6.28651246634623e-42)
--(axis cs:0.0286,1.95450635618193e-42)
--(axis cs:0.0288,6.06392937465401e-43)
--(axis cs:0.029,1.87746436861622e-43)
--(axis cs:0.0292,5.8009664131289e-44)
--(axis cs:0.0294,1.78875316998022e-44)
--(axis cs:0.0296,5.50467937292145e-45)
--(axis cs:0.0298,1.69065544747347e-45)
--(axis cs:0.03,5.18238148622274e-46)
--(axis cs:0.0302,1.58549314377919e-46)
--(axis cs:0.0304,4.84138269202927e-47)
--(axis cs:0.0306,1.47554910417118e-47)
--(axis cs:0.0308,4.48875706629181e-48)
--(axis cs:0.031,1.36299893327607e-48)
--(axis cs:0.0312,4.13114592599943e-49)
--(axis cs:0.0314,1.24985497245378e-49)
--(axis cs:0.0316,3.77460119646181e-50)
--(axis cs:0.0318,1.13792340123613e-50)
--(axis cs:0.032,3.4244708583103e-51)
--(axis cs:0.0322,1.02877463298859e-51)
--(axis cs:0.0324,3.08532582781497e-52)
--(axis cs:0.0326,9.2372645944798e-53)
--(axis cs:0.0328,2.7609255834434e-53)
--(axis cs:0.033,8.23838828545169e-54)
--(axis cs:0.0332,2.45421829531249e-54)
--(axis cs:0.0334,7.29918727184782e-55)
--(axis cs:0.0336,2.16737015903793e-55)
--(axis cs:0.0338,6.42533384294508e-56)
--(axis cs:0.034,1.90181805977008e-56)
--(axis cs:0.0342,5.62029897630528e-57)
--(axis cs:0.0344,1.65833954505777e-57)
--(axis cs:0.0346,4.88559400972066e-58)
--(axis cs:0.0348,1.43713429611765e-58)
--(axis cs:0.035,4.22104002158139e-59)
--(axis cs:0.0352,1.2379117751643e-59)
--(axis cs:0.0354,3.62504910675334e-60)
--(axis cs:0.0356,1.05998040843784e-60)
--(axis cs:0.0358,3.09490410293414e-61)
--(axis cs:0.036,9.02334461341669e-62)
--(axis cs:0.0362,2.62702593828628e-62)
--(axis cs:0.0364,7.63735603578234e-63)
--(axis cs:0.0366,2.21722043117976e-63)
--(axis cs:0.0368,6.42786990157014e-64)
--(axis cs:0.037,1.86089891413977e-64)
--(axis cs:0.0372,5.37998453912766e-65)
--(axis cs:0.0374,1.5532693565287e-65)
--(axis cs:0.0376,4.47842085405192e-66)
--(axis cs:0.0378,1.28949664552523e-66)
--(axis cs:0.038,3.70798048055983e-67)
--(axis cs:0.0382,1.06483231137532e-67)
--(axis cs:0.0384,3.05390932274284e-68)
--(axis cs:0.0386,8.74715240754835e-69)
--(axis cs:0.0388,2.50217293175214e-69)
--(axis cs:0.039,7.14845826172323e-70)
--(axis cs:0.0392,2.03965250032098e-70)
--(axis cs:0.0394,5.8123658209206e-71)
--cycle;
\path [draw=none, fill=blue, fill opacity=0.5]
(axis cs:0,0)
--(axis cs:0.0002,6.88149847089383e-283)
--(axis cs:0.0004,5.38329324547414e-232)
--(axis cs:0.0006,1.80526708928995e-202)
--(axis cs:0.0008,1.0802222727238e-181)
--(axis cs:0.001,1.03927455538975e-165)
--(axis cs:0.0012,9.284355067112e-153)
--(axis cs:0.0014,6.70264999764463e-142)
--(axis cs:0.0016,1.42270014450123e-132)
--(axis cs:0.0018,2.04241209795704e-124)
--(axis cs:0.002,3.50239373677996e-117)
--(axis cs:0.0022,1.0763994483646e-110)
--(axis cs:0.0024,7.99953899528515e-105)
--(axis cs:0.0026,1.80482808273425e-99)
--(axis cs:0.0028,1.47530594980937e-94)
--(axis cs:0.003,5.0266949465783e-90)
--(axis cs:0.0032,7.99309022920627e-86)
--(axis cs:0.0034,6.50610273957532e-82)
--(axis cs:0.0036,2.92653432604146e-78)
--(axis cs:0.0038,7.75652633853781e-75)
--(axis cs:0.004,1.27887150345601e-71)
--(axis cs:0.0042,1.37387136164765e-68)
--(axis cs:0.0044,1.00075911725486e-65)
--(axis cs:0.0046,5.11650284411508e-63)
--(axis cs:0.0048,1.89215762100807e-60)
--(axis cs:0.005,5.19708066546697e-58)
--(axis cs:0.0052,1.08519112913327e-55)
--(axis cs:0.0054,1.75864114867945e-53)
--(axis cs:0.0056,2.25305595468255e-51)
--(axis cs:0.0058,2.31978418006334e-49)
--(axis cs:0.006,1.94819763121604e-47)
--(axis cs:0.0062,1.35246061951633e-45)
--(axis cs:0.0064,7.85536854187431e-44)
--(axis cs:0.0066,3.85939041326377e-42)
--(axis cs:0.0068,1.61999834203236e-40)
--(axis cs:0.007,5.8628757428418e-39)
--(axis cs:0.0072,1.84471455956781e-37)
--(axis cs:0.0074,5.08503961140203e-36)
--(axis cs:0.0076,1.23670034865055e-34)
--(axis cs:0.0078,2.67091172323767e-33)
--(axis cs:0.008,5.15329732842381e-32)
--(axis cs:0.0082,8.93211136472859e-31)
--(axis cs:0.0084,1.39799146700368e-29)
--(axis cs:0.0086,1.98525615382389e-28)
--(axis cs:0.0088,2.56937152193057e-27)
--(axis cs:0.009,3.04327839977601e-26)
--(axis cs:0.0092,3.31167624895744e-25)
--(axis cs:0.0094,3.32294847816438e-24)
--(axis cs:0.0096,3.08494431084593e-23)
--(axis cs:0.0098,2.65831236795176e-22)
--(axis cs:0.01,2.1325663849262e-21)
--(axis cs:0.0102,1.59721161269042e-20)
--(axis cs:0.0104,1.11979829006939e-19)
--(axis cs:0.0106,7.36754354703253e-19)
--(axis cs:0.0108,4.55972127990545e-18)
--(axis cs:0.011,2.66046231163419e-17)
--(axis cs:0.0112,1.46655123482879e-16)
--(axis cs:0.0114,7.65293478264192e-16)
--(axis cs:0.0116,3.7876809871124e-15)
--(axis cs:0.0118,1.78120985335917e-14)
--(axis cs:0.012,7.97251566953794e-14)
--(axis cs:0.0122,3.4018858950948e-13)
--(axis cs:0.0124,1.38598721522646e-12)
--(axis cs:0.0126,5.39948902691054e-12)
--(axis cs:0.0128,2.01422837127883e-11)
--(axis cs:0.013,7.20456391440064e-11)
--(axis cs:0.0132,2.47402100952021e-10)
--(axis cs:0.0134,8.16628494582459e-10)
--(axis cs:0.0136,2.59403838921817e-09)
--(axis cs:0.0138,7.93856711154157e-09)
--(axis cs:0.014,2.34306300004452e-08)
--(axis cs:0.0142,6.67643356640233e-08)
--(axis cs:0.0144,1.83843314480245e-07)
--(axis cs:0.0146,4.89666471389706e-07)
--(axis cs:0.0148,1.2626761922186e-06)
--(axis cs:0.015,3.15498203452717e-06)
--(axis cs:0.0152,7.64494508653064e-06)
--(axis cs:0.0154,1.7979170213845e-05)
--(axis cs:0.0156,4.10690407641051e-05)
--(axis cs:0.0158,9.11860238376426e-05)
--(axis cs:0.016,0.000196932737783124)
--(axis cs:0.0162,0.000413979796989082)
--(axis cs:0.0164,0.000847611318286239)
--(axis cs:0.0166,0.00169139539232139)
--(axis cs:0.0168,0.003291462393694)
--(axis cs:0.017,0.0062500454828601)
--(axis cs:0.0172,0.0115870850932161)
--(axis cs:0.0174,0.0209845482354142)
--(axis cs:0.0176,0.0371440265188652)
--(axis cs:0.0178,0.0642930463763373)
--(axis cs:0.018,0.108877714983386)
--(axis cs:0.0182,0.180476707043854)
--(axis cs:0.0184,0.292961727437675)
--(axis cs:0.0186,0.465910058625537)
--(axis cs:0.0188,0.726243790315821)
--(axis cs:0.019,1.11002725226193)
--(axis cs:0.0192,1.66430041509408)
--(axis cs:0.0194,2.4487655468853)
--(axis cs:0.0196,3.5370840535338)
--(axis cs:0.0198,5.01748973197458)
--(axis cs:0.02,6.99239507315247)
--(axis cs:0.0202,9.57667065386176)
--(axis cs:0.0204,12.8943243600301)
--(axis cs:0.0206,17.0734035981651)
--(axis cs:0.0208,22.2390900744056)
--(axis cs:0.021,28.5051457064319)
--(axis cs:0.0212,27.7618954211491)
--(axis cs:0.0214,21.7895424162046)
--(axis cs:0.0216,16.9141687949919)
--(axis cs:0.0218,12.9880167701224)
--(axis cs:0.022,9.86752800549861)
--(axis cs:0.0222,7.41871492887164)
--(axis cs:0.0224,5.52055875933554)
--(axis cs:0.0226,4.06676018328376)
--(axis cs:0.0228,2.96619948938404)
--(axis cs:0.023,2.14245417264068)
--(axis cs:0.0232,1.53268653943172)
--(axis cs:0.0234,1.08616336780916)
--(axis cs:0.0236,0.76261366342503)
--(axis cs:0.0238,0.530575922349809)
--(axis cs:0.024,0.365837537964855)
--(axis cs:0.0242,0.250028440537005)
--(axis cs:0.0244,0.169399480962606)
--(axis cs:0.0246,0.113793080852648)
--(axis cs:0.0248,0.0757982021597045)
--(axis cs:0.025,0.0500723496819741)
--(axis cs:0.0252,0.0328086565119342)
--(axis cs:0.0254,0.0213247725075256)
--(axis cs:0.0256,0.0137511384350888)
--(axis cs:0.0258,0.00879838085750516)
--(axis cs:0.026,0.00558633645818186)
--(axis cs:0.0262,0.00352014751036091)
--(axis cs:0.0264,0.0022016697110189)
--(axis cs:0.0266,0.0013669354400426)
--(axis cs:0.0268,0.000842547340201024)
--(axis cs:0.027,0.000515627426063578)
--(axis cs:0.0272,0.000313340849114407)
--(axis cs:0.0274,0.000189095104711343)
--(axis cs:0.0276,0.00011333598801155)
--(axis cs:0.0278,6.74715624447029e-05)
--(axis cs:0.028,3.99005885881548e-05)
--(axis cs:0.0282,2.34413596984823e-05)
--(axis cs:0.0284,1.3682636249639e-05)
--(axis cs:0.0286,7.93556697071585e-06)
--(axis cs:0.0288,4.57345341808433e-06)
--(axis cs:0.029,2.61941815048323e-06)
--(axis cs:0.0292,1.49105988537527e-06)
--(axis cs:0.0294,8.43625509032096e-07)
--(axis cs:0.0296,4.74463336067587e-07)
--(axis cs:0.0298,2.65269451197919e-07)
--(axis cs:0.03,1.47447041872296e-07)
--(axis cs:0.0302,8.14856437416443e-08)
--(axis cs:0.0304,4.47768648879564e-08)
--(axis cs:0.0306,2.44672142554971e-08)
--(axis cs:0.0308,1.32954714741728e-08)
--(axis cs:0.031,7.18523313166044e-09)
--(axis cs:0.0312,3.86211184480844e-09)
--(axis cs:0.0314,2.0648290071526e-09)
--(axis cs:0.0316,1.09811155552866e-09)
--(axis cs:0.0318,5.80950707694493e-10)
--(axis cs:0.032,3.05766131800596e-10)
--(axis cs:0.0322,1.60111695949653e-10)
--(axis cs:0.0324,8.34192239271124e-11)
--(axis cs:0.0326,4.32458052135747e-11)
--(axis cs:0.0328,2.2309069997019e-11)
--(axis cs:0.033,1.14525713369952e-11)
--(axis cs:0.0332,5.85103354789937e-12)
--(axis cs:0.0334,2.97504653091987e-12)
--(axis cs:0.0336,1.50560016949624e-12)
--(axis cs:0.0338,7.58408127498507e-13)
--(axis cs:0.034,3.80273483880839e-13)
--(axis cs:0.0342,1.89806362745019e-13)
--(axis cs:0.0344,9.43123749013075e-14)
--(axis cs:0.0346,4.66542210861782e-14)
--(axis cs:0.0348,2.29772655367192e-14)
--(axis cs:0.035,1.12670783224797e-14)
--(axis cs:0.0352,5.50110500809886e-15)
--(axis cs:0.0354,2.67444670514748e-15)
--(axis cs:0.0356,1.29474057046822e-15)
--(axis cs:0.0358,6.24188034795496e-16)
--(axis cs:0.036,2.9967516537154e-16)
--(axis cs:0.0362,1.43287117184139e-16)
--(axis cs:0.0364,6.82343019931502e-17)
--(axis cs:0.0366,3.23634952024334e-17)
--(axis cs:0.0368,1.52891270231179e-17)
--(axis cs:0.037,7.19452002562637e-18)
--(axis cs:0.0372,3.37232801960942e-18)
--(axis cs:0.0374,1.57464751428699e-18)
--(axis cs:0.0376,7.32451525804251e-19)
--(axis cs:0.0378,3.39416222766471e-19)
--(axis cs:0.038,1.56696896785998e-19)
--(axis cs:0.0382,7.20738888345073e-20)
--(axis cs:0.0384,3.30294009683984e-20)
--(axis cs:0.0386,1.5081474411321e-20)
--(axis cs:0.0388,6.86154969376638e-21)
--(axis cs:0.039,3.11064639137136e-21)
--(axis cs:0.0392,1.40521760798034e-21)
--(axis cs:0.0394,6.32579760713982e-22)
--cycle;
\path [draw=none, fill=blue, fill opacity=0.5]
(axis cs:0,0)
--(axis cs:0.0002,6.88149847089383e-283)
--(axis cs:0.0004,5.38329324547414e-232)
--(axis cs:0.0006,1.80526708928995e-202)
--(axis cs:0.0008,1.0802222727238e-181)
--(axis cs:0.001,1.03927455538975e-165)
--(axis cs:0.0012,9.284355067112e-153)
--(axis cs:0.0014,6.70264999764463e-142)
--(axis cs:0.0016,1.42270014450123e-132)
--(axis cs:0.0018,2.04241209795704e-124)
--(axis cs:0.002,3.50239373677996e-117)
--(axis cs:0.0022,1.0763994483646e-110)
--(axis cs:0.0024,7.99953899528515e-105)
--(axis cs:0.0026,1.80482808273425e-99)
--(axis cs:0.0028,1.47530594980937e-94)
--(axis cs:0.003,5.0266949465783e-90)
--(axis cs:0.0032,7.99309022920627e-86)
--(axis cs:0.0034,6.50610273957532e-82)
--(axis cs:0.0036,2.92653432604146e-78)
--(axis cs:0.0038,7.75652633853781e-75)
--(axis cs:0.004,1.27887150345601e-71)
--(axis cs:0.0042,1.37387136164765e-68)
--(axis cs:0.0044,1.00075911725486e-65)
--(axis cs:0.0046,5.11650284411508e-63)
--(axis cs:0.0048,1.89215762100807e-60)
--(axis cs:0.005,5.19708066546697e-58)
--(axis cs:0.0052,1.08519112913327e-55)
--(axis cs:0.0054,1.75864114867945e-53)
--(axis cs:0.0056,2.25305595468255e-51)
--(axis cs:0.0058,2.31978418006334e-49)
--(axis cs:0.006,1.94819763121604e-47)
--(axis cs:0.0062,1.35246061951633e-45)
--(axis cs:0.0064,7.85536854187431e-44)
--(axis cs:0.0066,3.85939041326377e-42)
--(axis cs:0.0068,1.61999834203236e-40)
--(axis cs:0.007,5.8628757428418e-39)
--(axis cs:0.0072,1.84471455956781e-37)
--(axis cs:0.0074,5.08503961140203e-36)
--(axis cs:0.0076,1.23670034865055e-34)
--(axis cs:0.0078,2.67091172323767e-33)
--(axis cs:0.008,5.15329732842381e-32)
--(axis cs:0.0082,8.93211136472859e-31)
--(axis cs:0.0084,1.39799146700368e-29)
--(axis cs:0.0086,1.98525615382389e-28)
--(axis cs:0.0088,2.56937152193057e-27)
--(axis cs:0.009,3.04327839977601e-26)
--(axis cs:0.0092,3.31167624895744e-25)
--(axis cs:0.0094,3.32294847816438e-24)
--(axis cs:0.0096,3.08494431084593e-23)
--(axis cs:0.0098,2.65831236795176e-22)
--(axis cs:0.01,2.1325663849262e-21)
--(axis cs:0.0102,1.59721161269042e-20)
--(axis cs:0.0104,1.11979829006939e-19)
--(axis cs:0.0106,7.36754354703253e-19)
--(axis cs:0.0108,4.55972127990545e-18)
--(axis cs:0.011,2.66046231163419e-17)
--(axis cs:0.0112,1.46655123482879e-16)
--(axis cs:0.0114,7.65293478264192e-16)
--(axis cs:0.0116,3.7876809871124e-15)
--(axis cs:0.0118,1.78120985335917e-14)
--(axis cs:0.012,7.97251566953794e-14)
--(axis cs:0.0122,3.4018858950948e-13)
--(axis cs:0.0124,1.38598721522646e-12)
--(axis cs:0.0126,5.39948902691054e-12)
--(axis cs:0.0128,2.01422837127883e-11)
--(axis cs:0.013,7.20456391440064e-11)
--(axis cs:0.0132,2.47402100952021e-10)
--(axis cs:0.0134,8.16628494582459e-10)
--(axis cs:0.0136,2.59403838921817e-09)
--(axis cs:0.0138,7.93856711154157e-09)
--(axis cs:0.014,2.34306300004452e-08)
--(axis cs:0.0142,6.67643356640233e-08)
--(axis cs:0.0144,1.83843314480245e-07)
--(axis cs:0.0146,4.89666471389706e-07)
--(axis cs:0.0148,1.2626761922186e-06)
--(axis cs:0.015,3.15498203452717e-06)
--(axis cs:0.0152,7.64494508653064e-06)
--(axis cs:0.0154,1.7979170213845e-05)
--(axis cs:0.0156,4.10690407641051e-05)
--(axis cs:0.0158,9.11860238376426e-05)
--(axis cs:0.016,0.000196932737783124)
--(axis cs:0.0162,0.000413979796989082)
--(axis cs:0.0164,0.000847611318286239)
--(axis cs:0.0166,0.00169139539232139)
--(axis cs:0.0168,0.003291462393694)
--(axis cs:0.017,0.0062500454828601)
--(axis cs:0.0172,0.0115870850932161)
--(axis cs:0.0174,0.0209845482354142)
--(axis cs:0.0176,0.0371440265188652)
--(axis cs:0.0178,0.0642930463763373)
--(axis cs:0.018,0.108877714983386)
--(axis cs:0.0182,0.180476707043854)
--(axis cs:0.0184,0.292961727437675)
--(axis cs:0.0186,0.465910058625537)
--(axis cs:0.0188,0.726243790315821)
--(axis cs:0.019,1.11002725226193)
--(axis cs:0.0192,1.66430041509408)
--(axis cs:0.0194,2.4487655468853)
--(axis cs:0.0196,3.5370840535338)
--(axis cs:0.0198,5.01748973197458)
--(axis cs:0.02,6.99239507315247)
--(axis cs:0.0202,9.57667065386176)
--(axis cs:0.0204,12.8943243600301)
--(axis cs:0.0206,17.0734035981651)
--(axis cs:0.0208,22.2390900744056)
--(axis cs:0.021,28.5051457064319)
--(axis cs:0.0212,35.9640840071515)
--(axis cs:0.0214,44.6766605348988)
--(axis cs:0.0216,54.6614702084668)
--(axis cs:0.0218,65.8855783073964)
--(axis cs:0.022,78.2571685923711)
--(axis cs:0.0222,91.6211465226821)
--(axis cs:0.0224,103.118466326843)
--(axis cs:0.0226,88.8644303563046)
--(axis cs:0.0228,75.7215206023922)
--(axis cs:0.023,63.8107597647364)
--(axis cs:0.0232,53.1903606992744)
--(axis cs:0.0234,43.8647340785928)
--(axis cs:0.0236,35.794698231767)
--(axis cs:0.0238,28.9079762621495)
--(axis cs:0.024,23.1092422614582)
--(axis cs:0.0242,18.2891777620824)
--(axis cs:0.0244,14.3321973242304)
--(axis cs:0.0246,11.1226802899885)
--(axis cs:0.0248,8.5496927906303)
--(axis cs:0.025,6.51029473521079)
--(axis cs:0.0252,4.91160046880308)
--(axis cs:0.0254,3.67180249028864)
--(axis cs:0.0256,2.72038076443975)
--(axis cs:0.0258,1.9977124689199)
--(axis cs:0.026,1.4542752129583)
--(axis cs:0.0262,1.04960689175367)
--(axis cs:0.0264,0.751152344904728)
--(axis cs:0.0266,0.533094561978577)
--(axis cs:0.0268,0.37523879492738)
--(axis cs:0.027,0.261992996537101)
--(axis cs:0.0272,0.181468053245895)
--(axis cs:0.0274,0.124706245475069)
--(axis cs:0.0276,0.0850357725091174)
--(axis cs:0.0278,0.0575423270059376)
--(axis cs:0.028,0.038644824705325)
--(axis cs:0.0282,0.0257607381643917)
--(axis cs:0.0284,0.0170463826302373)
--(axis cs:0.0286,0.0111984039614772)
--(axis cs:0.0288,0.00730419057073215)
--(axis cs:0.029,0.0047306558591358)
--(axis cs:0.0292,0.00304259595678884)
--(axis cs:0.0294,0.00194348157065519)
--(axis cs:0.0296,0.0012330154694237)
--(axis cs:0.0298,0.000777045655683668)
--(axis cs:0.03,0.000486465062960411)
--(axis cs:0.0302,0.000302566146091308)
--(axis cs:0.0304,0.00018697704844155)
--(axis cs:0.0306,0.000114812871709676)
--(axis cs:0.0308,7.00586093683498e-05)
--(axis cs:0.031,4.2484916600035e-05)
--(axis cs:0.0312,2.56060962327199e-05)
--(axis cs:0.0314,1.53398004244612e-05)
--(axis cs:0.0316,9.13472646610638e-06)
--(axis cs:0.0318,5.40756879979802e-06)
--(axis cs:0.032,3.18251022839673e-06)
--(axis cs:0.0322,1.86221069963013e-06)
--(axis cs:0.0324,1.08344914359411e-06)
--(axis cs:0.0326,6.26812791986565e-07)
--(axis cs:0.0328,3.6061613226628e-07)
--(axis cs:0.033,2.06328055433611e-07)
--(axis cs:0.0332,1.17409868378581e-07)
--(axis cs:0.0334,6.64524433226212e-08)
--(axis cs:0.0336,3.74113788594143e-08)
--(axis cs:0.0338,2.09511944436424e-08)
--(axis cs:0.034,1.16721702412188e-08)
--(axis cs:0.0342,6.4692984389792e-09)
--(axis cs:0.0344,3.56738566186036e-09)
--(axis cs:0.0346,1.95728635927153e-09)
--(axis cs:0.0348,1.06854746121432e-09)
--(axis cs:0.035,5.80485940657903e-10)
--(axis cs:0.0352,3.13813005027762e-10)
--(axis cs:0.0354,1.68831712046916e-10)
--(axis cs:0.0356,9.03988896788645e-11)
--(axis cs:0.0358,4.81747989009494e-11)
--(axis cs:0.036,2.55532332739877e-11)
--(axis cs:0.0362,1.34915541309124e-11)
--(axis cs:0.0364,7.09069752729882e-12)
--(axis cs:0.0366,3.70977136869238e-12)
--(axis cs:0.0368,1.93221989411021e-12)
--(axis cs:0.037,1.00192879722832e-12)
--(axis cs:0.0372,5.17258269270647e-13)
--(axis cs:0.0374,2.65881030828973e-13)
--(axis cs:0.0376,1.36080326664345e-13)
--(axis cs:0.0378,6.93505420970246e-14)
--(axis cs:0.038,3.51940350459016e-14)
--(axis cs:0.0382,1.77856941332027e-14)
--(axis cs:0.0384,8.95102399128808e-15)
--(axis cs:0.0386,4.48633895026745e-15)
--(axis cs:0.0388,2.23947360167532e-15)
--(axis cs:0.039,1.11339969632567e-15)
--(axis cs:0.0392,5.51345784477401e-16)
--(axis cs:0.0394,2.71945011684331e-16)
--cycle;
\path [draw=none, fill=blue, fill opacity=0.5]
(axis cs:0,0)
--(axis cs:0.0002,6.88149847089383e-283)
--(axis cs:0.0004,5.38329324547414e-232)
--(axis cs:0.0006,1.80526708928995e-202)
--(axis cs:0.0008,1.0802222727238e-181)
--(axis cs:0.001,1.03927455538975e-165)
--(axis cs:0.0012,9.284355067112e-153)
--(axis cs:0.0014,6.70264999764463e-142)
--(axis cs:0.0016,1.42270014450123e-132)
--(axis cs:0.0018,2.04241209795704e-124)
--(axis cs:0.002,3.50239373677996e-117)
--(axis cs:0.0022,1.0763994483646e-110)
--(axis cs:0.0024,7.99953899528515e-105)
--(axis cs:0.0026,1.80482808273425e-99)
--(axis cs:0.0028,1.47530594980937e-94)
--(axis cs:0.003,5.0266949465783e-90)
--(axis cs:0.0032,7.99309022920627e-86)
--(axis cs:0.0034,6.50610273957532e-82)
--(axis cs:0.0036,2.92653432604146e-78)
--(axis cs:0.0038,7.75652633853781e-75)
--(axis cs:0.004,1.27887150345601e-71)
--(axis cs:0.0042,1.37387136164765e-68)
--(axis cs:0.0044,1.00075911725486e-65)
--(axis cs:0.0046,5.11650284411508e-63)
--(axis cs:0.0048,1.89215762100807e-60)
--(axis cs:0.005,5.19708066546697e-58)
--(axis cs:0.0052,1.08519112913327e-55)
--(axis cs:0.0054,1.75864114867945e-53)
--(axis cs:0.0056,2.25305595468255e-51)
--(axis cs:0.0058,2.31978418006334e-49)
--(axis cs:0.006,1.94819763121604e-47)
--(axis cs:0.0062,1.35246061951633e-45)
--(axis cs:0.0064,7.85536854187431e-44)
--(axis cs:0.0066,3.85939041326377e-42)
--(axis cs:0.0068,1.61999834203236e-40)
--(axis cs:0.007,5.8628757428418e-39)
--(axis cs:0.0072,1.84471455956781e-37)
--(axis cs:0.0074,5.08503961140203e-36)
--(axis cs:0.0076,1.23670034865055e-34)
--(axis cs:0.0078,2.67091172323767e-33)
--(axis cs:0.008,5.15329732842381e-32)
--(axis cs:0.0082,8.93211136472859e-31)
--(axis cs:0.0084,1.39799146700368e-29)
--(axis cs:0.0086,1.98525615382389e-28)
--(axis cs:0.0088,2.56937152193057e-27)
--(axis cs:0.009,3.04327839977601e-26)
--(axis cs:0.0092,3.31167624895744e-25)
--(axis cs:0.0094,3.32294847816438e-24)
--(axis cs:0.0096,3.08494431084593e-23)
--(axis cs:0.0098,2.65831236795176e-22)
--(axis cs:0.01,2.1325663849262e-21)
--(axis cs:0.0102,1.59721161269042e-20)
--(axis cs:0.0104,1.11979829006939e-19)
--(axis cs:0.0106,7.36754354703253e-19)
--(axis cs:0.0108,4.55972127990545e-18)
--(axis cs:0.011,2.66046231163419e-17)
--(axis cs:0.0112,1.46655123482879e-16)
--(axis cs:0.0114,7.65293478264192e-16)
--(axis cs:0.0116,3.7876809871124e-15)
--(axis cs:0.0118,1.78120985335917e-14)
--(axis cs:0.012,7.97251566953794e-14)
--(axis cs:0.0122,3.4018858950948e-13)
--(axis cs:0.0124,1.38598721522646e-12)
--(axis cs:0.0126,5.39948902691054e-12)
--(axis cs:0.0128,2.01422837127883e-11)
--(axis cs:0.013,7.20456391440064e-11)
--(axis cs:0.0132,2.47402100952021e-10)
--(axis cs:0.0134,8.16628494582459e-10)
--(axis cs:0.0136,2.59403838921817e-09)
--(axis cs:0.0138,7.93856711154157e-09)
--(axis cs:0.014,2.34306300004452e-08)
--(axis cs:0.0142,6.67643356640233e-08)
--(axis cs:0.0144,1.83843314480245e-07)
--(axis cs:0.0146,4.89666471389706e-07)
--(axis cs:0.0148,1.2626761922186e-06)
--(axis cs:0.015,3.15498203452717e-06)
--(axis cs:0.0152,7.64494508653064e-06)
--(axis cs:0.0154,1.7979170213845e-05)
--(axis cs:0.0156,3.7658079611177e-05)
--(axis cs:0.0158,1.87420165753465e-05)
--(axis cs:0.016,9.24250139028707e-06)
--(axis cs:0.0162,4.5172452326614e-06)
--(axis cs:0.0164,2.18857329389199e-06)
--(axis cs:0.0166,1.05133420160334e-06)
--(axis cs:0.0168,5.00839471261352e-07)
--(axis cs:0.017,2.36655772209254e-07)
--(axis cs:0.0172,1.10936966886697e-07)
--(axis cs:0.0174,5.16004121551136e-08)
--(axis cs:0.0176,2.38189257271155e-08)
--(axis cs:0.0178,1.09132770888478e-08)
--(axis cs:0.018,4.96388475445766e-09)
--(axis cs:0.0182,2.2417587444063e-09)
--(axis cs:0.0184,1.00536023534575e-09)
--(axis cs:0.0186,4.47798927374872e-10)
--(axis cs:0.0188,1.98122479510485e-10)
--(axis cs:0.019,8.7082869969346e-11)
--(axis cs:0.0192,3.80309388371811e-11)
--(axis cs:0.0194,1.65044850569294e-11)
--(axis cs:0.0196,7.11837908731441e-12)
--(axis cs:0.0198,3.05159149971245e-12)
--(axis cs:0.02,1.30043368831224e-12)
--(axis cs:0.0202,5.50954117508255e-13)
--(axis cs:0.0204,2.32089498543993e-13)
--(axis cs:0.0206,9.72197326187167e-14)
--(axis cs:0.0208,4.0500178081169e-14)
--(axis cs:0.021,1.67805596237452e-14)
--(axis cs:0.0212,6.91584302087708e-15)
--(axis cs:0.0214,2.83539771694205e-15)
--(axis cs:0.0216,1.15651924143567e-15)
--(axis cs:0.0218,4.69353977721349e-16)
--(axis cs:0.022,1.89537255017966e-16)
--(axis cs:0.0222,7.61678629358275e-17)
--(axis cs:0.0224,3.04626530052703e-17)
--(axis cs:0.0226,1.21259887645579e-17)
--(axis cs:0.0228,4.80456288052996e-18)
--(axis cs:0.023,1.8950068938661e-18)
--(axis cs:0.0232,7.44080551455369e-19)
--(axis cs:0.0234,2.90879145590291e-19)
--(axis cs:0.0236,1.1321894943238e-19)
--(axis cs:0.0238,4.38802622641073e-20)
--(axis cs:0.024,1.69352564892312e-20)
--(axis cs:0.0242,6.50901095377492e-21)
--(axis cs:0.0244,2.49153147957588e-21)
--(axis cs:0.0246,9.49888708849199e-22)
--(axis cs:0.0248,3.60711681893591e-22)
--(axis cs:0.025,1.3644397353969e-22)
--(axis cs:0.0252,5.14138469226384e-23)
--(axis cs:0.0254,1.93001978872092e-23)
--(axis cs:0.0256,7.21810340585825e-24)
--(axis cs:0.0258,2.68959762500427e-24)
--(axis cs:0.026,9.98566661792328e-25)
--(axis cs:0.0262,3.6941496807468e-25)
--(axis cs:0.0264,1.36182474916111e-25)
--(axis cs:0.0266,5.00286065871767e-26)
--(axis cs:0.0268,1.83158343383945e-26)
--(axis cs:0.027,6.68292097066354e-27)
--(axis cs:0.0272,2.43028394239313e-27)
--(axis cs:0.0274,8.80882743445791e-28)
--(axis cs:0.0276,3.18250047040028e-28)
--(axis cs:0.0278,1.14611085680212e-28)
--(axis cs:0.028,4.11443911103716e-29)
--(axis cs:0.0282,1.47244206536675e-29)
--(axis cs:0.0284,5.25323318970638e-30)
--(axis cs:0.0286,1.86849950728891e-30)
--(axis cs:0.0288,6.62603526420191e-31)
--(axis cs:0.029,2.34274598112145e-31)
--(axis cs:0.0292,8.25892046285214e-32)
--(axis cs:0.0294,2.90311060532944e-32)
--(axis cs:0.0296,1.01756315239963e-32)
--(axis cs:0.0298,3.5565721067613e-33)
--(axis cs:0.03,1.23962116880601e-33)
--(axis cs:0.0302,4.30871570850815e-34)
--(axis cs:0.0304,1.49355861358081e-34)
--(axis cs:0.0306,5.16328482435287e-35)
--(axis cs:0.0308,1.78021645790086e-35)
--(axis cs:0.031,6.12175196186357e-36)
--(axis cs:0.0312,2.09965456035204e-36)
--(axis cs:0.0314,7.18293501274318e-37)
--(axis cs:0.0316,2.45104063814212e-37)
--(axis cs:0.0318,8.34268518420225e-38)
--(axis cs:0.032,2.83256587034247e-38)
--(axis cs:0.0322,9.59367104341985e-39)
--(axis cs:0.0324,3.24139436756253e-39)
--(axis cs:0.0326,1.09252795039009e-39)
--(axis cs:0.0328,3.67365231123331e-40)
--(axis cs:0.033,1.23236492322482e-40)
--(axis cs:0.0332,4.12446063746452e-41)
--(axis cs:0.0334,1.37718452035046e-41)
--(axis cs:0.0336,4.58801283402551e-42)
--(axis cs:0.0338,1.52501754192857e-42)
--(axis cs:0.034,5.05769688891288e-43)
--(axis cs:0.0342,1.67366426563217e-43)
--(axis cs:0.0344,5.52625785272222e-44)
--(axis cs:0.0346,1.82075161131955e-44)
--(axis cs:0.0348,5.98599503975823e-45)
--(axis cs:0.035,1.9638000548463e-45)
--(axis cs:0.0352,6.42898577109859e-46)
--(axis cs:0.0354,2.1002974480877e-46)
--(axis cs:0.0356,6.84732590989572e-47)
--(axis cs:0.0358,2.22777646835226e-47)
--(axis cs:0.036,7.23337637178125e-48)
--(axis cs:0.0362,2.34389218482664e-48)
--(axis cs:0.0364,7.58000438263052e-49)
--(axis cs:0.0366,2.44649620718702e-49)
--(axis cs:0.0368,7.88080852298557e-50)
--(axis cs:0.037,2.53370407054759e-50)
--(axis cs:0.0372,8.13031751342969e-51)
--(axis cs:0.0374,2.60395289449275e-51)
--(axis cs:0.0376,8.32415418185093e-52)
--(axis cs:0.0378,2.6560469014955e-52)
--(axis cs:0.038,8.45915799726862e-53)
--(axis cs:0.0382,2.68918905048345e-53)
--(axis cs:0.0384,8.53346181345964e-54)
--(axis cs:0.0386,2.70299778679314e-54)
--(axis cs:0.0388,8.54652088167659e-55)
--(axis cs:0.039,2.69750867965481e-55)
--(axis cs:0.0392,8.49909460584531e-56)
--(axis cs:0.0394,2.67316146339825e-56)
--cycle;
\path [draw=none, fill=blue, fill opacity=0.5]
(axis cs:0,0)
--(axis cs:0.0002,6.88149847089383e-283)
--(axis cs:0.0004,5.38329324547414e-232)
--(axis cs:0.0006,1.80526708928995e-202)
--(axis cs:0.0008,1.0802222727238e-181)
--(axis cs:0.001,1.03927455538975e-165)
--(axis cs:0.0012,9.284355067112e-153)
--(axis cs:0.0014,6.70264999764463e-142)
--(axis cs:0.0016,1.42270014450123e-132)
--(axis cs:0.0018,2.04241209795704e-124)
--(axis cs:0.002,3.50239373677996e-117)
--(axis cs:0.0022,1.0763994483646e-110)
--(axis cs:0.0024,7.99953899528515e-105)
--(axis cs:0.0026,1.80482808273425e-99)
--(axis cs:0.0028,1.47530594980937e-94)
--(axis cs:0.003,5.0266949465783e-90)
--(axis cs:0.0032,7.99309022920627e-86)
--(axis cs:0.0034,6.50610273957532e-82)
--(axis cs:0.0036,2.92653432604146e-78)
--(axis cs:0.0038,7.75652633853781e-75)
--(axis cs:0.004,1.27887150345601e-71)
--(axis cs:0.0042,1.37387136164765e-68)
--(axis cs:0.0044,1.00075911725486e-65)
--(axis cs:0.0046,5.11650284411508e-63)
--(axis cs:0.0048,1.89215762100807e-60)
--(axis cs:0.005,5.19708066546697e-58)
--(axis cs:0.0052,1.08519112913327e-55)
--(axis cs:0.0054,1.75864114867945e-53)
--(axis cs:0.0056,2.25305595468255e-51)
--(axis cs:0.0058,2.31978418006334e-49)
--(axis cs:0.006,1.94819763121604e-47)
--(axis cs:0.0062,1.35246061951633e-45)
--(axis cs:0.0064,7.85536854187431e-44)
--(axis cs:0.0066,3.85939041326377e-42)
--(axis cs:0.0068,1.61999834203236e-40)
--(axis cs:0.007,5.8628757428418e-39)
--(axis cs:0.0072,1.84471455956781e-37)
--(axis cs:0.0074,5.08503961140203e-36)
--(axis cs:0.0076,1.23670034865055e-34)
--(axis cs:0.0078,2.67091172323767e-33)
--(axis cs:0.008,5.15329732842381e-32)
--(axis cs:0.0082,8.93211136472859e-31)
--(axis cs:0.0084,1.39799146700368e-29)
--(axis cs:0.0086,1.98525615382389e-28)
--(axis cs:0.0088,2.56937152193057e-27)
--(axis cs:0.009,3.04327839977601e-26)
--(axis cs:0.0092,3.31167624895744e-25)
--(axis cs:0.0094,3.32294847816438e-24)
--(axis cs:0.0096,3.08494431084593e-23)
--(axis cs:0.0098,2.65831236795176e-22)
--(axis cs:0.01,2.1325663849262e-21)
--(axis cs:0.0102,1.59721161269042e-20)
--(axis cs:0.0104,1.11979829006939e-19)
--(axis cs:0.0106,7.36754354703253e-19)
--(axis cs:0.0108,4.55972127990545e-18)
--(axis cs:0.011,2.66046231163419e-17)
--(axis cs:0.0112,1.46655123482879e-16)
--(axis cs:0.0114,7.65293478264192e-16)
--(axis cs:0.0116,3.7876809871124e-15)
--(axis cs:0.0118,1.78120985335917e-14)
--(axis cs:0.012,7.97251566953794e-14)
--(axis cs:0.0122,3.4018858950948e-13)
--(axis cs:0.0124,1.38598721522646e-12)
--(axis cs:0.0126,5.39948902691054e-12)
--(axis cs:0.0128,2.01422837127883e-11)
--(axis cs:0.013,7.20456391440064e-11)
--(axis cs:0.0132,2.47402100952021e-10)
--(axis cs:0.0134,8.16628494582459e-10)
--(axis cs:0.0136,2.59403838921817e-09)
--(axis cs:0.0138,7.93856711154157e-09)
--(axis cs:0.014,2.34306300004452e-08)
--(axis cs:0.0142,6.67643356640233e-08)
--(axis cs:0.0144,1.83843314480245e-07)
--(axis cs:0.0146,4.89666471389706e-07)
--(axis cs:0.0148,1.2626761922186e-06)
--(axis cs:0.015,3.15498203452717e-06)
--(axis cs:0.0152,7.64494508653064e-06)
--(axis cs:0.0154,1.7979170213845e-05)
--(axis cs:0.0156,4.10690407641051e-05)
--(axis cs:0.0158,9.11860238376426e-05)
--(axis cs:0.016,0.000196932737783124)
--(axis cs:0.0162,0.000413979796989082)
--(axis cs:0.0164,0.000847611318286239)
--(axis cs:0.0166,0.00169139539232139)
--(axis cs:0.0168,0.003291462393694)
--(axis cs:0.017,0.0062500454828601)
--(axis cs:0.0172,0.0115870850932161)
--(axis cs:0.0174,0.00778273944702365)
--(axis cs:0.0176,0.00448060903275038)
--(axis cs:0.0178,0.00255411761700844)
--(axis cs:0.018,0.00144190717862481)
--(axis cs:0.0182,0.000806336789362508)
--(axis cs:0.0184,0.000446751107565886)
--(axis cs:0.0186,0.000245283985078712)
--(axis cs:0.0188,0.00013347773643944)
--(axis cs:0.019,7.20051655177482e-05)
--(axis cs:0.0192,3.85132583475396e-05)
--(axis cs:0.0194,2.04278607692691e-05)
--(axis cs:0.0196,1.07466590998682e-05)
--(axis cs:0.0198,5.60830677274215e-06)
--(axis cs:0.02,2.90379129483597e-06)
--(axis cs:0.0202,1.49190101011772e-06)
--(axis cs:0.0204,7.60710253355381e-07)
--(axis cs:0.0206,3.85003786698849e-07)
--(axis cs:0.0208,1.93435964732349e-07)
--(axis cs:0.021,9.64926505877817e-08)
--(axis cs:0.0212,4.77961238059279e-08)
--(axis cs:0.0214,2.35118925731706e-08)
--(axis cs:0.0216,1.14876838412627e-08)
--(axis cs:0.0218,5.57544271889242e-09)
--(axis cs:0.022,2.68830683625505e-09)
--(axis cs:0.0222,1.28789403121594e-09)
--(axis cs:0.0224,6.13099745745081e-10)
--(axis cs:0.0226,2.90053740312067e-10)
--(axis cs:0.0228,1.36385285237948e-10)
--(axis cs:0.023,6.37444630840224e-11)
--(axis cs:0.0232,2.96173615640384e-11)
--(axis cs:0.0234,1.36811041615753e-11)
--(axis cs:0.0236,6.2835876296212e-12)
--(axis cs:0.0238,2.86976128942386e-12)
--(axis cs:0.024,1.3033896418131e-12)
--(axis cs:0.0242,5.88750121415524e-13)
--(axis cs:0.0244,2.64516597164137e-13)
--(axis cs:0.0246,1.18215938123978e-13)
--(axis cs:0.0248,5.255759287684e-14)
--(axis cs:0.025,2.32469252232671e-14)
--(axis cs:0.0252,1.02305659610502e-14)
--(axis cs:0.0254,4.47992316053469e-15)
--(axis cs:0.0256,1.95213630999053e-15)
--(axis cs:0.0258,8.46544011725541e-16)
--(axis cs:0.026,3.65358493949292e-16)
--(axis cs:0.0262,1.56945502083572e-16)
--(axis cs:0.0264,6.71069643194406e-17)
--(axis cs:0.0266,2.85629962894664e-17)
--(axis cs:0.0268,1.21027800607901e-17)
--(axis cs:0.027,5.10550684580449e-18)
--(axis cs:0.0272,2.14432921898816e-18)
--(axis cs:0.0274,8.96744853650233e-19)
--(axis cs:0.0276,3.73419105674198e-19)
--(axis cs:0.0278,1.54845725029218e-19)
--(axis cs:0.028,6.39442128938801e-20)
--(axis cs:0.0282,2.6298222127352e-20)
--(axis cs:0.0284,1.07720391140932e-20)
--(axis cs:0.0286,4.39479501814542e-21)
--(axis cs:0.0288,1.78595617606647e-21)
--(axis cs:0.029,7.2296335358632e-22)
--(axis cs:0.0292,2.91538871761007e-22)
--(axis cs:0.0294,1.17120333717999e-22)
--(axis cs:0.0296,4.6875326137853e-23)
--(axis cs:0.0298,1.86918625143203e-23)
--(axis cs:0.03,7.42637420595622e-24)
--(axis cs:0.0302,2.93992575451275e-24)
--(axis cs:0.0304,1.15971187226631e-24)
--(axis cs:0.0306,4.5586536851254e-25)
--(axis cs:0.0308,1.78572279395691e-25)
--(axis cs:0.031,6.97108311119898e-26)
--(axis cs:0.0312,2.71214399231177e-26)
--(axis cs:0.0314,1.05164374156872e-26)
--(axis cs:0.0316,4.06429084905467e-27)
--(axis cs:0.0318,1.56558863728127e-27)
--(axis cs:0.032,6.0112318400383e-28)
--(axis cs:0.0322,2.30069008420888e-28)
--(axis cs:0.0324,8.77762820435934e-29)
--(axis cs:0.0326,3.33838149293861e-29)
--(axis cs:0.0328,1.26575453175437e-29)
--(axis cs:0.033,4.78445538515688e-30)
--(axis cs:0.0332,1.80301565501958e-30)
--(axis cs:0.0334,6.77430386453264e-31)
--(axis cs:0.0336,2.53771003756134e-31)
--(axis cs:0.0338,9.47862151705772e-32)
--(axis cs:0.034,3.5301055865143e-32)
--(axis cs:0.0342,1.31093985672291e-32)
--(axis cs:0.0344,4.85448760330276e-33)
--(axis cs:0.0346,1.79259570964789e-33)
--(axis cs:0.0348,6.60103727962226e-34)
--(axis cs:0.035,2.42407039237887e-34)
--(axis cs:0.0352,8.87755963276196e-35)
--(axis cs:0.0354,3.24241957372043e-35)
--(axis cs:0.0356,1.18109214487747e-35)
--(axis cs:0.0358,4.29090419393295e-36)
--(axis cs:0.036,1.55480397682429e-36)
--(axis cs:0.0362,5.61921160918253e-37)
--(axis cs:0.0364,2.02562476103656e-37)
--(axis cs:0.0366,7.28345159341886e-38)
--(axis cs:0.0368,2.61228585498635e-38)
--(axis cs:0.037,9.34587126154641e-39)
--(axis cs:0.0372,3.33537578121633e-39)
--(axis cs:0.0374,1.18742368919572e-39)
--(axis cs:0.0376,4.2170866661322e-40)
--(axis cs:0.0378,1.49408388950585e-40)
--(axis cs:0.038,5.28083602364121e-41)
--(axis cs:0.0382,1.86210890398195e-41)
--(axis cs:0.0384,6.55075669093197e-42)
--(axis cs:0.0386,2.29916964220208e-42)
--(axis cs:0.0388,8.05105388170408e-43)
--(axis cs:0.039,2.81284300656489e-43)
--(axis cs:0.0392,9.80523563488412e-44)
--(axis cs:0.0394,3.41035006246725e-44)
--cycle;
\path [draw=none, fill=blue, fill opacity=0.5]
(axis cs:0,0)
--(axis cs:0.0002,7.07051949791944e-220)
--(axis cs:0.0004,3.35713224621954e-178)
--(axis cs:0.0006,4.55876472150082e-154)
--(axis cs:0.0008,4.06372766744917e-137)
--(axis cs:0.001,4.19927757126269e-124)
--(axis cs:0.0012,1.40567805458719e-113)
--(axis cs:0.0014,9.00643402552461e-105)
--(axis cs:0.0016,3.18925325198261e-97)
--(axis cs:0.0018,1.23690071259004e-90)
--(axis cs:0.002,8.38122637353824e-85)
--(axis cs:0.0022,1.38442183846587e-79)
--(axis cs:0.0024,7.12902189670533e-75)
--(axis cs:0.0026,1.3794522412844e-70)
--(axis cs:0.0028,1.15971222412979e-66)
--(axis cs:0.003,4.75273753659039e-63)
--(axis cs:0.0032,1.04179315001234e-59)
--(axis cs:0.0034,1.31771394306738e-56)
--(axis cs:0.0036,1.02415094765992e-53)
--(axis cs:0.0038,5.15560039198329e-51)
--(axis cs:0.004,1.75757764872752e-48)
--(axis cs:0.0042,4.2148638496568e-46)
--(axis cs:0.0044,7.3467295284001e-44)
--(axis cs:0.0046,9.57542435968325e-42)
--(axis cs:0.0048,9.56565522459126e-40)
--(axis cs:0.005,7.48492710605918e-38)
--(axis cs:0.0052,4.67617522469868e-36)
--(axis cs:0.0054,2.37244858785325e-34)
--(axis cs:0.0056,9.92369722206861e-33)
--(axis cs:0.0058,3.46894198997681e-31)
--(axis cs:0.006,1.02575999424703e-29)
--(axis cs:0.0062,2.59405173111155e-28)
--(axis cs:0.0064,5.66632176036028e-27)
--(axis cs:0.0066,1.07875184310474e-25)
--(axis cs:0.0068,1.80466875933379e-24)
--(axis cs:0.007,2.67286801654633e-23)
--(axis cs:0.0072,3.52886813743904e-22)
--(axis cs:0.0074,4.1792715817461e-21)
--(axis cs:0.0076,4.46564241233065e-20)
--(axis cs:0.0078,4.32813414557636e-19)
--(axis cs:0.008,3.82375668734256e-18)
--(axis cs:0.0082,3.09338087199957e-17)
--(axis cs:0.0084,2.3012710139591e-16)
--(axis cs:0.0086,1.58052451958739e-15)
--(axis cs:0.0088,1.00582834740469e-14)
--(axis cs:0.009,5.95139142432837e-14)
--(axis cs:0.0092,3.28450725704436e-13)
--(axis cs:0.0094,1.69580021234189e-12)
--(axis cs:0.0096,8.21383031334312e-12)
--(axis cs:0.0098,3.74215771035445e-11)
--(axis cs:0.01,1.60758325803006e-10)
--(axis cs:0.0102,6.52688622108691e-10)
--(axis cs:0.0104,2.50996299416417e-09)
--(axis cs:0.0106,9.16116446427591e-09)
--(axis cs:0.0108,3.17979701351146e-08)
--(axis cs:0.011,1.05150251268355e-07)
--(axis cs:0.0112,3.31846466176376e-07)
--(axis cs:0.0114,1.00113761423932e-06)
--(axis cs:0.0116,2.89172206747296e-06)
--(axis cs:0.0118,8.00879727229867e-06)
--(axis cs:0.012,2.12978350679836e-05)
--(axis cs:0.0122,5.44553092320697e-05)
--(axis cs:0.0124,0.000134039423465113)
--(axis cs:0.0126,0.000318007176388045)
--(axis cs:0.0128,0.000728036322089197)
--(axis cs:0.013,0.0016101168440237)
--(axis cs:0.0132,0.00344353575065258)
--(axis cs:0.0134,0.00712900268421548)
--(axis cs:0.0136,0.0143002873493135)
--(axis cs:0.0138,0.0278194928102673)
--(axis cs:0.014,0.0525317220624886)
--(axis cs:0.0142,0.0963663542846326)
--(axis cs:0.0144,0.171873415509561)
--(axis cs:0.0146,0.298266989051259)
--(axis cs:0.0148,0.504005145686858)
--(axis cs:0.015,0.829860797883638)
--(axis cs:0.0152,1.33232757972859)
--(axis cs:0.0154,2.08706366547733)
--(axis cs:0.0156,3.19191805935217)
--(axis cs:0.0158,4.76893233237364)
--(axis cs:0.016,6.96459902866247)
--(axis cs:0.0162,9.94762374169503)
--(axis cs:0.0164,13.9035166777831)
--(axis cs:0.0166,19.0255558911488)
--(axis cs:0.0168,25.5020229662806)
--(axis cs:0.017,33.5000909117679)
--(axis cs:0.0172,43.1472928224831)
--(axis cs:0.0174,54.5120428038473)
--(axis cs:0.0176,67.5851270066647)
--(axis cs:0.0178,82.2643417177921)
--(axis cs:0.018,98.3444542406121)
--(axis cs:0.0182,115.514361790022)
--(axis cs:0.0184,133.362730410417)
--(axis cs:0.0186,151.392565798958)
--(axis cs:0.0188,169.04420019866)
--(axis cs:0.019,180.643788201455)
--(axis cs:0.0192,162.494169280991)
--(axis cs:0.0194,144.186468575044)
--(axis cs:0.0196,126.24152132235)
--(axis cs:0.0198,109.090345237702)
--(axis cs:0.02,93.0654979661408)
--(axis cs:0.0202,78.4003417082175)
--(axis cs:0.0204,65.2348133355188)
--(axis cs:0.0206,53.6259974478233)
--(axis cs:0.0208,43.5617320532775)
--(axis cs:0.021,34.9755978219366)
--(axis cs:0.0212,27.7618954211491)
--(axis cs:0.0214,21.7895424162046)
--(axis cs:0.0216,16.9141687949919)
--(axis cs:0.0218,12.9880167701224)
--(axis cs:0.022,9.86752800549861)
--(axis cs:0.0222,7.41871492887164)
--(axis cs:0.0224,5.52055875933554)
--(axis cs:0.0226,4.06676018328376)
--(axis cs:0.0228,2.96619948938404)
--(axis cs:0.023,2.14245417264068)
--(axis cs:0.0232,1.53268653943172)
--(axis cs:0.0234,1.08616336780916)
--(axis cs:0.0236,0.76261366342503)
--(axis cs:0.0238,0.530575922349809)
--(axis cs:0.024,0.365837537964855)
--(axis cs:0.0242,0.250028440537005)
--(axis cs:0.0244,0.169399480962606)
--(axis cs:0.0246,0.113793080852648)
--(axis cs:0.0248,0.0757982021597045)
--(axis cs:0.025,0.0500723496819741)
--(axis cs:0.0252,0.0328086565119342)
--(axis cs:0.0254,0.0213247725075256)
--(axis cs:0.0256,0.0137511384350888)
--(axis cs:0.0258,0.00879838085750516)
--(axis cs:0.026,0.00558633645818186)
--(axis cs:0.0262,0.00352014751036091)
--(axis cs:0.0264,0.0022016697110189)
--(axis cs:0.0266,0.0013669354400426)
--(axis cs:0.0268,0.000842547340201024)
--(axis cs:0.027,0.000515627426063578)
--(axis cs:0.0272,0.000313340849114407)
--(axis cs:0.0274,0.000189095104711343)
--(axis cs:0.0276,0.00011333598801155)
--(axis cs:0.0278,6.74715624447029e-05)
--(axis cs:0.028,3.99005885881548e-05)
--(axis cs:0.0282,2.34413596984823e-05)
--(axis cs:0.0284,1.3682636249639e-05)
--(axis cs:0.0286,7.93556697071585e-06)
--(axis cs:0.0288,4.57345341808433e-06)
--(axis cs:0.029,2.61941815048323e-06)
--(axis cs:0.0292,1.49105988537527e-06)
--(axis cs:0.0294,8.43625509032096e-07)
--(axis cs:0.0296,4.74463336067587e-07)
--(axis cs:0.0298,2.65269451197919e-07)
--(axis cs:0.03,1.47447041872296e-07)
--(axis cs:0.0302,8.14856437416443e-08)
--(axis cs:0.0304,4.47768648879564e-08)
--(axis cs:0.0306,2.44672142554971e-08)
--(axis cs:0.0308,1.32954714741728e-08)
--(axis cs:0.031,7.18523313166044e-09)
--(axis cs:0.0312,3.86211184480844e-09)
--(axis cs:0.0314,2.0648290071526e-09)
--(axis cs:0.0316,1.09811155552866e-09)
--(axis cs:0.0318,5.80950707694493e-10)
--(axis cs:0.032,3.05766131800596e-10)
--(axis cs:0.0322,1.60111695949653e-10)
--(axis cs:0.0324,8.34192239271124e-11)
--(axis cs:0.0326,4.32458052135747e-11)
--(axis cs:0.0328,2.2309069997019e-11)
--(axis cs:0.033,1.14525713369952e-11)
--(axis cs:0.0332,5.85103354789937e-12)
--(axis cs:0.0334,2.97504653091987e-12)
--(axis cs:0.0336,1.50560016949624e-12)
--(axis cs:0.0338,7.58408127498507e-13)
--(axis cs:0.034,3.80273483880839e-13)
--(axis cs:0.0342,1.89806362745019e-13)
--(axis cs:0.0344,9.43123749013075e-14)
--(axis cs:0.0346,4.66542210861782e-14)
--(axis cs:0.0348,2.29772655367192e-14)
--(axis cs:0.035,1.12670783224797e-14)
--(axis cs:0.0352,5.50110500809886e-15)
--(axis cs:0.0354,2.67444670514748e-15)
--(axis cs:0.0356,1.29474057046822e-15)
--(axis cs:0.0358,6.24188034795496e-16)
--(axis cs:0.036,2.9967516537154e-16)
--(axis cs:0.0362,1.43287117184139e-16)
--(axis cs:0.0364,6.82343019931502e-17)
--(axis cs:0.0366,3.23634952024334e-17)
--(axis cs:0.0368,1.52891270231179e-17)
--(axis cs:0.037,7.19452002562637e-18)
--(axis cs:0.0372,3.37232801960942e-18)
--(axis cs:0.0374,1.57464751428699e-18)
--(axis cs:0.0376,7.32451525804251e-19)
--(axis cs:0.0378,3.39416222766471e-19)
--(axis cs:0.038,1.56696896785998e-19)
--(axis cs:0.0382,7.20738888345073e-20)
--(axis cs:0.0384,3.30294009683984e-20)
--(axis cs:0.0386,1.5081474411321e-20)
--(axis cs:0.0388,6.86154969376638e-21)
--(axis cs:0.039,3.11064639137136e-21)
--(axis cs:0.0392,1.40521760798034e-21)
--(axis cs:0.0394,6.32579760713982e-22)
--cycle;
\path [draw=none, fill=blue, fill opacity=0.5]
(axis cs:0,0)
--(axis cs:0.0002,1.24681940844559e-185)
--(axis cs:0.0004,3.78429989649157e-149)
--(axis cs:0.0006,4.69624390399748e-128)
--(axis cs:0.0008,2.91814488510357e-113)
--(axis cs:0.001,6.39937004086229e-102)
--(axis cs:0.0012,9.19292845684038e-93)
--(axis cs:0.0014,4.10935002282113e-85)
--(axis cs:0.0016,1.44888628628164e-78)
--(axis cs:0.0018,7.34152876082861e-73)
--(axis cs:0.002,8.0525019711728e-68)
--(axis cs:0.0022,2.56068941187556e-63)
--(axis cs:0.0024,2.9292421581097e-59)
--(axis cs:0.0026,1.41995830907816e-55)
--(axis cs:0.0028,3.31301642990214e-52)
--(axis cs:0.003,4.11568387030809e-49)
--(axis cs:0.0032,2.95307692152295e-46)
--(axis cs:0.0034,1.30806963691571e-43)
--(axis cs:0.0036,3.77970210701459e-41)
--(axis cs:0.0038,7.46130251996167e-39)
--(axis cs:0.004,1.04633990725249e-36)
--(axis cs:0.0042,1.07774752268699e-34)
--(axis cs:0.0044,8.3909217479038e-33)
--(axis cs:0.0046,5.06232673594148e-31)
--(axis cs:0.0048,2.41856883933146e-29)
--(axis cs:0.005,9.32610456511297e-28)
--(axis cs:0.0052,2.95167434916092e-26)
--(axis cs:0.0054,7.78268945730413e-25)
--(axis cs:0.0056,1.73238744737497e-23)
--(axis cs:0.0058,3.29434213604689e-22)
--(axis cs:0.006,5.40916986257712e-21)
--(axis cs:0.0062,7.74292596344275e-20)
--(axis cs:0.0064,9.74695105038983e-19)
--(axis cs:0.0066,1.08755078580298e-17)
--(axis cs:0.0068,1.08334855272526e-16)
--(axis cs:0.007,9.69782013606882e-16)
--(axis cs:0.0072,7.84830360507471e-15)
--(axis cs:0.0074,5.7738733862825e-14)
--(axis cs:0.0076,3.88106550443288e-13)
--(axis cs:0.0078,2.39473656171717e-12)
--(axis cs:0.008,1.36227283385691e-11)
--(axis cs:0.0082,7.17311040949162e-11)
--(axis cs:0.0084,3.50913326863578e-10)
--(axis cs:0.0086,1.6004392380367e-09)
--(axis cs:0.0088,6.82684778432269e-09)
--(axis cs:0.009,2.73176680030113e-08)
--(axis cs:0.0092,1.02831056135345e-07)
--(axis cs:0.0094,3.65088994324908e-07)
--(axis cs:0.0096,1.22555178042531e-06)
--(axis cs:0.0098,3.89871873397428e-06)
--(axis cs:0.01,1.17789817815586e-05)
--(axis cs:0.0102,3.3866611040304e-05)
--(axis cs:0.0104,9.28421083067902e-05)
--(axis cs:0.0106,0.000243114650517988)
--(axis cs:0.0108,0.000609129762155081)
--(axis cs:0.011,0.00146264862143454)
--(axis cs:0.0112,0.00337103179365444)
--(axis cs:0.0114,0.00746800143814087)
--(axis cs:0.0116,0.0159242143929695)
--(axis cs:0.0118,0.0327255277864527)
--(axis cs:0.012,0.0648970380843938)
--(axis cs:0.0122,0.12433136640818)
--(axis cs:0.0124,0.230376091159419)
--(axis cs:0.0126,0.367106247261684)
--(axis cs:0.0128,0.216826052554323)
--(axis cs:0.013,0.126306236283288)
--(axis cs:0.0132,0.0725957783827948)
--(axis cs:0.0134,0.0411853416623558)
--(axis cs:0.0136,0.0230718587213021)
--(axis cs:0.0138,0.012766968508524)
--(axis cs:0.014,0.00698084839717244)
--(axis cs:0.0142,0.00377300501691588)
--(axis cs:0.0144,0.0020163356605606)
--(axis cs:0.0146,0.00106577707534001)
--(axis cs:0.0148,0.000557345440476419)
--(axis cs:0.015,0.00028844194244243)
--(axis cs:0.0152,0.000147769489564178)
--(axis cs:0.0154,7.49575680292609e-05)
--(axis cs:0.0156,3.7658079611177e-05)
--(axis cs:0.0158,1.87420165753465e-05)
--(axis cs:0.016,9.24250139028707e-06)
--(axis cs:0.0162,4.5172452326614e-06)
--(axis cs:0.0164,2.18857329389199e-06)
--(axis cs:0.0166,1.05133420160334e-06)
--(axis cs:0.0168,5.00839471261352e-07)
--(axis cs:0.017,2.36655772209254e-07)
--(axis cs:0.0172,1.10936966886697e-07)
--(axis cs:0.0174,5.16004121551136e-08)
--(axis cs:0.0176,2.38189257271155e-08)
--(axis cs:0.0178,1.09132770888478e-08)
--(axis cs:0.018,4.96388475445766e-09)
--(axis cs:0.0182,2.2417587444063e-09)
--(axis cs:0.0184,1.00536023534575e-09)
--(axis cs:0.0186,4.47798927374872e-10)
--(axis cs:0.0188,1.98122479510485e-10)
--(axis cs:0.019,8.7082869969346e-11)
--(axis cs:0.0192,3.80309388371811e-11)
--(axis cs:0.0194,1.65044850569294e-11)
--(axis cs:0.0196,7.11837908731441e-12)
--(axis cs:0.0198,3.05159149971245e-12)
--(axis cs:0.02,1.30043368831224e-12)
--(axis cs:0.0202,5.50954117508255e-13)
--(axis cs:0.0204,2.32089498543993e-13)
--(axis cs:0.0206,9.72197326187167e-14)
--(axis cs:0.0208,4.0500178081169e-14)
--(axis cs:0.021,1.67805596237452e-14)
--(axis cs:0.0212,6.91584302087708e-15)
--(axis cs:0.0214,2.83539771694205e-15)
--(axis cs:0.0216,1.15651924143567e-15)
--(axis cs:0.0218,4.69353977721349e-16)
--(axis cs:0.022,1.89537255017966e-16)
--(axis cs:0.0222,7.61678629358275e-17)
--(axis cs:0.0224,3.04626530052703e-17)
--(axis cs:0.0226,1.21259887645579e-17)
--(axis cs:0.0228,4.80456288052996e-18)
--(axis cs:0.023,1.8950068938661e-18)
--(axis cs:0.0232,7.44080551455369e-19)
--(axis cs:0.0234,2.90879145590291e-19)
--(axis cs:0.0236,1.1321894943238e-19)
--(axis cs:0.0238,4.38802622641073e-20)
--(axis cs:0.024,1.69352564892312e-20)
--(axis cs:0.0242,6.50901095377492e-21)
--(axis cs:0.0244,2.49153147957588e-21)
--(axis cs:0.0246,9.49888708849199e-22)
--(axis cs:0.0248,3.60711681893591e-22)
--(axis cs:0.025,1.3644397353969e-22)
--(axis cs:0.0252,5.14138469226384e-23)
--(axis cs:0.0254,1.93001978872092e-23)
--(axis cs:0.0256,7.21810340585825e-24)
--(axis cs:0.0258,2.68959762500427e-24)
--(axis cs:0.026,9.98566661792328e-25)
--(axis cs:0.0262,3.6941496807468e-25)
--(axis cs:0.0264,1.36182474916111e-25)
--(axis cs:0.0266,5.00286065871767e-26)
--(axis cs:0.0268,1.83158343383945e-26)
--(axis cs:0.027,6.68292097066354e-27)
--(axis cs:0.0272,2.43028394239313e-27)
--(axis cs:0.0274,8.80882743445791e-28)
--(axis cs:0.0276,3.18250047040028e-28)
--(axis cs:0.0278,1.14611085680212e-28)
--(axis cs:0.028,4.11443911103716e-29)
--(axis cs:0.0282,1.47244206536675e-29)
--(axis cs:0.0284,5.25323318970638e-30)
--(axis cs:0.0286,1.86849950728891e-30)
--(axis cs:0.0288,6.62603526420191e-31)
--(axis cs:0.029,2.34274598112145e-31)
--(axis cs:0.0292,8.25892046285214e-32)
--(axis cs:0.0294,2.90311060532944e-32)
--(axis cs:0.0296,1.01756315239963e-32)
--(axis cs:0.0298,3.5565721067613e-33)
--(axis cs:0.03,1.23962116880601e-33)
--(axis cs:0.0302,4.30871570850815e-34)
--(axis cs:0.0304,1.49355861358081e-34)
--(axis cs:0.0306,5.16328482435287e-35)
--(axis cs:0.0308,1.78021645790086e-35)
--(axis cs:0.031,6.12175196186357e-36)
--(axis cs:0.0312,2.09965456035204e-36)
--(axis cs:0.0314,7.18293501274318e-37)
--(axis cs:0.0316,2.45104063814212e-37)
--(axis cs:0.0318,8.34268518420225e-38)
--(axis cs:0.032,2.83256587034247e-38)
--(axis cs:0.0322,9.59367104341985e-39)
--(axis cs:0.0324,3.24139436756253e-39)
--(axis cs:0.0326,1.09252795039009e-39)
--(axis cs:0.0328,3.67365231123331e-40)
--(axis cs:0.033,1.23236492322482e-40)
--(axis cs:0.0332,4.12446063746452e-41)
--(axis cs:0.0334,1.37718452035046e-41)
--(axis cs:0.0336,4.58801283402551e-42)
--(axis cs:0.0338,1.52501754192857e-42)
--(axis cs:0.034,5.05769688891288e-43)
--(axis cs:0.0342,1.67366426563217e-43)
--(axis cs:0.0344,5.52625785272222e-44)
--(axis cs:0.0346,1.82075161131955e-44)
--(axis cs:0.0348,5.98599503975823e-45)
--(axis cs:0.035,1.9638000548463e-45)
--(axis cs:0.0352,6.42898577109859e-46)
--(axis cs:0.0354,2.1002974480877e-46)
--(axis cs:0.0356,6.84732590989572e-47)
--(axis cs:0.0358,2.22777646835226e-47)
--(axis cs:0.036,7.23337637178125e-48)
--(axis cs:0.0362,2.34389218482664e-48)
--(axis cs:0.0364,7.58000438263052e-49)
--(axis cs:0.0366,2.44649620718702e-49)
--(axis cs:0.0368,7.88080852298557e-50)
--(axis cs:0.037,2.53370407054759e-50)
--(axis cs:0.0372,8.13031751342969e-51)
--(axis cs:0.0374,2.60395289449275e-51)
--(axis cs:0.0376,8.32415418185093e-52)
--(axis cs:0.0378,2.6560469014955e-52)
--(axis cs:0.038,8.45915799726862e-53)
--(axis cs:0.0382,2.68918905048345e-53)
--(axis cs:0.0384,8.53346181345964e-54)
--(axis cs:0.0386,2.70299778679314e-54)
--(axis cs:0.0388,8.54652088167659e-55)
--(axis cs:0.039,2.69750867965481e-55)
--(axis cs:0.0392,8.49909460584531e-56)
--(axis cs:0.0394,2.67316146339825e-56)
--cycle;
\path [draw=none, fill=blue, fill opacity=0.5]
(axis cs:0,0)
--(axis cs:0.0002,1.24681940844559e-185)
--(axis cs:0.0004,3.78429989649157e-149)
--(axis cs:0.0006,4.69624390399748e-128)
--(axis cs:0.0008,2.91814488510357e-113)
--(axis cs:0.001,6.39937004086229e-102)
--(axis cs:0.0012,9.19292845684038e-93)
--(axis cs:0.0014,4.10935002282113e-85)
--(axis cs:0.0016,1.44888628628164e-78)
--(axis cs:0.0018,7.34152876082861e-73)
--(axis cs:0.002,8.0525019711728e-68)
--(axis cs:0.0022,2.56068941187556e-63)
--(axis cs:0.0024,2.9292421581097e-59)
--(axis cs:0.0026,1.41995830907816e-55)
--(axis cs:0.0028,3.31301642990214e-52)
--(axis cs:0.003,4.11568387030809e-49)
--(axis cs:0.0032,2.95307692152295e-46)
--(axis cs:0.0034,1.30806963691571e-43)
--(axis cs:0.0036,3.77970210701459e-41)
--(axis cs:0.0038,7.46130251996167e-39)
--(axis cs:0.004,1.04633990725249e-36)
--(axis cs:0.0042,1.07774752268699e-34)
--(axis cs:0.0044,8.3909217479038e-33)
--(axis cs:0.0046,5.06232673594148e-31)
--(axis cs:0.0048,2.41856883933146e-29)
--(axis cs:0.005,9.32610456511297e-28)
--(axis cs:0.0052,2.95167434916092e-26)
--(axis cs:0.0054,7.78268945730413e-25)
--(axis cs:0.0056,1.73238744737497e-23)
--(axis cs:0.0058,3.29434213604689e-22)
--(axis cs:0.006,5.40916986257712e-21)
--(axis cs:0.0062,7.74292596344275e-20)
--(axis cs:0.0064,9.74695105038983e-19)
--(axis cs:0.0066,1.08755078580298e-17)
--(axis cs:0.0068,1.08334855272526e-16)
--(axis cs:0.007,9.69782013606882e-16)
--(axis cs:0.0072,7.84830360507471e-15)
--(axis cs:0.0074,5.7738733862825e-14)
--(axis cs:0.0076,3.88106550443288e-13)
--(axis cs:0.0078,2.39473656171717e-12)
--(axis cs:0.008,1.36227283385691e-11)
--(axis cs:0.0082,7.17311040949162e-11)
--(axis cs:0.0084,3.50913326863578e-10)
--(axis cs:0.0086,1.6004392380367e-09)
--(axis cs:0.0088,6.82684778432269e-09)
--(axis cs:0.009,2.73176680030113e-08)
--(axis cs:0.0092,1.02831056135345e-07)
--(axis cs:0.0094,3.65088994324908e-07)
--(axis cs:0.0096,1.22555178042531e-06)
--(axis cs:0.0098,3.89871873397428e-06)
--(axis cs:0.01,1.17789817815586e-05)
--(axis cs:0.0102,3.3866611040304e-05)
--(axis cs:0.0104,9.28421083067902e-05)
--(axis cs:0.0106,0.000243114650517988)
--(axis cs:0.0108,0.000609129762155081)
--(axis cs:0.011,0.00146264862143454)
--(axis cs:0.0112,0.00337103179365444)
--(axis cs:0.0114,0.00746800143814087)
--(axis cs:0.0116,0.0159242143929695)
--(axis cs:0.0118,0.0327255277864527)
--(axis cs:0.012,0.0648970380843938)
--(axis cs:0.0122,0.12433136640818)
--(axis cs:0.0124,0.230376091159419)
--(axis cs:0.0126,0.413290121770664)
--(axis cs:0.0128,0.71857463105485)
--(axis cs:0.013,1.21200914038453)
--(axis cs:0.0132,1.98497868160372)
--(axis cs:0.0134,3.15938477142859)
--(axis cs:0.0136,4.89113638839976)
--(axis cs:0.0138,7.37098493642559)
--(axis cs:0.014,10.8213805301847)
--(axis cs:0.0142,11.2785303309885)
--(axis cs:0.0144,7.89134633657271)
--(axis cs:0.0146,5.44110428079139)
--(axis cs:0.0148,3.69853799204927)
--(axis cs:0.015,2.4793800811449)
--(axis cs:0.0152,1.63976851873599)
--(axis cs:0.0154,1.07028313230223)
--(axis cs:0.0156,0.689661504730145)
--(axis cs:0.0158,0.438867480799633)
--(axis cs:0.016,0.275882802307981)
--(axis cs:0.0162,0.171371414177485)
--(axis cs:0.0164,0.105220080883349)
--(axis cs:0.0166,0.0638741074673723)
--(axis cs:0.0168,0.0383470703560727)
--(axis cs:0.017,0.0227736030712954)
--(axis cs:0.0172,0.0133822979151883)
--(axis cs:0.0174,0.00778273944702365)
--(axis cs:0.0176,0.00448060903275038)
--(axis cs:0.0178,0.00255411761700844)
--(axis cs:0.018,0.00144190717862481)
--(axis cs:0.0182,0.000806336789362508)
--(axis cs:0.0184,0.000446751107565886)
--(axis cs:0.0186,0.000245283985078712)
--(axis cs:0.0188,0.00013347773643944)
--(axis cs:0.019,7.20051655177482e-05)
--(axis cs:0.0192,3.85132583475396e-05)
--(axis cs:0.0194,2.04278607692691e-05)
--(axis cs:0.0196,1.07466590998682e-05)
--(axis cs:0.0198,5.60830677274215e-06)
--(axis cs:0.02,2.90379129483597e-06)
--(axis cs:0.0202,1.49190101011772e-06)
--(axis cs:0.0204,7.60710253355381e-07)
--(axis cs:0.0206,3.85003786698849e-07)
--(axis cs:0.0208,1.93435964732349e-07)
--(axis cs:0.021,9.64926505877817e-08)
--(axis cs:0.0212,4.77961238059279e-08)
--(axis cs:0.0214,2.35118925731706e-08)
--(axis cs:0.0216,1.14876838412627e-08)
--(axis cs:0.0218,5.57544271889242e-09)
--(axis cs:0.022,2.68830683625505e-09)
--(axis cs:0.0222,1.28789403121594e-09)
--(axis cs:0.0224,6.13099745745081e-10)
--(axis cs:0.0226,2.90053740312067e-10)
--(axis cs:0.0228,1.36385285237948e-10)
--(axis cs:0.023,6.37444630840224e-11)
--(axis cs:0.0232,2.96173615640384e-11)
--(axis cs:0.0234,1.36811041615753e-11)
--(axis cs:0.0236,6.2835876296212e-12)
--(axis cs:0.0238,2.86976128942386e-12)
--(axis cs:0.024,1.3033896418131e-12)
--(axis cs:0.0242,5.88750121415524e-13)
--(axis cs:0.0244,2.64516597164137e-13)
--(axis cs:0.0246,1.18215938123978e-13)
--(axis cs:0.0248,5.255759287684e-14)
--(axis cs:0.025,2.32469252232671e-14)
--(axis cs:0.0252,1.02305659610502e-14)
--(axis cs:0.0254,4.47992316053469e-15)
--(axis cs:0.0256,1.95213630999053e-15)
--(axis cs:0.0258,8.46544011725541e-16)
--(axis cs:0.026,3.65358493949292e-16)
--(axis cs:0.0262,1.56945502083572e-16)
--(axis cs:0.0264,6.71069643194406e-17)
--(axis cs:0.0266,2.85629962894664e-17)
--(axis cs:0.0268,1.21027800607901e-17)
--(axis cs:0.027,5.10550684580449e-18)
--(axis cs:0.0272,2.14432921898816e-18)
--(axis cs:0.0274,8.96744853650233e-19)
--(axis cs:0.0276,3.73419105674198e-19)
--(axis cs:0.0278,1.54845725029218e-19)
--(axis cs:0.028,6.39442128938801e-20)
--(axis cs:0.0282,2.6298222127352e-20)
--(axis cs:0.0284,1.07720391140932e-20)
--(axis cs:0.0286,4.39479501814542e-21)
--(axis cs:0.0288,1.78595617606647e-21)
--(axis cs:0.029,7.2296335358632e-22)
--(axis cs:0.0292,2.91538871761007e-22)
--(axis cs:0.0294,1.17120333717999e-22)
--(axis cs:0.0296,4.6875326137853e-23)
--(axis cs:0.0298,1.86918625143203e-23)
--(axis cs:0.03,7.42637420595622e-24)
--(axis cs:0.0302,2.93992575451275e-24)
--(axis cs:0.0304,1.15971187226631e-24)
--(axis cs:0.0306,4.5586536851254e-25)
--(axis cs:0.0308,1.78572279395691e-25)
--(axis cs:0.031,6.97108311119898e-26)
--(axis cs:0.0312,2.71214399231177e-26)
--(axis cs:0.0314,1.05164374156872e-26)
--(axis cs:0.0316,4.06429084905467e-27)
--(axis cs:0.0318,1.56558863728127e-27)
--(axis cs:0.032,6.0112318400383e-28)
--(axis cs:0.0322,2.30069008420888e-28)
--(axis cs:0.0324,8.77762820435934e-29)
--(axis cs:0.0326,3.33838149293861e-29)
--(axis cs:0.0328,1.26575453175437e-29)
--(axis cs:0.033,4.78445538515688e-30)
--(axis cs:0.0332,1.80301565501958e-30)
--(axis cs:0.0334,6.77430386453264e-31)
--(axis cs:0.0336,2.53771003756134e-31)
--(axis cs:0.0338,9.47862151705772e-32)
--(axis cs:0.034,3.5301055865143e-32)
--(axis cs:0.0342,1.31093985672291e-32)
--(axis cs:0.0344,4.85448760330276e-33)
--(axis cs:0.0346,1.79259570964789e-33)
--(axis cs:0.0348,6.60103727962226e-34)
--(axis cs:0.035,2.42407039237887e-34)
--(axis cs:0.0352,8.87755963276196e-35)
--(axis cs:0.0354,3.24241957372043e-35)
--(axis cs:0.0356,1.18109214487747e-35)
--(axis cs:0.0358,4.29090419393295e-36)
--(axis cs:0.036,1.55480397682429e-36)
--(axis cs:0.0362,5.61921160918253e-37)
--(axis cs:0.0364,2.02562476103656e-37)
--(axis cs:0.0366,7.28345159341886e-38)
--(axis cs:0.0368,2.61228585498635e-38)
--(axis cs:0.037,9.34587126154641e-39)
--(axis cs:0.0372,3.33537578121633e-39)
--(axis cs:0.0374,1.18742368919572e-39)
--(axis cs:0.0376,4.2170866661322e-40)
--(axis cs:0.0378,1.49408388950585e-40)
--(axis cs:0.038,5.28083602364121e-41)
--(axis cs:0.0382,1.86210890398195e-41)
--(axis cs:0.0384,6.55075669093197e-42)
--(axis cs:0.0386,2.29916964220208e-42)
--(axis cs:0.0388,8.05105388170408e-43)
--(axis cs:0.039,2.81284300656489e-43)
--(axis cs:0.0392,9.80523563488412e-44)
--(axis cs:0.0394,3.41035006246725e-44)
--cycle;
\path [draw=none, fill=blue, fill opacity=0.5]
(axis cs:0,0)
--(axis cs:0.0002,7.07051949791944e-220)
--(axis cs:0.0004,3.35713224621954e-178)
--(axis cs:0.0006,4.55876472150082e-154)
--(axis cs:0.0008,4.06372766744917e-137)
--(axis cs:0.001,4.19927757126269e-124)
--(axis cs:0.0012,1.40567805458719e-113)
--(axis cs:0.0014,9.00643402552461e-105)
--(axis cs:0.0016,3.18925325198261e-97)
--(axis cs:0.0018,1.23690071259004e-90)
--(axis cs:0.002,8.38122637353824e-85)
--(axis cs:0.0022,1.38442183846587e-79)
--(axis cs:0.0024,7.12902189670533e-75)
--(axis cs:0.0026,1.3794522412844e-70)
--(axis cs:0.0028,1.15971222412979e-66)
--(axis cs:0.003,4.75273753659039e-63)
--(axis cs:0.0032,1.04179315001234e-59)
--(axis cs:0.0034,1.31771394306738e-56)
--(axis cs:0.0036,1.02415094765992e-53)
--(axis cs:0.0038,5.15560039198329e-51)
--(axis cs:0.004,1.75757764872752e-48)
--(axis cs:0.0042,4.2148638496568e-46)
--(axis cs:0.0044,7.3467295284001e-44)
--(axis cs:0.0046,9.57542435968325e-42)
--(axis cs:0.0048,9.56565522459126e-40)
--(axis cs:0.005,7.48492710605918e-38)
--(axis cs:0.0052,4.67617522469868e-36)
--(axis cs:0.0054,2.37244858785325e-34)
--(axis cs:0.0056,9.92369722206861e-33)
--(axis cs:0.0058,3.46894198997681e-31)
--(axis cs:0.006,1.02575999424703e-29)
--(axis cs:0.0062,2.59405173111155e-28)
--(axis cs:0.0064,5.66632176036028e-27)
--(axis cs:0.0066,1.07875184310474e-25)
--(axis cs:0.0068,1.80466875933379e-24)
--(axis cs:0.007,2.67286801654633e-23)
--(axis cs:0.0072,3.52886813743904e-22)
--(axis cs:0.0074,4.1792715817461e-21)
--(axis cs:0.0076,4.46564241233065e-20)
--(axis cs:0.0078,4.32813414557636e-19)
--(axis cs:0.008,3.82375668734256e-18)
--(axis cs:0.0082,3.09338087199957e-17)
--(axis cs:0.0084,2.3012710139591e-16)
--(axis cs:0.0086,1.58052451958739e-15)
--(axis cs:0.0088,1.00582834740469e-14)
--(axis cs:0.009,5.95139142432837e-14)
--(axis cs:0.0092,3.28450725704436e-13)
--(axis cs:0.0094,1.69580021234189e-12)
--(axis cs:0.0096,8.21383031334312e-12)
--(axis cs:0.0098,3.74215771035445e-11)
--(axis cs:0.01,1.60758325803006e-10)
--(axis cs:0.0102,6.52688622108691e-10)
--(axis cs:0.0104,2.50996299416417e-09)
--(axis cs:0.0106,9.16116446427591e-09)
--(axis cs:0.0108,3.17979701351146e-08)
--(axis cs:0.011,1.05150251268355e-07)
--(axis cs:0.0112,3.31846466176376e-07)
--(axis cs:0.0114,1.00113761423932e-06)
--(axis cs:0.0116,2.89172206747296e-06)
--(axis cs:0.0118,8.00879727229867e-06)
--(axis cs:0.012,2.12978350679836e-05)
--(axis cs:0.0122,5.44553092320697e-05)
--(axis cs:0.0124,0.000134039423465113)
--(axis cs:0.0126,0.000318007176388045)
--(axis cs:0.0128,0.000728036322089197)
--(axis cs:0.013,0.0016101168440237)
--(axis cs:0.0132,0.00344353575065258)
--(axis cs:0.0134,0.00712900268421548)
--(axis cs:0.0136,0.0143002873493135)
--(axis cs:0.0138,0.012766968508524)
--(axis cs:0.014,0.00698084839717244)
--(axis cs:0.0142,0.00377300501691588)
--(axis cs:0.0144,0.0020163356605606)
--(axis cs:0.0146,0.00106577707534001)
--(axis cs:0.0148,0.000557345440476419)
--(axis cs:0.015,0.00028844194244243)
--(axis cs:0.0152,0.000147769489564178)
--(axis cs:0.0154,7.49575680292609e-05)
--(axis cs:0.0156,3.7658079611177e-05)
--(axis cs:0.0158,1.87420165753465e-05)
--(axis cs:0.016,9.24250139028707e-06)
--(axis cs:0.0162,4.5172452326614e-06)
--(axis cs:0.0164,2.18857329389199e-06)
--(axis cs:0.0166,1.05133420160334e-06)
--(axis cs:0.0168,5.00839471261352e-07)
--(axis cs:0.017,2.36655772209254e-07)
--(axis cs:0.0172,1.10936966886697e-07)
--(axis cs:0.0174,5.16004121551136e-08)
--(axis cs:0.0176,2.38189257271155e-08)
--(axis cs:0.0178,1.09132770888478e-08)
--(axis cs:0.018,4.96388475445766e-09)
--(axis cs:0.0182,2.2417587444063e-09)
--(axis cs:0.0184,1.00536023534575e-09)
--(axis cs:0.0186,4.47798927374872e-10)
--(axis cs:0.0188,1.98122479510485e-10)
--(axis cs:0.019,8.7082869969346e-11)
--(axis cs:0.0192,3.80309388371811e-11)
--(axis cs:0.0194,1.65044850569294e-11)
--(axis cs:0.0196,7.11837908731441e-12)
--(axis cs:0.0198,3.05159149971245e-12)
--(axis cs:0.02,1.30043368831224e-12)
--(axis cs:0.0202,5.50954117508255e-13)
--(axis cs:0.0204,2.32089498543993e-13)
--(axis cs:0.0206,9.72197326187167e-14)
--(axis cs:0.0208,4.0500178081169e-14)
--(axis cs:0.021,1.67805596237452e-14)
--(axis cs:0.0212,6.91584302087708e-15)
--(axis cs:0.0214,2.83539771694205e-15)
--(axis cs:0.0216,1.15651924143567e-15)
--(axis cs:0.0218,4.69353977721349e-16)
--(axis cs:0.022,1.89537255017966e-16)
--(axis cs:0.0222,7.61678629358275e-17)
--(axis cs:0.0224,3.04626530052703e-17)
--(axis cs:0.0226,1.21259887645579e-17)
--(axis cs:0.0228,4.80456288052996e-18)
--(axis cs:0.023,1.8950068938661e-18)
--(axis cs:0.0232,7.44080551455369e-19)
--(axis cs:0.0234,2.90879145590291e-19)
--(axis cs:0.0236,1.1321894943238e-19)
--(axis cs:0.0238,4.38802622641073e-20)
--(axis cs:0.024,1.69352564892312e-20)
--(axis cs:0.0242,6.50901095377492e-21)
--(axis cs:0.0244,2.49153147957588e-21)
--(axis cs:0.0246,9.49888708849199e-22)
--(axis cs:0.0248,3.60711681893591e-22)
--(axis cs:0.025,1.3644397353969e-22)
--(axis cs:0.0252,5.14138469226384e-23)
--(axis cs:0.0254,1.93001978872092e-23)
--(axis cs:0.0256,7.21810340585825e-24)
--(axis cs:0.0258,2.68959762500427e-24)
--(axis cs:0.026,9.98566661792328e-25)
--(axis cs:0.0262,3.6941496807468e-25)
--(axis cs:0.0264,1.36182474916111e-25)
--(axis cs:0.0266,5.00286065871767e-26)
--(axis cs:0.0268,1.83158343383945e-26)
--(axis cs:0.027,6.68292097066354e-27)
--(axis cs:0.0272,2.43028394239313e-27)
--(axis cs:0.0274,8.80882743445791e-28)
--(axis cs:0.0276,3.18250047040028e-28)
--(axis cs:0.0278,1.14611085680212e-28)
--(axis cs:0.028,4.11443911103716e-29)
--(axis cs:0.0282,1.47244206536675e-29)
--(axis cs:0.0284,5.25323318970638e-30)
--(axis cs:0.0286,1.86849950728891e-30)
--(axis cs:0.0288,6.62603526420191e-31)
--(axis cs:0.029,2.34274598112145e-31)
--(axis cs:0.0292,8.25892046285214e-32)
--(axis cs:0.0294,2.90311060532944e-32)
--(axis cs:0.0296,1.01756315239963e-32)
--(axis cs:0.0298,3.5565721067613e-33)
--(axis cs:0.03,1.23962116880601e-33)
--(axis cs:0.0302,4.30871570850815e-34)
--(axis cs:0.0304,1.49355861358081e-34)
--(axis cs:0.0306,5.16328482435287e-35)
--(axis cs:0.0308,1.78021645790086e-35)
--(axis cs:0.031,6.12175196186357e-36)
--(axis cs:0.0312,2.09965456035204e-36)
--(axis cs:0.0314,7.18293501274318e-37)
--(axis cs:0.0316,2.45104063814212e-37)
--(axis cs:0.0318,8.34268518420225e-38)
--(axis cs:0.032,2.83256587034247e-38)
--(axis cs:0.0322,9.59367104341985e-39)
--(axis cs:0.0324,3.24139436756253e-39)
--(axis cs:0.0326,1.09252795039009e-39)
--(axis cs:0.0328,3.67365231123331e-40)
--(axis cs:0.033,1.23236492322482e-40)
--(axis cs:0.0332,4.12446063746452e-41)
--(axis cs:0.0334,1.37718452035046e-41)
--(axis cs:0.0336,4.58801283402551e-42)
--(axis cs:0.0338,1.52501754192857e-42)
--(axis cs:0.034,5.05769688891288e-43)
--(axis cs:0.0342,1.67366426563217e-43)
--(axis cs:0.0344,5.52625785272222e-44)
--(axis cs:0.0346,1.82075161131955e-44)
--(axis cs:0.0348,5.98599503975823e-45)
--(axis cs:0.035,1.9638000548463e-45)
--(axis cs:0.0352,6.42898577109859e-46)
--(axis cs:0.0354,2.1002974480877e-46)
--(axis cs:0.0356,6.84732590989572e-47)
--(axis cs:0.0358,2.22777646835226e-47)
--(axis cs:0.036,7.23337637178125e-48)
--(axis cs:0.0362,2.34389218482664e-48)
--(axis cs:0.0364,7.58000438263052e-49)
--(axis cs:0.0366,2.44649620718702e-49)
--(axis cs:0.0368,7.88080852298557e-50)
--(axis cs:0.037,2.53370407054759e-50)
--(axis cs:0.0372,8.13031751342969e-51)
--(axis cs:0.0374,2.60395289449275e-51)
--(axis cs:0.0376,8.32415418185093e-52)
--(axis cs:0.0378,2.6560469014955e-52)
--(axis cs:0.038,8.45915799726862e-53)
--(axis cs:0.0382,2.68918905048345e-53)
--(axis cs:0.0384,8.53346181345964e-54)
--(axis cs:0.0386,2.70299778679314e-54)
--(axis cs:0.0388,8.54652088167659e-55)
--(axis cs:0.039,2.69750867965481e-55)
--(axis cs:0.0392,8.49909460584531e-56)
--(axis cs:0.0394,2.67316146339825e-56)
--cycle;
\path [draw=none, fill=blue, fill opacity=0.5]
(axis cs:0,0)
--(axis cs:0.0002,7.07051949791944e-220)
--(axis cs:0.0004,3.35713224621954e-178)
--(axis cs:0.0006,4.55876472150082e-154)
--(axis cs:0.0008,4.06372766744917e-137)
--(axis cs:0.001,4.19927757126269e-124)
--(axis cs:0.0012,1.40567805458719e-113)
--(axis cs:0.0014,9.00643402552461e-105)
--(axis cs:0.0016,3.18925325198261e-97)
--(axis cs:0.0018,1.23690071259004e-90)
--(axis cs:0.002,8.38122637353824e-85)
--(axis cs:0.0022,1.38442183846587e-79)
--(axis cs:0.0024,7.12902189670533e-75)
--(axis cs:0.0026,1.3794522412844e-70)
--(axis cs:0.0028,1.15971222412979e-66)
--(axis cs:0.003,4.75273753659039e-63)
--(axis cs:0.0032,1.04179315001234e-59)
--(axis cs:0.0034,1.31771394306738e-56)
--(axis cs:0.0036,1.02415094765992e-53)
--(axis cs:0.0038,5.15560039198329e-51)
--(axis cs:0.004,1.75757764872752e-48)
--(axis cs:0.0042,4.2148638496568e-46)
--(axis cs:0.0044,7.3467295284001e-44)
--(axis cs:0.0046,9.57542435968325e-42)
--(axis cs:0.0048,9.56565522459126e-40)
--(axis cs:0.005,7.48492710605918e-38)
--(axis cs:0.0052,4.67617522469868e-36)
--(axis cs:0.0054,2.37244858785325e-34)
--(axis cs:0.0056,9.92369722206861e-33)
--(axis cs:0.0058,3.46894198997681e-31)
--(axis cs:0.006,1.02575999424703e-29)
--(axis cs:0.0062,2.59405173111155e-28)
--(axis cs:0.0064,5.66632176036028e-27)
--(axis cs:0.0066,1.07875184310474e-25)
--(axis cs:0.0068,1.80466875933379e-24)
--(axis cs:0.007,2.67286801654633e-23)
--(axis cs:0.0072,3.52886813743904e-22)
--(axis cs:0.0074,4.1792715817461e-21)
--(axis cs:0.0076,4.46564241233065e-20)
--(axis cs:0.0078,4.32813414557636e-19)
--(axis cs:0.008,3.82375668734256e-18)
--(axis cs:0.0082,3.09338087199957e-17)
--(axis cs:0.0084,2.3012710139591e-16)
--(axis cs:0.0086,1.58052451958739e-15)
--(axis cs:0.0088,1.00582834740469e-14)
--(axis cs:0.009,5.95139142432837e-14)
--(axis cs:0.0092,3.28450725704436e-13)
--(axis cs:0.0094,1.69580021234189e-12)
--(axis cs:0.0096,8.21383031334312e-12)
--(axis cs:0.0098,3.74215771035445e-11)
--(axis cs:0.01,1.60758325803006e-10)
--(axis cs:0.0102,6.52688622108691e-10)
--(axis cs:0.0104,2.50996299416417e-09)
--(axis cs:0.0106,9.16116446427591e-09)
--(axis cs:0.0108,3.17979701351146e-08)
--(axis cs:0.011,1.05150251268355e-07)
--(axis cs:0.0112,3.31846466176376e-07)
--(axis cs:0.0114,1.00113761423932e-06)
--(axis cs:0.0116,2.89172206747296e-06)
--(axis cs:0.0118,8.00879727229867e-06)
--(axis cs:0.012,2.12978350679836e-05)
--(axis cs:0.0122,5.44553092320697e-05)
--(axis cs:0.0124,0.000134039423465113)
--(axis cs:0.0126,0.000318007176388045)
--(axis cs:0.0128,0.000728036322089197)
--(axis cs:0.013,0.0016101168440237)
--(axis cs:0.0132,0.00344353575065258)
--(axis cs:0.0134,0.00712900268421548)
--(axis cs:0.0136,0.0143002873493135)
--(axis cs:0.0138,0.0278194928102673)
--(axis cs:0.014,0.0525317220624886)
--(axis cs:0.0142,0.0963663542846326)
--(axis cs:0.0144,0.171873415509561)
--(axis cs:0.0146,0.298266989051259)
--(axis cs:0.0148,0.504005145686858)
--(axis cs:0.015,0.829860797883638)
--(axis cs:0.0152,1.33232757972859)
--(axis cs:0.0154,1.07028313230223)
--(axis cs:0.0156,0.689661504730145)
--(axis cs:0.0158,0.438867480799633)
--(axis cs:0.016,0.275882802307981)
--(axis cs:0.0162,0.171371414177485)
--(axis cs:0.0164,0.105220080883349)
--(axis cs:0.0166,0.0638741074673723)
--(axis cs:0.0168,0.0383470703560727)
--(axis cs:0.017,0.0227736030712954)
--(axis cs:0.0172,0.0133822979151883)
--(axis cs:0.0174,0.00778273944702365)
--(axis cs:0.0176,0.00448060903275038)
--(axis cs:0.0178,0.00255411761700844)
--(axis cs:0.018,0.00144190717862481)
--(axis cs:0.0182,0.000806336789362508)
--(axis cs:0.0184,0.000446751107565886)
--(axis cs:0.0186,0.000245283985078712)
--(axis cs:0.0188,0.00013347773643944)
--(axis cs:0.019,7.20051655177482e-05)
--(axis cs:0.0192,3.85132583475396e-05)
--(axis cs:0.0194,2.04278607692691e-05)
--(axis cs:0.0196,1.07466590998682e-05)
--(axis cs:0.0198,5.60830677274215e-06)
--(axis cs:0.02,2.90379129483597e-06)
--(axis cs:0.0202,1.49190101011772e-06)
--(axis cs:0.0204,7.60710253355381e-07)
--(axis cs:0.0206,3.85003786698849e-07)
--(axis cs:0.0208,1.93435964732349e-07)
--(axis cs:0.021,9.64926505877817e-08)
--(axis cs:0.0212,4.77961238059279e-08)
--(axis cs:0.0214,2.35118925731706e-08)
--(axis cs:0.0216,1.14876838412627e-08)
--(axis cs:0.0218,5.57544271889242e-09)
--(axis cs:0.022,2.68830683625505e-09)
--(axis cs:0.0222,1.28789403121594e-09)
--(axis cs:0.0224,6.13099745745081e-10)
--(axis cs:0.0226,2.90053740312067e-10)
--(axis cs:0.0228,1.36385285237948e-10)
--(axis cs:0.023,6.37444630840224e-11)
--(axis cs:0.0232,2.96173615640384e-11)
--(axis cs:0.0234,1.36811041615753e-11)
--(axis cs:0.0236,6.2835876296212e-12)
--(axis cs:0.0238,2.86976128942386e-12)
--(axis cs:0.024,1.3033896418131e-12)
--(axis cs:0.0242,5.88750121415524e-13)
--(axis cs:0.0244,2.64516597164137e-13)
--(axis cs:0.0246,1.18215938123978e-13)
--(axis cs:0.0248,5.255759287684e-14)
--(axis cs:0.025,2.32469252232671e-14)
--(axis cs:0.0252,1.02305659610502e-14)
--(axis cs:0.0254,4.47992316053469e-15)
--(axis cs:0.0256,1.95213630999053e-15)
--(axis cs:0.0258,8.46544011725541e-16)
--(axis cs:0.026,3.65358493949292e-16)
--(axis cs:0.0262,1.56945502083572e-16)
--(axis cs:0.0264,6.71069643194406e-17)
--(axis cs:0.0266,2.85629962894664e-17)
--(axis cs:0.0268,1.21027800607901e-17)
--(axis cs:0.027,5.10550684580449e-18)
--(axis cs:0.0272,2.14432921898816e-18)
--(axis cs:0.0274,8.96744853650233e-19)
--(axis cs:0.0276,3.73419105674198e-19)
--(axis cs:0.0278,1.54845725029218e-19)
--(axis cs:0.028,6.39442128938801e-20)
--(axis cs:0.0282,2.6298222127352e-20)
--(axis cs:0.0284,1.07720391140932e-20)
--(axis cs:0.0286,4.39479501814542e-21)
--(axis cs:0.0288,1.78595617606647e-21)
--(axis cs:0.029,7.2296335358632e-22)
--(axis cs:0.0292,2.91538871761007e-22)
--(axis cs:0.0294,1.17120333717999e-22)
--(axis cs:0.0296,4.6875326137853e-23)
--(axis cs:0.0298,1.86918625143203e-23)
--(axis cs:0.03,7.42637420595622e-24)
--(axis cs:0.0302,2.93992575451275e-24)
--(axis cs:0.0304,1.15971187226631e-24)
--(axis cs:0.0306,4.5586536851254e-25)
--(axis cs:0.0308,1.78572279395691e-25)
--(axis cs:0.031,6.97108311119898e-26)
--(axis cs:0.0312,2.71214399231177e-26)
--(axis cs:0.0314,1.05164374156872e-26)
--(axis cs:0.0316,4.06429084905467e-27)
--(axis cs:0.0318,1.56558863728127e-27)
--(axis cs:0.032,6.0112318400383e-28)
--(axis cs:0.0322,2.30069008420888e-28)
--(axis cs:0.0324,8.77762820435934e-29)
--(axis cs:0.0326,3.33838149293861e-29)
--(axis cs:0.0328,1.26575453175437e-29)
--(axis cs:0.033,4.78445538515688e-30)
--(axis cs:0.0332,1.80301565501958e-30)
--(axis cs:0.0334,6.77430386453264e-31)
--(axis cs:0.0336,2.53771003756134e-31)
--(axis cs:0.0338,9.47862151705772e-32)
--(axis cs:0.034,3.5301055865143e-32)
--(axis cs:0.0342,1.31093985672291e-32)
--(axis cs:0.0344,4.85448760330276e-33)
--(axis cs:0.0346,1.79259570964789e-33)
--(axis cs:0.0348,6.60103727962226e-34)
--(axis cs:0.035,2.42407039237887e-34)
--(axis cs:0.0352,8.87755963276196e-35)
--(axis cs:0.0354,3.24241957372043e-35)
--(axis cs:0.0356,1.18109214487747e-35)
--(axis cs:0.0358,4.29090419393295e-36)
--(axis cs:0.036,1.55480397682429e-36)
--(axis cs:0.0362,5.61921160918253e-37)
--(axis cs:0.0364,2.02562476103656e-37)
--(axis cs:0.0366,7.28345159341886e-38)
--(axis cs:0.0368,2.61228585498635e-38)
--(axis cs:0.037,9.34587126154641e-39)
--(axis cs:0.0372,3.33537578121633e-39)
--(axis cs:0.0374,1.18742368919572e-39)
--(axis cs:0.0376,4.2170866661322e-40)
--(axis cs:0.0378,1.49408388950585e-40)
--(axis cs:0.038,5.28083602364121e-41)
--(axis cs:0.0382,1.86210890398195e-41)
--(axis cs:0.0384,6.55075669093197e-42)
--(axis cs:0.0386,2.29916964220208e-42)
--(axis cs:0.0388,8.05105388170408e-43)
--(axis cs:0.039,2.81284300656489e-43)
--(axis cs:0.0392,9.80523563488412e-44)
--(axis cs:0.0394,3.41035006246725e-44)
--cycle;
\path [draw=none, fill=blue, fill opacity=0.5]
(axis cs:0,0)
--(axis cs:0.0002,2.55329303009598e-95)
--(axis cs:0.0004,1.68182412350504e-73)
--(axis cs:0.0006,5.49781289889967e-61)
--(axis cs:0.0008,2.78718791670709e-52)
--(axis cs:0.001,1.15539866519647e-45)
--(axis cs:0.0012,2.29045339557146e-40)
--(axis cs:0.0014,5.57050636268994e-36)
--(axis cs:0.0016,2.91664333325709e-32)
--(axis cs:0.0018,4.72521876543319e-29)
--(axis cs:0.002,3.03440752670528e-26)
--(axis cs:0.0022,9.2163699131261e-24)
--(axis cs:0.0024,1.50843826973494e-21)
--(axis cs:0.0026,1.46891166386691e-19)
--(axis cs:0.0028,9.19186255467381e-18)
--(axis cs:0.003,3.92879593928923e-16)
--(axis cs:0.0032,1.20485095111868e-14)
--(axis cs:0.0034,2.75996884779829e-13)
--(axis cs:0.0036,4.8825997596155e-12)
--(axis cs:0.0038,6.85966297166094e-11)
--(axis cs:0.004,7.83646193730588e-10)
--(axis cs:0.0042,7.42781149314522e-09)
--(axis cs:0.0044,5.94374689671225e-08)
--(axis cs:0.0046,4.07612920102254e-07)
--(axis cs:0.0048,2.42727662171799e-06)
--(axis cs:0.005,1.26961478962858e-05)
--(axis cs:0.0052,5.89271481250347e-05)
--(axis cs:0.0054,0.00024488352824238)
--(axis cs:0.0056,0.000918517912095324)
--(axis cs:0.0058,0.00313195835741374)
--(axis cs:0.006,0.00977111118024442)
--(axis cs:0.0062,0.0280540417034728)
--(axis cs:0.0064,0.0745167406233305)
--(axis cs:0.0066,0.183988958680021)
--(axis cs:0.0068,0.424127151084662)
--(axis cs:0.007,0.916409463495715)
--(axis cs:0.0072,1.86272770489421)
--(axis cs:0.0074,3.57374020060483)
--(axis cs:0.0076,6.49145456201812)
--(axis cs:0.0078,11.1952645063309)
--(axis cs:0.008,18.3795368595609)
--(axis cs:0.0082,28.7934129004416)
--(axis cs:0.0084,43.1404483402246)
--(axis cs:0.0086,61.9462231884466)
--(axis cs:0.0088,85.4135836108871)
--(axis cs:0.009,113.294124010367)
--(axis cs:0.0092,144.807380446516)
--(axis cs:0.0094,165.48788308252)
--(axis cs:0.0096,131.787026921522)
--(axis cs:0.0098,102.422833784496)
--(axis cs:0.01,77.7608857315053)
--(axis cs:0.0102,57.7249279146676)
--(axis cs:0.0104,41.9351047754794)
--(axis cs:0.0106,29.8371367797266)
--(axis cs:0.0108,20.8082131422306)
--(axis cs:0.011,14.2339510184086)
--(axis cs:0.0112,9.55711844836904)
--(axis cs:0.0114,6.30262201512299)
--(axis cs:0.0116,4.08484518709801)
--(axis cs:0.0118,2.60341577529304)
--(axis cs:0.012,1.63254744116085)
--(axis cs:0.0122,1.00779280227118)
--(axis cs:0.0124,0.612741798994738)
--(axis cs:0.0126,0.367106247261684)
--(axis cs:0.0128,0.216826052554323)
--(axis cs:0.013,0.126306236283288)
--(axis cs:0.0132,0.0725957783827948)
--(axis cs:0.0134,0.0411853416623558)
--(axis cs:0.0136,0.0230718587213021)
--(axis cs:0.0138,0.012766968508524)
--(axis cs:0.014,0.00698084839717244)
--(axis cs:0.0142,0.00377300501691588)
--(axis cs:0.0144,0.0020163356605606)
--(axis cs:0.0146,0.00106577707534001)
--(axis cs:0.0148,0.000557345440476419)
--(axis cs:0.015,0.00028844194244243)
--(axis cs:0.0152,0.000147769489564178)
--(axis cs:0.0154,7.49575680292609e-05)
--(axis cs:0.0156,3.7658079611177e-05)
--(axis cs:0.0158,1.87420165753465e-05)
--(axis cs:0.016,9.24250139028707e-06)
--(axis cs:0.0162,4.5172452326614e-06)
--(axis cs:0.0164,2.18857329389199e-06)
--(axis cs:0.0166,1.05133420160334e-06)
--(axis cs:0.0168,5.00839471261352e-07)
--(axis cs:0.017,2.36655772209254e-07)
--(axis cs:0.0172,1.10936966886697e-07)
--(axis cs:0.0174,5.16004121551136e-08)
--(axis cs:0.0176,2.38189257271155e-08)
--(axis cs:0.0178,1.09132770888478e-08)
--(axis cs:0.018,4.96388475445766e-09)
--(axis cs:0.0182,2.2417587444063e-09)
--(axis cs:0.0184,1.00536023534575e-09)
--(axis cs:0.0186,4.47798927374872e-10)
--(axis cs:0.0188,1.98122479510485e-10)
--(axis cs:0.019,8.7082869969346e-11)
--(axis cs:0.0192,3.80309388371811e-11)
--(axis cs:0.0194,1.65044850569294e-11)
--(axis cs:0.0196,7.11837908731441e-12)
--(axis cs:0.0198,3.05159149971245e-12)
--(axis cs:0.02,1.30043368831224e-12)
--(axis cs:0.0202,5.50954117508255e-13)
--(axis cs:0.0204,2.32089498543993e-13)
--(axis cs:0.0206,9.72197326187167e-14)
--(axis cs:0.0208,4.0500178081169e-14)
--(axis cs:0.021,1.67805596237452e-14)
--(axis cs:0.0212,6.91584302087708e-15)
--(axis cs:0.0214,2.83539771694205e-15)
--(axis cs:0.0216,1.15651924143567e-15)
--(axis cs:0.0218,4.69353977721349e-16)
--(axis cs:0.022,1.89537255017966e-16)
--(axis cs:0.0222,7.61678629358275e-17)
--(axis cs:0.0224,3.04626530052703e-17)
--(axis cs:0.0226,1.21259887645579e-17)
--(axis cs:0.0228,4.80456288052996e-18)
--(axis cs:0.023,1.8950068938661e-18)
--(axis cs:0.0232,7.44080551455369e-19)
--(axis cs:0.0234,2.90879145590291e-19)
--(axis cs:0.0236,1.1321894943238e-19)
--(axis cs:0.0238,4.38802622641073e-20)
--(axis cs:0.024,1.69352564892312e-20)
--(axis cs:0.0242,6.50901095377492e-21)
--(axis cs:0.0244,2.49153147957588e-21)
--(axis cs:0.0246,9.49888708849199e-22)
--(axis cs:0.0248,3.60711681893591e-22)
--(axis cs:0.025,1.3644397353969e-22)
--(axis cs:0.0252,5.14138469226384e-23)
--(axis cs:0.0254,1.93001978872092e-23)
--(axis cs:0.0256,7.21810340585825e-24)
--(axis cs:0.0258,2.68959762500427e-24)
--(axis cs:0.026,9.98566661792328e-25)
--(axis cs:0.0262,3.6941496807468e-25)
--(axis cs:0.0264,1.36182474916111e-25)
--(axis cs:0.0266,5.00286065871767e-26)
--(axis cs:0.0268,1.83158343383945e-26)
--(axis cs:0.027,6.68292097066354e-27)
--(axis cs:0.0272,2.43028394239313e-27)
--(axis cs:0.0274,8.80882743445791e-28)
--(axis cs:0.0276,3.18250047040028e-28)
--(axis cs:0.0278,1.14611085680212e-28)
--(axis cs:0.028,4.11443911103716e-29)
--(axis cs:0.0282,1.47244206536675e-29)
--(axis cs:0.0284,5.25323318970638e-30)
--(axis cs:0.0286,1.86849950728891e-30)
--(axis cs:0.0288,6.62603526420191e-31)
--(axis cs:0.029,2.34274598112145e-31)
--(axis cs:0.0292,8.25892046285214e-32)
--(axis cs:0.0294,2.90311060532944e-32)
--(axis cs:0.0296,1.01756315239963e-32)
--(axis cs:0.0298,3.5565721067613e-33)
--(axis cs:0.03,1.23962116880601e-33)
--(axis cs:0.0302,4.30871570850815e-34)
--(axis cs:0.0304,1.49355861358081e-34)
--(axis cs:0.0306,5.16328482435287e-35)
--(axis cs:0.0308,1.78021645790086e-35)
--(axis cs:0.031,6.12175196186357e-36)
--(axis cs:0.0312,2.09965456035204e-36)
--(axis cs:0.0314,7.18293501274318e-37)
--(axis cs:0.0316,2.45104063814212e-37)
--(axis cs:0.0318,8.34268518420225e-38)
--(axis cs:0.032,2.83256587034247e-38)
--(axis cs:0.0322,9.59367104341985e-39)
--(axis cs:0.0324,3.24139436756253e-39)
--(axis cs:0.0326,1.09252795039009e-39)
--(axis cs:0.0328,3.67365231123331e-40)
--(axis cs:0.033,1.23236492322482e-40)
--(axis cs:0.0332,4.12446063746452e-41)
--(axis cs:0.0334,1.37718452035046e-41)
--(axis cs:0.0336,4.58801283402551e-42)
--(axis cs:0.0338,1.52501754192857e-42)
--(axis cs:0.034,5.05769688891288e-43)
--(axis cs:0.0342,1.67366426563217e-43)
--(axis cs:0.0344,5.52625785272222e-44)
--(axis cs:0.0346,1.82075161131955e-44)
--(axis cs:0.0348,5.98599503975823e-45)
--(axis cs:0.035,1.9638000548463e-45)
--(axis cs:0.0352,6.42898577109859e-46)
--(axis cs:0.0354,2.1002974480877e-46)
--(axis cs:0.0356,6.84732590989572e-47)
--(axis cs:0.0358,2.22777646835226e-47)
--(axis cs:0.036,7.23337637178125e-48)
--(axis cs:0.0362,2.34389218482664e-48)
--(axis cs:0.0364,7.58000438263052e-49)
--(axis cs:0.0366,2.44649620718702e-49)
--(axis cs:0.0368,7.88080852298557e-50)
--(axis cs:0.037,2.53370407054759e-50)
--(axis cs:0.0372,8.13031751342969e-51)
--(axis cs:0.0374,2.60395289449275e-51)
--(axis cs:0.0376,8.32415418185093e-52)
--(axis cs:0.0378,2.6560469014955e-52)
--(axis cs:0.038,8.45915799726862e-53)
--(axis cs:0.0382,2.68918905048345e-53)
--(axis cs:0.0384,8.53346181345964e-54)
--(axis cs:0.0386,2.70299778679314e-54)
--(axis cs:0.0388,8.54652088167659e-55)
--(axis cs:0.039,2.69750867965481e-55)
--(axis cs:0.0392,8.49909460584531e-56)
--(axis cs:0.0394,2.67316146339825e-56)
--cycle;
\addplot [semithick, color0, opacity=0.8]
table {%
0 0
0.0002 3.95089986982401e-31
0.0004 1.7288904267305e-21
0.0006 4.2390984453977e-16
0.0008 1.88811660352377e-12
0.001 9.35180497548891e-10
0.0012 1.15441697355956e-07
0.0014 5.46251003336046e-06
0.0016 0.000128110137580322
0.0018 0.00175760125050102
0.002 0.0157962100273986
0.0022 0.100817953923259
0.0024 0.485020261218208
0.0026 1.84024622579568
0.0028 5.70381922363986
0.003 14.8505340782921
0.0032 33.2177029909548
0.0034 65.0182405218694
0.0036 113.071974323329
0.0038 176.958291931182
0.004 251.92521080782
0.0042 329.27516283
0.0044 398.268505124031
0.0046 448.85515884413
0.0048 474.189815847679
0.005 472.060411156707
0.0052 444.894558756359
0.0054 398.581697445687
0.0056 340.698284065176
0.0058 278.765390132612
0.006 218.979357069213
0.0062 165.58315457169
0.0064 120.815079026538
0.0066 85.2442631844142
0.0068 58.2787711177048
0.007 38.6762174669473
0.0072 24.9566689864978
0.0074 15.6819348518085
0.0076 9.6092950684639
0.0078 5.74941026071925
0.008 3.36289542641486
0.0082 1.92504865939838
0.0084 1.07957702041241
0.0086 0.593694418792337
0.0088 0.320446113786265
0.009 0.169898314826288
0.0092 0.0885524876070683
0.0094 0.0454050383416075
0.0096 0.0229188611207476
0.0098 0.0113957692002469
0.01 0.00558489580279418
0.0102 0.00269929974117084
0.0104 0.00128730466490826
0.0106 0.000606069215412055
0.0108 0.000281824066500681
0.011 0.000129491710086973
0.0112 5.88161400591037e-05
0.0114 2.64188958878439e-05
0.0116 1.17397891884797e-05
0.0118 5.16284297470597e-06
0.012 2.24775082783557e-06
0.0122 9.69121820503786e-07
0.0124 4.13916758159382e-07
0.0126 1.75177831434875e-07
0.0128 7.34848547104344e-08
0.013 3.05621933299358e-08
0.0132 1.26051877673596e-08
0.0134 5.1570032813465e-09
0.0136 2.0932893944091e-09
0.0138 8.43220198538763e-10
0.014 3.37151058369145e-10
0.0142 1.33834435916904e-10
0.0144 5.27539961091201e-11
0.0146 2.06522717779017e-11
0.0148 8.03124382031559e-12
0.015 3.10294950655172e-12
0.0152 1.19128308015436e-12
0.0154 4.54540149913224e-13
0.0156 1.72390181140141e-13
0.0158 6.49978903194613e-14
0.016 2.43665206508037e-14
0.0162 9.08351819191971e-15
0.0164 3.36773254243571e-15
0.0166 1.24193420842092e-15
0.0168 4.55606253832252e-16
0.017 1.66288257913556e-16
0.0172 6.03897535029679e-17
0.0174 2.1824381482602e-17
0.0176 7.84952309420417e-18
0.0178 2.81003232437693e-18
0.018 1.00135639304338e-18
0.0182 3.55235641677016e-19
0.0184 1.25468504813723e-19
0.0186 4.4124701636192e-20
0.0188 1.54523668801194e-20
0.019 5.38902904610328e-21
0.0192 1.87181662615934e-21
0.0194 6.47570756759223e-22
0.0196 2.23159383585354e-22
0.0198 7.66088454499077e-23
0.02 2.62004893068498e-23
0.0202 8.92763785971532e-24
0.0204 3.03102471455776e-24
0.0206 1.02540718757071e-24
0.0208 3.45688238253561e-25
0.021 1.1613973381258e-25
0.0212 3.88875765368101e-26
0.0214 1.29777643437928e-26
0.0216 4.31690271137589e-27
0.0218 1.43136948614258e-27
0.022 4.7310926805422e-28
0.0222 1.55891999605845e-28
0.0224 5.12107018312939e-29
0.0226 1.67723311744901e-29
0.0228 5.47699873987991e-30
0.023 1.78331691502149e-30
0.0232 5.78989560321019e-31
0.0234 1.87451269749009e-31
0.0236 6.05201690680936e-32
0.0238 1.94860647371264e-32
0.024 6.25717244805115e-33
0.0242 2.0039152602773e-33
0.0244 6.40095343780388e-34
0.0246 2.0393440465356e-34
0.0248 6.48086350573449e-35
0.025 2.05441148839129e-35
0.0252 6.4963504541668e-36
0.0254 2.04924464279847e-36
0.0256 6.44874307040056e-37
0.0258 2.02454508309857e-37
0.026 6.34110320078355e-38
0.0262 1.98153040354014e-38
0.0264 6.17800786984118e-39
0.0266 1.92185633197062e-39
0.0268 5.96527926936136e-40
0.027 1.84752537803359e-40
0.0272 5.70968188505042e-41
0.0274 1.76078815528641e-41
0.0276 5.4186059585716e-42
0.0278 1.66404327671261e-42
0.028 5.09975510408039e-43
0.0282 1.5597411127119e-43
0.0284 4.76085349466722e-44
0.0286 1.45029581675223e-44
0.0288 4.40938493130639e-45
0.029 1.3380089717475e-45
0.0292 4.05237263667289e-46
0.0294 1.22500709199272e-46
0.0296 3.69620508597466e-47
0.0298 1.11319412170508e-47
0.03 3.34650985758505e-48
0.0302 1.00421907518375e-48
0.0304 3.00807455797737e-49
0.0306 8.99458118226471e-50
0.0308 2.68481148237362e-50
0.031 8.00009727731514e-51
0.0312 2.37976088390367e-51
0.0314 7.06701098749733e-52
0.0316 2.0951265513022e-52
0.0318 6.2010369128817e-53
0.032 1.83233680229848e-53
0.0322 5.40555705491807e-54
0.0324 1.59212391595571e-54
0.0326 4.68189317030407e-55
0.0328 1.37461535925932e-55
0.033 4.0296066309506e-56
0.0332 1.17943080786081e-56
0.0334 3.44680810294073e-57
0.0336 1.00577981451736e-57
0.0338 2.93046227125017e-58
0.034 8.52555945485244e-59
0.0342 2.4766759715603e-59
0.0344 7.184242020946e-60
0.0346 2.08096120856607e-60
0.0348 6.01899505301586e-61
0.035 1.73846745830563e-61
0.0352 5.01414895848345e-62
0.0354 1.4441802412468e-62
0.0356 4.15378859254108e-63
0.0358 1.1930851351353e-63
0.036 3.42221805823752e-64
0.0362 9.80298138120522e-65
0.0364 2.80432216538558e-65
0.0366 8.01164595555668e-66
0.0368 2.28583311371841e-66
0.037 6.51329797203411e-67
0.0372 1.85351319674645e-67
0.0374 5.26784882758811e-68
0.0376 1.4952654973008e-68
0.0378 4.23891920517638e-69
0.038 1.2001848555619e-69
0.0382 3.39392008576943e-70
0.0384 9.58561031136903e-71
0.0386 2.70400050940451e-71
0.0388 7.61845138342213e-72
0.039 2.14389538326699e-72
0.0392 6.02589274852925e-73
0.0394 1.69170264255059e-73
};
\addlegendentry{notebook}
\addplot [semithick, color1, opacity=0.8]
table {%
0 0
0.0002 4.04368514862087e-35
0.0004 1.25272008938949e-24
0.0006 9.65346259577257e-19
0.0008 9.69094321532076e-15
0.001 9.01653559458301e-12
0.0012 1.86323886206438e-09
0.0014 1.36308759054038e-07
0.0016 4.66297027012948e-06
0.0018 8.92566476238349e-05
0.002 0.0010806496464931
0.0022 0.00903151459729266
0.0024 0.0555775579214406
0.0026 0.264477804740357
0.0028 1.01106053297254
0.003 3.20022716358246
0.0032 8.59359582485187
0.0034 19.9714205468096
0.0036 40.8364337262247
0.0038 74.489190934246
0.004 122.63737500972
0.0042 184.064660928411
0.0044 254.018066296762
0.0046 324.74056978027
0.0048 387.083461543562
0.005 432.655136750648
0.0052 455.753734287534
0.0054 454.469278713062
0.0056 430.713393972366
0.0058 389.335391130023
0.006 336.74120792832
0.0062 279.482379356421
0.0064 223.166139852648
0.0066 171.847592783236
0.0068 127.889117288277
0.007 92.161852432548
0.0072 64.4283023771624
0.0074 43.7649384616828
0.0076 28.9307060251247
0.0078 18.6372741031152
0.008 11.7154187861733
0.0082 7.19457466602162
0.0084 4.32122161225997
0.0086 2.54103263182437
0.0088 1.4643115661178
0.009 0.82768562598747
0.0092 0.459271463321069
0.0094 0.250371931866057
0.0096 0.134193806359963
0.0098 0.0707634126579589
0.01 0.0367362846068369
0.0102 0.018786932501781
0.0104 0.00946976789378442
0.0106 0.00470738837305254
0.0108 0.00230886868815283
0.011 0.00111790791988685
0.0112 0.000534562153913613
0.0114 0.00025255910834419
0.0116 0.000117944587665254
0.0118 5.44642522814749e-05
0.012 2.48784617872186e-05
0.0122 1.12451673411679e-05
0.0124 5.03133566191059e-06
0.0126 2.22901629785383e-06
0.0128 9.78108167672577e-07
0.013 4.25235215155964e-07
0.0132 1.8321436994974e-07
0.0134 7.82513271224556e-08
0.0136 3.31386328456154e-08
0.0138 1.39184931767362e-08
0.014 5.79914781556753e-09
0.0142 2.39742918516806e-09
0.0144 9.83622316412938e-10
0.0146 4.00589423659466e-10
0.0148 1.61973166427244e-10
0.015 6.50338909349618e-11
0.0152 2.59338417965968e-11
0.0154 1.02730351130093e-11
0.0156 4.04302985151058e-12
0.0158 1.58110282220569e-12
0.016 6.14503562618277e-13
0.0162 2.37390584422077e-13
0.0164 9.1167215395404e-14
0.0166 3.48104066439517e-14
0.0168 1.32169630292309e-14
0.017 4.99069887932052e-15
0.0172 1.87435212674958e-15
0.0174 7.00248674727646e-16
0.0176 2.60264010969948e-16
0.0178 9.62463769271557e-17
0.018 3.54167940614942e-17
0.0182 1.29698003774648e-17
0.0184 4.7271530918389e-18
0.0186 1.71494467053479e-18
0.0188 6.19334149097625e-19
0.019 2.22670930626025e-19
0.0192 7.97082893690277e-20
0.0194 2.84106797975296e-20
0.0196 1.00840155791236e-20
0.0198 3.56445901111221e-21
0.02 1.25486042699487e-21
0.0202 4.40018920549441e-22
0.0204 1.53692577618315e-22
0.0206 5.34774408776781e-23
0.0208 1.85376292933918e-23
0.021 6.40224732564722e-24
0.0212 2.20309578999997e-24
0.0214 7.55412564088324e-25
0.0216 2.58113884798824e-25
0.0218 8.7890227501203e-26
0.022 2.98261281271697e-26
0.0222 1.00879882995762e-26
0.0224 3.40084889151921e-27
0.0226 1.14279470199847e-27
0.0228 3.82797991013378e-28
0.023 1.27824314593684e-28
0.0232 4.25520963387255e-29
0.0234 1.41225408013708e-29
0.0236 4.67314546043816e-30
0.0238 1.54180704611887e-30
0.024 5.07217470679387e-31
0.0242 1.66387440219997e-31
0.0244 5.4428614463191e-32
0.0246 1.77554717500868e-32
0.0248 5.77633952463157e-33
0.025 1.87415588932812e-33
0.0252 6.06467804373849e-34
0.0254 1.95737920655727e-34
0.0256 6.30119810772174e-35
0.0258 2.02333425052184e-35
0.026 6.48072365427423e-36
0.0262 2.07064396562142e-36
0.0264 6.59975431842602e-37
0.0266 2.09847858540786e-37
0.0268 6.65655398417971e-38
0.027 2.10657012662075e-38
0.0272 6.65115446267256e-39
0.0274 2.0952006066426e-39
0.0276 6.5852791193173e-40
0.0278 2.06516627344732e-40
0.028 6.46219586689164e-41
0.0282 2.0177214295937e-41
0.0284 6.28651246634623e-42
0.0286 1.95450635618193e-42
0.0288 6.06392937465401e-43
0.029 1.87746436861622e-43
0.0292 5.8009664131289e-44
0.0294 1.78875316998022e-44
0.0296 5.50467937292145e-45
0.0298 1.69065544747347e-45
0.03 5.18238148622274e-46
0.0302 1.58549314377919e-46
0.0304 4.84138269202927e-47
0.0306 1.47554910417118e-47
0.0308 4.48875706629181e-48
0.031 1.36299893327607e-48
0.0312 4.13114592599943e-49
0.0314 1.24985497245378e-49
0.0316 3.77460119646181e-50
0.0318 1.13792340123613e-50
0.032 3.4244708583103e-51
0.0322 1.02877463298859e-51
0.0324 3.08532582781497e-52
0.0326 9.2372645944798e-53
0.0328 2.7609255834434e-53
0.033 8.23838828545169e-54
0.0332 2.45421829531249e-54
0.0334 7.29918727184782e-55
0.0336 2.16737015903793e-55
0.0338 6.42533384294508e-56
0.034 1.90181805977008e-56
0.0342 5.62029897630528e-57
0.0344 1.65833954505777e-57
0.0346 4.88559400972066e-58
0.0348 1.43713429611765e-58
0.035 4.22104002158139e-59
0.0352 1.2379117751643e-59
0.0354 3.62504910675334e-60
0.0356 1.05998040843784e-60
0.0358 3.09490410293414e-61
0.036 9.02334461341669e-62
0.0362 2.62702593828628e-62
0.0364 7.63735603578234e-63
0.0366 2.21722043117976e-63
0.0368 6.42786990157014e-64
0.037 1.86089891413977e-64
0.0372 5.37998453912766e-65
0.0374 1.5532693565287e-65
0.0376 4.47842085405192e-66
0.0378 1.28949664552523e-66
0.038 3.70798048055983e-67
0.0382 1.06483231137532e-67
0.0384 3.05390932274284e-68
0.0386 8.74715240754835e-69
0.0388 2.50217293175214e-69
0.039 7.14845826172323e-70
0.0392 2.03965250032098e-70
0.0394 5.8123658209206e-71
};
\addlegendentry{home theater}
\addplot [semithick, color2, opacity=0.8]
table {%
0 0
0.0002 6.88149847089383e-283
0.0004 5.38329324547414e-232
0.0006 1.80526708928995e-202
0.0008 1.0802222727238e-181
0.001 1.03927455538975e-165
0.0012 9.284355067112e-153
0.0014 6.70264999764463e-142
0.0016 1.42270014450123e-132
0.0018 2.04241209795704e-124
0.002 3.50239373677996e-117
0.0022 1.0763994483646e-110
0.0024 7.99953899528515e-105
0.0026 1.80482808273425e-99
0.0028 1.47530594980937e-94
0.003 5.0266949465783e-90
0.0032 7.99309022920627e-86
0.0034 6.50610273957532e-82
0.0036 2.92653432604146e-78
0.0038 7.75652633853781e-75
0.004 1.27887150345601e-71
0.0042 1.37387136164765e-68
0.0044 1.00075911725486e-65
0.0046 5.11650284411508e-63
0.0048 1.89215762100807e-60
0.005 5.19708066546697e-58
0.0052 1.08519112913327e-55
0.0054 1.75864114867945e-53
0.0056 2.25305595468255e-51
0.0058 2.31978418006334e-49
0.006 1.94819763121604e-47
0.0062 1.35246061951633e-45
0.0064 7.85536854187431e-44
0.0066 3.85939041326377e-42
0.0068 1.61999834203236e-40
0.007 5.8628757428418e-39
0.0072 1.84471455956781e-37
0.0074 5.08503961140203e-36
0.0076 1.23670034865055e-34
0.0078 2.67091172323767e-33
0.008 5.15329732842381e-32
0.0082 8.93211136472859e-31
0.0084 1.39799146700368e-29
0.0086 1.98525615382389e-28
0.0088 2.56937152193057e-27
0.009 3.04327839977601e-26
0.0092 3.31167624895744e-25
0.0094 3.32294847816438e-24
0.0096 3.08494431084593e-23
0.0098 2.65831236795176e-22
0.01 2.1325663849262e-21
0.0102 1.59721161269042e-20
0.0104 1.11979829006939e-19
0.0106 7.36754354703253e-19
0.0108 4.55972127990545e-18
0.011 2.66046231163419e-17
0.0112 1.46655123482879e-16
0.0114 7.65293478264192e-16
0.0116 3.7876809871124e-15
0.0118 1.78120985335917e-14
0.012 7.97251566953794e-14
0.0122 3.4018858950948e-13
0.0124 1.38598721522646e-12
0.0126 5.39948902691054e-12
0.0128 2.01422837127883e-11
0.013 7.20456391440064e-11
0.0132 2.47402100952021e-10
0.0134 8.16628494582459e-10
0.0136 2.59403838921817e-09
0.0138 7.93856711154157e-09
0.014 2.34306300004452e-08
0.0142 6.67643356640233e-08
0.0144 1.83843314480245e-07
0.0146 4.89666471389706e-07
0.0148 1.2626761922186e-06
0.015 3.15498203452717e-06
0.0152 7.64494508653064e-06
0.0154 1.7979170213845e-05
0.0156 4.10690407641051e-05
0.0158 9.11860238376426e-05
0.016 0.000196932737783124
0.0162 0.000413979796989082
0.0164 0.000847611318286239
0.0166 0.00169139539232139
0.0168 0.003291462393694
0.017 0.0062500454828601
0.0172 0.0115870850932161
0.0174 0.0209845482354142
0.0176 0.0371440265188652
0.0178 0.0642930463763373
0.018 0.108877714983386
0.0182 0.180476707043854
0.0184 0.292961727437675
0.0186 0.465910058625537
0.0188 0.726243790315821
0.019 1.11002725226193
0.0192 1.66430041509408
0.0194 2.4487655468853
0.0196 3.5370840535338
0.0198 5.01748973197458
0.02 6.99239507315247
0.0202 9.57667065386176
0.0204 12.8943243600301
0.0206 17.0734035981651
0.0208 22.2390900744056
0.021 28.5051457064319
0.0212 35.9640840071515
0.0214 44.6766605348988
0.0216 54.6614702084668
0.0218 65.8855783073964
0.022 78.2571685923711
0.0222 91.6211465226821
0.0224 105.758479911965
0.0226 120.38979954693
0.0228 135.183438916364
0.023 149.767698722649
0.0232 163.746720685541
0.0234 176.718992149162
0.0236 188.297221225538
0.0238 198.128155922669
0.024 205.910890617481
0.0242 211.412313680501
0.0244 214.478588567525
0.0246 215.041899952673
0.0248 213.122098284032
0.025 208.823296702023
0.0252 202.325869917506
0.0254 193.874637245711
0.0256 183.764253081408
0.0258 172.322961516423
0.026 159.895894406816
0.0262 146.829012911792
0.0264 133.454629913765
0.0266 120.079229733793
0.0268 106.974049847052
0.027 94.3686338448545
0.0272 82.4473290789113
0.0274 71.3485041235013
0.0276 61.1661117241199
0.0278 51.9531269734913
0.028 43.726346859192
0.0282 36.4720401617675
0.0284 30.1519768750148
0.0286 24.7094331770909
0.0288 20.0748505940625
0.029 16.1709163678962
0.0292 12.9169178555411
0.0294 10.2323008361552
0.0296 8.03942588731629
0.0298 6.26556655955109
0.03 4.84422771277173
0.0302 3.71588317288378
0.0304 2.82824078929698
0.0306 2.13614245624748
0.0308 1.60119925386587
0.031 1.19124998031315
0.0312 0.879717081508035
0.0314 0.644919030540246
0.0316 0.469383820414796
0.0318 0.339195253214369
0.032 0.243392617438687
0.0322 0.173435333766309
0.0324 0.122737192065047
0.0326 0.0862697274977952
0.0328 0.0602308341323727
0.033 0.0417725969301809
0.0332 0.0287812426330238
0.0334 0.019701794893277
0.0336 0.0134002335232519
0.0338 0.00905650894641977
0.034 0.00608250052212452
0.0342 0.00405982042295815
0.0344 0.00269317607639867
0.0346 0.00177576432103223
0.0348 0.00116385116920287
0.035 0.000758279830267585
0.0352 0.00049114463290525
0.0354 0.00031627478765365
0.0356 0.000202498579474155
0.0358 0.000128916390980941
0.036 8.16110779963494e-05
0.0362 5.13772129292449e-05
0.0364 3.21660422831698e-05
0.0366 2.00288110665282e-05
0.0368 1.24041655210445e-05
0.037 7.64115074510302e-06
0.0372 4.68222646366409e-06
0.0374 2.85411663334271e-06
0.0376 1.73077064148876e-06
0.0378 1.04418706414512e-06
0.038 6.2677289956069e-07
0.0382 3.74332040693653e-07
0.0384 2.22453948782417e-07
0.0386 1.31546913680614e-07
0.0388 7.74104168757179e-08
0.039 4.53332442018682e-08
0.0392 2.64212268867879e-08
0.0394 1.53259631452647e-08
};
\addlegendentry{desktop computer}
\addplot [semithick, color3, opacity=0.8]
table {%
0 0
0.0002 1.24681940844559e-185
0.0004 3.78429989649157e-149
0.0006 4.69624390399748e-128
0.0008 2.91814488510357e-113
0.001 6.39937004086229e-102
0.0012 9.19292845684038e-93
0.0014 4.10935002282113e-85
0.0016 1.44888628628164e-78
0.0018 7.34152876082861e-73
0.002 8.0525019711728e-68
0.0022 2.56068941187556e-63
0.0024 2.9292421581097e-59
0.0026 1.41995830907816e-55
0.0028 3.31301642990214e-52
0.003 4.11568387030809e-49
0.0032 2.95307692152295e-46
0.0034 1.30806963691571e-43
0.0036 3.77970210701459e-41
0.0038 7.46130251996167e-39
0.004 1.04633990725249e-36
0.0042 1.07774752268699e-34
0.0044 8.3909217479038e-33
0.0046 5.06232673594148e-31
0.0048 2.41856883933146e-29
0.005 9.32610456511297e-28
0.0052 2.95167434916092e-26
0.0054 7.78268945730413e-25
0.0056 1.73238744737497e-23
0.0058 3.29434213604689e-22
0.006 5.40916986257712e-21
0.0062 7.74292596344275e-20
0.0064 9.74695105038983e-19
0.0066 1.08755078580298e-17
0.0068 1.08334855272526e-16
0.007 9.69782013606882e-16
0.0072 7.84830360507471e-15
0.0074 5.7738733862825e-14
0.0076 3.88106550443288e-13
0.0078 2.39473656171717e-12
0.008 1.36227283385691e-11
0.0082 7.17311040949162e-11
0.0084 3.50913326863578e-10
0.0086 1.6004392380367e-09
0.0088 6.82684778432269e-09
0.009 2.73176680030113e-08
0.0092 1.02831056135345e-07
0.0094 3.65088994324908e-07
0.0096 1.22555178042531e-06
0.0098 3.89871873397428e-06
0.01 1.17789817815586e-05
0.0102 3.3866611040304e-05
0.0104 9.28421083067902e-05
0.0106 0.000243114650517988
0.0108 0.000609129762155081
0.011 0.00146264862143454
0.0112 0.00337103179365444
0.0114 0.00746800143814087
0.0116 0.0159242143929695
0.0118 0.0327255277864527
0.012 0.0648970380843938
0.0122 0.12433136640818
0.0124 0.230376091159419
0.0126 0.413290121770664
0.0128 0.71857463105485
0.013 1.21200914038453
0.0132 1.98497868160372
0.0134 3.15938477142859
0.0136 4.89113638839976
0.0138 7.37098493642559
0.014 10.8213805301847
0.0142 15.4881639756421
0.0144 21.6263228554087
0.0146 29.4797377603284
0.0148 39.2557701993348
0.015 51.0965758321095
0.0152 65.0499917528924
0.0154 81.043547656562
0.0156 98.8654077980709
0.0158 118.155743089958
0.016 138.411133486056
0.0162 159.003193096051
0.0164 179.210880628243
0.0166 198.264165244307
0.0168 215.395147924022
0.017 229.891647574822
0.0172 241.147828318187
0.0174 248.706737842329
0.0176 252.290593545106
0.0178 251.816133107019
0.018 247.394104883597
0.0182 239.313747960238
0.0184 228.014656534806
0.0186 214.049549881916
0.0188 198.042070256027
0.019 180.643788201455
0.0192 162.494169280991
0.0194 144.186468575044
0.0196 126.24152132235
0.0198 109.090345237702
0.02 93.0654979661408
0.0202 78.4003417082175
0.0204 65.2348133355188
0.0206 53.6259974478233
0.0208 43.5617320532775
0.021 34.9755978219366
0.0212 27.7618954211491
0.0214 21.7895424162046
0.0216 16.9141687949919
0.0218 12.9880167701224
0.022 9.86752800549861
0.0222 7.41871492887164
0.0224 5.52055875933554
0.0226 4.06676018328376
0.0228 2.96619948938404
0.023 2.14245417264068
0.0232 1.53268653943172
0.0234 1.08616336780916
0.0236 0.76261366342503
0.0238 0.530575922349809
0.024 0.365837537964855
0.0242 0.250028440537005
0.0244 0.169399480962606
0.0246 0.113793080852648
0.0248 0.0757982021597045
0.025 0.0500723496819741
0.0252 0.0328086565119342
0.0254 0.0213247725075256
0.0256 0.0137511384350888
0.0258 0.00879838085750516
0.026 0.00558633645818186
0.0262 0.00352014751036091
0.0264 0.0022016697110189
0.0266 0.0013669354400426
0.0268 0.000842547340201024
0.027 0.000515627426063578
0.0272 0.000313340849114407
0.0274 0.000189095104711343
0.0276 0.00011333598801155
0.0278 6.74715624447029e-05
0.028 3.99005885881548e-05
0.0282 2.34413596984823e-05
0.0284 1.3682636249639e-05
0.0286 7.93556697071585e-06
0.0288 4.57345341808433e-06
0.029 2.61941815048323e-06
0.0292 1.49105988537527e-06
0.0294 8.43625509032096e-07
0.0296 4.74463336067587e-07
0.0298 2.65269451197919e-07
0.03 1.47447041872296e-07
0.0302 8.14856437416443e-08
0.0304 4.47768648879564e-08
0.0306 2.44672142554971e-08
0.0308 1.32954714741728e-08
0.031 7.18523313166044e-09
0.0312 3.86211184480844e-09
0.0314 2.0648290071526e-09
0.0316 1.09811155552866e-09
0.0318 5.80950707694493e-10
0.032 3.05766131800596e-10
0.0322 1.60111695949653e-10
0.0324 8.34192239271124e-11
0.0326 4.32458052135747e-11
0.0328 2.2309069997019e-11
0.033 1.14525713369952e-11
0.0332 5.85103354789937e-12
0.0334 2.97504653091987e-12
0.0336 1.50560016949624e-12
0.0338 7.58408127498507e-13
0.034 3.80273483880839e-13
0.0342 1.89806362745019e-13
0.0344 9.43123749013075e-14
0.0346 4.66542210861782e-14
0.0348 2.29772655367192e-14
0.035 1.12670783224797e-14
0.0352 5.50110500809886e-15
0.0354 2.67444670514748e-15
0.0356 1.29474057046822e-15
0.0358 6.24188034795496e-16
0.036 2.9967516537154e-16
0.0362 1.43287117184139e-16
0.0364 6.82343019931502e-17
0.0366 3.23634952024334e-17
0.0368 1.52891270231179e-17
0.037 7.19452002562637e-18
0.0372 3.37232801960942e-18
0.0374 1.57464751428699e-18
0.0376 7.32451525804251e-19
0.0378 3.39416222766471e-19
0.038 1.56696896785998e-19
0.0382 7.20738888345073e-20
0.0384 3.30294009683984e-20
0.0386 1.5081474411321e-20
0.0388 6.86154969376638e-21
0.039 3.11064639137136e-21
0.0392 1.40521760798034e-21
0.0394 6.32579760713982e-22
};
\addlegendentry{monitor}
\addplot [semithick, color4, opacity=0.8]
table {%
0 0
0.0002 7.07051949791944e-220
0.0004 3.35713224621954e-178
0.0006 4.55876472150082e-154
0.0008 4.06372766744917e-137
0.001 4.19927757126269e-124
0.0012 1.40567805458719e-113
0.0014 9.00643402552461e-105
0.0016 3.18925325198261e-97
0.0018 1.23690071259004e-90
0.002 8.38122637353824e-85
0.0022 1.38442183846587e-79
0.0024 7.12902189670533e-75
0.0026 1.3794522412844e-70
0.0028 1.15971222412979e-66
0.003 4.75273753659039e-63
0.0032 1.04179315001234e-59
0.0034 1.31771394306738e-56
0.0036 1.02415094765992e-53
0.0038 5.15560039198329e-51
0.004 1.75757764872752e-48
0.0042 4.2148638496568e-46
0.0044 7.3467295284001e-44
0.0046 9.57542435968325e-42
0.0048 9.56565522459126e-40
0.005 7.48492710605918e-38
0.0052 4.67617522469868e-36
0.0054 2.37244858785325e-34
0.0056 9.92369722206861e-33
0.0058 3.46894198997681e-31
0.006 1.02575999424703e-29
0.0062 2.59405173111155e-28
0.0064 5.66632176036028e-27
0.0066 1.07875184310474e-25
0.0068 1.80466875933379e-24
0.007 2.67286801654633e-23
0.0072 3.52886813743904e-22
0.0074 4.1792715817461e-21
0.0076 4.46564241233065e-20
0.0078 4.32813414557636e-19
0.008 3.82375668734256e-18
0.0082 3.09338087199957e-17
0.0084 2.3012710139591e-16
0.0086 1.58052451958739e-15
0.0088 1.00582834740469e-14
0.009 5.95139142432837e-14
0.0092 3.28450725704436e-13
0.0094 1.69580021234189e-12
0.0096 8.21383031334312e-12
0.0098 3.74215771035445e-11
0.01 1.60758325803006e-10
0.0102 6.52688622108691e-10
0.0104 2.50996299416417e-09
0.0106 9.16116446427591e-09
0.0108 3.17979701351146e-08
0.011 1.05150251268355e-07
0.0112 3.31846466176376e-07
0.0114 1.00113761423932e-06
0.0116 2.89172206747296e-06
0.0118 8.00879727229867e-06
0.012 2.12978350679836e-05
0.0122 5.44553092320697e-05
0.0124 0.000134039423465113
0.0126 0.000318007176388045
0.0128 0.000728036322089197
0.013 0.0016101168440237
0.0132 0.00344353575065258
0.0134 0.00712900268421548
0.0136 0.0143002873493135
0.0138 0.0278194928102673
0.014 0.0525317220624886
0.0142 0.0963663542846326
0.0144 0.171873415509561
0.0146 0.298266989051259
0.0148 0.504005145686858
0.015 0.829860797883638
0.0152 1.33232757972859
0.0154 2.08706366547733
0.0156 3.19191805935217
0.0158 4.76893233237364
0.016 6.96459902866247
0.0162 9.94762374169503
0.0164 13.9035166777831
0.0166 19.0255558911488
0.0168 25.5020229662806
0.017 33.5000909117679
0.0172 43.1472928224831
0.0174 54.5120428038473
0.0176 67.5851270066647
0.0178 82.2643417177921
0.018 98.3444542406121
0.0182 115.514361790022
0.0184 133.362730410417
0.0186 151.392565798958
0.0188 169.04420019866
0.019 185.725202628365
0.0192 200.84487054017
0.0194 213.850362049618
0.0196 224.261266652776
0.0198 231.69952683913
0.02 235.912096199222
0.0202 236.784483492525
0.0204 234.344280206117
0.0206 228.754774702605
0.0208 220.299692533207
0.021 209.360862067862
0.0212 196.391112204931
0.0214 181.884929654833
0.0216 166.349342768124
0.0218 150.277197440381
0.022 134.124513155393
0.0222 118.293030857027
0.0224 103.118466326843
0.0226 88.8644303563046
0.0228 75.7215206023922
0.023 63.8107597647364
0.0232 53.1903606992744
0.0234 43.8647340785928
0.0236 35.794698231767
0.0238 28.9079762621495
0.024 23.1092422614582
0.0242 18.2891777620824
0.0244 14.3321973242304
0.0246 11.1226802899885
0.0248 8.5496927906303
0.025 6.51029473521079
0.0252 4.91160046880308
0.0254 3.67180249028864
0.0256 2.72038076443975
0.0258 1.9977124689199
0.026 1.4542752129583
0.0262 1.04960689175367
0.0264 0.751152344904728
0.0266 0.533094561978577
0.0268 0.37523879492738
0.027 0.261992996537101
0.0272 0.181468053245895
0.0274 0.124706245475069
0.0276 0.0850357725091174
0.0278 0.0575423270059376
0.028 0.038644824705325
0.0282 0.0257607381643917
0.0284 0.0170463826302373
0.0286 0.0111984039614772
0.0288 0.00730419057073215
0.029 0.0047306558591358
0.0292 0.00304259595678884
0.0294 0.00194348157065519
0.0296 0.0012330154694237
0.0298 0.000777045655683668
0.03 0.000486465062960411
0.0302 0.000302566146091308
0.0304 0.00018697704844155
0.0306 0.000114812871709676
0.0308 7.00586093683498e-05
0.031 4.2484916600035e-05
0.0312 2.56060962327199e-05
0.0314 1.53398004244612e-05
0.0316 9.13472646610638e-06
0.0318 5.40756879979802e-06
0.032 3.18251022839673e-06
0.0322 1.86221069963013e-06
0.0324 1.08344914359411e-06
0.0326 6.26812791986565e-07
0.0328 3.6061613226628e-07
0.033 2.06328055433611e-07
0.0332 1.17409868378581e-07
0.0334 6.64524433226212e-08
0.0336 3.74113788594143e-08
0.0338 2.09511944436424e-08
0.034 1.16721702412188e-08
0.0342 6.4692984389792e-09
0.0344 3.56738566186036e-09
0.0346 1.95728635927153e-09
0.0348 1.06854746121432e-09
0.035 5.80485940657903e-10
0.0352 3.13813005027762e-10
0.0354 1.68831712046916e-10
0.0356 9.03988896788645e-11
0.0358 4.81747989009494e-11
0.036 2.55532332739877e-11
0.0362 1.34915541309124e-11
0.0364 7.09069752729882e-12
0.0366 3.70977136869238e-12
0.0368 1.93221989411021e-12
0.037 1.00192879722832e-12
0.0372 5.17258269270647e-13
0.0374 2.65881030828973e-13
0.0376 1.36080326664345e-13
0.0378 6.93505420970246e-14
0.038 3.51940350459016e-14
0.0382 1.77856941332027e-14
0.0384 8.95102399128808e-15
0.0386 4.48633895026745e-15
0.0388 2.23947360167532e-15
0.039 1.11339969632567e-15
0.0392 5.51345784477401e-16
0.0394 2.71945011684331e-16
};
\addlegendentry{screen}
\addplot [semithick, color5, opacity=0.8]
table {%
0 0
0.0002 1.59862054051401e-63
0.0004 2.01433889039574e-47
0.0006 2.97077629654634e-38
0.0008 6.36171904282589e-32
0.001 3.79438340819427e-27
0.0012 2.34966409367811e-23
0.0014 3.04782019614733e-20
0.0016 1.25904990534363e-17
0.0018 2.17034474693635e-15
0.002 1.87750330629904e-13
0.0022 9.2972654226903e-12
0.0024 2.90438635134726e-10
0.0026 6.16207408914511e-09
0.0028 9.40339740634871e-08
0.003 1.0801313742615e-06
0.0032 9.68774339936991e-06
0.0034 6.99108734785651e-05
0.0036 0.000416130640341037
0.0038 0.00208599494885405
0.004 0.00896271696907974
0.0042 0.0335068515028188
0.0044 0.110410042257394
0.0046 0.32428663564024
0.0048 0.857310496430777
0.005 2.0575910593128
0.0052 4.51727747541394
0.0054 9.13280377204685
0.0056 17.105512518158
0.0058 29.8396920784007
0.006 48.7151828613912
0.0062 74.7526085244504
0.0064 108.238434806541
0.0066 148.413644172392
0.0068 193.334360833966
0.007 239.977541268642
0.0072 284.599042113547
0.0074 323.277812243375
0.0076 352.525101961038
0.0078 369.820182835286
0.008 373.95799434267
0.0082 365.148310561923
0.0084 344.870503591877
0.0086 315.542489549099
0.0088 280.093413973026
0.009 241.533319073515
0.0092 202.594437566125
0.0094 165.48788308252
0.0096 131.787026921522
0.0098 102.422833784496
0.01 77.7608857315053
0.0102 57.7249279146676
0.0104 41.9351047754794
0.0106 29.8371367797266
0.0108 20.8082131422306
0.011 14.2339510184086
0.0112 9.55711844836904
0.0114 6.30262201512299
0.0116 4.08484518709801
0.0118 2.60341577529304
0.012 1.63254744116085
0.0122 1.00779280227118
0.0124 0.612741798994738
0.0126 0.367106247261684
0.0128 0.216826052554323
0.013 0.126306236283288
0.0132 0.0725957783827948
0.0134 0.0411853416623558
0.0136 0.0230718587213021
0.0138 0.012766968508524
0.014 0.00698084839717244
0.0142 0.00377300501691588
0.0144 0.0020163356605606
0.0146 0.00106577707534001
0.0148 0.000557345440476419
0.015 0.00028844194244243
0.0152 0.000147769489564178
0.0154 7.49575680292609e-05
0.0156 3.7658079611177e-05
0.0158 1.87420165753465e-05
0.016 9.24250139028707e-06
0.0162 4.5172452326614e-06
0.0164 2.18857329389199e-06
0.0166 1.05133420160334e-06
0.0168 5.00839471261352e-07
0.017 2.36655772209254e-07
0.0172 1.10936966886697e-07
0.0174 5.16004121551136e-08
0.0176 2.38189257271155e-08
0.0178 1.09132770888478e-08
0.018 4.96388475445766e-09
0.0182 2.2417587444063e-09
0.0184 1.00536023534575e-09
0.0186 4.47798927374872e-10
0.0188 1.98122479510485e-10
0.019 8.7082869969346e-11
0.0192 3.80309388371811e-11
0.0194 1.65044850569294e-11
0.0196 7.11837908731441e-12
0.0198 3.05159149971245e-12
0.02 1.30043368831224e-12
0.0202 5.50954117508255e-13
0.0204 2.32089498543993e-13
0.0206 9.72197326187167e-14
0.0208 4.0500178081169e-14
0.021 1.67805596237452e-14
0.0212 6.91584302087708e-15
0.0214 2.83539771694205e-15
0.0216 1.15651924143567e-15
0.0218 4.69353977721349e-16
0.022 1.89537255017966e-16
0.0222 7.61678629358275e-17
0.0224 3.04626530052703e-17
0.0226 1.21259887645579e-17
0.0228 4.80456288052996e-18
0.023 1.8950068938661e-18
0.0232 7.44080551455369e-19
0.0234 2.90879145590291e-19
0.0236 1.1321894943238e-19
0.0238 4.38802622641073e-20
0.024 1.69352564892312e-20
0.0242 6.50901095377492e-21
0.0244 2.49153147957588e-21
0.0246 9.49888708849199e-22
0.0248 3.60711681893591e-22
0.025 1.3644397353969e-22
0.0252 5.14138469226384e-23
0.0254 1.93001978872092e-23
0.0256 7.21810340585825e-24
0.0258 2.68959762500427e-24
0.026 9.98566661792328e-25
0.0262 3.6941496807468e-25
0.0264 1.36182474916111e-25
0.0266 5.00286065871767e-26
0.0268 1.83158343383945e-26
0.027 6.68292097066354e-27
0.0272 2.43028394239313e-27
0.0274 8.80882743445791e-28
0.0276 3.18250047040028e-28
0.0278 1.14611085680212e-28
0.028 4.11443911103716e-29
0.0282 1.47244206536675e-29
0.0284 5.25323318970638e-30
0.0286 1.86849950728891e-30
0.0288 6.62603526420191e-31
0.029 2.34274598112145e-31
0.0292 8.25892046285214e-32
0.0294 2.90311060532944e-32
0.0296 1.01756315239963e-32
0.0298 3.5565721067613e-33
0.03 1.23962116880601e-33
0.0302 4.30871570850815e-34
0.0304 1.49355861358081e-34
0.0306 5.16328482435287e-35
0.0308 1.78021645790086e-35
0.031 6.12175196186357e-36
0.0312 2.09965456035204e-36
0.0314 7.18293501274318e-37
0.0316 2.45104063814212e-37
0.0318 8.34268518420225e-38
0.032 2.83256587034247e-38
0.0322 9.59367104341985e-39
0.0324 3.24139436756253e-39
0.0326 1.09252795039009e-39
0.0328 3.67365231123331e-40
0.033 1.23236492322482e-40
0.0332 4.12446063746452e-41
0.0334 1.37718452035046e-41
0.0336 4.58801283402551e-42
0.0338 1.52501754192857e-42
0.034 5.05769688891288e-43
0.0342 1.67366426563217e-43
0.0344 5.52625785272222e-44
0.0346 1.82075161131955e-44
0.0348 5.98599503975823e-45
0.035 1.9638000548463e-45
0.0352 6.42898577109859e-46
0.0354 2.1002974480877e-46
0.0356 6.84732590989572e-47
0.0358 2.22777646835226e-47
0.036 7.23337637178125e-48
0.0362 2.34389218482664e-48
0.0364 7.58000438263052e-49
0.0366 2.44649620718702e-49
0.0368 7.88080852298557e-50
0.037 2.53370407054759e-50
0.0372 8.13031751342969e-51
0.0374 2.60395289449275e-51
0.0376 8.32415418185093e-52
0.0378 2.6560469014955e-52
0.038 8.45915799726862e-53
0.0382 2.68918905048345e-53
0.0384 8.53346181345964e-54
0.0386 2.70299778679314e-54
0.0388 8.54652088167659e-55
0.039 2.69750867965481e-55
0.0392 8.49909460584531e-56
0.0394 2.67316146339825e-56
};
\addlegendentry{entertainment center}
\addplot [semithick, color6, opacity=0.8]
table {%
0 0
0.0002 2.55329303009598e-95
0.0004 1.68182412350504e-73
0.0006 5.49781289889967e-61
0.0008 2.78718791670709e-52
0.001 1.15539866519647e-45
0.0012 2.29045339557146e-40
0.0014 5.57050636268994e-36
0.0016 2.91664333325709e-32
0.0018 4.72521876543319e-29
0.002 3.03440752670528e-26
0.0022 9.2163699131261e-24
0.0024 1.50843826973494e-21
0.0026 1.46891166386691e-19
0.0028 9.19186255467381e-18
0.003 3.92879593928923e-16
0.0032 1.20485095111868e-14
0.0034 2.75996884779829e-13
0.0036 4.8825997596155e-12
0.0038 6.85966297166094e-11
0.004 7.83646193730588e-10
0.0042 7.42781149314522e-09
0.0044 5.94374689671225e-08
0.0046 4.07612920102254e-07
0.0048 2.42727662171799e-06
0.005 1.26961478962858e-05
0.0052 5.89271481250347e-05
0.0054 0.00024488352824238
0.0056 0.000918517912095324
0.0058 0.00313195835741374
0.006 0.00977111118024442
0.0062 0.0280540417034728
0.0064 0.0745167406233305
0.0066 0.183988958680021
0.0068 0.424127151084662
0.007 0.916409463495715
0.0072 1.86272770489421
0.0074 3.57374020060483
0.0076 6.49145456201812
0.0078 11.1952645063309
0.008 18.3795368595609
0.0082 28.7934129004416
0.0084 43.1404483402246
0.0086 61.9462231884466
0.0088 85.4135836108871
0.009 113.294124010367
0.0092 144.807380446516
0.0094 178.633955532491
0.0096 212.995813988652
0.0098 245.819176375318
0.01 274.957380384007
0.0102 298.437587579442
0.0104 314.689804767927
0.0106 322.720544684573
0.0108 322.205200696454
0.011 313.489536716451
0.0112 297.507459103879
0.0114 275.635706083237
0.0116 249.513846594636
0.0118 220.859322105232
0.012 191.303056940233
0.0122 162.263310146721
0.0124 134.866214122589
0.0126 109.912871811589
0.0128 87.8863632630511
0.013 68.9881738239281
0.0132 53.1923420081801
0.0134 40.3065007250231
0.0136 30.0311924250773
0.0138 22.0116143616537
0.014 15.8786768394761
0.0142 11.2785303309885
0.0144 7.89134633657271
0.0146 5.44110428079139
0.0148 3.69853799204927
0.015 2.4793800811449
0.0152 1.63976851873599
0.0154 1.07028313230223
0.0156 0.689661504730145
0.0158 0.438867480799633
0.016 0.275882802307981
0.0162 0.171371414177485
0.0164 0.105220080883349
0.0166 0.0638741074673723
0.0168 0.0383470703560727
0.017 0.0227736030712954
0.0172 0.0133822979151883
0.0174 0.00778273944702365
0.0176 0.00448060903275038
0.0178 0.00255411761700844
0.018 0.00144190717862481
0.0182 0.000806336789362508
0.0184 0.000446751107565886
0.0186 0.000245283985078712
0.0188 0.00013347773643944
0.019 7.20051655177482e-05
0.0192 3.85132583475396e-05
0.0194 2.04278607692691e-05
0.0196 1.07466590998682e-05
0.0198 5.60830677274215e-06
0.02 2.90379129483597e-06
0.0202 1.49190101011772e-06
0.0204 7.60710253355381e-07
0.0206 3.85003786698849e-07
0.0208 1.93435964732349e-07
0.021 9.64926505877817e-08
0.0212 4.77961238059279e-08
0.0214 2.35118925731706e-08
0.0216 1.14876838412627e-08
0.0218 5.57544271889242e-09
0.022 2.68830683625505e-09
0.0222 1.28789403121594e-09
0.0224 6.13099745745081e-10
0.0226 2.90053740312067e-10
0.0228 1.36385285237948e-10
0.023 6.37444630840224e-11
0.0232 2.96173615640384e-11
0.0234 1.36811041615753e-11
0.0236 6.2835876296212e-12
0.0238 2.86976128942386e-12
0.024 1.3033896418131e-12
0.0242 5.88750121415524e-13
0.0244 2.64516597164137e-13
0.0246 1.18215938123978e-13
0.0248 5.255759287684e-14
0.025 2.32469252232671e-14
0.0252 1.02305659610502e-14
0.0254 4.47992316053469e-15
0.0256 1.95213630999053e-15
0.0258 8.46544011725541e-16
0.026 3.65358493949292e-16
0.0262 1.56945502083572e-16
0.0264 6.71069643194406e-17
0.0266 2.85629962894664e-17
0.0268 1.21027800607901e-17
0.027 5.10550684580449e-18
0.0272 2.14432921898816e-18
0.0274 8.96744853650233e-19
0.0276 3.73419105674198e-19
0.0278 1.54845725029218e-19
0.028 6.39442128938801e-20
0.0282 2.6298222127352e-20
0.0284 1.07720391140932e-20
0.0286 4.39479501814542e-21
0.0288 1.78595617606647e-21
0.029 7.2296335358632e-22
0.0292 2.91538871761007e-22
0.0294 1.17120333717999e-22
0.0296 4.6875326137853e-23
0.0298 1.86918625143203e-23
0.03 7.42637420595622e-24
0.0302 2.93992575451275e-24
0.0304 1.15971187226631e-24
0.0306 4.5586536851254e-25
0.0308 1.78572279395691e-25
0.031 6.97108311119898e-26
0.0312 2.71214399231177e-26
0.0314 1.05164374156872e-26
0.0316 4.06429084905467e-27
0.0318 1.56558863728127e-27
0.032 6.0112318400383e-28
0.0322 2.30069008420888e-28
0.0324 8.77762820435934e-29
0.0326 3.33838149293861e-29
0.0328 1.26575453175437e-29
0.033 4.78445538515688e-30
0.0332 1.80301565501958e-30
0.0334 6.77430386453264e-31
0.0336 2.53771003756134e-31
0.0338 9.47862151705772e-32
0.034 3.5301055865143e-32
0.0342 1.31093985672291e-32
0.0344 4.85448760330276e-33
0.0346 1.79259570964789e-33
0.0348 6.60103727962226e-34
0.035 2.42407039237887e-34
0.0352 8.87755963276196e-35
0.0354 3.24241957372043e-35
0.0356 1.18109214487747e-35
0.0358 4.29090419393295e-36
0.036 1.55480397682429e-36
0.0362 5.61921160918253e-37
0.0364 2.02562476103656e-37
0.0366 7.28345159341886e-38
0.0368 2.61228585498635e-38
0.037 9.34587126154641e-39
0.0372 3.33537578121633e-39
0.0374 1.18742368919572e-39
0.0376 4.2170866661322e-40
0.0378 1.49408388950585e-40
0.038 5.28083602364121e-41
0.0382 1.86210890398195e-41
0.0384 6.55075669093197e-42
0.0386 2.29916964220208e-42
0.0388 8.05105388170408e-43
0.039 2.81284300656489e-43
0.0392 9.80523563488412e-44
0.0394 3.41035006246725e-44
};
\addlegendentry{television}

\nextgroupplot[
height=\figheight,
legend cell align={left},
legend style={fill opacity=0.8, draw opacity=1, text opacity=1, draw=white!80!black},
tick align=outside,
tick pos=both,
width=\figwidth,
x grid style={white!69.0196078431373!black},
xmin=-0.0443, xmax=0.9303,
xtick style={color=black},
xtick={-0.2,0,0.2,0.4,0.6,0.8,1},
xticklabels={−0.2,0.0,0.2,0.4,0.6,0.8,1.0},
y grid style={white!69.0196078431373!black},
ymin=-114.098517279488, ymax=2396.06886286925,
ytick style={color=black}
]
\path [draw=none, fill=blue, fill opacity=0.5]
(axis cs:0,0)
--(axis cs:0.001,322.58271626249)
--(axis cs:0.002,4.5485439001751e-06)
--(axis cs:0.003,1.08331330228178e-16)
--(axis cs:0.004,1.85338944612998e-28)
--(axis cs:0.005,7.37735891461139e-41)
--(axis cs:0.006,1.15314008937915e-53)
--(axis cs:0.007,9.35747394104202e-67)
--(axis cs:0.008,4.65685972914961e-80)
--(axis cs:0.009,1.58252448966646e-93)
--(axis cs:0.01,3.95169462233851e-107)
--(axis cs:0.011,7.64016500319958e-121)
--(axis cs:0.012,1.18869226225084e-134)
--(axis cs:0.013,1.53251456003575e-148)
--(axis cs:0.014,1.67490368336033e-162)
--(axis cs:0.015,1.57998245946121e-176)
--(axis cs:0.016,1.30525638137975e-190)
--(axis cs:0.017,9.55588966251468e-205)
--(axis cs:0.018,6.26093883958636e-219)
--(axis cs:0.019,3.70139413359828e-233)
--(axis cs:0.02,1.98819757673142e-247)
--(axis cs:0.021,9.76084292719275e-262)
--(axis cs:0.022,4.40201301358189e-276)
--(axis cs:0.023,1.83170729292297e-290)
--(axis cs:0.024,7.05925369579821e-305)
--(axis cs:0.025,2.52818331633424e-319)
--(axis cs:0.026,0)
--(axis cs:0.027,0)
--(axis cs:0.028,0)
--(axis cs:0.029,0)
--(axis cs:0.03,0)
--(axis cs:0.031,0)
--(axis cs:0.032,0)
--(axis cs:0.033,0)
--(axis cs:0.034,0)
--(axis cs:0.035,0)
--(axis cs:0.036,0)
--(axis cs:0.037,0)
--(axis cs:0.038,0)
--(axis cs:0.039,0)
--(axis cs:0.04,0)
--(axis cs:0.041,0)
--(axis cs:0.042,0)
--(axis cs:0.043,0)
--(axis cs:0.044,0)
--(axis cs:0.045,0)
--(axis cs:0.046,0)
--(axis cs:0.047,0)
--(axis cs:0.048,0)
--(axis cs:0.049,0)
--(axis cs:0.05,0)
--(axis cs:0.051,0)
--(axis cs:0.052,0)
--(axis cs:0.053,0)
--(axis cs:0.054,0)
--(axis cs:0.055,0)
--(axis cs:0.056,0)
--(axis cs:0.057,0)
--(axis cs:0.058,0)
--(axis cs:0.059,0)
--(axis cs:0.06,0)
--(axis cs:0.061,0)
--(axis cs:0.062,0)
--(axis cs:0.063,0)
--(axis cs:0.064,0)
--(axis cs:0.065,0)
--(axis cs:0.066,0)
--(axis cs:0.067,0)
--(axis cs:0.068,0)
--(axis cs:0.069,0)
--(axis cs:0.07,0)
--(axis cs:0.071,0)
--(axis cs:0.072,0)
--(axis cs:0.073,0)
--(axis cs:0.074,0)
--(axis cs:0.075,0)
--(axis cs:0.076,0)
--(axis cs:0.077,0)
--(axis cs:0.078,0)
--(axis cs:0.079,0)
--(axis cs:0.08,0)
--(axis cs:0.081,0)
--(axis cs:0.082,0)
--(axis cs:0.083,0)
--(axis cs:0.084,0)
--(axis cs:0.085,0)
--(axis cs:0.086,0)
--(axis cs:0.087,0)
--(axis cs:0.088,0)
--(axis cs:0.089,0)
--(axis cs:0.09,0)
--(axis cs:0.091,0)
--(axis cs:0.092,0)
--(axis cs:0.093,0)
--(axis cs:0.094,0)
--(axis cs:0.095,0)
--(axis cs:0.096,0)
--(axis cs:0.097,0)
--(axis cs:0.098,0)
--(axis cs:0.099,0)
--(axis cs:0.1,0)
--(axis cs:0.101,0)
--(axis cs:0.102,0)
--(axis cs:0.103,0)
--(axis cs:0.104,0)
--(axis cs:0.105,0)
--(axis cs:0.106,0)
--(axis cs:0.107,0)
--(axis cs:0.108,0)
--(axis cs:0.109,0)
--(axis cs:0.11,0)
--(axis cs:0.111,0)
--(axis cs:0.112,0)
--(axis cs:0.113,0)
--(axis cs:0.114,0)
--(axis cs:0.115,0)
--(axis cs:0.116,0)
--(axis cs:0.117,0)
--(axis cs:0.118,0)
--(axis cs:0.119,0)
--(axis cs:0.12,0)
--(axis cs:0.121,0)
--(axis cs:0.122,0)
--(axis cs:0.123,0)
--(axis cs:0.124,0)
--(axis cs:0.125,0)
--(axis cs:0.126,0)
--(axis cs:0.127,0)
--(axis cs:0.128,0)
--(axis cs:0.129,0)
--(axis cs:0.13,0)
--(axis cs:0.131,0)
--(axis cs:0.132,0)
--(axis cs:0.133,0)
--(axis cs:0.134,0)
--(axis cs:0.135,0)
--(axis cs:0.136,0)
--(axis cs:0.137,0)
--(axis cs:0.138,0)
--(axis cs:0.139,0)
--(axis cs:0.14,0)
--(axis cs:0.141,0)
--(axis cs:0.142,0)
--(axis cs:0.143,0)
--(axis cs:0.144,0)
--(axis cs:0.145,0)
--(axis cs:0.146,0)
--(axis cs:0.147,0)
--(axis cs:0.148,0)
--(axis cs:0.149,0)
--(axis cs:0.15,0)
--(axis cs:0.151,0)
--(axis cs:0.152,0)
--(axis cs:0.153,0)
--(axis cs:0.154,0)
--(axis cs:0.155,0)
--(axis cs:0.156,0)
--(axis cs:0.157,0)
--(axis cs:0.158,0)
--(axis cs:0.159,0)
--(axis cs:0.16,0)
--(axis cs:0.161,0)
--(axis cs:0.162,0)
--(axis cs:0.163,0)
--(axis cs:0.164,0)
--(axis cs:0.165,0)
--(axis cs:0.166,0)
--(axis cs:0.167,0)
--(axis cs:0.168,0)
--(axis cs:0.169,0)
--(axis cs:0.17,0)
--(axis cs:0.171,0)
--(axis cs:0.172,0)
--(axis cs:0.173,0)
--(axis cs:0.174,0)
--(axis cs:0.175,0)
--(axis cs:0.176,0)
--(axis cs:0.177,0)
--(axis cs:0.178,0)
--(axis cs:0.179,0)
--(axis cs:0.18,0)
--(axis cs:0.181,0)
--(axis cs:0.182,0)
--(axis cs:0.183,0)
--(axis cs:0.184,0)
--(axis cs:0.185,0)
--(axis cs:0.186,0)
--(axis cs:0.187,0)
--(axis cs:0.188,0)
--(axis cs:0.189,0)
--(axis cs:0.19,0)
--(axis cs:0.191,0)
--(axis cs:0.192,0)
--(axis cs:0.193,0)
--(axis cs:0.194,0)
--(axis cs:0.195,0)
--(axis cs:0.196,0)
--(axis cs:0.197,0)
--(axis cs:0.198,0)
--(axis cs:0.199,0)
--(axis cs:0.2,0)
--(axis cs:0.201,0)
--(axis cs:0.202,0)
--(axis cs:0.203,0)
--(axis cs:0.204,0)
--(axis cs:0.205,0)
--(axis cs:0.206,0)
--(axis cs:0.207,0)
--(axis cs:0.208,0)
--(axis cs:0.209,0)
--(axis cs:0.21,0)
--(axis cs:0.211,0)
--(axis cs:0.212,0)
--(axis cs:0.213,0)
--(axis cs:0.214,0)
--(axis cs:0.215,0)
--(axis cs:0.216,0)
--(axis cs:0.217,0)
--(axis cs:0.218,0)
--(axis cs:0.219,0)
--(axis cs:0.22,0)
--(axis cs:0.221,0)
--(axis cs:0.222,0)
--(axis cs:0.223,0)
--(axis cs:0.224,0)
--(axis cs:0.225,0)
--(axis cs:0.226,0)
--(axis cs:0.227,0)
--(axis cs:0.228,0)
--(axis cs:0.229,0)
--(axis cs:0.23,0)
--(axis cs:0.231,0)
--(axis cs:0.232,0)
--(axis cs:0.233,0)
--(axis cs:0.234,0)
--(axis cs:0.235,0)
--(axis cs:0.236,0)
--(axis cs:0.237,0)
--(axis cs:0.238,0)
--(axis cs:0.239,0)
--(axis cs:0.24,0)
--(axis cs:0.241,0)
--(axis cs:0.242,0)
--(axis cs:0.243,0)
--(axis cs:0.244,0)
--(axis cs:0.245,0)
--(axis cs:0.246,0)
--(axis cs:0.247,0)
--(axis cs:0.248,0)
--(axis cs:0.249,0)
--(axis cs:0.25,0)
--(axis cs:0.251,0)
--(axis cs:0.252,0)
--(axis cs:0.253,0)
--(axis cs:0.254,0)
--(axis cs:0.255,0)
--(axis cs:0.256,0)
--(axis cs:0.257,0)
--(axis cs:0.258,0)
--(axis cs:0.259,0)
--(axis cs:0.26,0)
--(axis cs:0.261,0)
--(axis cs:0.262,0)
--(axis cs:0.263,0)
--(axis cs:0.264,0)
--(axis cs:0.265,0)
--(axis cs:0.266,0)
--(axis cs:0.267,0)
--(axis cs:0.268,0)
--(axis cs:0.269,0)
--(axis cs:0.27,0)
--(axis cs:0.271,0)
--(axis cs:0.272,0)
--(axis cs:0.273,0)
--(axis cs:0.274,0)
--(axis cs:0.275,0)
--(axis cs:0.276,0)
--(axis cs:0.277,0)
--(axis cs:0.278,0)
--(axis cs:0.279,0)
--(axis cs:0.28,0)
--(axis cs:0.281,0)
--(axis cs:0.282,0)
--(axis cs:0.283,0)
--(axis cs:0.284,0)
--(axis cs:0.285,0)
--(axis cs:0.286,0)
--(axis cs:0.287,0)
--(axis cs:0.288,0)
--(axis cs:0.289,0)
--(axis cs:0.29,0)
--(axis cs:0.291,0)
--(axis cs:0.292,0)
--(axis cs:0.293,0)
--(axis cs:0.294,0)
--(axis cs:0.295,0)
--(axis cs:0.296,0)
--(axis cs:0.297,0)
--(axis cs:0.298,0)
--(axis cs:0.299,0)
--(axis cs:0.3,0)
--(axis cs:0.301,0)
--(axis cs:0.302,0)
--(axis cs:0.303,0)
--(axis cs:0.304,0)
--(axis cs:0.305,0)
--(axis cs:0.306,0)
--(axis cs:0.307,0)
--(axis cs:0.308,0)
--(axis cs:0.309,0)
--(axis cs:0.31,0)
--(axis cs:0.311,0)
--(axis cs:0.312,0)
--(axis cs:0.313,0)
--(axis cs:0.314,0)
--(axis cs:0.315,0)
--(axis cs:0.316,0)
--(axis cs:0.317,0)
--(axis cs:0.318,0)
--(axis cs:0.319,0)
--(axis cs:0.32,0)
--(axis cs:0.321,0)
--(axis cs:0.322,0)
--(axis cs:0.323,0)
--(axis cs:0.324,0)
--(axis cs:0.325,0)
--(axis cs:0.326,0)
--(axis cs:0.327,0)
--(axis cs:0.328,0)
--(axis cs:0.329,0)
--(axis cs:0.33,0)
--(axis cs:0.331,0)
--(axis cs:0.332,0)
--(axis cs:0.333,0)
--(axis cs:0.334,0)
--(axis cs:0.335,0)
--(axis cs:0.336,0)
--(axis cs:0.337,0)
--(axis cs:0.338,0)
--(axis cs:0.339,0)
--(axis cs:0.34,0)
--(axis cs:0.341,0)
--(axis cs:0.342,0)
--(axis cs:0.343,0)
--(axis cs:0.344,0)
--(axis cs:0.345,0)
--(axis cs:0.346,0)
--(axis cs:0.347,0)
--(axis cs:0.348,0)
--(axis cs:0.349,0)
--(axis cs:0.35,0)
--(axis cs:0.351,0)
--(axis cs:0.352,0)
--(axis cs:0.353,0)
--(axis cs:0.354,0)
--(axis cs:0.355,0)
--(axis cs:0.356,0)
--(axis cs:0.357,0)
--(axis cs:0.358,0)
--(axis cs:0.359,0)
--(axis cs:0.36,0)
--(axis cs:0.361,0)
--(axis cs:0.362,0)
--(axis cs:0.363,0)
--(axis cs:0.364,0)
--(axis cs:0.365,0)
--(axis cs:0.366,0)
--(axis cs:0.367,0)
--(axis cs:0.368,0)
--(axis cs:0.369,0)
--(axis cs:0.37,0)
--(axis cs:0.371,0)
--(axis cs:0.372,0)
--(axis cs:0.373,0)
--(axis cs:0.374,0)
--(axis cs:0.375,0)
--(axis cs:0.376,0)
--(axis cs:0.377,0)
--(axis cs:0.378,0)
--(axis cs:0.379,0)
--(axis cs:0.38,0)
--(axis cs:0.381,0)
--(axis cs:0.382,0)
--(axis cs:0.383,0)
--(axis cs:0.384,0)
--(axis cs:0.385,0)
--(axis cs:0.386,0)
--(axis cs:0.387,0)
--(axis cs:0.388,0)
--(axis cs:0.389,0)
--(axis cs:0.39,0)
--(axis cs:0.391,0)
--(axis cs:0.392,0)
--(axis cs:0.393,0)
--(axis cs:0.394,0)
--(axis cs:0.395,0)
--(axis cs:0.396,0)
--(axis cs:0.397,0)
--(axis cs:0.398,0)
--(axis cs:0.399,0)
--(axis cs:0.4,0)
--(axis cs:0.401,0)
--(axis cs:0.402,0)
--(axis cs:0.403,0)
--(axis cs:0.404,0)
--(axis cs:0.405,0)
--(axis cs:0.406,0)
--(axis cs:0.407,0)
--(axis cs:0.408,0)
--(axis cs:0.409,0)
--(axis cs:0.41,0)
--(axis cs:0.411,0)
--(axis cs:0.412,0)
--(axis cs:0.413,0)
--(axis cs:0.414,0)
--(axis cs:0.415,0)
--(axis cs:0.416,0)
--(axis cs:0.417,0)
--(axis cs:0.418,0)
--(axis cs:0.419,0)
--(axis cs:0.42,0)
--(axis cs:0.421,0)
--(axis cs:0.422,0)
--(axis cs:0.423,0)
--(axis cs:0.424,0)
--(axis cs:0.425,0)
--(axis cs:0.426,0)
--(axis cs:0.427,0)
--(axis cs:0.428,0)
--(axis cs:0.429,0)
--(axis cs:0.43,0)
--(axis cs:0.431,0)
--(axis cs:0.432,0)
--(axis cs:0.433,0)
--(axis cs:0.434,0)
--(axis cs:0.435,0)
--(axis cs:0.436,0)
--(axis cs:0.437,0)
--(axis cs:0.438,0)
--(axis cs:0.439,0)
--(axis cs:0.44,0)
--(axis cs:0.441,0)
--(axis cs:0.442,0)
--(axis cs:0.443,0)
--(axis cs:0.444,0)
--(axis cs:0.445,0)
--(axis cs:0.446,0)
--(axis cs:0.447,0)
--(axis cs:0.448,0)
--(axis cs:0.449,0)
--(axis cs:0.45,0)
--(axis cs:0.451,0)
--(axis cs:0.452,0)
--(axis cs:0.453,0)
--(axis cs:0.454,0)
--(axis cs:0.455,0)
--(axis cs:0.456,0)
--(axis cs:0.457,0)
--(axis cs:0.458,0)
--(axis cs:0.459,0)
--(axis cs:0.46,0)
--(axis cs:0.461,0)
--(axis cs:0.462,0)
--(axis cs:0.463,0)
--(axis cs:0.464,0)
--(axis cs:0.465,0)
--(axis cs:0.466,0)
--(axis cs:0.467,0)
--(axis cs:0.468,0)
--(axis cs:0.469,0)
--(axis cs:0.47,0)
--(axis cs:0.471,0)
--(axis cs:0.472,0)
--(axis cs:0.473,0)
--(axis cs:0.474,0)
--(axis cs:0.475,0)
--(axis cs:0.476,0)
--(axis cs:0.477,0)
--(axis cs:0.478,0)
--(axis cs:0.479,0)
--(axis cs:0.48,0)
--(axis cs:0.481,0)
--(axis cs:0.482,0)
--(axis cs:0.483,0)
--(axis cs:0.484,0)
--(axis cs:0.485,0)
--(axis cs:0.486,0)
--(axis cs:0.487,0)
--(axis cs:0.488,0)
--(axis cs:0.489,0)
--(axis cs:0.49,0)
--(axis cs:0.491,0)
--(axis cs:0.492,0)
--(axis cs:0.493,0)
--(axis cs:0.494,0)
--(axis cs:0.495,0)
--(axis cs:0.496,0)
--(axis cs:0.497,0)
--(axis cs:0.498,0)
--(axis cs:0.499,0)
--(axis cs:0.5,0)
--(axis cs:0.501,0)
--(axis cs:0.502,0)
--(axis cs:0.503,0)
--(axis cs:0.504,0)
--(axis cs:0.505,0)
--(axis cs:0.506,0)
--(axis cs:0.507,0)
--(axis cs:0.508,0)
--(axis cs:0.509,0)
--(axis cs:0.51,0)
--(axis cs:0.511,0)
--(axis cs:0.512,0)
--(axis cs:0.513,0)
--(axis cs:0.514,0)
--(axis cs:0.515,0)
--(axis cs:0.516,0)
--(axis cs:0.517,0)
--(axis cs:0.518,0)
--(axis cs:0.519,0)
--(axis cs:0.52,0)
--(axis cs:0.521,0)
--(axis cs:0.522,0)
--(axis cs:0.523,0)
--(axis cs:0.524,0)
--(axis cs:0.525,0)
--(axis cs:0.526,0)
--(axis cs:0.527,0)
--(axis cs:0.528,0)
--(axis cs:0.529,0)
--(axis cs:0.53,0)
--(axis cs:0.531,0)
--(axis cs:0.532,0)
--(axis cs:0.533,0)
--(axis cs:0.534,0)
--(axis cs:0.535,0)
--(axis cs:0.536,0)
--(axis cs:0.537,0)
--(axis cs:0.538,0)
--(axis cs:0.539,0)
--(axis cs:0.54,0)
--(axis cs:0.541,0)
--(axis cs:0.542,0)
--(axis cs:0.543,0)
--(axis cs:0.544,0)
--(axis cs:0.545,0)
--(axis cs:0.546,0)
--(axis cs:0.547,0)
--(axis cs:0.548,0)
--(axis cs:0.549,0)
--(axis cs:0.55,0)
--(axis cs:0.551,0)
--(axis cs:0.552,0)
--(axis cs:0.553,0)
--(axis cs:0.554,0)
--(axis cs:0.555,0)
--(axis cs:0.556,0)
--(axis cs:0.557,0)
--(axis cs:0.558,0)
--(axis cs:0.559,0)
--(axis cs:0.56,0)
--(axis cs:0.561,0)
--(axis cs:0.562,0)
--(axis cs:0.563,0)
--(axis cs:0.564,0)
--(axis cs:0.565,0)
--(axis cs:0.566,0)
--(axis cs:0.567,0)
--(axis cs:0.568,0)
--(axis cs:0.569,0)
--(axis cs:0.57,0)
--(axis cs:0.571,0)
--(axis cs:0.572,0)
--(axis cs:0.573,0)
--(axis cs:0.574,0)
--(axis cs:0.575,0)
--(axis cs:0.576,0)
--(axis cs:0.577,0)
--(axis cs:0.578,0)
--(axis cs:0.579,0)
--(axis cs:0.58,0)
--(axis cs:0.581,0)
--(axis cs:0.582,0)
--(axis cs:0.583,0)
--(axis cs:0.584,0)
--(axis cs:0.585,0)
--(axis cs:0.586,0)
--(axis cs:0.587,0)
--(axis cs:0.588,0)
--(axis cs:0.589,0)
--(axis cs:0.59,0)
--(axis cs:0.591,0)
--(axis cs:0.592,0)
--(axis cs:0.593,0)
--(axis cs:0.594,0)
--(axis cs:0.595,0)
--(axis cs:0.596,0)
--(axis cs:0.597,0)
--(axis cs:0.598,0)
--(axis cs:0.599,0)
--(axis cs:0.6,0)
--(axis cs:0.601,0)
--(axis cs:0.602,0)
--(axis cs:0.603,0)
--(axis cs:0.604,0)
--(axis cs:0.605,0)
--(axis cs:0.606,0)
--(axis cs:0.607,0)
--(axis cs:0.608,0)
--(axis cs:0.609,0)
--(axis cs:0.61,0)
--(axis cs:0.611,0)
--(axis cs:0.612,0)
--(axis cs:0.613,0)
--(axis cs:0.614,0)
--(axis cs:0.615,0)
--(axis cs:0.616,0)
--(axis cs:0.617,0)
--(axis cs:0.618,0)
--(axis cs:0.619,0)
--(axis cs:0.62,0)
--(axis cs:0.621,0)
--(axis cs:0.622,0)
--(axis cs:0.623,0)
--(axis cs:0.624,0)
--(axis cs:0.625,0)
--(axis cs:0.626,0)
--(axis cs:0.627,0)
--(axis cs:0.628,0)
--(axis cs:0.629,0)
--(axis cs:0.63,0)
--(axis cs:0.631,0)
--(axis cs:0.632,0)
--(axis cs:0.633,0)
--(axis cs:0.634,0)
--(axis cs:0.635,0)
--(axis cs:0.636,0)
--(axis cs:0.637,0)
--(axis cs:0.638,0)
--(axis cs:0.639,0)
--(axis cs:0.64,0)
--(axis cs:0.641,0)
--(axis cs:0.642,0)
--(axis cs:0.643,0)
--(axis cs:0.644,0)
--(axis cs:0.645,0)
--(axis cs:0.646,0)
--(axis cs:0.647,0)
--(axis cs:0.648,0)
--(axis cs:0.649,0)
--(axis cs:0.65,0)
--(axis cs:0.651,0)
--(axis cs:0.652,0)
--(axis cs:0.653,0)
--(axis cs:0.654,0)
--(axis cs:0.655,0)
--(axis cs:0.656,0)
--(axis cs:0.657,0)
--(axis cs:0.658,0)
--(axis cs:0.659,0)
--(axis cs:0.66,0)
--(axis cs:0.661,0)
--(axis cs:0.662,0)
--(axis cs:0.663,0)
--(axis cs:0.664,0)
--(axis cs:0.665,0)
--(axis cs:0.666,0)
--(axis cs:0.667,0)
--(axis cs:0.668,0)
--(axis cs:0.669,0)
--(axis cs:0.67,0)
--(axis cs:0.671,0)
--(axis cs:0.672,0)
--(axis cs:0.673,0)
--(axis cs:0.674,0)
--(axis cs:0.675,0)
--(axis cs:0.676,0)
--(axis cs:0.677,0)
--(axis cs:0.678,0)
--(axis cs:0.679,0)
--(axis cs:0.68,0)
--(axis cs:0.681,0)
--(axis cs:0.682,0)
--(axis cs:0.683,0)
--(axis cs:0.684,0)
--(axis cs:0.685,0)
--(axis cs:0.686,0)
--(axis cs:0.687,0)
--(axis cs:0.688,0)
--(axis cs:0.689,0)
--(axis cs:0.69,0)
--(axis cs:0.691,0)
--(axis cs:0.692,0)
--(axis cs:0.693,0)
--(axis cs:0.694,0)
--(axis cs:0.695,0)
--(axis cs:0.696,0)
--(axis cs:0.697,0)
--(axis cs:0.698,0)
--(axis cs:0.699,0)
--(axis cs:0.7,0)
--(axis cs:0.701,0)
--(axis cs:0.702,0)
--(axis cs:0.703,0)
--(axis cs:0.704,0)
--(axis cs:0.705,0)
--(axis cs:0.706,0)
--(axis cs:0.707,0)
--(axis cs:0.708,0)
--(axis cs:0.709,0)
--(axis cs:0.71,0)
--(axis cs:0.711,0)
--(axis cs:0.712,0)
--(axis cs:0.713,0)
--(axis cs:0.714,0)
--(axis cs:0.715,0)
--(axis cs:0.716,0)
--(axis cs:0.717,0)
--(axis cs:0.718,0)
--(axis cs:0.719,0)
--(axis cs:0.72,0)
--(axis cs:0.721,0)
--(axis cs:0.722,0)
--(axis cs:0.723,0)
--(axis cs:0.724,0)
--(axis cs:0.725,0)
--(axis cs:0.726,0)
--(axis cs:0.727,0)
--(axis cs:0.728,0)
--(axis cs:0.729,0)
--(axis cs:0.73,0)
--(axis cs:0.731,0)
--(axis cs:0.732,0)
--(axis cs:0.733,0)
--(axis cs:0.734,0)
--(axis cs:0.735,0)
--(axis cs:0.736,0)
--(axis cs:0.737,0)
--(axis cs:0.738,0)
--(axis cs:0.739,0)
--(axis cs:0.74,0)
--(axis cs:0.741,0)
--(axis cs:0.742,0)
--(axis cs:0.743,0)
--(axis cs:0.744,0)
--(axis cs:0.745,0)
--(axis cs:0.746,0)
--(axis cs:0.747,0)
--(axis cs:0.748,0)
--(axis cs:0.749,0)
--(axis cs:0.75,0)
--(axis cs:0.751,0)
--(axis cs:0.752,0)
--(axis cs:0.753,0)
--(axis cs:0.754,0)
--(axis cs:0.755,0)
--(axis cs:0.756,0)
--(axis cs:0.757,0)
--(axis cs:0.758,0)
--(axis cs:0.759,0)
--(axis cs:0.76,0)
--(axis cs:0.761,0)
--(axis cs:0.762,0)
--(axis cs:0.763,0)
--(axis cs:0.764,0)
--(axis cs:0.765,0)
--(axis cs:0.766,0)
--(axis cs:0.767,0)
--(axis cs:0.768,0)
--(axis cs:0.769,0)
--(axis cs:0.77,0)
--(axis cs:0.771,0)
--(axis cs:0.772,0)
--(axis cs:0.773,0)
--(axis cs:0.774,0)
--(axis cs:0.775,0)
--(axis cs:0.776,0)
--(axis cs:0.777,0)
--(axis cs:0.778,0)
--(axis cs:0.779,0)
--(axis cs:0.78,0)
--(axis cs:0.781,0)
--(axis cs:0.782,0)
--(axis cs:0.783,0)
--(axis cs:0.784,0)
--(axis cs:0.785,0)
--(axis cs:0.786,0)
--(axis cs:0.787,0)
--(axis cs:0.788,0)
--(axis cs:0.789,0)
--(axis cs:0.79,0)
--(axis cs:0.791,0)
--(axis cs:0.792,0)
--(axis cs:0.793,0)
--(axis cs:0.794,0)
--(axis cs:0.795,0)
--(axis cs:0.796,0)
--(axis cs:0.797,0)
--(axis cs:0.798,0)
--(axis cs:0.799,0)
--(axis cs:0.8,0)
--(axis cs:0.801,0)
--(axis cs:0.802,0)
--(axis cs:0.803,0)
--(axis cs:0.804,0)
--(axis cs:0.805,0)
--(axis cs:0.806,0)
--(axis cs:0.807,0)
--(axis cs:0.808,0)
--(axis cs:0.809,0)
--(axis cs:0.81,0)
--(axis cs:0.811,0)
--(axis cs:0.812,0)
--(axis cs:0.813,0)
--(axis cs:0.814,0)
--(axis cs:0.815,0)
--(axis cs:0.816,0)
--(axis cs:0.817,0)
--(axis cs:0.818,0)
--(axis cs:0.819,0)
--(axis cs:0.82,0)
--(axis cs:0.821,0)
--(axis cs:0.822,0)
--(axis cs:0.823,0)
--(axis cs:0.824,0)
--(axis cs:0.825,0)
--(axis cs:0.826,0)
--(axis cs:0.827,0)
--(axis cs:0.828,0)
--(axis cs:0.829,0)
--(axis cs:0.83,0)
--(axis cs:0.831,0)
--(axis cs:0.832,0)
--(axis cs:0.833,0)
--(axis cs:0.834,0)
--(axis cs:0.835,0)
--(axis cs:0.836,0)
--(axis cs:0.837,0)
--(axis cs:0.838,0)
--(axis cs:0.839,0)
--(axis cs:0.84,0)
--(axis cs:0.841,0)
--(axis cs:0.842,0)
--(axis cs:0.843,0)
--(axis cs:0.844,0)
--(axis cs:0.845,0)
--(axis cs:0.846,0)
--(axis cs:0.847,0)
--(axis cs:0.848,0)
--(axis cs:0.849,0)
--(axis cs:0.85,0)
--(axis cs:0.851,0)
--(axis cs:0.852,0)
--(axis cs:0.853,0)
--(axis cs:0.854,0)
--(axis cs:0.855,0)
--(axis cs:0.856,0)
--(axis cs:0.857,0)
--(axis cs:0.858,0)
--(axis cs:0.859,0)
--(axis cs:0.86,0)
--(axis cs:0.861,0)
--(axis cs:0.862,0)
--(axis cs:0.863,0)
--(axis cs:0.864,0)
--(axis cs:0.865,0)
--(axis cs:0.866,0)
--(axis cs:0.867,0)
--(axis cs:0.868,0)
--(axis cs:0.869,0)
--(axis cs:0.87,0)
--(axis cs:0.871,0)
--(axis cs:0.872,0)
--(axis cs:0.873,0)
--(axis cs:0.874,0)
--(axis cs:0.875,0)
--(axis cs:0.876,0)
--(axis cs:0.877,0)
--(axis cs:0.878,0)
--(axis cs:0.879,0)
--(axis cs:0.88,0)
--(axis cs:0.881,0)
--(axis cs:0.882,0)
--(axis cs:0.883,0)
--(axis cs:0.884,0)
--(axis cs:0.885,0)
--(axis cs:0.886,0)
--cycle;
\path [draw=none, fill=blue, fill opacity=0.5]
(axis cs:0,0)
--(axis cs:0.001,322.58271626249)
--(axis cs:0.002,4.5485439001751e-06)
--(axis cs:0.003,1.08331330228178e-16)
--(axis cs:0.004,1.85338944612998e-28)
--(axis cs:0.005,7.37735891461139e-41)
--(axis cs:0.006,1.15314008937915e-53)
--(axis cs:0.007,9.35747394104202e-67)
--(axis cs:0.008,4.65685972914961e-80)
--(axis cs:0.009,1.58252448966646e-93)
--(axis cs:0.01,3.95169462233851e-107)
--(axis cs:0.011,7.64016500319958e-121)
--(axis cs:0.012,1.18869226225084e-134)
--(axis cs:0.013,1.53251456003575e-148)
--(axis cs:0.014,1.67490368336033e-162)
--(axis cs:0.015,1.57998245946121e-176)
--(axis cs:0.016,1.30525638137975e-190)
--(axis cs:0.017,9.55588966251468e-205)
--(axis cs:0.018,6.26093883958636e-219)
--(axis cs:0.019,3.70139413359828e-233)
--(axis cs:0.02,1.98819757673142e-247)
--(axis cs:0.021,9.76084292719275e-262)
--(axis cs:0.022,4.40201301358189e-276)
--(axis cs:0.023,1.83170729292297e-290)
--(axis cs:0.024,7.05925369579821e-305)
--(axis cs:0.025,2.52818331633424e-319)
--(axis cs:0.026,0)
--(axis cs:0.027,0)
--(axis cs:0.028,0)
--(axis cs:0.029,0)
--(axis cs:0.03,0)
--(axis cs:0.031,0)
--(axis cs:0.032,0)
--(axis cs:0.033,0)
--(axis cs:0.034,0)
--(axis cs:0.035,0)
--(axis cs:0.036,0)
--(axis cs:0.037,0)
--(axis cs:0.038,0)
--(axis cs:0.039,0)
--(axis cs:0.04,0)
--(axis cs:0.041,0)
--(axis cs:0.042,0)
--(axis cs:0.043,0)
--(axis cs:0.044,0)
--(axis cs:0.045,0)
--(axis cs:0.046,0)
--(axis cs:0.047,0)
--(axis cs:0.048,0)
--(axis cs:0.049,0)
--(axis cs:0.05,0)
--(axis cs:0.051,0)
--(axis cs:0.052,0)
--(axis cs:0.053,0)
--(axis cs:0.054,0)
--(axis cs:0.055,0)
--(axis cs:0.056,0)
--(axis cs:0.057,0)
--(axis cs:0.058,0)
--(axis cs:0.059,0)
--(axis cs:0.06,0)
--(axis cs:0.061,0)
--(axis cs:0.062,0)
--(axis cs:0.063,0)
--(axis cs:0.064,0)
--(axis cs:0.065,0)
--(axis cs:0.066,0)
--(axis cs:0.067,0)
--(axis cs:0.068,0)
--(axis cs:0.069,0)
--(axis cs:0.07,0)
--(axis cs:0.071,0)
--(axis cs:0.072,0)
--(axis cs:0.073,0)
--(axis cs:0.074,0)
--(axis cs:0.075,0)
--(axis cs:0.076,0)
--(axis cs:0.077,0)
--(axis cs:0.078,0)
--(axis cs:0.079,0)
--(axis cs:0.08,0)
--(axis cs:0.081,0)
--(axis cs:0.082,0)
--(axis cs:0.083,0)
--(axis cs:0.084,0)
--(axis cs:0.085,0)
--(axis cs:0.086,0)
--(axis cs:0.087,0)
--(axis cs:0.088,0)
--(axis cs:0.089,0)
--(axis cs:0.09,0)
--(axis cs:0.091,0)
--(axis cs:0.092,0)
--(axis cs:0.093,0)
--(axis cs:0.094,0)
--(axis cs:0.095,0)
--(axis cs:0.096,0)
--(axis cs:0.097,0)
--(axis cs:0.098,0)
--(axis cs:0.099,0)
--(axis cs:0.1,0)
--(axis cs:0.101,0)
--(axis cs:0.102,0)
--(axis cs:0.103,0)
--(axis cs:0.104,0)
--(axis cs:0.105,0)
--(axis cs:0.106,0)
--(axis cs:0.107,0)
--(axis cs:0.108,0)
--(axis cs:0.109,0)
--(axis cs:0.11,0)
--(axis cs:0.111,0)
--(axis cs:0.112,0)
--(axis cs:0.113,0)
--(axis cs:0.114,0)
--(axis cs:0.115,0)
--(axis cs:0.116,0)
--(axis cs:0.117,0)
--(axis cs:0.118,0)
--(axis cs:0.119,0)
--(axis cs:0.12,0)
--(axis cs:0.121,0)
--(axis cs:0.122,0)
--(axis cs:0.123,0)
--(axis cs:0.124,0)
--(axis cs:0.125,0)
--(axis cs:0.126,0)
--(axis cs:0.127,0)
--(axis cs:0.128,0)
--(axis cs:0.129,0)
--(axis cs:0.13,0)
--(axis cs:0.131,0)
--(axis cs:0.132,0)
--(axis cs:0.133,0)
--(axis cs:0.134,0)
--(axis cs:0.135,0)
--(axis cs:0.136,0)
--(axis cs:0.137,0)
--(axis cs:0.138,0)
--(axis cs:0.139,0)
--(axis cs:0.14,0)
--(axis cs:0.141,0)
--(axis cs:0.142,0)
--(axis cs:0.143,0)
--(axis cs:0.144,0)
--(axis cs:0.145,0)
--(axis cs:0.146,0)
--(axis cs:0.147,0)
--(axis cs:0.148,0)
--(axis cs:0.149,0)
--(axis cs:0.15,0)
--(axis cs:0.151,0)
--(axis cs:0.152,0)
--(axis cs:0.153,0)
--(axis cs:0.154,0)
--(axis cs:0.155,0)
--(axis cs:0.156,0)
--(axis cs:0.157,0)
--(axis cs:0.158,0)
--(axis cs:0.159,0)
--(axis cs:0.16,0)
--(axis cs:0.161,0)
--(axis cs:0.162,0)
--(axis cs:0.163,0)
--(axis cs:0.164,0)
--(axis cs:0.165,0)
--(axis cs:0.166,0)
--(axis cs:0.167,0)
--(axis cs:0.168,0)
--(axis cs:0.169,0)
--(axis cs:0.17,0)
--(axis cs:0.171,0)
--(axis cs:0.172,0)
--(axis cs:0.173,0)
--(axis cs:0.174,0)
--(axis cs:0.175,0)
--(axis cs:0.176,0)
--(axis cs:0.177,0)
--(axis cs:0.178,0)
--(axis cs:0.179,0)
--(axis cs:0.18,0)
--(axis cs:0.181,0)
--(axis cs:0.182,0)
--(axis cs:0.183,0)
--(axis cs:0.184,0)
--(axis cs:0.185,0)
--(axis cs:0.186,0)
--(axis cs:0.187,0)
--(axis cs:0.188,0)
--(axis cs:0.189,0)
--(axis cs:0.19,0)
--(axis cs:0.191,0)
--(axis cs:0.192,0)
--(axis cs:0.193,0)
--(axis cs:0.194,0)
--(axis cs:0.195,0)
--(axis cs:0.196,0)
--(axis cs:0.197,0)
--(axis cs:0.198,0)
--(axis cs:0.199,0)
--(axis cs:0.2,0)
--(axis cs:0.201,0)
--(axis cs:0.202,0)
--(axis cs:0.203,0)
--(axis cs:0.204,0)
--(axis cs:0.205,0)
--(axis cs:0.206,0)
--(axis cs:0.207,0)
--(axis cs:0.208,0)
--(axis cs:0.209,0)
--(axis cs:0.21,0)
--(axis cs:0.211,0)
--(axis cs:0.212,0)
--(axis cs:0.213,0)
--(axis cs:0.214,0)
--(axis cs:0.215,0)
--(axis cs:0.216,0)
--(axis cs:0.217,0)
--(axis cs:0.218,0)
--(axis cs:0.219,0)
--(axis cs:0.22,0)
--(axis cs:0.221,0)
--(axis cs:0.222,0)
--(axis cs:0.223,0)
--(axis cs:0.224,0)
--(axis cs:0.225,0)
--(axis cs:0.226,0)
--(axis cs:0.227,0)
--(axis cs:0.228,0)
--(axis cs:0.229,0)
--(axis cs:0.23,0)
--(axis cs:0.231,0)
--(axis cs:0.232,0)
--(axis cs:0.233,0)
--(axis cs:0.234,0)
--(axis cs:0.235,0)
--(axis cs:0.236,0)
--(axis cs:0.237,0)
--(axis cs:0.238,0)
--(axis cs:0.239,0)
--(axis cs:0.24,0)
--(axis cs:0.241,0)
--(axis cs:0.242,0)
--(axis cs:0.243,0)
--(axis cs:0.244,0)
--(axis cs:0.245,0)
--(axis cs:0.246,0)
--(axis cs:0.247,0)
--(axis cs:0.248,0)
--(axis cs:0.249,0)
--(axis cs:0.25,0)
--(axis cs:0.251,0)
--(axis cs:0.252,0)
--(axis cs:0.253,0)
--(axis cs:0.254,0)
--(axis cs:0.255,0)
--(axis cs:0.256,0)
--(axis cs:0.257,0)
--(axis cs:0.258,0)
--(axis cs:0.259,0)
--(axis cs:0.26,0)
--(axis cs:0.261,0)
--(axis cs:0.262,0)
--(axis cs:0.263,0)
--(axis cs:0.264,0)
--(axis cs:0.265,0)
--(axis cs:0.266,0)
--(axis cs:0.267,0)
--(axis cs:0.268,0)
--(axis cs:0.269,0)
--(axis cs:0.27,0)
--(axis cs:0.271,0)
--(axis cs:0.272,0)
--(axis cs:0.273,0)
--(axis cs:0.274,0)
--(axis cs:0.275,0)
--(axis cs:0.276,0)
--(axis cs:0.277,0)
--(axis cs:0.278,0)
--(axis cs:0.279,0)
--(axis cs:0.28,0)
--(axis cs:0.281,0)
--(axis cs:0.282,0)
--(axis cs:0.283,0)
--(axis cs:0.284,0)
--(axis cs:0.285,0)
--(axis cs:0.286,0)
--(axis cs:0.287,0)
--(axis cs:0.288,0)
--(axis cs:0.289,0)
--(axis cs:0.29,0)
--(axis cs:0.291,0)
--(axis cs:0.292,0)
--(axis cs:0.293,0)
--(axis cs:0.294,0)
--(axis cs:0.295,0)
--(axis cs:0.296,0)
--(axis cs:0.297,0)
--(axis cs:0.298,0)
--(axis cs:0.299,0)
--(axis cs:0.3,0)
--(axis cs:0.301,0)
--(axis cs:0.302,0)
--(axis cs:0.303,0)
--(axis cs:0.304,0)
--(axis cs:0.305,0)
--(axis cs:0.306,0)
--(axis cs:0.307,0)
--(axis cs:0.308,0)
--(axis cs:0.309,0)
--(axis cs:0.31,0)
--(axis cs:0.311,0)
--(axis cs:0.312,0)
--(axis cs:0.313,0)
--(axis cs:0.314,0)
--(axis cs:0.315,0)
--(axis cs:0.316,0)
--(axis cs:0.317,0)
--(axis cs:0.318,0)
--(axis cs:0.319,0)
--(axis cs:0.32,0)
--(axis cs:0.321,0)
--(axis cs:0.322,0)
--(axis cs:0.323,0)
--(axis cs:0.324,0)
--(axis cs:0.325,0)
--(axis cs:0.326,0)
--(axis cs:0.327,0)
--(axis cs:0.328,0)
--(axis cs:0.329,0)
--(axis cs:0.33,0)
--(axis cs:0.331,0)
--(axis cs:0.332,0)
--(axis cs:0.333,0)
--(axis cs:0.334,0)
--(axis cs:0.335,0)
--(axis cs:0.336,0)
--(axis cs:0.337,0)
--(axis cs:0.338,0)
--(axis cs:0.339,0)
--(axis cs:0.34,0)
--(axis cs:0.341,0)
--(axis cs:0.342,0)
--(axis cs:0.343,0)
--(axis cs:0.344,0)
--(axis cs:0.345,0)
--(axis cs:0.346,0)
--(axis cs:0.347,0)
--(axis cs:0.348,0)
--(axis cs:0.349,0)
--(axis cs:0.35,0)
--(axis cs:0.351,0)
--(axis cs:0.352,0)
--(axis cs:0.353,0)
--(axis cs:0.354,0)
--(axis cs:0.355,0)
--(axis cs:0.356,0)
--(axis cs:0.357,0)
--(axis cs:0.358,0)
--(axis cs:0.359,0)
--(axis cs:0.36,0)
--(axis cs:0.361,0)
--(axis cs:0.362,0)
--(axis cs:0.363,0)
--(axis cs:0.364,0)
--(axis cs:0.365,0)
--(axis cs:0.366,0)
--(axis cs:0.367,0)
--(axis cs:0.368,0)
--(axis cs:0.369,0)
--(axis cs:0.37,0)
--(axis cs:0.371,0)
--(axis cs:0.372,0)
--(axis cs:0.373,0)
--(axis cs:0.374,0)
--(axis cs:0.375,0)
--(axis cs:0.376,0)
--(axis cs:0.377,0)
--(axis cs:0.378,0)
--(axis cs:0.379,0)
--(axis cs:0.38,0)
--(axis cs:0.381,0)
--(axis cs:0.382,0)
--(axis cs:0.383,0)
--(axis cs:0.384,0)
--(axis cs:0.385,0)
--(axis cs:0.386,0)
--(axis cs:0.387,0)
--(axis cs:0.388,0)
--(axis cs:0.389,0)
--(axis cs:0.39,0)
--(axis cs:0.391,0)
--(axis cs:0.392,0)
--(axis cs:0.393,0)
--(axis cs:0.394,0)
--(axis cs:0.395,0)
--(axis cs:0.396,0)
--(axis cs:0.397,0)
--(axis cs:0.398,0)
--(axis cs:0.399,0)
--(axis cs:0.4,0)
--(axis cs:0.401,0)
--(axis cs:0.402,0)
--(axis cs:0.403,0)
--(axis cs:0.404,0)
--(axis cs:0.405,0)
--(axis cs:0.406,0)
--(axis cs:0.407,0)
--(axis cs:0.408,0)
--(axis cs:0.409,0)
--(axis cs:0.41,0)
--(axis cs:0.411,0)
--(axis cs:0.412,0)
--(axis cs:0.413,0)
--(axis cs:0.414,0)
--(axis cs:0.415,0)
--(axis cs:0.416,0)
--(axis cs:0.417,0)
--(axis cs:0.418,0)
--(axis cs:0.419,0)
--(axis cs:0.42,0)
--(axis cs:0.421,0)
--(axis cs:0.422,0)
--(axis cs:0.423,0)
--(axis cs:0.424,0)
--(axis cs:0.425,0)
--(axis cs:0.426,0)
--(axis cs:0.427,0)
--(axis cs:0.428,0)
--(axis cs:0.429,0)
--(axis cs:0.43,0)
--(axis cs:0.431,0)
--(axis cs:0.432,0)
--(axis cs:0.433,0)
--(axis cs:0.434,0)
--(axis cs:0.435,0)
--(axis cs:0.436,0)
--(axis cs:0.437,0)
--(axis cs:0.438,0)
--(axis cs:0.439,0)
--(axis cs:0.44,0)
--(axis cs:0.441,0)
--(axis cs:0.442,0)
--(axis cs:0.443,0)
--(axis cs:0.444,0)
--(axis cs:0.445,0)
--(axis cs:0.446,0)
--(axis cs:0.447,0)
--(axis cs:0.448,0)
--(axis cs:0.449,0)
--(axis cs:0.45,0)
--(axis cs:0.451,0)
--(axis cs:0.452,0)
--(axis cs:0.453,0)
--(axis cs:0.454,0)
--(axis cs:0.455,0)
--(axis cs:0.456,0)
--(axis cs:0.457,0)
--(axis cs:0.458,0)
--(axis cs:0.459,0)
--(axis cs:0.46,0)
--(axis cs:0.461,0)
--(axis cs:0.462,0)
--(axis cs:0.463,0)
--(axis cs:0.464,0)
--(axis cs:0.465,0)
--(axis cs:0.466,0)
--(axis cs:0.467,0)
--(axis cs:0.468,0)
--(axis cs:0.469,0)
--(axis cs:0.47,0)
--(axis cs:0.471,0)
--(axis cs:0.472,0)
--(axis cs:0.473,0)
--(axis cs:0.474,0)
--(axis cs:0.475,0)
--(axis cs:0.476,0)
--(axis cs:0.477,0)
--(axis cs:0.478,0)
--(axis cs:0.479,0)
--(axis cs:0.48,0)
--(axis cs:0.481,0)
--(axis cs:0.482,0)
--(axis cs:0.483,0)
--(axis cs:0.484,0)
--(axis cs:0.485,0)
--(axis cs:0.486,0)
--(axis cs:0.487,0)
--(axis cs:0.488,0)
--(axis cs:0.489,0)
--(axis cs:0.49,0)
--(axis cs:0.491,0)
--(axis cs:0.492,0)
--(axis cs:0.493,0)
--(axis cs:0.494,0)
--(axis cs:0.495,0)
--(axis cs:0.496,0)
--(axis cs:0.497,0)
--(axis cs:0.498,0)
--(axis cs:0.499,0)
--(axis cs:0.5,0)
--(axis cs:0.501,0)
--(axis cs:0.502,0)
--(axis cs:0.503,0)
--(axis cs:0.504,0)
--(axis cs:0.505,0)
--(axis cs:0.506,0)
--(axis cs:0.507,0)
--(axis cs:0.508,0)
--(axis cs:0.509,0)
--(axis cs:0.51,0)
--(axis cs:0.511,0)
--(axis cs:0.512,0)
--(axis cs:0.513,0)
--(axis cs:0.514,0)
--(axis cs:0.515,0)
--(axis cs:0.516,0)
--(axis cs:0.517,0)
--(axis cs:0.518,0)
--(axis cs:0.519,0)
--(axis cs:0.52,0)
--(axis cs:0.521,0)
--(axis cs:0.522,0)
--(axis cs:0.523,0)
--(axis cs:0.524,0)
--(axis cs:0.525,0)
--(axis cs:0.526,0)
--(axis cs:0.527,0)
--(axis cs:0.528,0)
--(axis cs:0.529,0)
--(axis cs:0.53,0)
--(axis cs:0.531,0)
--(axis cs:0.532,0)
--(axis cs:0.533,0)
--(axis cs:0.534,0)
--(axis cs:0.535,0)
--(axis cs:0.536,0)
--(axis cs:0.537,0)
--(axis cs:0.538,0)
--(axis cs:0.539,0)
--(axis cs:0.54,0)
--(axis cs:0.541,0)
--(axis cs:0.542,0)
--(axis cs:0.543,0)
--(axis cs:0.544,0)
--(axis cs:0.545,0)
--(axis cs:0.546,0)
--(axis cs:0.547,0)
--(axis cs:0.548,0)
--(axis cs:0.549,0)
--(axis cs:0.55,0)
--(axis cs:0.551,0)
--(axis cs:0.552,0)
--(axis cs:0.553,0)
--(axis cs:0.554,0)
--(axis cs:0.555,0)
--(axis cs:0.556,0)
--(axis cs:0.557,0)
--(axis cs:0.558,0)
--(axis cs:0.559,0)
--(axis cs:0.56,0)
--(axis cs:0.561,0)
--(axis cs:0.562,0)
--(axis cs:0.563,0)
--(axis cs:0.564,0)
--(axis cs:0.565,0)
--(axis cs:0.566,0)
--(axis cs:0.567,0)
--(axis cs:0.568,0)
--(axis cs:0.569,0)
--(axis cs:0.57,0)
--(axis cs:0.571,0)
--(axis cs:0.572,0)
--(axis cs:0.573,0)
--(axis cs:0.574,0)
--(axis cs:0.575,0)
--(axis cs:0.576,0)
--(axis cs:0.577,0)
--(axis cs:0.578,0)
--(axis cs:0.579,0)
--(axis cs:0.58,0)
--(axis cs:0.581,0)
--(axis cs:0.582,0)
--(axis cs:0.583,0)
--(axis cs:0.584,0)
--(axis cs:0.585,0)
--(axis cs:0.586,0)
--(axis cs:0.587,0)
--(axis cs:0.588,0)
--(axis cs:0.589,0)
--(axis cs:0.59,0)
--(axis cs:0.591,0)
--(axis cs:0.592,0)
--(axis cs:0.593,0)
--(axis cs:0.594,0)
--(axis cs:0.595,0)
--(axis cs:0.596,0)
--(axis cs:0.597,0)
--(axis cs:0.598,0)
--(axis cs:0.599,0)
--(axis cs:0.6,0)
--(axis cs:0.601,0)
--(axis cs:0.602,0)
--(axis cs:0.603,0)
--(axis cs:0.604,0)
--(axis cs:0.605,0)
--(axis cs:0.606,0)
--(axis cs:0.607,0)
--(axis cs:0.608,0)
--(axis cs:0.609,0)
--(axis cs:0.61,0)
--(axis cs:0.611,0)
--(axis cs:0.612,0)
--(axis cs:0.613,0)
--(axis cs:0.614,0)
--(axis cs:0.615,0)
--(axis cs:0.616,0)
--(axis cs:0.617,0)
--(axis cs:0.618,0)
--(axis cs:0.619,0)
--(axis cs:0.62,0)
--(axis cs:0.621,0)
--(axis cs:0.622,0)
--(axis cs:0.623,0)
--(axis cs:0.624,0)
--(axis cs:0.625,0)
--(axis cs:0.626,0)
--(axis cs:0.627,0)
--(axis cs:0.628,0)
--(axis cs:0.629,0)
--(axis cs:0.63,0)
--(axis cs:0.631,0)
--(axis cs:0.632,0)
--(axis cs:0.633,0)
--(axis cs:0.634,0)
--(axis cs:0.635,0)
--(axis cs:0.636,0)
--(axis cs:0.637,0)
--(axis cs:0.638,0)
--(axis cs:0.639,0)
--(axis cs:0.64,0)
--(axis cs:0.641,0)
--(axis cs:0.642,0)
--(axis cs:0.643,0)
--(axis cs:0.644,0)
--(axis cs:0.645,0)
--(axis cs:0.646,0)
--(axis cs:0.647,0)
--(axis cs:0.648,0)
--(axis cs:0.649,0)
--(axis cs:0.65,0)
--(axis cs:0.651,0)
--(axis cs:0.652,0)
--(axis cs:0.653,0)
--(axis cs:0.654,0)
--(axis cs:0.655,0)
--(axis cs:0.656,0)
--(axis cs:0.657,0)
--(axis cs:0.658,0)
--(axis cs:0.659,0)
--(axis cs:0.66,0)
--(axis cs:0.661,0)
--(axis cs:0.662,0)
--(axis cs:0.663,0)
--(axis cs:0.664,0)
--(axis cs:0.665,0)
--(axis cs:0.666,0)
--(axis cs:0.667,0)
--(axis cs:0.668,0)
--(axis cs:0.669,0)
--(axis cs:0.67,0)
--(axis cs:0.671,0)
--(axis cs:0.672,0)
--(axis cs:0.673,0)
--(axis cs:0.674,0)
--(axis cs:0.675,0)
--(axis cs:0.676,0)
--(axis cs:0.677,0)
--(axis cs:0.678,0)
--(axis cs:0.679,0)
--(axis cs:0.68,0)
--(axis cs:0.681,0)
--(axis cs:0.682,0)
--(axis cs:0.683,0)
--(axis cs:0.684,0)
--(axis cs:0.685,0)
--(axis cs:0.686,0)
--(axis cs:0.687,0)
--(axis cs:0.688,0)
--(axis cs:0.689,0)
--(axis cs:0.69,0)
--(axis cs:0.691,0)
--(axis cs:0.692,0)
--(axis cs:0.693,0)
--(axis cs:0.694,0)
--(axis cs:0.695,0)
--(axis cs:0.696,0)
--(axis cs:0.697,0)
--(axis cs:0.698,0)
--(axis cs:0.699,0)
--(axis cs:0.7,0)
--(axis cs:0.701,0)
--(axis cs:0.702,0)
--(axis cs:0.703,0)
--(axis cs:0.704,0)
--(axis cs:0.705,0)
--(axis cs:0.706,0)
--(axis cs:0.707,0)
--(axis cs:0.708,0)
--(axis cs:0.709,0)
--(axis cs:0.71,0)
--(axis cs:0.711,0)
--(axis cs:0.712,0)
--(axis cs:0.713,0)
--(axis cs:0.714,0)
--(axis cs:0.715,0)
--(axis cs:0.716,0)
--(axis cs:0.717,0)
--(axis cs:0.718,0)
--(axis cs:0.719,0)
--(axis cs:0.72,0)
--(axis cs:0.721,0)
--(axis cs:0.722,0)
--(axis cs:0.723,0)
--(axis cs:0.724,0)
--(axis cs:0.725,0)
--(axis cs:0.726,0)
--(axis cs:0.727,0)
--(axis cs:0.728,0)
--(axis cs:0.729,0)
--(axis cs:0.73,0)
--(axis cs:0.731,0)
--(axis cs:0.732,0)
--(axis cs:0.733,0)
--(axis cs:0.734,0)
--(axis cs:0.735,0)
--(axis cs:0.736,0)
--(axis cs:0.737,0)
--(axis cs:0.738,0)
--(axis cs:0.739,0)
--(axis cs:0.74,0)
--(axis cs:0.741,0)
--(axis cs:0.742,0)
--(axis cs:0.743,0)
--(axis cs:0.744,0)
--(axis cs:0.745,0)
--(axis cs:0.746,0)
--(axis cs:0.747,0)
--(axis cs:0.748,0)
--(axis cs:0.749,0)
--(axis cs:0.75,0)
--(axis cs:0.751,0)
--(axis cs:0.752,0)
--(axis cs:0.753,0)
--(axis cs:0.754,0)
--(axis cs:0.755,0)
--(axis cs:0.756,0)
--(axis cs:0.757,0)
--(axis cs:0.758,0)
--(axis cs:0.759,0)
--(axis cs:0.76,0)
--(axis cs:0.761,0)
--(axis cs:0.762,0)
--(axis cs:0.763,0)
--(axis cs:0.764,0)
--(axis cs:0.765,0)
--(axis cs:0.766,0)
--(axis cs:0.767,0)
--(axis cs:0.768,0)
--(axis cs:0.769,0)
--(axis cs:0.77,0)
--(axis cs:0.771,0)
--(axis cs:0.772,0)
--(axis cs:0.773,0)
--(axis cs:0.774,0)
--(axis cs:0.775,0)
--(axis cs:0.776,0)
--(axis cs:0.777,0)
--(axis cs:0.778,0)
--(axis cs:0.779,0)
--(axis cs:0.78,0)
--(axis cs:0.781,0)
--(axis cs:0.782,0)
--(axis cs:0.783,0)
--(axis cs:0.784,0)
--(axis cs:0.785,0)
--(axis cs:0.786,0)
--(axis cs:0.787,0)
--(axis cs:0.788,0)
--(axis cs:0.789,0)
--(axis cs:0.79,0)
--(axis cs:0.791,0)
--(axis cs:0.792,0)
--(axis cs:0.793,0)
--(axis cs:0.794,0)
--(axis cs:0.795,0)
--(axis cs:0.796,0)
--(axis cs:0.797,0)
--(axis cs:0.798,0)
--(axis cs:0.799,0)
--(axis cs:0.8,0)
--(axis cs:0.801,0)
--(axis cs:0.802,0)
--(axis cs:0.803,0)
--(axis cs:0.804,0)
--(axis cs:0.805,0)
--(axis cs:0.806,0)
--(axis cs:0.807,0)
--(axis cs:0.808,0)
--(axis cs:0.809,0)
--(axis cs:0.81,0)
--(axis cs:0.811,0)
--(axis cs:0.812,0)
--(axis cs:0.813,0)
--(axis cs:0.814,0)
--(axis cs:0.815,0)
--(axis cs:0.816,0)
--(axis cs:0.817,0)
--(axis cs:0.818,0)
--(axis cs:0.819,0)
--(axis cs:0.82,0)
--(axis cs:0.821,0)
--(axis cs:0.822,0)
--(axis cs:0.823,0)
--(axis cs:0.824,0)
--(axis cs:0.825,0)
--(axis cs:0.826,0)
--(axis cs:0.827,0)
--(axis cs:0.828,0)
--(axis cs:0.829,0)
--(axis cs:0.83,0)
--(axis cs:0.831,0)
--(axis cs:0.832,0)
--(axis cs:0.833,0)
--(axis cs:0.834,0)
--(axis cs:0.835,0)
--(axis cs:0.836,0)
--(axis cs:0.837,0)
--(axis cs:0.838,0)
--(axis cs:0.839,0)
--(axis cs:0.84,0)
--(axis cs:0.841,0)
--(axis cs:0.842,0)
--(axis cs:0.843,0)
--(axis cs:0.844,0)
--(axis cs:0.845,0)
--(axis cs:0.846,0)
--(axis cs:0.847,0)
--(axis cs:0.848,0)
--(axis cs:0.849,0)
--(axis cs:0.85,0)
--(axis cs:0.851,0)
--(axis cs:0.852,0)
--(axis cs:0.853,0)
--(axis cs:0.854,0)
--(axis cs:0.855,0)
--(axis cs:0.856,0)
--(axis cs:0.857,0)
--(axis cs:0.858,0)
--(axis cs:0.859,0)
--(axis cs:0.86,0)
--(axis cs:0.861,0)
--(axis cs:0.862,0)
--(axis cs:0.863,0)
--(axis cs:0.864,0)
--(axis cs:0.865,0)
--(axis cs:0.866,0)
--(axis cs:0.867,0)
--(axis cs:0.868,0)
--(axis cs:0.869,0)
--(axis cs:0.87,0)
--(axis cs:0.871,0)
--(axis cs:0.872,0)
--(axis cs:0.873,0)
--(axis cs:0.874,0)
--(axis cs:0.875,0)
--(axis cs:0.876,0)
--(axis cs:0.877,0)
--(axis cs:0.878,0)
--(axis cs:0.879,0)
--(axis cs:0.88,0)
--(axis cs:0.881,0)
--(axis cs:0.882,0)
--(axis cs:0.883,0)
--(axis cs:0.884,0)
--(axis cs:0.885,0)
--(axis cs:0.886,0)
--cycle;
\path [draw=none, fill=blue, fill opacity=0.5]
(axis cs:0,0)
--(axis cs:0.001,0)
--(axis cs:0.002,3.79114210771532e-305)
--(axis cs:0.003,2.27295435483742e-220)
--(axis cs:0.004,2.22041196223942e-164)
--(axis cs:0.005,3.59392471299512e-124)
--(axis cs:0.006,5.99467998609246e-94)
--(axis cs:0.007,1.26935745406657e-70)
--(axis cs:0.008,4.65685972914961e-80)
--(axis cs:0.009,1.58252448966646e-93)
--(axis cs:0.01,3.95169462233851e-107)
--(axis cs:0.011,7.64016500319958e-121)
--(axis cs:0.012,1.18869226225084e-134)
--(axis cs:0.013,1.53251456003575e-148)
--(axis cs:0.014,1.67490368336033e-162)
--(axis cs:0.015,1.57998245946121e-176)
--(axis cs:0.016,1.30525638137975e-190)
--(axis cs:0.017,9.55588966251468e-205)
--(axis cs:0.018,6.26093883958636e-219)
--(axis cs:0.019,3.70139413359828e-233)
--(axis cs:0.02,1.98819757673142e-247)
--(axis cs:0.021,9.76084292719275e-262)
--(axis cs:0.022,4.40201301358189e-276)
--(axis cs:0.023,1.83170729292297e-290)
--(axis cs:0.024,7.05925369579821e-305)
--(axis cs:0.025,2.52818331633424e-319)
--(axis cs:0.026,0)
--(axis cs:0.027,0)
--(axis cs:0.028,0)
--(axis cs:0.029,0)
--(axis cs:0.03,0)
--(axis cs:0.031,0)
--(axis cs:0.032,0)
--(axis cs:0.033,0)
--(axis cs:0.034,0)
--(axis cs:0.035,0)
--(axis cs:0.036,0)
--(axis cs:0.037,0)
--(axis cs:0.038,0)
--(axis cs:0.039,0)
--(axis cs:0.04,0)
--(axis cs:0.041,0)
--(axis cs:0.042,0)
--(axis cs:0.043,0)
--(axis cs:0.044,0)
--(axis cs:0.045,0)
--(axis cs:0.046,0)
--(axis cs:0.047,0)
--(axis cs:0.048,0)
--(axis cs:0.049,0)
--(axis cs:0.05,0)
--(axis cs:0.051,0)
--(axis cs:0.052,0)
--(axis cs:0.053,0)
--(axis cs:0.054,0)
--(axis cs:0.055,0)
--(axis cs:0.056,0)
--(axis cs:0.057,0)
--(axis cs:0.058,0)
--(axis cs:0.059,0)
--(axis cs:0.06,0)
--(axis cs:0.061,0)
--(axis cs:0.062,0)
--(axis cs:0.063,0)
--(axis cs:0.064,0)
--(axis cs:0.065,0)
--(axis cs:0.066,0)
--(axis cs:0.067,0)
--(axis cs:0.068,0)
--(axis cs:0.069,0)
--(axis cs:0.07,0)
--(axis cs:0.071,0)
--(axis cs:0.072,0)
--(axis cs:0.073,0)
--(axis cs:0.074,0)
--(axis cs:0.075,0)
--(axis cs:0.076,0)
--(axis cs:0.077,0)
--(axis cs:0.078,0)
--(axis cs:0.079,0)
--(axis cs:0.08,0)
--(axis cs:0.081,0)
--(axis cs:0.082,0)
--(axis cs:0.083,0)
--(axis cs:0.084,0)
--(axis cs:0.085,0)
--(axis cs:0.086,0)
--(axis cs:0.087,0)
--(axis cs:0.088,0)
--(axis cs:0.089,0)
--(axis cs:0.09,0)
--(axis cs:0.091,0)
--(axis cs:0.092,0)
--(axis cs:0.093,0)
--(axis cs:0.094,0)
--(axis cs:0.095,0)
--(axis cs:0.096,0)
--(axis cs:0.097,0)
--(axis cs:0.098,0)
--(axis cs:0.099,0)
--(axis cs:0.1,0)
--(axis cs:0.101,0)
--(axis cs:0.102,0)
--(axis cs:0.103,0)
--(axis cs:0.104,0)
--(axis cs:0.105,0)
--(axis cs:0.106,0)
--(axis cs:0.107,0)
--(axis cs:0.108,0)
--(axis cs:0.109,0)
--(axis cs:0.11,0)
--(axis cs:0.111,0)
--(axis cs:0.112,0)
--(axis cs:0.113,0)
--(axis cs:0.114,0)
--(axis cs:0.115,0)
--(axis cs:0.116,0)
--(axis cs:0.117,0)
--(axis cs:0.118,0)
--(axis cs:0.119,0)
--(axis cs:0.12,0)
--(axis cs:0.121,0)
--(axis cs:0.122,0)
--(axis cs:0.123,0)
--(axis cs:0.124,0)
--(axis cs:0.125,0)
--(axis cs:0.126,0)
--(axis cs:0.127,0)
--(axis cs:0.128,0)
--(axis cs:0.129,0)
--(axis cs:0.13,0)
--(axis cs:0.131,0)
--(axis cs:0.132,0)
--(axis cs:0.133,0)
--(axis cs:0.134,0)
--(axis cs:0.135,0)
--(axis cs:0.136,0)
--(axis cs:0.137,0)
--(axis cs:0.138,0)
--(axis cs:0.139,0)
--(axis cs:0.14,0)
--(axis cs:0.141,0)
--(axis cs:0.142,0)
--(axis cs:0.143,0)
--(axis cs:0.144,0)
--(axis cs:0.145,0)
--(axis cs:0.146,0)
--(axis cs:0.147,0)
--(axis cs:0.148,0)
--(axis cs:0.149,0)
--(axis cs:0.15,0)
--(axis cs:0.151,0)
--(axis cs:0.152,0)
--(axis cs:0.153,0)
--(axis cs:0.154,0)
--(axis cs:0.155,0)
--(axis cs:0.156,0)
--(axis cs:0.157,0)
--(axis cs:0.158,0)
--(axis cs:0.159,0)
--(axis cs:0.16,0)
--(axis cs:0.161,0)
--(axis cs:0.162,0)
--(axis cs:0.163,0)
--(axis cs:0.164,0)
--(axis cs:0.165,0)
--(axis cs:0.166,0)
--(axis cs:0.167,0)
--(axis cs:0.168,0)
--(axis cs:0.169,0)
--(axis cs:0.17,0)
--(axis cs:0.171,0)
--(axis cs:0.172,0)
--(axis cs:0.173,0)
--(axis cs:0.174,0)
--(axis cs:0.175,0)
--(axis cs:0.176,0)
--(axis cs:0.177,0)
--(axis cs:0.178,0)
--(axis cs:0.179,0)
--(axis cs:0.18,0)
--(axis cs:0.181,0)
--(axis cs:0.182,0)
--(axis cs:0.183,0)
--(axis cs:0.184,0)
--(axis cs:0.185,0)
--(axis cs:0.186,0)
--(axis cs:0.187,0)
--(axis cs:0.188,0)
--(axis cs:0.189,0)
--(axis cs:0.19,0)
--(axis cs:0.191,0)
--(axis cs:0.192,0)
--(axis cs:0.193,0)
--(axis cs:0.194,0)
--(axis cs:0.195,0)
--(axis cs:0.196,0)
--(axis cs:0.197,0)
--(axis cs:0.198,0)
--(axis cs:0.199,0)
--(axis cs:0.2,0)
--(axis cs:0.201,0)
--(axis cs:0.202,0)
--(axis cs:0.203,0)
--(axis cs:0.204,0)
--(axis cs:0.205,0)
--(axis cs:0.206,0)
--(axis cs:0.207,0)
--(axis cs:0.208,0)
--(axis cs:0.209,0)
--(axis cs:0.21,0)
--(axis cs:0.211,0)
--(axis cs:0.212,0)
--(axis cs:0.213,0)
--(axis cs:0.214,0)
--(axis cs:0.215,0)
--(axis cs:0.216,0)
--(axis cs:0.217,0)
--(axis cs:0.218,0)
--(axis cs:0.219,0)
--(axis cs:0.22,0)
--(axis cs:0.221,0)
--(axis cs:0.222,0)
--(axis cs:0.223,0)
--(axis cs:0.224,0)
--(axis cs:0.225,0)
--(axis cs:0.226,0)
--(axis cs:0.227,0)
--(axis cs:0.228,0)
--(axis cs:0.229,0)
--(axis cs:0.23,0)
--(axis cs:0.231,0)
--(axis cs:0.232,0)
--(axis cs:0.233,0)
--(axis cs:0.234,0)
--(axis cs:0.235,0)
--(axis cs:0.236,0)
--(axis cs:0.237,0)
--(axis cs:0.238,0)
--(axis cs:0.239,0)
--(axis cs:0.24,0)
--(axis cs:0.241,0)
--(axis cs:0.242,0)
--(axis cs:0.243,0)
--(axis cs:0.244,0)
--(axis cs:0.245,0)
--(axis cs:0.246,0)
--(axis cs:0.247,0)
--(axis cs:0.248,0)
--(axis cs:0.249,0)
--(axis cs:0.25,0)
--(axis cs:0.251,0)
--(axis cs:0.252,0)
--(axis cs:0.253,0)
--(axis cs:0.254,0)
--(axis cs:0.255,0)
--(axis cs:0.256,0)
--(axis cs:0.257,0)
--(axis cs:0.258,0)
--(axis cs:0.259,0)
--(axis cs:0.26,0)
--(axis cs:0.261,0)
--(axis cs:0.262,0)
--(axis cs:0.263,0)
--(axis cs:0.264,0)
--(axis cs:0.265,0)
--(axis cs:0.266,0)
--(axis cs:0.267,0)
--(axis cs:0.268,0)
--(axis cs:0.269,0)
--(axis cs:0.27,0)
--(axis cs:0.271,0)
--(axis cs:0.272,0)
--(axis cs:0.273,0)
--(axis cs:0.274,0)
--(axis cs:0.275,0)
--(axis cs:0.276,0)
--(axis cs:0.277,0)
--(axis cs:0.278,0)
--(axis cs:0.279,0)
--(axis cs:0.28,0)
--(axis cs:0.281,0)
--(axis cs:0.282,0)
--(axis cs:0.283,0)
--(axis cs:0.284,0)
--(axis cs:0.285,0)
--(axis cs:0.286,0)
--(axis cs:0.287,0)
--(axis cs:0.288,0)
--(axis cs:0.289,0)
--(axis cs:0.29,0)
--(axis cs:0.291,0)
--(axis cs:0.292,0)
--(axis cs:0.293,0)
--(axis cs:0.294,0)
--(axis cs:0.295,0)
--(axis cs:0.296,0)
--(axis cs:0.297,0)
--(axis cs:0.298,0)
--(axis cs:0.299,0)
--(axis cs:0.3,0)
--(axis cs:0.301,0)
--(axis cs:0.302,0)
--(axis cs:0.303,0)
--(axis cs:0.304,0)
--(axis cs:0.305,0)
--(axis cs:0.306,0)
--(axis cs:0.307,0)
--(axis cs:0.308,0)
--(axis cs:0.309,0)
--(axis cs:0.31,0)
--(axis cs:0.311,0)
--(axis cs:0.312,0)
--(axis cs:0.313,0)
--(axis cs:0.314,0)
--(axis cs:0.315,0)
--(axis cs:0.316,0)
--(axis cs:0.317,0)
--(axis cs:0.318,0)
--(axis cs:0.319,0)
--(axis cs:0.32,0)
--(axis cs:0.321,0)
--(axis cs:0.322,0)
--(axis cs:0.323,0)
--(axis cs:0.324,0)
--(axis cs:0.325,0)
--(axis cs:0.326,0)
--(axis cs:0.327,0)
--(axis cs:0.328,0)
--(axis cs:0.329,0)
--(axis cs:0.33,0)
--(axis cs:0.331,0)
--(axis cs:0.332,0)
--(axis cs:0.333,0)
--(axis cs:0.334,0)
--(axis cs:0.335,0)
--(axis cs:0.336,0)
--(axis cs:0.337,0)
--(axis cs:0.338,0)
--(axis cs:0.339,0)
--(axis cs:0.34,0)
--(axis cs:0.341,0)
--(axis cs:0.342,0)
--(axis cs:0.343,0)
--(axis cs:0.344,0)
--(axis cs:0.345,0)
--(axis cs:0.346,0)
--(axis cs:0.347,0)
--(axis cs:0.348,0)
--(axis cs:0.349,0)
--(axis cs:0.35,0)
--(axis cs:0.351,0)
--(axis cs:0.352,0)
--(axis cs:0.353,0)
--(axis cs:0.354,0)
--(axis cs:0.355,0)
--(axis cs:0.356,0)
--(axis cs:0.357,0)
--(axis cs:0.358,0)
--(axis cs:0.359,0)
--(axis cs:0.36,0)
--(axis cs:0.361,0)
--(axis cs:0.362,0)
--(axis cs:0.363,0)
--(axis cs:0.364,0)
--(axis cs:0.365,0)
--(axis cs:0.366,0)
--(axis cs:0.367,0)
--(axis cs:0.368,0)
--(axis cs:0.369,0)
--(axis cs:0.37,0)
--(axis cs:0.371,0)
--(axis cs:0.372,0)
--(axis cs:0.373,0)
--(axis cs:0.374,0)
--(axis cs:0.375,0)
--(axis cs:0.376,0)
--(axis cs:0.377,0)
--(axis cs:0.378,0)
--(axis cs:0.379,0)
--(axis cs:0.38,0)
--(axis cs:0.381,0)
--(axis cs:0.382,0)
--(axis cs:0.383,0)
--(axis cs:0.384,0)
--(axis cs:0.385,0)
--(axis cs:0.386,0)
--(axis cs:0.387,0)
--(axis cs:0.388,0)
--(axis cs:0.389,0)
--(axis cs:0.39,0)
--(axis cs:0.391,0)
--(axis cs:0.392,0)
--(axis cs:0.393,0)
--(axis cs:0.394,0)
--(axis cs:0.395,0)
--(axis cs:0.396,0)
--(axis cs:0.397,0)
--(axis cs:0.398,0)
--(axis cs:0.399,0)
--(axis cs:0.4,0)
--(axis cs:0.401,0)
--(axis cs:0.402,0)
--(axis cs:0.403,0)
--(axis cs:0.404,0)
--(axis cs:0.405,0)
--(axis cs:0.406,0)
--(axis cs:0.407,0)
--(axis cs:0.408,0)
--(axis cs:0.409,0)
--(axis cs:0.41,0)
--(axis cs:0.411,0)
--(axis cs:0.412,0)
--(axis cs:0.413,0)
--(axis cs:0.414,0)
--(axis cs:0.415,0)
--(axis cs:0.416,0)
--(axis cs:0.417,0)
--(axis cs:0.418,0)
--(axis cs:0.419,0)
--(axis cs:0.42,0)
--(axis cs:0.421,0)
--(axis cs:0.422,0)
--(axis cs:0.423,0)
--(axis cs:0.424,0)
--(axis cs:0.425,0)
--(axis cs:0.426,0)
--(axis cs:0.427,0)
--(axis cs:0.428,0)
--(axis cs:0.429,0)
--(axis cs:0.43,0)
--(axis cs:0.431,0)
--(axis cs:0.432,0)
--(axis cs:0.433,0)
--(axis cs:0.434,0)
--(axis cs:0.435,0)
--(axis cs:0.436,0)
--(axis cs:0.437,0)
--(axis cs:0.438,0)
--(axis cs:0.439,0)
--(axis cs:0.44,0)
--(axis cs:0.441,0)
--(axis cs:0.442,0)
--(axis cs:0.443,0)
--(axis cs:0.444,0)
--(axis cs:0.445,0)
--(axis cs:0.446,0)
--(axis cs:0.447,0)
--(axis cs:0.448,0)
--(axis cs:0.449,0)
--(axis cs:0.45,0)
--(axis cs:0.451,0)
--(axis cs:0.452,0)
--(axis cs:0.453,0)
--(axis cs:0.454,0)
--(axis cs:0.455,0)
--(axis cs:0.456,0)
--(axis cs:0.457,0)
--(axis cs:0.458,0)
--(axis cs:0.459,0)
--(axis cs:0.46,0)
--(axis cs:0.461,0)
--(axis cs:0.462,0)
--(axis cs:0.463,0)
--(axis cs:0.464,0)
--(axis cs:0.465,0)
--(axis cs:0.466,0)
--(axis cs:0.467,0)
--(axis cs:0.468,0)
--(axis cs:0.469,0)
--(axis cs:0.47,0)
--(axis cs:0.471,0)
--(axis cs:0.472,0)
--(axis cs:0.473,0)
--(axis cs:0.474,0)
--(axis cs:0.475,0)
--(axis cs:0.476,0)
--(axis cs:0.477,0)
--(axis cs:0.478,0)
--(axis cs:0.479,0)
--(axis cs:0.48,0)
--(axis cs:0.481,0)
--(axis cs:0.482,0)
--(axis cs:0.483,0)
--(axis cs:0.484,0)
--(axis cs:0.485,0)
--(axis cs:0.486,0)
--(axis cs:0.487,0)
--(axis cs:0.488,0)
--(axis cs:0.489,0)
--(axis cs:0.49,0)
--(axis cs:0.491,0)
--(axis cs:0.492,0)
--(axis cs:0.493,0)
--(axis cs:0.494,0)
--(axis cs:0.495,0)
--(axis cs:0.496,0)
--(axis cs:0.497,0)
--(axis cs:0.498,0)
--(axis cs:0.499,0)
--(axis cs:0.5,0)
--(axis cs:0.501,0)
--(axis cs:0.502,0)
--(axis cs:0.503,0)
--(axis cs:0.504,0)
--(axis cs:0.505,0)
--(axis cs:0.506,0)
--(axis cs:0.507,0)
--(axis cs:0.508,0)
--(axis cs:0.509,0)
--(axis cs:0.51,0)
--(axis cs:0.511,0)
--(axis cs:0.512,0)
--(axis cs:0.513,0)
--(axis cs:0.514,0)
--(axis cs:0.515,0)
--(axis cs:0.516,0)
--(axis cs:0.517,0)
--(axis cs:0.518,0)
--(axis cs:0.519,0)
--(axis cs:0.52,0)
--(axis cs:0.521,0)
--(axis cs:0.522,0)
--(axis cs:0.523,0)
--(axis cs:0.524,0)
--(axis cs:0.525,0)
--(axis cs:0.526,0)
--(axis cs:0.527,0)
--(axis cs:0.528,0)
--(axis cs:0.529,0)
--(axis cs:0.53,0)
--(axis cs:0.531,0)
--(axis cs:0.532,0)
--(axis cs:0.533,0)
--(axis cs:0.534,0)
--(axis cs:0.535,0)
--(axis cs:0.536,0)
--(axis cs:0.537,0)
--(axis cs:0.538,0)
--(axis cs:0.539,0)
--(axis cs:0.54,0)
--(axis cs:0.541,0)
--(axis cs:0.542,0)
--(axis cs:0.543,0)
--(axis cs:0.544,0)
--(axis cs:0.545,0)
--(axis cs:0.546,0)
--(axis cs:0.547,0)
--(axis cs:0.548,0)
--(axis cs:0.549,0)
--(axis cs:0.55,0)
--(axis cs:0.551,0)
--(axis cs:0.552,0)
--(axis cs:0.553,0)
--(axis cs:0.554,0)
--(axis cs:0.555,0)
--(axis cs:0.556,0)
--(axis cs:0.557,0)
--(axis cs:0.558,0)
--(axis cs:0.559,0)
--(axis cs:0.56,0)
--(axis cs:0.561,0)
--(axis cs:0.562,0)
--(axis cs:0.563,0)
--(axis cs:0.564,0)
--(axis cs:0.565,0)
--(axis cs:0.566,0)
--(axis cs:0.567,0)
--(axis cs:0.568,0)
--(axis cs:0.569,0)
--(axis cs:0.57,0)
--(axis cs:0.571,0)
--(axis cs:0.572,0)
--(axis cs:0.573,0)
--(axis cs:0.574,0)
--(axis cs:0.575,0)
--(axis cs:0.576,0)
--(axis cs:0.577,0)
--(axis cs:0.578,0)
--(axis cs:0.579,0)
--(axis cs:0.58,0)
--(axis cs:0.581,0)
--(axis cs:0.582,0)
--(axis cs:0.583,0)
--(axis cs:0.584,0)
--(axis cs:0.585,0)
--(axis cs:0.586,0)
--(axis cs:0.587,0)
--(axis cs:0.588,0)
--(axis cs:0.589,0)
--(axis cs:0.59,0)
--(axis cs:0.591,0)
--(axis cs:0.592,0)
--(axis cs:0.593,0)
--(axis cs:0.594,0)
--(axis cs:0.595,0)
--(axis cs:0.596,0)
--(axis cs:0.597,0)
--(axis cs:0.598,0)
--(axis cs:0.599,0)
--(axis cs:0.6,0)
--(axis cs:0.601,0)
--(axis cs:0.602,0)
--(axis cs:0.603,0)
--(axis cs:0.604,0)
--(axis cs:0.605,0)
--(axis cs:0.606,0)
--(axis cs:0.607,0)
--(axis cs:0.608,0)
--(axis cs:0.609,0)
--(axis cs:0.61,0)
--(axis cs:0.611,0)
--(axis cs:0.612,0)
--(axis cs:0.613,0)
--(axis cs:0.614,0)
--(axis cs:0.615,0)
--(axis cs:0.616,0)
--(axis cs:0.617,0)
--(axis cs:0.618,0)
--(axis cs:0.619,0)
--(axis cs:0.62,0)
--(axis cs:0.621,0)
--(axis cs:0.622,0)
--(axis cs:0.623,0)
--(axis cs:0.624,0)
--(axis cs:0.625,0)
--(axis cs:0.626,0)
--(axis cs:0.627,0)
--(axis cs:0.628,0)
--(axis cs:0.629,0)
--(axis cs:0.63,0)
--(axis cs:0.631,0)
--(axis cs:0.632,0)
--(axis cs:0.633,0)
--(axis cs:0.634,0)
--(axis cs:0.635,0)
--(axis cs:0.636,0)
--(axis cs:0.637,0)
--(axis cs:0.638,0)
--(axis cs:0.639,0)
--(axis cs:0.64,0)
--(axis cs:0.641,0)
--(axis cs:0.642,0)
--(axis cs:0.643,0)
--(axis cs:0.644,0)
--(axis cs:0.645,0)
--(axis cs:0.646,0)
--(axis cs:0.647,0)
--(axis cs:0.648,0)
--(axis cs:0.649,0)
--(axis cs:0.65,0)
--(axis cs:0.651,0)
--(axis cs:0.652,0)
--(axis cs:0.653,0)
--(axis cs:0.654,0)
--(axis cs:0.655,0)
--(axis cs:0.656,0)
--(axis cs:0.657,0)
--(axis cs:0.658,0)
--(axis cs:0.659,0)
--(axis cs:0.66,0)
--(axis cs:0.661,0)
--(axis cs:0.662,0)
--(axis cs:0.663,0)
--(axis cs:0.664,0)
--(axis cs:0.665,0)
--(axis cs:0.666,0)
--(axis cs:0.667,0)
--(axis cs:0.668,0)
--(axis cs:0.669,0)
--(axis cs:0.67,0)
--(axis cs:0.671,0)
--(axis cs:0.672,0)
--(axis cs:0.673,0)
--(axis cs:0.674,0)
--(axis cs:0.675,0)
--(axis cs:0.676,0)
--(axis cs:0.677,0)
--(axis cs:0.678,0)
--(axis cs:0.679,0)
--(axis cs:0.68,0)
--(axis cs:0.681,0)
--(axis cs:0.682,0)
--(axis cs:0.683,0)
--(axis cs:0.684,0)
--(axis cs:0.685,0)
--(axis cs:0.686,0)
--(axis cs:0.687,0)
--(axis cs:0.688,0)
--(axis cs:0.689,0)
--(axis cs:0.69,0)
--(axis cs:0.691,0)
--(axis cs:0.692,0)
--(axis cs:0.693,0)
--(axis cs:0.694,0)
--(axis cs:0.695,0)
--(axis cs:0.696,0)
--(axis cs:0.697,0)
--(axis cs:0.698,0)
--(axis cs:0.699,0)
--(axis cs:0.7,0)
--(axis cs:0.701,0)
--(axis cs:0.702,0)
--(axis cs:0.703,0)
--(axis cs:0.704,0)
--(axis cs:0.705,0)
--(axis cs:0.706,0)
--(axis cs:0.707,0)
--(axis cs:0.708,0)
--(axis cs:0.709,0)
--(axis cs:0.71,0)
--(axis cs:0.711,0)
--(axis cs:0.712,0)
--(axis cs:0.713,0)
--(axis cs:0.714,0)
--(axis cs:0.715,0)
--(axis cs:0.716,0)
--(axis cs:0.717,0)
--(axis cs:0.718,0)
--(axis cs:0.719,0)
--(axis cs:0.72,0)
--(axis cs:0.721,0)
--(axis cs:0.722,0)
--(axis cs:0.723,0)
--(axis cs:0.724,0)
--(axis cs:0.725,0)
--(axis cs:0.726,0)
--(axis cs:0.727,0)
--(axis cs:0.728,0)
--(axis cs:0.729,0)
--(axis cs:0.73,0)
--(axis cs:0.731,0)
--(axis cs:0.732,0)
--(axis cs:0.733,0)
--(axis cs:0.734,0)
--(axis cs:0.735,0)
--(axis cs:0.736,0)
--(axis cs:0.737,0)
--(axis cs:0.738,0)
--(axis cs:0.739,0)
--(axis cs:0.74,0)
--(axis cs:0.741,0)
--(axis cs:0.742,0)
--(axis cs:0.743,0)
--(axis cs:0.744,0)
--(axis cs:0.745,0)
--(axis cs:0.746,0)
--(axis cs:0.747,0)
--(axis cs:0.748,0)
--(axis cs:0.749,0)
--(axis cs:0.75,0)
--(axis cs:0.751,0)
--(axis cs:0.752,0)
--(axis cs:0.753,0)
--(axis cs:0.754,0)
--(axis cs:0.755,0)
--(axis cs:0.756,0)
--(axis cs:0.757,0)
--(axis cs:0.758,0)
--(axis cs:0.759,0)
--(axis cs:0.76,0)
--(axis cs:0.761,0)
--(axis cs:0.762,0)
--(axis cs:0.763,0)
--(axis cs:0.764,0)
--(axis cs:0.765,0)
--(axis cs:0.766,0)
--(axis cs:0.767,0)
--(axis cs:0.768,0)
--(axis cs:0.769,0)
--(axis cs:0.77,0)
--(axis cs:0.771,0)
--(axis cs:0.772,0)
--(axis cs:0.773,0)
--(axis cs:0.774,0)
--(axis cs:0.775,0)
--(axis cs:0.776,0)
--(axis cs:0.777,0)
--(axis cs:0.778,0)
--(axis cs:0.779,0)
--(axis cs:0.78,0)
--(axis cs:0.781,0)
--(axis cs:0.782,0)
--(axis cs:0.783,0)
--(axis cs:0.784,0)
--(axis cs:0.785,0)
--(axis cs:0.786,0)
--(axis cs:0.787,0)
--(axis cs:0.788,0)
--(axis cs:0.789,0)
--(axis cs:0.79,0)
--(axis cs:0.791,0)
--(axis cs:0.792,0)
--(axis cs:0.793,0)
--(axis cs:0.794,0)
--(axis cs:0.795,0)
--(axis cs:0.796,0)
--(axis cs:0.797,0)
--(axis cs:0.798,0)
--(axis cs:0.799,0)
--(axis cs:0.8,0)
--(axis cs:0.801,0)
--(axis cs:0.802,0)
--(axis cs:0.803,0)
--(axis cs:0.804,0)
--(axis cs:0.805,0)
--(axis cs:0.806,0)
--(axis cs:0.807,0)
--(axis cs:0.808,0)
--(axis cs:0.809,0)
--(axis cs:0.81,0)
--(axis cs:0.811,0)
--(axis cs:0.812,0)
--(axis cs:0.813,0)
--(axis cs:0.814,0)
--(axis cs:0.815,0)
--(axis cs:0.816,0)
--(axis cs:0.817,0)
--(axis cs:0.818,0)
--(axis cs:0.819,0)
--(axis cs:0.82,0)
--(axis cs:0.821,0)
--(axis cs:0.822,0)
--(axis cs:0.823,0)
--(axis cs:0.824,0)
--(axis cs:0.825,0)
--(axis cs:0.826,0)
--(axis cs:0.827,0)
--(axis cs:0.828,0)
--(axis cs:0.829,0)
--(axis cs:0.83,0)
--(axis cs:0.831,0)
--(axis cs:0.832,0)
--(axis cs:0.833,0)
--(axis cs:0.834,0)
--(axis cs:0.835,0)
--(axis cs:0.836,0)
--(axis cs:0.837,0)
--(axis cs:0.838,0)
--(axis cs:0.839,0)
--(axis cs:0.84,0)
--(axis cs:0.841,0)
--(axis cs:0.842,0)
--(axis cs:0.843,0)
--(axis cs:0.844,0)
--(axis cs:0.845,0)
--(axis cs:0.846,0)
--(axis cs:0.847,0)
--(axis cs:0.848,0)
--(axis cs:0.849,0)
--(axis cs:0.85,0)
--(axis cs:0.851,0)
--(axis cs:0.852,0)
--(axis cs:0.853,0)
--(axis cs:0.854,0)
--(axis cs:0.855,0)
--(axis cs:0.856,0)
--(axis cs:0.857,0)
--(axis cs:0.858,0)
--(axis cs:0.859,0)
--(axis cs:0.86,0)
--(axis cs:0.861,0)
--(axis cs:0.862,0)
--(axis cs:0.863,0)
--(axis cs:0.864,0)
--(axis cs:0.865,0)
--(axis cs:0.866,0)
--(axis cs:0.867,0)
--(axis cs:0.868,0)
--(axis cs:0.869,0)
--(axis cs:0.87,0)
--(axis cs:0.871,0)
--(axis cs:0.872,0)
--(axis cs:0.873,0)
--(axis cs:0.874,0)
--(axis cs:0.875,0)
--(axis cs:0.876,0)
--(axis cs:0.877,0)
--(axis cs:0.878,0)
--(axis cs:0.879,0)
--(axis cs:0.88,0)
--(axis cs:0.881,0)
--(axis cs:0.882,0)
--(axis cs:0.883,0)
--(axis cs:0.884,0)
--(axis cs:0.885,0)
--(axis cs:0.886,0)
--cycle;
\path [draw=none, fill=blue, fill opacity=0.5]
(axis cs:0,0)
--(axis cs:0.001,0)
--(axis cs:0.002,0)
--(axis cs:0.003,0)
--(axis cs:0.004,0)
--(axis cs:0.005,0)
--(axis cs:0.006,0)
--(axis cs:0.007,0)
--(axis cs:0.008,0)
--(axis cs:0.009,0)
--(axis cs:0.01,0)
--(axis cs:0.011,0)
--(axis cs:0.012,0)
--(axis cs:0.013,0)
--(axis cs:0.014,0)
--(axis cs:0.015,0)
--(axis cs:0.016,0)
--(axis cs:0.017,0)
--(axis cs:0.018,0)
--(axis cs:0.019,0)
--(axis cs:0.02,0)
--(axis cs:0.021,0)
--(axis cs:0.022,0)
--(axis cs:0.023,0)
--(axis cs:0.024,0)
--(axis cs:0.025,0)
--(axis cs:0.026,0)
--(axis cs:0.027,0)
--(axis cs:0.028,0)
--(axis cs:0.029,0)
--(axis cs:0.03,0)
--(axis cs:0.031,0)
--(axis cs:0.032,0)
--(axis cs:0.033,0)
--(axis cs:0.034,0)
--(axis cs:0.035,0)
--(axis cs:0.036,0)
--(axis cs:0.037,0)
--(axis cs:0.038,0)
--(axis cs:0.039,0)
--(axis cs:0.04,0)
--(axis cs:0.041,0)
--(axis cs:0.042,0)
--(axis cs:0.043,0)
--(axis cs:0.044,0)
--(axis cs:0.045,0)
--(axis cs:0.046,0)
--(axis cs:0.047,0)
--(axis cs:0.048,0)
--(axis cs:0.049,0)
--(axis cs:0.05,0)
--(axis cs:0.051,0)
--(axis cs:0.052,0)
--(axis cs:0.053,0)
--(axis cs:0.054,0)
--(axis cs:0.055,0)
--(axis cs:0.056,0)
--(axis cs:0.057,0)
--(axis cs:0.058,0)
--(axis cs:0.059,0)
--(axis cs:0.06,0)
--(axis cs:0.061,0)
--(axis cs:0.062,0)
--(axis cs:0.063,0)
--(axis cs:0.064,0)
--(axis cs:0.065,0)
--(axis cs:0.066,0)
--(axis cs:0.067,0)
--(axis cs:0.068,0)
--(axis cs:0.069,0)
--(axis cs:0.07,0)
--(axis cs:0.071,0)
--(axis cs:0.072,0)
--(axis cs:0.073,0)
--(axis cs:0.074,0)
--(axis cs:0.075,0)
--(axis cs:0.076,0)
--(axis cs:0.077,0)
--(axis cs:0.078,0)
--(axis cs:0.079,0)
--(axis cs:0.08,0)
--(axis cs:0.081,0)
--(axis cs:0.082,0)
--(axis cs:0.083,0)
--(axis cs:0.084,0)
--(axis cs:0.085,0)
--(axis cs:0.086,0)
--(axis cs:0.087,0)
--(axis cs:0.088,0)
--(axis cs:0.089,0)
--(axis cs:0.09,0)
--(axis cs:0.091,0)
--(axis cs:0.092,0)
--(axis cs:0.093,0)
--(axis cs:0.094,0)
--(axis cs:0.095,0)
--(axis cs:0.096,0)
--(axis cs:0.097,0)
--(axis cs:0.098,0)
--(axis cs:0.099,0)
--(axis cs:0.1,0)
--(axis cs:0.101,0)
--(axis cs:0.102,0)
--(axis cs:0.103,0)
--(axis cs:0.104,0)
--(axis cs:0.105,0)
--(axis cs:0.106,0)
--(axis cs:0.107,0)
--(axis cs:0.108,0)
--(axis cs:0.109,0)
--(axis cs:0.11,0)
--(axis cs:0.111,0)
--(axis cs:0.112,0)
--(axis cs:0.113,0)
--(axis cs:0.114,0)
--(axis cs:0.115,0)
--(axis cs:0.116,0)
--(axis cs:0.117,0)
--(axis cs:0.118,0)
--(axis cs:0.119,0)
--(axis cs:0.12,0)
--(axis cs:0.121,0)
--(axis cs:0.122,0)
--(axis cs:0.123,0)
--(axis cs:0.124,0)
--(axis cs:0.125,0)
--(axis cs:0.126,0)
--(axis cs:0.127,0)
--(axis cs:0.128,0)
--(axis cs:0.129,0)
--(axis cs:0.13,0)
--(axis cs:0.131,0)
--(axis cs:0.132,0)
--(axis cs:0.133,0)
--(axis cs:0.134,0)
--(axis cs:0.135,0)
--(axis cs:0.136,0)
--(axis cs:0.137,0)
--(axis cs:0.138,0)
--(axis cs:0.139,0)
--(axis cs:0.14,0)
--(axis cs:0.141,0)
--(axis cs:0.142,0)
--(axis cs:0.143,0)
--(axis cs:0.144,0)
--(axis cs:0.145,0)
--(axis cs:0.146,0)
--(axis cs:0.147,0)
--(axis cs:0.148,0)
--(axis cs:0.149,0)
--(axis cs:0.15,0)
--(axis cs:0.151,0)
--(axis cs:0.152,0)
--(axis cs:0.153,0)
--(axis cs:0.154,0)
--(axis cs:0.155,0)
--(axis cs:0.156,0)
--(axis cs:0.157,0)
--(axis cs:0.158,0)
--(axis cs:0.159,0)
--(axis cs:0.16,0)
--(axis cs:0.161,0)
--(axis cs:0.162,0)
--(axis cs:0.163,0)
--(axis cs:0.164,0)
--(axis cs:0.165,0)
--(axis cs:0.166,0)
--(axis cs:0.167,0)
--(axis cs:0.168,0)
--(axis cs:0.169,0)
--(axis cs:0.17,0)
--(axis cs:0.171,0)
--(axis cs:0.172,0)
--(axis cs:0.173,0)
--(axis cs:0.174,0)
--(axis cs:0.175,0)
--(axis cs:0.176,0)
--(axis cs:0.177,0)
--(axis cs:0.178,0)
--(axis cs:0.179,0)
--(axis cs:0.18,0)
--(axis cs:0.181,0)
--(axis cs:0.182,0)
--(axis cs:0.183,0)
--(axis cs:0.184,0)
--(axis cs:0.185,0)
--(axis cs:0.186,0)
--(axis cs:0.187,0)
--(axis cs:0.188,0)
--(axis cs:0.189,0)
--(axis cs:0.19,0)
--(axis cs:0.191,0)
--(axis cs:0.192,0)
--(axis cs:0.193,0)
--(axis cs:0.194,0)
--(axis cs:0.195,0)
--(axis cs:0.196,0)
--(axis cs:0.197,0)
--(axis cs:0.198,0)
--(axis cs:0.199,0)
--(axis cs:0.2,0)
--(axis cs:0.201,0)
--(axis cs:0.202,0)
--(axis cs:0.203,0)
--(axis cs:0.204,0)
--(axis cs:0.205,0)
--(axis cs:0.206,0)
--(axis cs:0.207,0)
--(axis cs:0.208,0)
--(axis cs:0.209,0)
--(axis cs:0.21,0)
--(axis cs:0.211,0)
--(axis cs:0.212,0)
--(axis cs:0.213,0)
--(axis cs:0.214,0)
--(axis cs:0.215,0)
--(axis cs:0.216,0)
--(axis cs:0.217,0)
--(axis cs:0.218,0)
--(axis cs:0.219,0)
--(axis cs:0.22,0)
--(axis cs:0.221,0)
--(axis cs:0.222,0)
--(axis cs:0.223,0)
--(axis cs:0.224,0)
--(axis cs:0.225,0)
--(axis cs:0.226,0)
--(axis cs:0.227,0)
--(axis cs:0.228,0)
--(axis cs:0.229,0)
--(axis cs:0.23,0)
--(axis cs:0.231,0)
--(axis cs:0.232,0)
--(axis cs:0.233,0)
--(axis cs:0.234,0)
--(axis cs:0.235,0)
--(axis cs:0.236,0)
--(axis cs:0.237,0)
--(axis cs:0.238,0)
--(axis cs:0.239,0)
--(axis cs:0.24,0)
--(axis cs:0.241,0)
--(axis cs:0.242,0)
--(axis cs:0.243,0)
--(axis cs:0.244,0)
--(axis cs:0.245,0)
--(axis cs:0.246,0)
--(axis cs:0.247,0)
--(axis cs:0.248,0)
--(axis cs:0.249,0)
--(axis cs:0.25,0)
--(axis cs:0.251,0)
--(axis cs:0.252,0)
--(axis cs:0.253,0)
--(axis cs:0.254,0)
--(axis cs:0.255,0)
--(axis cs:0.256,0)
--(axis cs:0.257,0)
--(axis cs:0.258,0)
--(axis cs:0.259,0)
--(axis cs:0.26,0)
--(axis cs:0.261,0)
--(axis cs:0.262,0)
--(axis cs:0.263,0)
--(axis cs:0.264,0)
--(axis cs:0.265,0)
--(axis cs:0.266,0)
--(axis cs:0.267,0)
--(axis cs:0.268,0)
--(axis cs:0.269,0)
--(axis cs:0.27,0)
--(axis cs:0.271,0)
--(axis cs:0.272,0)
--(axis cs:0.273,0)
--(axis cs:0.274,0)
--(axis cs:0.275,0)
--(axis cs:0.276,0)
--(axis cs:0.277,0)
--(axis cs:0.278,0)
--(axis cs:0.279,0)
--(axis cs:0.28,0)
--(axis cs:0.281,0)
--(axis cs:0.282,0)
--(axis cs:0.283,0)
--(axis cs:0.284,0)
--(axis cs:0.285,0)
--(axis cs:0.286,0)
--(axis cs:0.287,0)
--(axis cs:0.288,0)
--(axis cs:0.289,0)
--(axis cs:0.29,0)
--(axis cs:0.291,0)
--(axis cs:0.292,0)
--(axis cs:0.293,0)
--(axis cs:0.294,0)
--(axis cs:0.295,0)
--(axis cs:0.296,0)
--(axis cs:0.297,0)
--(axis cs:0.298,0)
--(axis cs:0.299,0)
--(axis cs:0.3,0)
--(axis cs:0.301,0)
--(axis cs:0.302,0)
--(axis cs:0.303,0)
--(axis cs:0.304,0)
--(axis cs:0.305,0)
--(axis cs:0.306,0)
--(axis cs:0.307,0)
--(axis cs:0.308,0)
--(axis cs:0.309,0)
--(axis cs:0.31,0)
--(axis cs:0.311,0)
--(axis cs:0.312,0)
--(axis cs:0.313,0)
--(axis cs:0.314,0)
--(axis cs:0.315,0)
--(axis cs:0.316,0)
--(axis cs:0.317,0)
--(axis cs:0.318,0)
--(axis cs:0.319,0)
--(axis cs:0.32,0)
--(axis cs:0.321,0)
--(axis cs:0.322,0)
--(axis cs:0.323,0)
--(axis cs:0.324,0)
--(axis cs:0.325,0)
--(axis cs:0.326,0)
--(axis cs:0.327,0)
--(axis cs:0.328,0)
--(axis cs:0.329,0)
--(axis cs:0.33,0)
--(axis cs:0.331,0)
--(axis cs:0.332,0)
--(axis cs:0.333,0)
--(axis cs:0.334,0)
--(axis cs:0.335,0)
--(axis cs:0.336,0)
--(axis cs:0.337,0)
--(axis cs:0.338,0)
--(axis cs:0.339,0)
--(axis cs:0.34,0)
--(axis cs:0.341,0)
--(axis cs:0.342,0)
--(axis cs:0.343,0)
--(axis cs:0.344,0)
--(axis cs:0.345,0)
--(axis cs:0.346,0)
--(axis cs:0.347,0)
--(axis cs:0.348,0)
--(axis cs:0.349,0)
--(axis cs:0.35,0)
--(axis cs:0.351,0)
--(axis cs:0.352,0)
--(axis cs:0.353,0)
--(axis cs:0.354,0)
--(axis cs:0.355,0)
--(axis cs:0.356,0)
--(axis cs:0.357,0)
--(axis cs:0.358,0)
--(axis cs:0.359,0)
--(axis cs:0.36,0)
--(axis cs:0.361,0)
--(axis cs:0.362,0)
--(axis cs:0.363,0)
--(axis cs:0.364,0)
--(axis cs:0.365,0)
--(axis cs:0.366,0)
--(axis cs:0.367,0)
--(axis cs:0.368,0)
--(axis cs:0.369,0)
--(axis cs:0.37,0)
--(axis cs:0.371,0)
--(axis cs:0.372,0)
--(axis cs:0.373,0)
--(axis cs:0.374,0)
--(axis cs:0.375,0)
--(axis cs:0.376,0)
--(axis cs:0.377,0)
--(axis cs:0.378,0)
--(axis cs:0.379,0)
--(axis cs:0.38,0)
--(axis cs:0.381,0)
--(axis cs:0.382,0)
--(axis cs:0.383,0)
--(axis cs:0.384,0)
--(axis cs:0.385,0)
--(axis cs:0.386,0)
--(axis cs:0.387,0)
--(axis cs:0.388,0)
--(axis cs:0.389,0)
--(axis cs:0.39,0)
--(axis cs:0.391,0)
--(axis cs:0.392,0)
--(axis cs:0.393,0)
--(axis cs:0.394,0)
--(axis cs:0.395,0)
--(axis cs:0.396,0)
--(axis cs:0.397,0)
--(axis cs:0.398,0)
--(axis cs:0.399,0)
--(axis cs:0.4,0)
--(axis cs:0.401,0)
--(axis cs:0.402,0)
--(axis cs:0.403,0)
--(axis cs:0.404,0)
--(axis cs:0.405,0)
--(axis cs:0.406,0)
--(axis cs:0.407,0)
--(axis cs:0.408,0)
--(axis cs:0.409,0)
--(axis cs:0.41,0)
--(axis cs:0.411,0)
--(axis cs:0.412,0)
--(axis cs:0.413,0)
--(axis cs:0.414,0)
--(axis cs:0.415,0)
--(axis cs:0.416,0)
--(axis cs:0.417,0)
--(axis cs:0.418,0)
--(axis cs:0.419,0)
--(axis cs:0.42,0)
--(axis cs:0.421,0)
--(axis cs:0.422,0)
--(axis cs:0.423,0)
--(axis cs:0.424,0)
--(axis cs:0.425,0)
--(axis cs:0.426,0)
--(axis cs:0.427,0)
--(axis cs:0.428,0)
--(axis cs:0.429,0)
--(axis cs:0.43,0)
--(axis cs:0.431,0)
--(axis cs:0.432,0)
--(axis cs:0.433,0)
--(axis cs:0.434,0)
--(axis cs:0.435,0)
--(axis cs:0.436,0)
--(axis cs:0.437,0)
--(axis cs:0.438,0)
--(axis cs:0.439,0)
--(axis cs:0.44,0)
--(axis cs:0.441,0)
--(axis cs:0.442,0)
--(axis cs:0.443,0)
--(axis cs:0.444,0)
--(axis cs:0.445,0)
--(axis cs:0.446,0)
--(axis cs:0.447,0)
--(axis cs:0.448,0)
--(axis cs:0.449,0)
--(axis cs:0.45,0)
--(axis cs:0.451,0)
--(axis cs:0.452,0)
--(axis cs:0.453,0)
--(axis cs:0.454,0)
--(axis cs:0.455,0)
--(axis cs:0.456,0)
--(axis cs:0.457,0)
--(axis cs:0.458,0)
--(axis cs:0.459,0)
--(axis cs:0.46,0)
--(axis cs:0.461,0)
--(axis cs:0.462,0)
--(axis cs:0.463,0)
--(axis cs:0.464,0)
--(axis cs:0.465,0)
--(axis cs:0.466,0)
--(axis cs:0.467,0)
--(axis cs:0.468,0)
--(axis cs:0.469,0)
--(axis cs:0.47,0)
--(axis cs:0.471,0)
--(axis cs:0.472,0)
--(axis cs:0.473,0)
--(axis cs:0.474,0)
--(axis cs:0.475,0)
--(axis cs:0.476,0)
--(axis cs:0.477,0)
--(axis cs:0.478,0)
--(axis cs:0.479,0)
--(axis cs:0.48,0)
--(axis cs:0.481,0)
--(axis cs:0.482,0)
--(axis cs:0.483,0)
--(axis cs:0.484,0)
--(axis cs:0.485,0)
--(axis cs:0.486,0)
--(axis cs:0.487,0)
--(axis cs:0.488,0)
--(axis cs:0.489,0)
--(axis cs:0.49,0)
--(axis cs:0.491,0)
--(axis cs:0.492,0)
--(axis cs:0.493,0)
--(axis cs:0.494,0)
--(axis cs:0.495,0)
--(axis cs:0.496,0)
--(axis cs:0.497,0)
--(axis cs:0.498,0)
--(axis cs:0.499,0)
--(axis cs:0.5,0)
--(axis cs:0.501,0)
--(axis cs:0.502,0)
--(axis cs:0.503,0)
--(axis cs:0.504,0)
--(axis cs:0.505,0)
--(axis cs:0.506,0)
--(axis cs:0.507,0)
--(axis cs:0.508,0)
--(axis cs:0.509,0)
--(axis cs:0.51,0)
--(axis cs:0.511,0)
--(axis cs:0.512,0)
--(axis cs:0.513,0)
--(axis cs:0.514,0)
--(axis cs:0.515,0)
--(axis cs:0.516,0)
--(axis cs:0.517,0)
--(axis cs:0.518,0)
--(axis cs:0.519,0)
--(axis cs:0.52,0)
--(axis cs:0.521,0)
--(axis cs:0.522,0)
--(axis cs:0.523,0)
--(axis cs:0.524,0)
--(axis cs:0.525,0)
--(axis cs:0.526,0)
--(axis cs:0.527,0)
--(axis cs:0.528,0)
--(axis cs:0.529,0)
--(axis cs:0.53,0)
--(axis cs:0.531,0)
--(axis cs:0.532,0)
--(axis cs:0.533,0)
--(axis cs:0.534,0)
--(axis cs:0.535,0)
--(axis cs:0.536,0)
--(axis cs:0.537,0)
--(axis cs:0.538,0)
--(axis cs:0.539,0)
--(axis cs:0.54,0)
--(axis cs:0.541,0)
--(axis cs:0.542,0)
--(axis cs:0.543,0)
--(axis cs:0.544,0)
--(axis cs:0.545,0)
--(axis cs:0.546,0)
--(axis cs:0.547,0)
--(axis cs:0.548,0)
--(axis cs:0.549,0)
--(axis cs:0.55,0)
--(axis cs:0.551,0)
--(axis cs:0.552,0)
--(axis cs:0.553,0)
--(axis cs:0.554,0)
--(axis cs:0.555,0)
--(axis cs:0.556,0)
--(axis cs:0.557,0)
--(axis cs:0.558,0)
--(axis cs:0.559,0)
--(axis cs:0.56,0)
--(axis cs:0.561,0)
--(axis cs:0.562,0)
--(axis cs:0.563,0)
--(axis cs:0.564,0)
--(axis cs:0.565,0)
--(axis cs:0.566,0)
--(axis cs:0.567,0)
--(axis cs:0.568,0)
--(axis cs:0.569,0)
--(axis cs:0.57,0)
--(axis cs:0.571,0)
--(axis cs:0.572,0)
--(axis cs:0.573,0)
--(axis cs:0.574,0)
--(axis cs:0.575,0)
--(axis cs:0.576,0)
--(axis cs:0.577,0)
--(axis cs:0.578,0)
--(axis cs:0.579,0)
--(axis cs:0.58,0)
--(axis cs:0.581,0)
--(axis cs:0.582,0)
--(axis cs:0.583,0)
--(axis cs:0.584,0)
--(axis cs:0.585,0)
--(axis cs:0.586,0)
--(axis cs:0.587,0)
--(axis cs:0.588,0)
--(axis cs:0.589,0)
--(axis cs:0.59,0)
--(axis cs:0.591,0)
--(axis cs:0.592,0)
--(axis cs:0.593,0)
--(axis cs:0.594,0)
--(axis cs:0.595,0)
--(axis cs:0.596,0)
--(axis cs:0.597,0)
--(axis cs:0.598,0)
--(axis cs:0.599,0)
--(axis cs:0.6,0)
--(axis cs:0.601,0)
--(axis cs:0.602,0)
--(axis cs:0.603,0)
--(axis cs:0.604,0)
--(axis cs:0.605,0)
--(axis cs:0.606,0)
--(axis cs:0.607,0)
--(axis cs:0.608,0)
--(axis cs:0.609,0)
--(axis cs:0.61,0)
--(axis cs:0.611,0)
--(axis cs:0.612,0)
--(axis cs:0.613,0)
--(axis cs:0.614,0)
--(axis cs:0.615,0)
--(axis cs:0.616,0)
--(axis cs:0.617,0)
--(axis cs:0.618,0)
--(axis cs:0.619,0)
--(axis cs:0.62,0)
--(axis cs:0.621,0)
--(axis cs:0.622,0)
--(axis cs:0.623,0)
--(axis cs:0.624,0)
--(axis cs:0.625,0)
--(axis cs:0.626,0)
--(axis cs:0.627,0)
--(axis cs:0.628,0)
--(axis cs:0.629,0)
--(axis cs:0.63,0)
--(axis cs:0.631,0)
--(axis cs:0.632,0)
--(axis cs:0.633,0)
--(axis cs:0.634,0)
--(axis cs:0.635,0)
--(axis cs:0.636,0)
--(axis cs:0.637,0)
--(axis cs:0.638,0)
--(axis cs:0.639,0)
--(axis cs:0.64,0)
--(axis cs:0.641,0)
--(axis cs:0.642,0)
--(axis cs:0.643,0)
--(axis cs:0.644,0)
--(axis cs:0.645,0)
--(axis cs:0.646,0)
--(axis cs:0.647,0)
--(axis cs:0.648,0)
--(axis cs:0.649,0)
--(axis cs:0.65,0)
--(axis cs:0.651,0)
--(axis cs:0.652,0)
--(axis cs:0.653,0)
--(axis cs:0.654,0)
--(axis cs:0.655,0)
--(axis cs:0.656,0)
--(axis cs:0.657,0)
--(axis cs:0.658,0)
--(axis cs:0.659,0)
--(axis cs:0.66,0)
--(axis cs:0.661,0)
--(axis cs:0.662,0)
--(axis cs:0.663,0)
--(axis cs:0.664,0)
--(axis cs:0.665,0)
--(axis cs:0.666,0)
--(axis cs:0.667,0)
--(axis cs:0.668,0)
--(axis cs:0.669,0)
--(axis cs:0.67,0)
--(axis cs:0.671,0)
--(axis cs:0.672,0)
--(axis cs:0.673,0)
--(axis cs:0.674,0)
--(axis cs:0.675,0)
--(axis cs:0.676,0)
--(axis cs:0.677,0)
--(axis cs:0.678,0)
--(axis cs:0.679,0)
--(axis cs:0.68,0)
--(axis cs:0.681,0)
--(axis cs:0.682,0)
--(axis cs:0.683,0)
--(axis cs:0.684,0)
--(axis cs:0.685,0)
--(axis cs:0.686,0)
--(axis cs:0.687,0)
--(axis cs:0.688,0)
--(axis cs:0.689,0)
--(axis cs:0.69,0)
--(axis cs:0.691,0)
--(axis cs:0.692,0)
--(axis cs:0.693,0)
--(axis cs:0.694,0)
--(axis cs:0.695,0)
--(axis cs:0.696,0)
--(axis cs:0.697,0)
--(axis cs:0.698,0)
--(axis cs:0.699,0)
--(axis cs:0.7,0)
--(axis cs:0.701,0)
--(axis cs:0.702,0)
--(axis cs:0.703,0)
--(axis cs:0.704,0)
--(axis cs:0.705,0)
--(axis cs:0.706,0)
--(axis cs:0.707,0)
--(axis cs:0.708,0)
--(axis cs:0.709,0)
--(axis cs:0.71,0)
--(axis cs:0.711,0)
--(axis cs:0.712,0)
--(axis cs:0.713,0)
--(axis cs:0.714,0)
--(axis cs:0.715,0)
--(axis cs:0.716,0)
--(axis cs:0.717,0)
--(axis cs:0.718,0)
--(axis cs:0.719,0)
--(axis cs:0.72,0)
--(axis cs:0.721,0)
--(axis cs:0.722,0)
--(axis cs:0.723,0)
--(axis cs:0.724,0)
--(axis cs:0.725,0)
--(axis cs:0.726,0)
--(axis cs:0.727,0)
--(axis cs:0.728,0)
--(axis cs:0.729,0)
--(axis cs:0.73,0)
--(axis cs:0.731,0)
--(axis cs:0.732,0)
--(axis cs:0.733,0)
--(axis cs:0.734,0)
--(axis cs:0.735,0)
--(axis cs:0.736,0)
--(axis cs:0.737,0)
--(axis cs:0.738,0)
--(axis cs:0.739,0)
--(axis cs:0.74,0)
--(axis cs:0.741,0)
--(axis cs:0.742,0)
--(axis cs:0.743,0)
--(axis cs:0.744,0)
--(axis cs:0.745,0)
--(axis cs:0.746,0)
--(axis cs:0.747,0)
--(axis cs:0.748,0)
--(axis cs:0.749,0)
--(axis cs:0.75,0)
--(axis cs:0.751,0)
--(axis cs:0.752,0)
--(axis cs:0.753,0)
--(axis cs:0.754,0)
--(axis cs:0.755,0)
--(axis cs:0.756,0)
--(axis cs:0.757,0)
--(axis cs:0.758,0)
--(axis cs:0.759,0)
--(axis cs:0.76,0)
--(axis cs:0.761,0)
--(axis cs:0.762,0)
--(axis cs:0.763,0)
--(axis cs:0.764,0)
--(axis cs:0.765,0)
--(axis cs:0.766,0)
--(axis cs:0.767,0)
--(axis cs:0.768,0)
--(axis cs:0.769,0)
--(axis cs:0.77,0)
--(axis cs:0.771,0)
--(axis cs:0.772,0)
--(axis cs:0.773,0)
--(axis cs:0.774,0)
--(axis cs:0.775,0)
--(axis cs:0.776,0)
--(axis cs:0.777,0)
--(axis cs:0.778,0)
--(axis cs:0.779,0)
--(axis cs:0.78,0)
--(axis cs:0.781,0)
--(axis cs:0.782,0)
--(axis cs:0.783,0)
--(axis cs:0.784,0)
--(axis cs:0.785,0)
--(axis cs:0.786,0)
--(axis cs:0.787,0)
--(axis cs:0.788,0)
--(axis cs:0.789,0)
--(axis cs:0.79,0)
--(axis cs:0.791,0)
--(axis cs:0.792,0)
--(axis cs:0.793,0)
--(axis cs:0.794,0)
--(axis cs:0.795,0)
--(axis cs:0.796,0)
--(axis cs:0.797,0)
--(axis cs:0.798,0)
--(axis cs:0.799,0)
--(axis cs:0.8,0)
--(axis cs:0.801,0)
--(axis cs:0.802,0)
--(axis cs:0.803,0)
--(axis cs:0.804,0)
--(axis cs:0.805,0)
--(axis cs:0.806,0)
--(axis cs:0.807,0)
--(axis cs:0.808,0)
--(axis cs:0.809,0)
--(axis cs:0.81,0)
--(axis cs:0.811,0)
--(axis cs:0.812,0)
--(axis cs:0.813,0)
--(axis cs:0.814,0)
--(axis cs:0.815,0)
--(axis cs:0.816,0)
--(axis cs:0.817,0)
--(axis cs:0.818,0)
--(axis cs:0.819,0)
--(axis cs:0.82,0)
--(axis cs:0.821,0)
--(axis cs:0.822,0)
--(axis cs:0.823,0)
--(axis cs:0.824,0)
--(axis cs:0.825,0)
--(axis cs:0.826,0)
--(axis cs:0.827,0)
--(axis cs:0.828,0)
--(axis cs:0.829,0)
--(axis cs:0.83,0)
--(axis cs:0.831,0)
--(axis cs:0.832,0)
--(axis cs:0.833,0)
--(axis cs:0.834,0)
--(axis cs:0.835,0)
--(axis cs:0.836,0)
--(axis cs:0.837,0)
--(axis cs:0.838,0)
--(axis cs:0.839,0)
--(axis cs:0.84,0)
--(axis cs:0.841,0)
--(axis cs:0.842,0)
--(axis cs:0.843,0)
--(axis cs:0.844,0)
--(axis cs:0.845,0)
--(axis cs:0.846,0)
--(axis cs:0.847,0)
--(axis cs:0.848,0)
--(axis cs:0.849,0)
--(axis cs:0.85,0)
--(axis cs:0.851,0)
--(axis cs:0.852,0)
--(axis cs:0.853,0)
--(axis cs:0.854,0)
--(axis cs:0.855,0)
--(axis cs:0.856,0)
--(axis cs:0.857,0)
--(axis cs:0.858,0)
--(axis cs:0.859,0)
--(axis cs:0.86,0)
--(axis cs:0.861,0)
--(axis cs:0.862,0)
--(axis cs:0.863,0)
--(axis cs:0.864,0)
--(axis cs:0.865,0)
--(axis cs:0.866,0)
--(axis cs:0.867,0)
--(axis cs:0.868,0)
--(axis cs:0.869,0)
--(axis cs:0.87,0)
--(axis cs:0.871,0)
--(axis cs:0.872,0)
--(axis cs:0.873,0)
--(axis cs:0.874,0)
--(axis cs:0.875,0)
--(axis cs:0.876,0)
--(axis cs:0.877,0)
--(axis cs:0.878,0)
--(axis cs:0.879,0)
--(axis cs:0.88,0)
--(axis cs:0.881,0)
--(axis cs:0.882,0)
--(axis cs:0.883,0)
--(axis cs:0.884,0)
--(axis cs:0.885,0)
--(axis cs:0.886,0)
--cycle;
\path [draw=none, fill=blue, fill opacity=0.5]
(axis cs:0,0)
--(axis cs:0.001,2228.86834404618)
--(axis cs:0.002,0.0194954987105334)
--(axis cs:0.003,2.00479132047621e-11)
--(axis cs:0.004,4.97379362068654e-22)
--(axis cs:0.005,1.57902148944301e-33)
--(axis cs:0.006,1.3486800860086e-45)
--(axis cs:0.007,4.6067588085775e-58)
--(axis cs:0.008,7.97243351467814e-71)
--(axis cs:0.009,8.14247048492395e-84)
--(axis cs:0.01,5.44658606605447e-97)
--(axis cs:0.011,2.57009820807579e-110)
--(axis cs:0.012,9.03732129532875e-124)
--(axis cs:0.013,2.46869121432058e-137)
--(axis cs:0.014,5.41093175319026e-151)
--(axis cs:0.015,9.76293036005148e-165)
--(axis cs:0.016,1.48031238377385e-178)
--(axis cs:0.017,1.91829987292819e-192)
--(axis cs:0.018,2.15442792335058e-206)
--(axis cs:0.019,2.1216464556491e-220)
--(axis cs:0.02,1.85023475651152e-234)
--(axis cs:0.021,1.44094734573719e-248)
--(axis cs:0.022,1.0094313702969e-262)
--(axis cs:0.023,6.40073259964631e-277)
--(axis cs:0.024,3.69379232441528e-291)
--(axis cs:0.025,1.94929302582764e-305)
--(axis cs:0.026,9.44653514848463e-320)
--(axis cs:0.027,0)
--(axis cs:0.028,0)
--(axis cs:0.029,0)
--(axis cs:0.03,0)
--(axis cs:0.031,0)
--(axis cs:0.032,0)
--(axis cs:0.033,0)
--(axis cs:0.034,0)
--(axis cs:0.035,0)
--(axis cs:0.036,0)
--(axis cs:0.037,0)
--(axis cs:0.038,0)
--(axis cs:0.039,0)
--(axis cs:0.04,0)
--(axis cs:0.041,0)
--(axis cs:0.042,0)
--(axis cs:0.043,0)
--(axis cs:0.044,0)
--(axis cs:0.045,0)
--(axis cs:0.046,0)
--(axis cs:0.047,0)
--(axis cs:0.048,0)
--(axis cs:0.049,0)
--(axis cs:0.05,0)
--(axis cs:0.051,0)
--(axis cs:0.052,0)
--(axis cs:0.053,0)
--(axis cs:0.054,0)
--(axis cs:0.055,0)
--(axis cs:0.056,0)
--(axis cs:0.057,0)
--(axis cs:0.058,0)
--(axis cs:0.059,0)
--(axis cs:0.06,0)
--(axis cs:0.061,0)
--(axis cs:0.062,0)
--(axis cs:0.063,0)
--(axis cs:0.064,0)
--(axis cs:0.065,0)
--(axis cs:0.066,0)
--(axis cs:0.067,0)
--(axis cs:0.068,0)
--(axis cs:0.069,0)
--(axis cs:0.07,0)
--(axis cs:0.071,0)
--(axis cs:0.072,0)
--(axis cs:0.073,0)
--(axis cs:0.074,0)
--(axis cs:0.075,0)
--(axis cs:0.076,0)
--(axis cs:0.077,0)
--(axis cs:0.078,0)
--(axis cs:0.079,0)
--(axis cs:0.08,0)
--(axis cs:0.081,0)
--(axis cs:0.082,0)
--(axis cs:0.083,0)
--(axis cs:0.084,0)
--(axis cs:0.085,0)
--(axis cs:0.086,0)
--(axis cs:0.087,0)
--(axis cs:0.088,0)
--(axis cs:0.089,0)
--(axis cs:0.09,0)
--(axis cs:0.091,0)
--(axis cs:0.092,0)
--(axis cs:0.093,0)
--(axis cs:0.094,0)
--(axis cs:0.095,0)
--(axis cs:0.096,0)
--(axis cs:0.097,0)
--(axis cs:0.098,0)
--(axis cs:0.099,0)
--(axis cs:0.1,0)
--(axis cs:0.101,0)
--(axis cs:0.102,0)
--(axis cs:0.103,0)
--(axis cs:0.104,0)
--(axis cs:0.105,0)
--(axis cs:0.106,0)
--(axis cs:0.107,0)
--(axis cs:0.108,0)
--(axis cs:0.109,0)
--(axis cs:0.11,0)
--(axis cs:0.111,0)
--(axis cs:0.112,0)
--(axis cs:0.113,0)
--(axis cs:0.114,0)
--(axis cs:0.115,0)
--(axis cs:0.116,0)
--(axis cs:0.117,0)
--(axis cs:0.118,0)
--(axis cs:0.119,0)
--(axis cs:0.12,0)
--(axis cs:0.121,0)
--(axis cs:0.122,0)
--(axis cs:0.123,0)
--(axis cs:0.124,0)
--(axis cs:0.125,0)
--(axis cs:0.126,0)
--(axis cs:0.127,0)
--(axis cs:0.128,0)
--(axis cs:0.129,0)
--(axis cs:0.13,0)
--(axis cs:0.131,0)
--(axis cs:0.132,0)
--(axis cs:0.133,0)
--(axis cs:0.134,0)
--(axis cs:0.135,0)
--(axis cs:0.136,0)
--(axis cs:0.137,0)
--(axis cs:0.138,0)
--(axis cs:0.139,0)
--(axis cs:0.14,0)
--(axis cs:0.141,0)
--(axis cs:0.142,0)
--(axis cs:0.143,0)
--(axis cs:0.144,0)
--(axis cs:0.145,0)
--(axis cs:0.146,0)
--(axis cs:0.147,0)
--(axis cs:0.148,0)
--(axis cs:0.149,0)
--(axis cs:0.15,0)
--(axis cs:0.151,0)
--(axis cs:0.152,0)
--(axis cs:0.153,0)
--(axis cs:0.154,0)
--(axis cs:0.155,0)
--(axis cs:0.156,0)
--(axis cs:0.157,0)
--(axis cs:0.158,0)
--(axis cs:0.159,0)
--(axis cs:0.16,0)
--(axis cs:0.161,0)
--(axis cs:0.162,0)
--(axis cs:0.163,0)
--(axis cs:0.164,0)
--(axis cs:0.165,0)
--(axis cs:0.166,0)
--(axis cs:0.167,0)
--(axis cs:0.168,0)
--(axis cs:0.169,0)
--(axis cs:0.17,0)
--(axis cs:0.171,0)
--(axis cs:0.172,0)
--(axis cs:0.173,0)
--(axis cs:0.174,0)
--(axis cs:0.175,0)
--(axis cs:0.176,0)
--(axis cs:0.177,0)
--(axis cs:0.178,0)
--(axis cs:0.179,0)
--(axis cs:0.18,0)
--(axis cs:0.181,0)
--(axis cs:0.182,0)
--(axis cs:0.183,0)
--(axis cs:0.184,0)
--(axis cs:0.185,0)
--(axis cs:0.186,0)
--(axis cs:0.187,0)
--(axis cs:0.188,0)
--(axis cs:0.189,0)
--(axis cs:0.19,0)
--(axis cs:0.191,0)
--(axis cs:0.192,0)
--(axis cs:0.193,0)
--(axis cs:0.194,0)
--(axis cs:0.195,0)
--(axis cs:0.196,0)
--(axis cs:0.197,0)
--(axis cs:0.198,0)
--(axis cs:0.199,0)
--(axis cs:0.2,0)
--(axis cs:0.201,0)
--(axis cs:0.202,0)
--(axis cs:0.203,0)
--(axis cs:0.204,0)
--(axis cs:0.205,0)
--(axis cs:0.206,0)
--(axis cs:0.207,0)
--(axis cs:0.208,0)
--(axis cs:0.209,0)
--(axis cs:0.21,0)
--(axis cs:0.211,0)
--(axis cs:0.212,0)
--(axis cs:0.213,0)
--(axis cs:0.214,0)
--(axis cs:0.215,0)
--(axis cs:0.216,0)
--(axis cs:0.217,0)
--(axis cs:0.218,0)
--(axis cs:0.219,0)
--(axis cs:0.22,0)
--(axis cs:0.221,0)
--(axis cs:0.222,0)
--(axis cs:0.223,0)
--(axis cs:0.224,0)
--(axis cs:0.225,0)
--(axis cs:0.226,0)
--(axis cs:0.227,0)
--(axis cs:0.228,0)
--(axis cs:0.229,0)
--(axis cs:0.23,0)
--(axis cs:0.231,0)
--(axis cs:0.232,0)
--(axis cs:0.233,0)
--(axis cs:0.234,0)
--(axis cs:0.235,0)
--(axis cs:0.236,0)
--(axis cs:0.237,0)
--(axis cs:0.238,0)
--(axis cs:0.239,0)
--(axis cs:0.24,0)
--(axis cs:0.241,0)
--(axis cs:0.242,0)
--(axis cs:0.243,0)
--(axis cs:0.244,0)
--(axis cs:0.245,0)
--(axis cs:0.246,0)
--(axis cs:0.247,0)
--(axis cs:0.248,0)
--(axis cs:0.249,0)
--(axis cs:0.25,0)
--(axis cs:0.251,0)
--(axis cs:0.252,0)
--(axis cs:0.253,0)
--(axis cs:0.254,0)
--(axis cs:0.255,0)
--(axis cs:0.256,0)
--(axis cs:0.257,0)
--(axis cs:0.258,0)
--(axis cs:0.259,0)
--(axis cs:0.26,0)
--(axis cs:0.261,0)
--(axis cs:0.262,0)
--(axis cs:0.263,0)
--(axis cs:0.264,0)
--(axis cs:0.265,0)
--(axis cs:0.266,0)
--(axis cs:0.267,0)
--(axis cs:0.268,0)
--(axis cs:0.269,0)
--(axis cs:0.27,0)
--(axis cs:0.271,0)
--(axis cs:0.272,0)
--(axis cs:0.273,0)
--(axis cs:0.274,0)
--(axis cs:0.275,0)
--(axis cs:0.276,0)
--(axis cs:0.277,0)
--(axis cs:0.278,0)
--(axis cs:0.279,0)
--(axis cs:0.28,0)
--(axis cs:0.281,0)
--(axis cs:0.282,0)
--(axis cs:0.283,0)
--(axis cs:0.284,0)
--(axis cs:0.285,0)
--(axis cs:0.286,0)
--(axis cs:0.287,0)
--(axis cs:0.288,0)
--(axis cs:0.289,0)
--(axis cs:0.29,0)
--(axis cs:0.291,0)
--(axis cs:0.292,0)
--(axis cs:0.293,0)
--(axis cs:0.294,0)
--(axis cs:0.295,0)
--(axis cs:0.296,0)
--(axis cs:0.297,0)
--(axis cs:0.298,0)
--(axis cs:0.299,0)
--(axis cs:0.3,0)
--(axis cs:0.301,0)
--(axis cs:0.302,0)
--(axis cs:0.303,0)
--(axis cs:0.304,0)
--(axis cs:0.305,0)
--(axis cs:0.306,0)
--(axis cs:0.307,0)
--(axis cs:0.308,0)
--(axis cs:0.309,0)
--(axis cs:0.31,0)
--(axis cs:0.311,0)
--(axis cs:0.312,0)
--(axis cs:0.313,0)
--(axis cs:0.314,0)
--(axis cs:0.315,0)
--(axis cs:0.316,0)
--(axis cs:0.317,0)
--(axis cs:0.318,0)
--(axis cs:0.319,0)
--(axis cs:0.32,0)
--(axis cs:0.321,0)
--(axis cs:0.322,0)
--(axis cs:0.323,0)
--(axis cs:0.324,0)
--(axis cs:0.325,0)
--(axis cs:0.326,0)
--(axis cs:0.327,0)
--(axis cs:0.328,0)
--(axis cs:0.329,0)
--(axis cs:0.33,0)
--(axis cs:0.331,0)
--(axis cs:0.332,0)
--(axis cs:0.333,0)
--(axis cs:0.334,0)
--(axis cs:0.335,0)
--(axis cs:0.336,0)
--(axis cs:0.337,0)
--(axis cs:0.338,0)
--(axis cs:0.339,0)
--(axis cs:0.34,0)
--(axis cs:0.341,0)
--(axis cs:0.342,0)
--(axis cs:0.343,0)
--(axis cs:0.344,0)
--(axis cs:0.345,0)
--(axis cs:0.346,0)
--(axis cs:0.347,0)
--(axis cs:0.348,0)
--(axis cs:0.349,0)
--(axis cs:0.35,0)
--(axis cs:0.351,0)
--(axis cs:0.352,0)
--(axis cs:0.353,0)
--(axis cs:0.354,0)
--(axis cs:0.355,0)
--(axis cs:0.356,0)
--(axis cs:0.357,0)
--(axis cs:0.358,0)
--(axis cs:0.359,0)
--(axis cs:0.36,0)
--(axis cs:0.361,0)
--(axis cs:0.362,0)
--(axis cs:0.363,0)
--(axis cs:0.364,0)
--(axis cs:0.365,0)
--(axis cs:0.366,0)
--(axis cs:0.367,0)
--(axis cs:0.368,0)
--(axis cs:0.369,0)
--(axis cs:0.37,0)
--(axis cs:0.371,0)
--(axis cs:0.372,0)
--(axis cs:0.373,0)
--(axis cs:0.374,0)
--(axis cs:0.375,0)
--(axis cs:0.376,0)
--(axis cs:0.377,0)
--(axis cs:0.378,0)
--(axis cs:0.379,0)
--(axis cs:0.38,0)
--(axis cs:0.381,0)
--(axis cs:0.382,0)
--(axis cs:0.383,0)
--(axis cs:0.384,0)
--(axis cs:0.385,0)
--(axis cs:0.386,0)
--(axis cs:0.387,0)
--(axis cs:0.388,0)
--(axis cs:0.389,0)
--(axis cs:0.39,0)
--(axis cs:0.391,0)
--(axis cs:0.392,0)
--(axis cs:0.393,0)
--(axis cs:0.394,0)
--(axis cs:0.395,0)
--(axis cs:0.396,0)
--(axis cs:0.397,0)
--(axis cs:0.398,0)
--(axis cs:0.399,0)
--(axis cs:0.4,0)
--(axis cs:0.401,0)
--(axis cs:0.402,0)
--(axis cs:0.403,0)
--(axis cs:0.404,0)
--(axis cs:0.405,0)
--(axis cs:0.406,0)
--(axis cs:0.407,0)
--(axis cs:0.408,0)
--(axis cs:0.409,0)
--(axis cs:0.41,0)
--(axis cs:0.411,0)
--(axis cs:0.412,0)
--(axis cs:0.413,0)
--(axis cs:0.414,0)
--(axis cs:0.415,0)
--(axis cs:0.416,0)
--(axis cs:0.417,0)
--(axis cs:0.418,0)
--(axis cs:0.419,0)
--(axis cs:0.42,0)
--(axis cs:0.421,0)
--(axis cs:0.422,0)
--(axis cs:0.423,0)
--(axis cs:0.424,0)
--(axis cs:0.425,0)
--(axis cs:0.426,0)
--(axis cs:0.427,0)
--(axis cs:0.428,0)
--(axis cs:0.429,0)
--(axis cs:0.43,0)
--(axis cs:0.431,0)
--(axis cs:0.432,0)
--(axis cs:0.433,0)
--(axis cs:0.434,0)
--(axis cs:0.435,0)
--(axis cs:0.436,0)
--(axis cs:0.437,0)
--(axis cs:0.438,0)
--(axis cs:0.439,0)
--(axis cs:0.44,0)
--(axis cs:0.441,0)
--(axis cs:0.442,0)
--(axis cs:0.443,0)
--(axis cs:0.444,0)
--(axis cs:0.445,0)
--(axis cs:0.446,0)
--(axis cs:0.447,0)
--(axis cs:0.448,0)
--(axis cs:0.449,0)
--(axis cs:0.45,0)
--(axis cs:0.451,0)
--(axis cs:0.452,0)
--(axis cs:0.453,0)
--(axis cs:0.454,0)
--(axis cs:0.455,0)
--(axis cs:0.456,0)
--(axis cs:0.457,0)
--(axis cs:0.458,0)
--(axis cs:0.459,0)
--(axis cs:0.46,0)
--(axis cs:0.461,0)
--(axis cs:0.462,0)
--(axis cs:0.463,0)
--(axis cs:0.464,0)
--(axis cs:0.465,0)
--(axis cs:0.466,0)
--(axis cs:0.467,0)
--(axis cs:0.468,0)
--(axis cs:0.469,0)
--(axis cs:0.47,0)
--(axis cs:0.471,0)
--(axis cs:0.472,0)
--(axis cs:0.473,0)
--(axis cs:0.474,0)
--(axis cs:0.475,0)
--(axis cs:0.476,0)
--(axis cs:0.477,0)
--(axis cs:0.478,0)
--(axis cs:0.479,0)
--(axis cs:0.48,0)
--(axis cs:0.481,0)
--(axis cs:0.482,0)
--(axis cs:0.483,0)
--(axis cs:0.484,0)
--(axis cs:0.485,0)
--(axis cs:0.486,0)
--(axis cs:0.487,0)
--(axis cs:0.488,0)
--(axis cs:0.489,0)
--(axis cs:0.49,0)
--(axis cs:0.491,0)
--(axis cs:0.492,0)
--(axis cs:0.493,0)
--(axis cs:0.494,0)
--(axis cs:0.495,0)
--(axis cs:0.496,0)
--(axis cs:0.497,0)
--(axis cs:0.498,0)
--(axis cs:0.499,0)
--(axis cs:0.5,0)
--(axis cs:0.501,0)
--(axis cs:0.502,0)
--(axis cs:0.503,0)
--(axis cs:0.504,0)
--(axis cs:0.505,0)
--(axis cs:0.506,0)
--(axis cs:0.507,0)
--(axis cs:0.508,0)
--(axis cs:0.509,0)
--(axis cs:0.51,0)
--(axis cs:0.511,0)
--(axis cs:0.512,0)
--(axis cs:0.513,0)
--(axis cs:0.514,0)
--(axis cs:0.515,0)
--(axis cs:0.516,0)
--(axis cs:0.517,0)
--(axis cs:0.518,0)
--(axis cs:0.519,0)
--(axis cs:0.52,0)
--(axis cs:0.521,0)
--(axis cs:0.522,0)
--(axis cs:0.523,0)
--(axis cs:0.524,0)
--(axis cs:0.525,0)
--(axis cs:0.526,0)
--(axis cs:0.527,0)
--(axis cs:0.528,0)
--(axis cs:0.529,0)
--(axis cs:0.53,0)
--(axis cs:0.531,0)
--(axis cs:0.532,0)
--(axis cs:0.533,0)
--(axis cs:0.534,0)
--(axis cs:0.535,0)
--(axis cs:0.536,0)
--(axis cs:0.537,0)
--(axis cs:0.538,0)
--(axis cs:0.539,0)
--(axis cs:0.54,0)
--(axis cs:0.541,0)
--(axis cs:0.542,0)
--(axis cs:0.543,0)
--(axis cs:0.544,0)
--(axis cs:0.545,0)
--(axis cs:0.546,0)
--(axis cs:0.547,0)
--(axis cs:0.548,0)
--(axis cs:0.549,0)
--(axis cs:0.55,0)
--(axis cs:0.551,0)
--(axis cs:0.552,0)
--(axis cs:0.553,0)
--(axis cs:0.554,0)
--(axis cs:0.555,0)
--(axis cs:0.556,0)
--(axis cs:0.557,0)
--(axis cs:0.558,0)
--(axis cs:0.559,0)
--(axis cs:0.56,0)
--(axis cs:0.561,0)
--(axis cs:0.562,0)
--(axis cs:0.563,0)
--(axis cs:0.564,0)
--(axis cs:0.565,0)
--(axis cs:0.566,0)
--(axis cs:0.567,0)
--(axis cs:0.568,0)
--(axis cs:0.569,0)
--(axis cs:0.57,0)
--(axis cs:0.571,0)
--(axis cs:0.572,0)
--(axis cs:0.573,0)
--(axis cs:0.574,0)
--(axis cs:0.575,0)
--(axis cs:0.576,0)
--(axis cs:0.577,0)
--(axis cs:0.578,0)
--(axis cs:0.579,0)
--(axis cs:0.58,0)
--(axis cs:0.581,0)
--(axis cs:0.582,0)
--(axis cs:0.583,0)
--(axis cs:0.584,0)
--(axis cs:0.585,0)
--(axis cs:0.586,0)
--(axis cs:0.587,0)
--(axis cs:0.588,0)
--(axis cs:0.589,0)
--(axis cs:0.59,0)
--(axis cs:0.591,0)
--(axis cs:0.592,0)
--(axis cs:0.593,0)
--(axis cs:0.594,0)
--(axis cs:0.595,0)
--(axis cs:0.596,0)
--(axis cs:0.597,0)
--(axis cs:0.598,0)
--(axis cs:0.599,0)
--(axis cs:0.6,0)
--(axis cs:0.601,0)
--(axis cs:0.602,0)
--(axis cs:0.603,0)
--(axis cs:0.604,0)
--(axis cs:0.605,0)
--(axis cs:0.606,0)
--(axis cs:0.607,0)
--(axis cs:0.608,0)
--(axis cs:0.609,0)
--(axis cs:0.61,0)
--(axis cs:0.611,0)
--(axis cs:0.612,0)
--(axis cs:0.613,0)
--(axis cs:0.614,0)
--(axis cs:0.615,0)
--(axis cs:0.616,0)
--(axis cs:0.617,0)
--(axis cs:0.618,0)
--(axis cs:0.619,0)
--(axis cs:0.62,0)
--(axis cs:0.621,0)
--(axis cs:0.622,0)
--(axis cs:0.623,0)
--(axis cs:0.624,0)
--(axis cs:0.625,0)
--(axis cs:0.626,0)
--(axis cs:0.627,0)
--(axis cs:0.628,0)
--(axis cs:0.629,0)
--(axis cs:0.63,0)
--(axis cs:0.631,0)
--(axis cs:0.632,0)
--(axis cs:0.633,0)
--(axis cs:0.634,0)
--(axis cs:0.635,0)
--(axis cs:0.636,0)
--(axis cs:0.637,0)
--(axis cs:0.638,0)
--(axis cs:0.639,0)
--(axis cs:0.64,0)
--(axis cs:0.641,0)
--(axis cs:0.642,0)
--(axis cs:0.643,0)
--(axis cs:0.644,0)
--(axis cs:0.645,0)
--(axis cs:0.646,0)
--(axis cs:0.647,0)
--(axis cs:0.648,0)
--(axis cs:0.649,0)
--(axis cs:0.65,0)
--(axis cs:0.651,0)
--(axis cs:0.652,0)
--(axis cs:0.653,0)
--(axis cs:0.654,0)
--(axis cs:0.655,0)
--(axis cs:0.656,0)
--(axis cs:0.657,0)
--(axis cs:0.658,0)
--(axis cs:0.659,0)
--(axis cs:0.66,0)
--(axis cs:0.661,0)
--(axis cs:0.662,0)
--(axis cs:0.663,0)
--(axis cs:0.664,0)
--(axis cs:0.665,0)
--(axis cs:0.666,0)
--(axis cs:0.667,0)
--(axis cs:0.668,0)
--(axis cs:0.669,0)
--(axis cs:0.67,0)
--(axis cs:0.671,0)
--(axis cs:0.672,0)
--(axis cs:0.673,0)
--(axis cs:0.674,0)
--(axis cs:0.675,0)
--(axis cs:0.676,0)
--(axis cs:0.677,0)
--(axis cs:0.678,0)
--(axis cs:0.679,0)
--(axis cs:0.68,0)
--(axis cs:0.681,0)
--(axis cs:0.682,0)
--(axis cs:0.683,0)
--(axis cs:0.684,0)
--(axis cs:0.685,0)
--(axis cs:0.686,0)
--(axis cs:0.687,0)
--(axis cs:0.688,0)
--(axis cs:0.689,0)
--(axis cs:0.69,0)
--(axis cs:0.691,0)
--(axis cs:0.692,0)
--(axis cs:0.693,0)
--(axis cs:0.694,0)
--(axis cs:0.695,0)
--(axis cs:0.696,0)
--(axis cs:0.697,0)
--(axis cs:0.698,0)
--(axis cs:0.699,0)
--(axis cs:0.7,0)
--(axis cs:0.701,0)
--(axis cs:0.702,0)
--(axis cs:0.703,0)
--(axis cs:0.704,0)
--(axis cs:0.705,0)
--(axis cs:0.706,0)
--(axis cs:0.707,0)
--(axis cs:0.708,0)
--(axis cs:0.709,0)
--(axis cs:0.71,0)
--(axis cs:0.711,0)
--(axis cs:0.712,0)
--(axis cs:0.713,0)
--(axis cs:0.714,0)
--(axis cs:0.715,0)
--(axis cs:0.716,0)
--(axis cs:0.717,0)
--(axis cs:0.718,0)
--(axis cs:0.719,0)
--(axis cs:0.72,0)
--(axis cs:0.721,0)
--(axis cs:0.722,0)
--(axis cs:0.723,0)
--(axis cs:0.724,0)
--(axis cs:0.725,0)
--(axis cs:0.726,0)
--(axis cs:0.727,0)
--(axis cs:0.728,0)
--(axis cs:0.729,0)
--(axis cs:0.73,0)
--(axis cs:0.731,0)
--(axis cs:0.732,0)
--(axis cs:0.733,0)
--(axis cs:0.734,0)
--(axis cs:0.735,0)
--(axis cs:0.736,0)
--(axis cs:0.737,0)
--(axis cs:0.738,0)
--(axis cs:0.739,0)
--(axis cs:0.74,0)
--(axis cs:0.741,0)
--(axis cs:0.742,0)
--(axis cs:0.743,0)
--(axis cs:0.744,0)
--(axis cs:0.745,0)
--(axis cs:0.746,0)
--(axis cs:0.747,0)
--(axis cs:0.748,0)
--(axis cs:0.749,0)
--(axis cs:0.75,0)
--(axis cs:0.751,0)
--(axis cs:0.752,0)
--(axis cs:0.753,0)
--(axis cs:0.754,0)
--(axis cs:0.755,0)
--(axis cs:0.756,0)
--(axis cs:0.757,0)
--(axis cs:0.758,0)
--(axis cs:0.759,0)
--(axis cs:0.76,0)
--(axis cs:0.761,0)
--(axis cs:0.762,0)
--(axis cs:0.763,0)
--(axis cs:0.764,0)
--(axis cs:0.765,0)
--(axis cs:0.766,0)
--(axis cs:0.767,0)
--(axis cs:0.768,0)
--(axis cs:0.769,0)
--(axis cs:0.77,0)
--(axis cs:0.771,0)
--(axis cs:0.772,0)
--(axis cs:0.773,0)
--(axis cs:0.774,0)
--(axis cs:0.775,0)
--(axis cs:0.776,0)
--(axis cs:0.777,0)
--(axis cs:0.778,0)
--(axis cs:0.779,0)
--(axis cs:0.78,0)
--(axis cs:0.781,0)
--(axis cs:0.782,0)
--(axis cs:0.783,0)
--(axis cs:0.784,0)
--(axis cs:0.785,0)
--(axis cs:0.786,0)
--(axis cs:0.787,0)
--(axis cs:0.788,0)
--(axis cs:0.789,0)
--(axis cs:0.79,0)
--(axis cs:0.791,0)
--(axis cs:0.792,0)
--(axis cs:0.793,0)
--(axis cs:0.794,0)
--(axis cs:0.795,0)
--(axis cs:0.796,0)
--(axis cs:0.797,0)
--(axis cs:0.798,0)
--(axis cs:0.799,0)
--(axis cs:0.8,0)
--(axis cs:0.801,0)
--(axis cs:0.802,0)
--(axis cs:0.803,0)
--(axis cs:0.804,0)
--(axis cs:0.805,0)
--(axis cs:0.806,0)
--(axis cs:0.807,0)
--(axis cs:0.808,0)
--(axis cs:0.809,0)
--(axis cs:0.81,0)
--(axis cs:0.811,0)
--(axis cs:0.812,0)
--(axis cs:0.813,0)
--(axis cs:0.814,0)
--(axis cs:0.815,0)
--(axis cs:0.816,0)
--(axis cs:0.817,0)
--(axis cs:0.818,0)
--(axis cs:0.819,0)
--(axis cs:0.82,0)
--(axis cs:0.821,0)
--(axis cs:0.822,0)
--(axis cs:0.823,0)
--(axis cs:0.824,0)
--(axis cs:0.825,0)
--(axis cs:0.826,0)
--(axis cs:0.827,0)
--(axis cs:0.828,0)
--(axis cs:0.829,0)
--(axis cs:0.83,0)
--(axis cs:0.831,0)
--(axis cs:0.832,0)
--(axis cs:0.833,0)
--(axis cs:0.834,0)
--(axis cs:0.835,0)
--(axis cs:0.836,0)
--(axis cs:0.837,0)
--(axis cs:0.838,0)
--(axis cs:0.839,0)
--(axis cs:0.84,0)
--(axis cs:0.841,0)
--(axis cs:0.842,0)
--(axis cs:0.843,0)
--(axis cs:0.844,0)
--(axis cs:0.845,0)
--(axis cs:0.846,0)
--(axis cs:0.847,0)
--(axis cs:0.848,0)
--(axis cs:0.849,0)
--(axis cs:0.85,0)
--(axis cs:0.851,0)
--(axis cs:0.852,0)
--(axis cs:0.853,0)
--(axis cs:0.854,0)
--(axis cs:0.855,0)
--(axis cs:0.856,0)
--(axis cs:0.857,0)
--(axis cs:0.858,0)
--(axis cs:0.859,0)
--(axis cs:0.86,0)
--(axis cs:0.861,0)
--(axis cs:0.862,0)
--(axis cs:0.863,0)
--(axis cs:0.864,0)
--(axis cs:0.865,0)
--(axis cs:0.866,0)
--(axis cs:0.867,0)
--(axis cs:0.868,0)
--(axis cs:0.869,0)
--(axis cs:0.87,0)
--(axis cs:0.871,0)
--(axis cs:0.872,0)
--(axis cs:0.873,0)
--(axis cs:0.874,0)
--(axis cs:0.875,0)
--(axis cs:0.876,0)
--(axis cs:0.877,0)
--(axis cs:0.878,0)
--(axis cs:0.879,0)
--(axis cs:0.88,0)
--(axis cs:0.881,0)
--(axis cs:0.882,0)
--(axis cs:0.883,0)
--(axis cs:0.884,0)
--(axis cs:0.885,0)
--(axis cs:0.886,0)
--cycle;
\path [draw=none, fill=blue, fill opacity=0.5]
(axis cs:0,0)
--(axis cs:0.001,0)
--(axis cs:0.002,3.79114210771532e-305)
--(axis cs:0.003,2.27295435483742e-220)
--(axis cs:0.004,2.22041196223942e-164)
--(axis cs:0.005,3.59392471299512e-124)
--(axis cs:0.006,5.99467998609246e-94)
--(axis cs:0.007,1.26935745406657e-70)
--(axis cs:0.008,7.97243351467814e-71)
--(axis cs:0.009,8.14247048492395e-84)
--(axis cs:0.01,5.44658606605447e-97)
--(axis cs:0.011,2.57009820807579e-110)
--(axis cs:0.012,9.03732129532875e-124)
--(axis cs:0.013,2.46869121432058e-137)
--(axis cs:0.014,5.41093175319026e-151)
--(axis cs:0.015,9.76293036005148e-165)
--(axis cs:0.016,1.48031238377385e-178)
--(axis cs:0.017,1.91829987292819e-192)
--(axis cs:0.018,2.15442792335058e-206)
--(axis cs:0.019,2.1216464556491e-220)
--(axis cs:0.02,1.85023475651152e-234)
--(axis cs:0.021,1.44094734573719e-248)
--(axis cs:0.022,1.0094313702969e-262)
--(axis cs:0.023,6.40073259964631e-277)
--(axis cs:0.024,3.69379232441528e-291)
--(axis cs:0.025,1.94929302582764e-305)
--(axis cs:0.026,9.44653514848463e-320)
--(axis cs:0.027,0)
--(axis cs:0.028,0)
--(axis cs:0.029,0)
--(axis cs:0.03,0)
--(axis cs:0.031,0)
--(axis cs:0.032,0)
--(axis cs:0.033,0)
--(axis cs:0.034,0)
--(axis cs:0.035,0)
--(axis cs:0.036,0)
--(axis cs:0.037,0)
--(axis cs:0.038,0)
--(axis cs:0.039,0)
--(axis cs:0.04,0)
--(axis cs:0.041,0)
--(axis cs:0.042,0)
--(axis cs:0.043,0)
--(axis cs:0.044,0)
--(axis cs:0.045,0)
--(axis cs:0.046,0)
--(axis cs:0.047,0)
--(axis cs:0.048,0)
--(axis cs:0.049,0)
--(axis cs:0.05,0)
--(axis cs:0.051,0)
--(axis cs:0.052,0)
--(axis cs:0.053,0)
--(axis cs:0.054,0)
--(axis cs:0.055,0)
--(axis cs:0.056,0)
--(axis cs:0.057,0)
--(axis cs:0.058,0)
--(axis cs:0.059,0)
--(axis cs:0.06,0)
--(axis cs:0.061,0)
--(axis cs:0.062,0)
--(axis cs:0.063,0)
--(axis cs:0.064,0)
--(axis cs:0.065,0)
--(axis cs:0.066,0)
--(axis cs:0.067,0)
--(axis cs:0.068,0)
--(axis cs:0.069,0)
--(axis cs:0.07,0)
--(axis cs:0.071,0)
--(axis cs:0.072,0)
--(axis cs:0.073,0)
--(axis cs:0.074,0)
--(axis cs:0.075,0)
--(axis cs:0.076,0)
--(axis cs:0.077,0)
--(axis cs:0.078,0)
--(axis cs:0.079,0)
--(axis cs:0.08,0)
--(axis cs:0.081,0)
--(axis cs:0.082,0)
--(axis cs:0.083,0)
--(axis cs:0.084,0)
--(axis cs:0.085,0)
--(axis cs:0.086,0)
--(axis cs:0.087,0)
--(axis cs:0.088,0)
--(axis cs:0.089,0)
--(axis cs:0.09,0)
--(axis cs:0.091,0)
--(axis cs:0.092,0)
--(axis cs:0.093,0)
--(axis cs:0.094,0)
--(axis cs:0.095,0)
--(axis cs:0.096,0)
--(axis cs:0.097,0)
--(axis cs:0.098,0)
--(axis cs:0.099,0)
--(axis cs:0.1,0)
--(axis cs:0.101,0)
--(axis cs:0.102,0)
--(axis cs:0.103,0)
--(axis cs:0.104,0)
--(axis cs:0.105,0)
--(axis cs:0.106,0)
--(axis cs:0.107,0)
--(axis cs:0.108,0)
--(axis cs:0.109,0)
--(axis cs:0.11,0)
--(axis cs:0.111,0)
--(axis cs:0.112,0)
--(axis cs:0.113,0)
--(axis cs:0.114,0)
--(axis cs:0.115,0)
--(axis cs:0.116,0)
--(axis cs:0.117,0)
--(axis cs:0.118,0)
--(axis cs:0.119,0)
--(axis cs:0.12,0)
--(axis cs:0.121,0)
--(axis cs:0.122,0)
--(axis cs:0.123,0)
--(axis cs:0.124,0)
--(axis cs:0.125,0)
--(axis cs:0.126,0)
--(axis cs:0.127,0)
--(axis cs:0.128,0)
--(axis cs:0.129,0)
--(axis cs:0.13,0)
--(axis cs:0.131,0)
--(axis cs:0.132,0)
--(axis cs:0.133,0)
--(axis cs:0.134,0)
--(axis cs:0.135,0)
--(axis cs:0.136,0)
--(axis cs:0.137,0)
--(axis cs:0.138,0)
--(axis cs:0.139,0)
--(axis cs:0.14,0)
--(axis cs:0.141,0)
--(axis cs:0.142,0)
--(axis cs:0.143,0)
--(axis cs:0.144,0)
--(axis cs:0.145,0)
--(axis cs:0.146,0)
--(axis cs:0.147,0)
--(axis cs:0.148,0)
--(axis cs:0.149,0)
--(axis cs:0.15,0)
--(axis cs:0.151,0)
--(axis cs:0.152,0)
--(axis cs:0.153,0)
--(axis cs:0.154,0)
--(axis cs:0.155,0)
--(axis cs:0.156,0)
--(axis cs:0.157,0)
--(axis cs:0.158,0)
--(axis cs:0.159,0)
--(axis cs:0.16,0)
--(axis cs:0.161,0)
--(axis cs:0.162,0)
--(axis cs:0.163,0)
--(axis cs:0.164,0)
--(axis cs:0.165,0)
--(axis cs:0.166,0)
--(axis cs:0.167,0)
--(axis cs:0.168,0)
--(axis cs:0.169,0)
--(axis cs:0.17,0)
--(axis cs:0.171,0)
--(axis cs:0.172,0)
--(axis cs:0.173,0)
--(axis cs:0.174,0)
--(axis cs:0.175,0)
--(axis cs:0.176,0)
--(axis cs:0.177,0)
--(axis cs:0.178,0)
--(axis cs:0.179,0)
--(axis cs:0.18,0)
--(axis cs:0.181,0)
--(axis cs:0.182,0)
--(axis cs:0.183,0)
--(axis cs:0.184,0)
--(axis cs:0.185,0)
--(axis cs:0.186,0)
--(axis cs:0.187,0)
--(axis cs:0.188,0)
--(axis cs:0.189,0)
--(axis cs:0.19,0)
--(axis cs:0.191,0)
--(axis cs:0.192,0)
--(axis cs:0.193,0)
--(axis cs:0.194,0)
--(axis cs:0.195,0)
--(axis cs:0.196,0)
--(axis cs:0.197,0)
--(axis cs:0.198,0)
--(axis cs:0.199,0)
--(axis cs:0.2,0)
--(axis cs:0.201,0)
--(axis cs:0.202,0)
--(axis cs:0.203,0)
--(axis cs:0.204,0)
--(axis cs:0.205,0)
--(axis cs:0.206,0)
--(axis cs:0.207,0)
--(axis cs:0.208,0)
--(axis cs:0.209,0)
--(axis cs:0.21,0)
--(axis cs:0.211,0)
--(axis cs:0.212,0)
--(axis cs:0.213,0)
--(axis cs:0.214,0)
--(axis cs:0.215,0)
--(axis cs:0.216,0)
--(axis cs:0.217,0)
--(axis cs:0.218,0)
--(axis cs:0.219,0)
--(axis cs:0.22,0)
--(axis cs:0.221,0)
--(axis cs:0.222,0)
--(axis cs:0.223,0)
--(axis cs:0.224,0)
--(axis cs:0.225,0)
--(axis cs:0.226,0)
--(axis cs:0.227,0)
--(axis cs:0.228,0)
--(axis cs:0.229,0)
--(axis cs:0.23,0)
--(axis cs:0.231,0)
--(axis cs:0.232,0)
--(axis cs:0.233,0)
--(axis cs:0.234,0)
--(axis cs:0.235,0)
--(axis cs:0.236,0)
--(axis cs:0.237,0)
--(axis cs:0.238,0)
--(axis cs:0.239,0)
--(axis cs:0.24,0)
--(axis cs:0.241,0)
--(axis cs:0.242,0)
--(axis cs:0.243,0)
--(axis cs:0.244,0)
--(axis cs:0.245,0)
--(axis cs:0.246,0)
--(axis cs:0.247,0)
--(axis cs:0.248,0)
--(axis cs:0.249,0)
--(axis cs:0.25,0)
--(axis cs:0.251,0)
--(axis cs:0.252,0)
--(axis cs:0.253,0)
--(axis cs:0.254,0)
--(axis cs:0.255,0)
--(axis cs:0.256,0)
--(axis cs:0.257,0)
--(axis cs:0.258,0)
--(axis cs:0.259,0)
--(axis cs:0.26,0)
--(axis cs:0.261,0)
--(axis cs:0.262,0)
--(axis cs:0.263,0)
--(axis cs:0.264,0)
--(axis cs:0.265,0)
--(axis cs:0.266,0)
--(axis cs:0.267,0)
--(axis cs:0.268,0)
--(axis cs:0.269,0)
--(axis cs:0.27,0)
--(axis cs:0.271,0)
--(axis cs:0.272,0)
--(axis cs:0.273,0)
--(axis cs:0.274,0)
--(axis cs:0.275,0)
--(axis cs:0.276,0)
--(axis cs:0.277,0)
--(axis cs:0.278,0)
--(axis cs:0.279,0)
--(axis cs:0.28,0)
--(axis cs:0.281,0)
--(axis cs:0.282,0)
--(axis cs:0.283,0)
--(axis cs:0.284,0)
--(axis cs:0.285,0)
--(axis cs:0.286,0)
--(axis cs:0.287,0)
--(axis cs:0.288,0)
--(axis cs:0.289,0)
--(axis cs:0.29,0)
--(axis cs:0.291,0)
--(axis cs:0.292,0)
--(axis cs:0.293,0)
--(axis cs:0.294,0)
--(axis cs:0.295,0)
--(axis cs:0.296,0)
--(axis cs:0.297,0)
--(axis cs:0.298,0)
--(axis cs:0.299,0)
--(axis cs:0.3,0)
--(axis cs:0.301,0)
--(axis cs:0.302,0)
--(axis cs:0.303,0)
--(axis cs:0.304,0)
--(axis cs:0.305,0)
--(axis cs:0.306,0)
--(axis cs:0.307,0)
--(axis cs:0.308,0)
--(axis cs:0.309,0)
--(axis cs:0.31,0)
--(axis cs:0.311,0)
--(axis cs:0.312,0)
--(axis cs:0.313,0)
--(axis cs:0.314,0)
--(axis cs:0.315,0)
--(axis cs:0.316,0)
--(axis cs:0.317,0)
--(axis cs:0.318,0)
--(axis cs:0.319,0)
--(axis cs:0.32,0)
--(axis cs:0.321,0)
--(axis cs:0.322,0)
--(axis cs:0.323,0)
--(axis cs:0.324,0)
--(axis cs:0.325,0)
--(axis cs:0.326,0)
--(axis cs:0.327,0)
--(axis cs:0.328,0)
--(axis cs:0.329,0)
--(axis cs:0.33,0)
--(axis cs:0.331,0)
--(axis cs:0.332,0)
--(axis cs:0.333,0)
--(axis cs:0.334,0)
--(axis cs:0.335,0)
--(axis cs:0.336,0)
--(axis cs:0.337,0)
--(axis cs:0.338,0)
--(axis cs:0.339,0)
--(axis cs:0.34,0)
--(axis cs:0.341,0)
--(axis cs:0.342,0)
--(axis cs:0.343,0)
--(axis cs:0.344,0)
--(axis cs:0.345,0)
--(axis cs:0.346,0)
--(axis cs:0.347,0)
--(axis cs:0.348,0)
--(axis cs:0.349,0)
--(axis cs:0.35,0)
--(axis cs:0.351,0)
--(axis cs:0.352,0)
--(axis cs:0.353,0)
--(axis cs:0.354,0)
--(axis cs:0.355,0)
--(axis cs:0.356,0)
--(axis cs:0.357,0)
--(axis cs:0.358,0)
--(axis cs:0.359,0)
--(axis cs:0.36,0)
--(axis cs:0.361,0)
--(axis cs:0.362,0)
--(axis cs:0.363,0)
--(axis cs:0.364,0)
--(axis cs:0.365,0)
--(axis cs:0.366,0)
--(axis cs:0.367,0)
--(axis cs:0.368,0)
--(axis cs:0.369,0)
--(axis cs:0.37,0)
--(axis cs:0.371,0)
--(axis cs:0.372,0)
--(axis cs:0.373,0)
--(axis cs:0.374,0)
--(axis cs:0.375,0)
--(axis cs:0.376,0)
--(axis cs:0.377,0)
--(axis cs:0.378,0)
--(axis cs:0.379,0)
--(axis cs:0.38,0)
--(axis cs:0.381,0)
--(axis cs:0.382,0)
--(axis cs:0.383,0)
--(axis cs:0.384,0)
--(axis cs:0.385,0)
--(axis cs:0.386,0)
--(axis cs:0.387,0)
--(axis cs:0.388,0)
--(axis cs:0.389,0)
--(axis cs:0.39,0)
--(axis cs:0.391,0)
--(axis cs:0.392,0)
--(axis cs:0.393,0)
--(axis cs:0.394,0)
--(axis cs:0.395,0)
--(axis cs:0.396,0)
--(axis cs:0.397,0)
--(axis cs:0.398,0)
--(axis cs:0.399,0)
--(axis cs:0.4,0)
--(axis cs:0.401,0)
--(axis cs:0.402,0)
--(axis cs:0.403,0)
--(axis cs:0.404,0)
--(axis cs:0.405,0)
--(axis cs:0.406,0)
--(axis cs:0.407,0)
--(axis cs:0.408,0)
--(axis cs:0.409,0)
--(axis cs:0.41,0)
--(axis cs:0.411,0)
--(axis cs:0.412,0)
--(axis cs:0.413,0)
--(axis cs:0.414,0)
--(axis cs:0.415,0)
--(axis cs:0.416,0)
--(axis cs:0.417,0)
--(axis cs:0.418,0)
--(axis cs:0.419,0)
--(axis cs:0.42,0)
--(axis cs:0.421,0)
--(axis cs:0.422,0)
--(axis cs:0.423,0)
--(axis cs:0.424,0)
--(axis cs:0.425,0)
--(axis cs:0.426,0)
--(axis cs:0.427,0)
--(axis cs:0.428,0)
--(axis cs:0.429,0)
--(axis cs:0.43,0)
--(axis cs:0.431,0)
--(axis cs:0.432,0)
--(axis cs:0.433,0)
--(axis cs:0.434,0)
--(axis cs:0.435,0)
--(axis cs:0.436,0)
--(axis cs:0.437,0)
--(axis cs:0.438,0)
--(axis cs:0.439,0)
--(axis cs:0.44,0)
--(axis cs:0.441,0)
--(axis cs:0.442,0)
--(axis cs:0.443,0)
--(axis cs:0.444,0)
--(axis cs:0.445,0)
--(axis cs:0.446,0)
--(axis cs:0.447,0)
--(axis cs:0.448,0)
--(axis cs:0.449,0)
--(axis cs:0.45,0)
--(axis cs:0.451,0)
--(axis cs:0.452,0)
--(axis cs:0.453,0)
--(axis cs:0.454,0)
--(axis cs:0.455,0)
--(axis cs:0.456,0)
--(axis cs:0.457,0)
--(axis cs:0.458,0)
--(axis cs:0.459,0)
--(axis cs:0.46,0)
--(axis cs:0.461,0)
--(axis cs:0.462,0)
--(axis cs:0.463,0)
--(axis cs:0.464,0)
--(axis cs:0.465,0)
--(axis cs:0.466,0)
--(axis cs:0.467,0)
--(axis cs:0.468,0)
--(axis cs:0.469,0)
--(axis cs:0.47,0)
--(axis cs:0.471,0)
--(axis cs:0.472,0)
--(axis cs:0.473,0)
--(axis cs:0.474,0)
--(axis cs:0.475,0)
--(axis cs:0.476,0)
--(axis cs:0.477,0)
--(axis cs:0.478,0)
--(axis cs:0.479,0)
--(axis cs:0.48,0)
--(axis cs:0.481,0)
--(axis cs:0.482,0)
--(axis cs:0.483,0)
--(axis cs:0.484,0)
--(axis cs:0.485,0)
--(axis cs:0.486,0)
--(axis cs:0.487,0)
--(axis cs:0.488,0)
--(axis cs:0.489,0)
--(axis cs:0.49,0)
--(axis cs:0.491,0)
--(axis cs:0.492,0)
--(axis cs:0.493,0)
--(axis cs:0.494,0)
--(axis cs:0.495,0)
--(axis cs:0.496,0)
--(axis cs:0.497,0)
--(axis cs:0.498,0)
--(axis cs:0.499,0)
--(axis cs:0.5,0)
--(axis cs:0.501,0)
--(axis cs:0.502,0)
--(axis cs:0.503,0)
--(axis cs:0.504,0)
--(axis cs:0.505,0)
--(axis cs:0.506,0)
--(axis cs:0.507,0)
--(axis cs:0.508,0)
--(axis cs:0.509,0)
--(axis cs:0.51,0)
--(axis cs:0.511,0)
--(axis cs:0.512,0)
--(axis cs:0.513,0)
--(axis cs:0.514,0)
--(axis cs:0.515,0)
--(axis cs:0.516,0)
--(axis cs:0.517,0)
--(axis cs:0.518,0)
--(axis cs:0.519,0)
--(axis cs:0.52,0)
--(axis cs:0.521,0)
--(axis cs:0.522,0)
--(axis cs:0.523,0)
--(axis cs:0.524,0)
--(axis cs:0.525,0)
--(axis cs:0.526,0)
--(axis cs:0.527,0)
--(axis cs:0.528,0)
--(axis cs:0.529,0)
--(axis cs:0.53,0)
--(axis cs:0.531,0)
--(axis cs:0.532,0)
--(axis cs:0.533,0)
--(axis cs:0.534,0)
--(axis cs:0.535,0)
--(axis cs:0.536,0)
--(axis cs:0.537,0)
--(axis cs:0.538,0)
--(axis cs:0.539,0)
--(axis cs:0.54,0)
--(axis cs:0.541,0)
--(axis cs:0.542,0)
--(axis cs:0.543,0)
--(axis cs:0.544,0)
--(axis cs:0.545,0)
--(axis cs:0.546,0)
--(axis cs:0.547,0)
--(axis cs:0.548,0)
--(axis cs:0.549,0)
--(axis cs:0.55,0)
--(axis cs:0.551,0)
--(axis cs:0.552,0)
--(axis cs:0.553,0)
--(axis cs:0.554,0)
--(axis cs:0.555,0)
--(axis cs:0.556,0)
--(axis cs:0.557,0)
--(axis cs:0.558,0)
--(axis cs:0.559,0)
--(axis cs:0.56,0)
--(axis cs:0.561,0)
--(axis cs:0.562,0)
--(axis cs:0.563,0)
--(axis cs:0.564,0)
--(axis cs:0.565,0)
--(axis cs:0.566,0)
--(axis cs:0.567,0)
--(axis cs:0.568,0)
--(axis cs:0.569,0)
--(axis cs:0.57,0)
--(axis cs:0.571,0)
--(axis cs:0.572,0)
--(axis cs:0.573,0)
--(axis cs:0.574,0)
--(axis cs:0.575,0)
--(axis cs:0.576,0)
--(axis cs:0.577,0)
--(axis cs:0.578,0)
--(axis cs:0.579,0)
--(axis cs:0.58,0)
--(axis cs:0.581,0)
--(axis cs:0.582,0)
--(axis cs:0.583,0)
--(axis cs:0.584,0)
--(axis cs:0.585,0)
--(axis cs:0.586,0)
--(axis cs:0.587,0)
--(axis cs:0.588,0)
--(axis cs:0.589,0)
--(axis cs:0.59,0)
--(axis cs:0.591,0)
--(axis cs:0.592,0)
--(axis cs:0.593,0)
--(axis cs:0.594,0)
--(axis cs:0.595,0)
--(axis cs:0.596,0)
--(axis cs:0.597,0)
--(axis cs:0.598,0)
--(axis cs:0.599,0)
--(axis cs:0.6,0)
--(axis cs:0.601,0)
--(axis cs:0.602,0)
--(axis cs:0.603,0)
--(axis cs:0.604,0)
--(axis cs:0.605,0)
--(axis cs:0.606,0)
--(axis cs:0.607,0)
--(axis cs:0.608,0)
--(axis cs:0.609,0)
--(axis cs:0.61,0)
--(axis cs:0.611,0)
--(axis cs:0.612,0)
--(axis cs:0.613,0)
--(axis cs:0.614,0)
--(axis cs:0.615,0)
--(axis cs:0.616,0)
--(axis cs:0.617,0)
--(axis cs:0.618,0)
--(axis cs:0.619,0)
--(axis cs:0.62,0)
--(axis cs:0.621,0)
--(axis cs:0.622,0)
--(axis cs:0.623,0)
--(axis cs:0.624,0)
--(axis cs:0.625,0)
--(axis cs:0.626,0)
--(axis cs:0.627,0)
--(axis cs:0.628,0)
--(axis cs:0.629,0)
--(axis cs:0.63,0)
--(axis cs:0.631,0)
--(axis cs:0.632,0)
--(axis cs:0.633,0)
--(axis cs:0.634,0)
--(axis cs:0.635,0)
--(axis cs:0.636,0)
--(axis cs:0.637,0)
--(axis cs:0.638,0)
--(axis cs:0.639,0)
--(axis cs:0.64,0)
--(axis cs:0.641,0)
--(axis cs:0.642,0)
--(axis cs:0.643,0)
--(axis cs:0.644,0)
--(axis cs:0.645,0)
--(axis cs:0.646,0)
--(axis cs:0.647,0)
--(axis cs:0.648,0)
--(axis cs:0.649,0)
--(axis cs:0.65,0)
--(axis cs:0.651,0)
--(axis cs:0.652,0)
--(axis cs:0.653,0)
--(axis cs:0.654,0)
--(axis cs:0.655,0)
--(axis cs:0.656,0)
--(axis cs:0.657,0)
--(axis cs:0.658,0)
--(axis cs:0.659,0)
--(axis cs:0.66,0)
--(axis cs:0.661,0)
--(axis cs:0.662,0)
--(axis cs:0.663,0)
--(axis cs:0.664,0)
--(axis cs:0.665,0)
--(axis cs:0.666,0)
--(axis cs:0.667,0)
--(axis cs:0.668,0)
--(axis cs:0.669,0)
--(axis cs:0.67,0)
--(axis cs:0.671,0)
--(axis cs:0.672,0)
--(axis cs:0.673,0)
--(axis cs:0.674,0)
--(axis cs:0.675,0)
--(axis cs:0.676,0)
--(axis cs:0.677,0)
--(axis cs:0.678,0)
--(axis cs:0.679,0)
--(axis cs:0.68,0)
--(axis cs:0.681,0)
--(axis cs:0.682,0)
--(axis cs:0.683,0)
--(axis cs:0.684,0)
--(axis cs:0.685,0)
--(axis cs:0.686,0)
--(axis cs:0.687,0)
--(axis cs:0.688,0)
--(axis cs:0.689,0)
--(axis cs:0.69,0)
--(axis cs:0.691,0)
--(axis cs:0.692,0)
--(axis cs:0.693,0)
--(axis cs:0.694,0)
--(axis cs:0.695,0)
--(axis cs:0.696,0)
--(axis cs:0.697,0)
--(axis cs:0.698,0)
--(axis cs:0.699,0)
--(axis cs:0.7,0)
--(axis cs:0.701,0)
--(axis cs:0.702,0)
--(axis cs:0.703,0)
--(axis cs:0.704,0)
--(axis cs:0.705,0)
--(axis cs:0.706,0)
--(axis cs:0.707,0)
--(axis cs:0.708,0)
--(axis cs:0.709,0)
--(axis cs:0.71,0)
--(axis cs:0.711,0)
--(axis cs:0.712,0)
--(axis cs:0.713,0)
--(axis cs:0.714,0)
--(axis cs:0.715,0)
--(axis cs:0.716,0)
--(axis cs:0.717,0)
--(axis cs:0.718,0)
--(axis cs:0.719,0)
--(axis cs:0.72,0)
--(axis cs:0.721,0)
--(axis cs:0.722,0)
--(axis cs:0.723,0)
--(axis cs:0.724,0)
--(axis cs:0.725,0)
--(axis cs:0.726,0)
--(axis cs:0.727,0)
--(axis cs:0.728,0)
--(axis cs:0.729,0)
--(axis cs:0.73,0)
--(axis cs:0.731,0)
--(axis cs:0.732,0)
--(axis cs:0.733,0)
--(axis cs:0.734,0)
--(axis cs:0.735,0)
--(axis cs:0.736,0)
--(axis cs:0.737,0)
--(axis cs:0.738,0)
--(axis cs:0.739,0)
--(axis cs:0.74,0)
--(axis cs:0.741,0)
--(axis cs:0.742,0)
--(axis cs:0.743,0)
--(axis cs:0.744,0)
--(axis cs:0.745,0)
--(axis cs:0.746,0)
--(axis cs:0.747,0)
--(axis cs:0.748,0)
--(axis cs:0.749,0)
--(axis cs:0.75,0)
--(axis cs:0.751,0)
--(axis cs:0.752,0)
--(axis cs:0.753,0)
--(axis cs:0.754,0)
--(axis cs:0.755,0)
--(axis cs:0.756,0)
--(axis cs:0.757,0)
--(axis cs:0.758,0)
--(axis cs:0.759,0)
--(axis cs:0.76,0)
--(axis cs:0.761,0)
--(axis cs:0.762,0)
--(axis cs:0.763,0)
--(axis cs:0.764,0)
--(axis cs:0.765,0)
--(axis cs:0.766,0)
--(axis cs:0.767,0)
--(axis cs:0.768,0)
--(axis cs:0.769,0)
--(axis cs:0.77,0)
--(axis cs:0.771,0)
--(axis cs:0.772,0)
--(axis cs:0.773,0)
--(axis cs:0.774,0)
--(axis cs:0.775,0)
--(axis cs:0.776,0)
--(axis cs:0.777,0)
--(axis cs:0.778,0)
--(axis cs:0.779,0)
--(axis cs:0.78,0)
--(axis cs:0.781,0)
--(axis cs:0.782,0)
--(axis cs:0.783,0)
--(axis cs:0.784,0)
--(axis cs:0.785,0)
--(axis cs:0.786,0)
--(axis cs:0.787,0)
--(axis cs:0.788,0)
--(axis cs:0.789,0)
--(axis cs:0.79,0)
--(axis cs:0.791,0)
--(axis cs:0.792,0)
--(axis cs:0.793,0)
--(axis cs:0.794,0)
--(axis cs:0.795,0)
--(axis cs:0.796,0)
--(axis cs:0.797,0)
--(axis cs:0.798,0)
--(axis cs:0.799,0)
--(axis cs:0.8,0)
--(axis cs:0.801,0)
--(axis cs:0.802,0)
--(axis cs:0.803,0)
--(axis cs:0.804,0)
--(axis cs:0.805,0)
--(axis cs:0.806,0)
--(axis cs:0.807,0)
--(axis cs:0.808,0)
--(axis cs:0.809,0)
--(axis cs:0.81,0)
--(axis cs:0.811,0)
--(axis cs:0.812,0)
--(axis cs:0.813,0)
--(axis cs:0.814,0)
--(axis cs:0.815,0)
--(axis cs:0.816,0)
--(axis cs:0.817,0)
--(axis cs:0.818,0)
--(axis cs:0.819,0)
--(axis cs:0.82,0)
--(axis cs:0.821,0)
--(axis cs:0.822,0)
--(axis cs:0.823,0)
--(axis cs:0.824,0)
--(axis cs:0.825,0)
--(axis cs:0.826,0)
--(axis cs:0.827,0)
--(axis cs:0.828,0)
--(axis cs:0.829,0)
--(axis cs:0.83,0)
--(axis cs:0.831,0)
--(axis cs:0.832,0)
--(axis cs:0.833,0)
--(axis cs:0.834,0)
--(axis cs:0.835,0)
--(axis cs:0.836,0)
--(axis cs:0.837,0)
--(axis cs:0.838,0)
--(axis cs:0.839,0)
--(axis cs:0.84,0)
--(axis cs:0.841,0)
--(axis cs:0.842,0)
--(axis cs:0.843,0)
--(axis cs:0.844,0)
--(axis cs:0.845,0)
--(axis cs:0.846,0)
--(axis cs:0.847,0)
--(axis cs:0.848,0)
--(axis cs:0.849,0)
--(axis cs:0.85,0)
--(axis cs:0.851,0)
--(axis cs:0.852,0)
--(axis cs:0.853,0)
--(axis cs:0.854,0)
--(axis cs:0.855,0)
--(axis cs:0.856,0)
--(axis cs:0.857,0)
--(axis cs:0.858,0)
--(axis cs:0.859,0)
--(axis cs:0.86,0)
--(axis cs:0.861,0)
--(axis cs:0.862,0)
--(axis cs:0.863,0)
--(axis cs:0.864,0)
--(axis cs:0.865,0)
--(axis cs:0.866,0)
--(axis cs:0.867,0)
--(axis cs:0.868,0)
--(axis cs:0.869,0)
--(axis cs:0.87,0)
--(axis cs:0.871,0)
--(axis cs:0.872,0)
--(axis cs:0.873,0)
--(axis cs:0.874,0)
--(axis cs:0.875,0)
--(axis cs:0.876,0)
--(axis cs:0.877,0)
--(axis cs:0.878,0)
--(axis cs:0.879,0)
--(axis cs:0.88,0)
--(axis cs:0.881,0)
--(axis cs:0.882,0)
--(axis cs:0.883,0)
--(axis cs:0.884,0)
--(axis cs:0.885,0)
--(axis cs:0.886,0)
--cycle;
\path [draw=none, fill=blue, fill opacity=0.5]
(axis cs:0,0)
--(axis cs:0.001,0)
--(axis cs:0.002,0)
--(axis cs:0.003,0)
--(axis cs:0.004,0)
--(axis cs:0.005,0)
--(axis cs:0.006,0)
--(axis cs:0.007,0)
--(axis cs:0.008,0)
--(axis cs:0.009,0)
--(axis cs:0.01,0)
--(axis cs:0.011,0)
--(axis cs:0.012,0)
--(axis cs:0.013,0)
--(axis cs:0.014,0)
--(axis cs:0.015,0)
--(axis cs:0.016,0)
--(axis cs:0.017,0)
--(axis cs:0.018,0)
--(axis cs:0.019,0)
--(axis cs:0.02,0)
--(axis cs:0.021,0)
--(axis cs:0.022,0)
--(axis cs:0.023,0)
--(axis cs:0.024,0)
--(axis cs:0.025,0)
--(axis cs:0.026,0)
--(axis cs:0.027,0)
--(axis cs:0.028,0)
--(axis cs:0.029,0)
--(axis cs:0.03,0)
--(axis cs:0.031,0)
--(axis cs:0.032,0)
--(axis cs:0.033,0)
--(axis cs:0.034,0)
--(axis cs:0.035,0)
--(axis cs:0.036,0)
--(axis cs:0.037,0)
--(axis cs:0.038,0)
--(axis cs:0.039,0)
--(axis cs:0.04,0)
--(axis cs:0.041,0)
--(axis cs:0.042,0)
--(axis cs:0.043,0)
--(axis cs:0.044,0)
--(axis cs:0.045,0)
--(axis cs:0.046,0)
--(axis cs:0.047,0)
--(axis cs:0.048,0)
--(axis cs:0.049,0)
--(axis cs:0.05,0)
--(axis cs:0.051,0)
--(axis cs:0.052,0)
--(axis cs:0.053,0)
--(axis cs:0.054,0)
--(axis cs:0.055,0)
--(axis cs:0.056,0)
--(axis cs:0.057,0)
--(axis cs:0.058,0)
--(axis cs:0.059,0)
--(axis cs:0.06,0)
--(axis cs:0.061,0)
--(axis cs:0.062,0)
--(axis cs:0.063,0)
--(axis cs:0.064,0)
--(axis cs:0.065,0)
--(axis cs:0.066,0)
--(axis cs:0.067,0)
--(axis cs:0.068,0)
--(axis cs:0.069,0)
--(axis cs:0.07,0)
--(axis cs:0.071,0)
--(axis cs:0.072,0)
--(axis cs:0.073,0)
--(axis cs:0.074,0)
--(axis cs:0.075,0)
--(axis cs:0.076,0)
--(axis cs:0.077,0)
--(axis cs:0.078,0)
--(axis cs:0.079,0)
--(axis cs:0.08,0)
--(axis cs:0.081,0)
--(axis cs:0.082,0)
--(axis cs:0.083,0)
--(axis cs:0.084,0)
--(axis cs:0.085,0)
--(axis cs:0.086,0)
--(axis cs:0.087,0)
--(axis cs:0.088,0)
--(axis cs:0.089,0)
--(axis cs:0.09,0)
--(axis cs:0.091,0)
--(axis cs:0.092,0)
--(axis cs:0.093,0)
--(axis cs:0.094,0)
--(axis cs:0.095,0)
--(axis cs:0.096,0)
--(axis cs:0.097,0)
--(axis cs:0.098,0)
--(axis cs:0.099,0)
--(axis cs:0.1,0)
--(axis cs:0.101,0)
--(axis cs:0.102,0)
--(axis cs:0.103,0)
--(axis cs:0.104,0)
--(axis cs:0.105,0)
--(axis cs:0.106,0)
--(axis cs:0.107,0)
--(axis cs:0.108,0)
--(axis cs:0.109,0)
--(axis cs:0.11,0)
--(axis cs:0.111,0)
--(axis cs:0.112,0)
--(axis cs:0.113,0)
--(axis cs:0.114,0)
--(axis cs:0.115,0)
--(axis cs:0.116,0)
--(axis cs:0.117,0)
--(axis cs:0.118,0)
--(axis cs:0.119,0)
--(axis cs:0.12,0)
--(axis cs:0.121,0)
--(axis cs:0.122,0)
--(axis cs:0.123,0)
--(axis cs:0.124,0)
--(axis cs:0.125,0)
--(axis cs:0.126,0)
--(axis cs:0.127,0)
--(axis cs:0.128,0)
--(axis cs:0.129,0)
--(axis cs:0.13,0)
--(axis cs:0.131,0)
--(axis cs:0.132,0)
--(axis cs:0.133,0)
--(axis cs:0.134,0)
--(axis cs:0.135,0)
--(axis cs:0.136,0)
--(axis cs:0.137,0)
--(axis cs:0.138,0)
--(axis cs:0.139,0)
--(axis cs:0.14,0)
--(axis cs:0.141,0)
--(axis cs:0.142,0)
--(axis cs:0.143,0)
--(axis cs:0.144,0)
--(axis cs:0.145,0)
--(axis cs:0.146,0)
--(axis cs:0.147,0)
--(axis cs:0.148,0)
--(axis cs:0.149,0)
--(axis cs:0.15,0)
--(axis cs:0.151,0)
--(axis cs:0.152,0)
--(axis cs:0.153,0)
--(axis cs:0.154,0)
--(axis cs:0.155,0)
--(axis cs:0.156,0)
--(axis cs:0.157,0)
--(axis cs:0.158,0)
--(axis cs:0.159,0)
--(axis cs:0.16,0)
--(axis cs:0.161,0)
--(axis cs:0.162,0)
--(axis cs:0.163,0)
--(axis cs:0.164,0)
--(axis cs:0.165,0)
--(axis cs:0.166,0)
--(axis cs:0.167,0)
--(axis cs:0.168,0)
--(axis cs:0.169,0)
--(axis cs:0.17,0)
--(axis cs:0.171,0)
--(axis cs:0.172,0)
--(axis cs:0.173,0)
--(axis cs:0.174,0)
--(axis cs:0.175,0)
--(axis cs:0.176,0)
--(axis cs:0.177,0)
--(axis cs:0.178,0)
--(axis cs:0.179,0)
--(axis cs:0.18,0)
--(axis cs:0.181,0)
--(axis cs:0.182,0)
--(axis cs:0.183,0)
--(axis cs:0.184,0)
--(axis cs:0.185,0)
--(axis cs:0.186,0)
--(axis cs:0.187,0)
--(axis cs:0.188,0)
--(axis cs:0.189,0)
--(axis cs:0.19,0)
--(axis cs:0.191,0)
--(axis cs:0.192,0)
--(axis cs:0.193,0)
--(axis cs:0.194,0)
--(axis cs:0.195,0)
--(axis cs:0.196,0)
--(axis cs:0.197,0)
--(axis cs:0.198,0)
--(axis cs:0.199,0)
--(axis cs:0.2,0)
--(axis cs:0.201,0)
--(axis cs:0.202,0)
--(axis cs:0.203,0)
--(axis cs:0.204,0)
--(axis cs:0.205,0)
--(axis cs:0.206,0)
--(axis cs:0.207,0)
--(axis cs:0.208,0)
--(axis cs:0.209,0)
--(axis cs:0.21,0)
--(axis cs:0.211,0)
--(axis cs:0.212,0)
--(axis cs:0.213,0)
--(axis cs:0.214,0)
--(axis cs:0.215,0)
--(axis cs:0.216,0)
--(axis cs:0.217,0)
--(axis cs:0.218,0)
--(axis cs:0.219,0)
--(axis cs:0.22,0)
--(axis cs:0.221,0)
--(axis cs:0.222,0)
--(axis cs:0.223,0)
--(axis cs:0.224,0)
--(axis cs:0.225,0)
--(axis cs:0.226,0)
--(axis cs:0.227,0)
--(axis cs:0.228,0)
--(axis cs:0.229,0)
--(axis cs:0.23,0)
--(axis cs:0.231,0)
--(axis cs:0.232,0)
--(axis cs:0.233,0)
--(axis cs:0.234,0)
--(axis cs:0.235,0)
--(axis cs:0.236,0)
--(axis cs:0.237,0)
--(axis cs:0.238,0)
--(axis cs:0.239,0)
--(axis cs:0.24,0)
--(axis cs:0.241,0)
--(axis cs:0.242,0)
--(axis cs:0.243,0)
--(axis cs:0.244,0)
--(axis cs:0.245,0)
--(axis cs:0.246,0)
--(axis cs:0.247,0)
--(axis cs:0.248,0)
--(axis cs:0.249,0)
--(axis cs:0.25,0)
--(axis cs:0.251,0)
--(axis cs:0.252,0)
--(axis cs:0.253,0)
--(axis cs:0.254,0)
--(axis cs:0.255,0)
--(axis cs:0.256,0)
--(axis cs:0.257,0)
--(axis cs:0.258,0)
--(axis cs:0.259,0)
--(axis cs:0.26,0)
--(axis cs:0.261,0)
--(axis cs:0.262,0)
--(axis cs:0.263,0)
--(axis cs:0.264,0)
--(axis cs:0.265,0)
--(axis cs:0.266,0)
--(axis cs:0.267,0)
--(axis cs:0.268,0)
--(axis cs:0.269,0)
--(axis cs:0.27,0)
--(axis cs:0.271,0)
--(axis cs:0.272,0)
--(axis cs:0.273,0)
--(axis cs:0.274,0)
--(axis cs:0.275,0)
--(axis cs:0.276,0)
--(axis cs:0.277,0)
--(axis cs:0.278,0)
--(axis cs:0.279,0)
--(axis cs:0.28,0)
--(axis cs:0.281,0)
--(axis cs:0.282,0)
--(axis cs:0.283,0)
--(axis cs:0.284,0)
--(axis cs:0.285,0)
--(axis cs:0.286,0)
--(axis cs:0.287,0)
--(axis cs:0.288,0)
--(axis cs:0.289,0)
--(axis cs:0.29,0)
--(axis cs:0.291,0)
--(axis cs:0.292,0)
--(axis cs:0.293,0)
--(axis cs:0.294,0)
--(axis cs:0.295,0)
--(axis cs:0.296,0)
--(axis cs:0.297,0)
--(axis cs:0.298,0)
--(axis cs:0.299,0)
--(axis cs:0.3,0)
--(axis cs:0.301,0)
--(axis cs:0.302,0)
--(axis cs:0.303,0)
--(axis cs:0.304,0)
--(axis cs:0.305,0)
--(axis cs:0.306,0)
--(axis cs:0.307,0)
--(axis cs:0.308,0)
--(axis cs:0.309,0)
--(axis cs:0.31,0)
--(axis cs:0.311,0)
--(axis cs:0.312,0)
--(axis cs:0.313,0)
--(axis cs:0.314,0)
--(axis cs:0.315,0)
--(axis cs:0.316,0)
--(axis cs:0.317,0)
--(axis cs:0.318,0)
--(axis cs:0.319,0)
--(axis cs:0.32,0)
--(axis cs:0.321,0)
--(axis cs:0.322,0)
--(axis cs:0.323,0)
--(axis cs:0.324,0)
--(axis cs:0.325,0)
--(axis cs:0.326,0)
--(axis cs:0.327,0)
--(axis cs:0.328,0)
--(axis cs:0.329,0)
--(axis cs:0.33,0)
--(axis cs:0.331,0)
--(axis cs:0.332,0)
--(axis cs:0.333,0)
--(axis cs:0.334,0)
--(axis cs:0.335,0)
--(axis cs:0.336,0)
--(axis cs:0.337,0)
--(axis cs:0.338,0)
--(axis cs:0.339,0)
--(axis cs:0.34,0)
--(axis cs:0.341,0)
--(axis cs:0.342,0)
--(axis cs:0.343,0)
--(axis cs:0.344,0)
--(axis cs:0.345,0)
--(axis cs:0.346,0)
--(axis cs:0.347,0)
--(axis cs:0.348,0)
--(axis cs:0.349,0)
--(axis cs:0.35,0)
--(axis cs:0.351,0)
--(axis cs:0.352,0)
--(axis cs:0.353,0)
--(axis cs:0.354,0)
--(axis cs:0.355,0)
--(axis cs:0.356,0)
--(axis cs:0.357,0)
--(axis cs:0.358,0)
--(axis cs:0.359,0)
--(axis cs:0.36,0)
--(axis cs:0.361,0)
--(axis cs:0.362,0)
--(axis cs:0.363,0)
--(axis cs:0.364,0)
--(axis cs:0.365,0)
--(axis cs:0.366,0)
--(axis cs:0.367,0)
--(axis cs:0.368,0)
--(axis cs:0.369,0)
--(axis cs:0.37,0)
--(axis cs:0.371,0)
--(axis cs:0.372,0)
--(axis cs:0.373,0)
--(axis cs:0.374,0)
--(axis cs:0.375,0)
--(axis cs:0.376,0)
--(axis cs:0.377,0)
--(axis cs:0.378,0)
--(axis cs:0.379,0)
--(axis cs:0.38,0)
--(axis cs:0.381,0)
--(axis cs:0.382,0)
--(axis cs:0.383,0)
--(axis cs:0.384,0)
--(axis cs:0.385,0)
--(axis cs:0.386,0)
--(axis cs:0.387,0)
--(axis cs:0.388,0)
--(axis cs:0.389,0)
--(axis cs:0.39,0)
--(axis cs:0.391,0)
--(axis cs:0.392,0)
--(axis cs:0.393,0)
--(axis cs:0.394,0)
--(axis cs:0.395,0)
--(axis cs:0.396,0)
--(axis cs:0.397,0)
--(axis cs:0.398,0)
--(axis cs:0.399,0)
--(axis cs:0.4,0)
--(axis cs:0.401,0)
--(axis cs:0.402,0)
--(axis cs:0.403,0)
--(axis cs:0.404,0)
--(axis cs:0.405,0)
--(axis cs:0.406,0)
--(axis cs:0.407,0)
--(axis cs:0.408,0)
--(axis cs:0.409,0)
--(axis cs:0.41,0)
--(axis cs:0.411,0)
--(axis cs:0.412,0)
--(axis cs:0.413,0)
--(axis cs:0.414,0)
--(axis cs:0.415,0)
--(axis cs:0.416,0)
--(axis cs:0.417,0)
--(axis cs:0.418,0)
--(axis cs:0.419,0)
--(axis cs:0.42,0)
--(axis cs:0.421,0)
--(axis cs:0.422,0)
--(axis cs:0.423,0)
--(axis cs:0.424,0)
--(axis cs:0.425,0)
--(axis cs:0.426,0)
--(axis cs:0.427,0)
--(axis cs:0.428,0)
--(axis cs:0.429,0)
--(axis cs:0.43,0)
--(axis cs:0.431,0)
--(axis cs:0.432,0)
--(axis cs:0.433,0)
--(axis cs:0.434,0)
--(axis cs:0.435,0)
--(axis cs:0.436,0)
--(axis cs:0.437,0)
--(axis cs:0.438,0)
--(axis cs:0.439,0)
--(axis cs:0.44,0)
--(axis cs:0.441,0)
--(axis cs:0.442,0)
--(axis cs:0.443,0)
--(axis cs:0.444,0)
--(axis cs:0.445,0)
--(axis cs:0.446,0)
--(axis cs:0.447,0)
--(axis cs:0.448,0)
--(axis cs:0.449,0)
--(axis cs:0.45,0)
--(axis cs:0.451,0)
--(axis cs:0.452,0)
--(axis cs:0.453,0)
--(axis cs:0.454,0)
--(axis cs:0.455,0)
--(axis cs:0.456,0)
--(axis cs:0.457,0)
--(axis cs:0.458,0)
--(axis cs:0.459,0)
--(axis cs:0.46,0)
--(axis cs:0.461,0)
--(axis cs:0.462,0)
--(axis cs:0.463,0)
--(axis cs:0.464,0)
--(axis cs:0.465,0)
--(axis cs:0.466,0)
--(axis cs:0.467,0)
--(axis cs:0.468,0)
--(axis cs:0.469,0)
--(axis cs:0.47,0)
--(axis cs:0.471,0)
--(axis cs:0.472,0)
--(axis cs:0.473,0)
--(axis cs:0.474,0)
--(axis cs:0.475,0)
--(axis cs:0.476,0)
--(axis cs:0.477,0)
--(axis cs:0.478,0)
--(axis cs:0.479,0)
--(axis cs:0.48,0)
--(axis cs:0.481,0)
--(axis cs:0.482,0)
--(axis cs:0.483,0)
--(axis cs:0.484,0)
--(axis cs:0.485,0)
--(axis cs:0.486,0)
--(axis cs:0.487,0)
--(axis cs:0.488,0)
--(axis cs:0.489,0)
--(axis cs:0.49,0)
--(axis cs:0.491,0)
--(axis cs:0.492,0)
--(axis cs:0.493,0)
--(axis cs:0.494,0)
--(axis cs:0.495,0)
--(axis cs:0.496,0)
--(axis cs:0.497,0)
--(axis cs:0.498,0)
--(axis cs:0.499,0)
--(axis cs:0.5,0)
--(axis cs:0.501,0)
--(axis cs:0.502,0)
--(axis cs:0.503,0)
--(axis cs:0.504,0)
--(axis cs:0.505,0)
--(axis cs:0.506,0)
--(axis cs:0.507,0)
--(axis cs:0.508,0)
--(axis cs:0.509,0)
--(axis cs:0.51,0)
--(axis cs:0.511,0)
--(axis cs:0.512,0)
--(axis cs:0.513,0)
--(axis cs:0.514,0)
--(axis cs:0.515,0)
--(axis cs:0.516,0)
--(axis cs:0.517,0)
--(axis cs:0.518,0)
--(axis cs:0.519,0)
--(axis cs:0.52,0)
--(axis cs:0.521,0)
--(axis cs:0.522,0)
--(axis cs:0.523,0)
--(axis cs:0.524,0)
--(axis cs:0.525,0)
--(axis cs:0.526,0)
--(axis cs:0.527,0)
--(axis cs:0.528,0)
--(axis cs:0.529,0)
--(axis cs:0.53,0)
--(axis cs:0.531,0)
--(axis cs:0.532,0)
--(axis cs:0.533,0)
--(axis cs:0.534,0)
--(axis cs:0.535,0)
--(axis cs:0.536,0)
--(axis cs:0.537,0)
--(axis cs:0.538,0)
--(axis cs:0.539,0)
--(axis cs:0.54,0)
--(axis cs:0.541,0)
--(axis cs:0.542,0)
--(axis cs:0.543,0)
--(axis cs:0.544,0)
--(axis cs:0.545,0)
--(axis cs:0.546,0)
--(axis cs:0.547,0)
--(axis cs:0.548,0)
--(axis cs:0.549,0)
--(axis cs:0.55,0)
--(axis cs:0.551,0)
--(axis cs:0.552,0)
--(axis cs:0.553,0)
--(axis cs:0.554,0)
--(axis cs:0.555,0)
--(axis cs:0.556,0)
--(axis cs:0.557,0)
--(axis cs:0.558,0)
--(axis cs:0.559,0)
--(axis cs:0.56,0)
--(axis cs:0.561,0)
--(axis cs:0.562,0)
--(axis cs:0.563,0)
--(axis cs:0.564,0)
--(axis cs:0.565,0)
--(axis cs:0.566,0)
--(axis cs:0.567,0)
--(axis cs:0.568,0)
--(axis cs:0.569,0)
--(axis cs:0.57,0)
--(axis cs:0.571,0)
--(axis cs:0.572,0)
--(axis cs:0.573,0)
--(axis cs:0.574,0)
--(axis cs:0.575,0)
--(axis cs:0.576,0)
--(axis cs:0.577,0)
--(axis cs:0.578,0)
--(axis cs:0.579,0)
--(axis cs:0.58,0)
--(axis cs:0.581,0)
--(axis cs:0.582,0)
--(axis cs:0.583,0)
--(axis cs:0.584,0)
--(axis cs:0.585,0)
--(axis cs:0.586,0)
--(axis cs:0.587,0)
--(axis cs:0.588,0)
--(axis cs:0.589,0)
--(axis cs:0.59,0)
--(axis cs:0.591,0)
--(axis cs:0.592,0)
--(axis cs:0.593,0)
--(axis cs:0.594,0)
--(axis cs:0.595,0)
--(axis cs:0.596,0)
--(axis cs:0.597,0)
--(axis cs:0.598,0)
--(axis cs:0.599,0)
--(axis cs:0.6,0)
--(axis cs:0.601,0)
--(axis cs:0.602,0)
--(axis cs:0.603,0)
--(axis cs:0.604,0)
--(axis cs:0.605,0)
--(axis cs:0.606,0)
--(axis cs:0.607,0)
--(axis cs:0.608,0)
--(axis cs:0.609,0)
--(axis cs:0.61,0)
--(axis cs:0.611,0)
--(axis cs:0.612,0)
--(axis cs:0.613,0)
--(axis cs:0.614,0)
--(axis cs:0.615,0)
--(axis cs:0.616,0)
--(axis cs:0.617,0)
--(axis cs:0.618,0)
--(axis cs:0.619,0)
--(axis cs:0.62,0)
--(axis cs:0.621,0)
--(axis cs:0.622,0)
--(axis cs:0.623,0)
--(axis cs:0.624,0)
--(axis cs:0.625,0)
--(axis cs:0.626,0)
--(axis cs:0.627,0)
--(axis cs:0.628,0)
--(axis cs:0.629,0)
--(axis cs:0.63,0)
--(axis cs:0.631,0)
--(axis cs:0.632,0)
--(axis cs:0.633,0)
--(axis cs:0.634,0)
--(axis cs:0.635,0)
--(axis cs:0.636,0)
--(axis cs:0.637,0)
--(axis cs:0.638,0)
--(axis cs:0.639,0)
--(axis cs:0.64,0)
--(axis cs:0.641,0)
--(axis cs:0.642,0)
--(axis cs:0.643,0)
--(axis cs:0.644,0)
--(axis cs:0.645,0)
--(axis cs:0.646,0)
--(axis cs:0.647,0)
--(axis cs:0.648,0)
--(axis cs:0.649,0)
--(axis cs:0.65,0)
--(axis cs:0.651,0)
--(axis cs:0.652,0)
--(axis cs:0.653,0)
--(axis cs:0.654,0)
--(axis cs:0.655,0)
--(axis cs:0.656,0)
--(axis cs:0.657,0)
--(axis cs:0.658,0)
--(axis cs:0.659,0)
--(axis cs:0.66,0)
--(axis cs:0.661,0)
--(axis cs:0.662,0)
--(axis cs:0.663,0)
--(axis cs:0.664,0)
--(axis cs:0.665,0)
--(axis cs:0.666,0)
--(axis cs:0.667,0)
--(axis cs:0.668,0)
--(axis cs:0.669,0)
--(axis cs:0.67,0)
--(axis cs:0.671,0)
--(axis cs:0.672,0)
--(axis cs:0.673,0)
--(axis cs:0.674,0)
--(axis cs:0.675,0)
--(axis cs:0.676,0)
--(axis cs:0.677,0)
--(axis cs:0.678,0)
--(axis cs:0.679,0)
--(axis cs:0.68,0)
--(axis cs:0.681,0)
--(axis cs:0.682,0)
--(axis cs:0.683,0)
--(axis cs:0.684,0)
--(axis cs:0.685,0)
--(axis cs:0.686,0)
--(axis cs:0.687,0)
--(axis cs:0.688,0)
--(axis cs:0.689,0)
--(axis cs:0.69,0)
--(axis cs:0.691,0)
--(axis cs:0.692,0)
--(axis cs:0.693,0)
--(axis cs:0.694,0)
--(axis cs:0.695,0)
--(axis cs:0.696,0)
--(axis cs:0.697,0)
--(axis cs:0.698,0)
--(axis cs:0.699,0)
--(axis cs:0.7,0)
--(axis cs:0.701,0)
--(axis cs:0.702,0)
--(axis cs:0.703,0)
--(axis cs:0.704,0)
--(axis cs:0.705,0)
--(axis cs:0.706,0)
--(axis cs:0.707,0)
--(axis cs:0.708,0)
--(axis cs:0.709,0)
--(axis cs:0.71,0)
--(axis cs:0.711,0)
--(axis cs:0.712,0)
--(axis cs:0.713,0)
--(axis cs:0.714,0)
--(axis cs:0.715,0)
--(axis cs:0.716,0)
--(axis cs:0.717,0)
--(axis cs:0.718,0)
--(axis cs:0.719,0)
--(axis cs:0.72,0)
--(axis cs:0.721,0)
--(axis cs:0.722,0)
--(axis cs:0.723,0)
--(axis cs:0.724,0)
--(axis cs:0.725,0)
--(axis cs:0.726,0)
--(axis cs:0.727,0)
--(axis cs:0.728,0)
--(axis cs:0.729,0)
--(axis cs:0.73,0)
--(axis cs:0.731,0)
--(axis cs:0.732,0)
--(axis cs:0.733,0)
--(axis cs:0.734,0)
--(axis cs:0.735,0)
--(axis cs:0.736,0)
--(axis cs:0.737,0)
--(axis cs:0.738,0)
--(axis cs:0.739,0)
--(axis cs:0.74,0)
--(axis cs:0.741,0)
--(axis cs:0.742,0)
--(axis cs:0.743,0)
--(axis cs:0.744,0)
--(axis cs:0.745,0)
--(axis cs:0.746,0)
--(axis cs:0.747,0)
--(axis cs:0.748,0)
--(axis cs:0.749,0)
--(axis cs:0.75,0)
--(axis cs:0.751,0)
--(axis cs:0.752,0)
--(axis cs:0.753,0)
--(axis cs:0.754,0)
--(axis cs:0.755,0)
--(axis cs:0.756,0)
--(axis cs:0.757,0)
--(axis cs:0.758,0)
--(axis cs:0.759,0)
--(axis cs:0.76,0)
--(axis cs:0.761,0)
--(axis cs:0.762,0)
--(axis cs:0.763,0)
--(axis cs:0.764,0)
--(axis cs:0.765,0)
--(axis cs:0.766,0)
--(axis cs:0.767,0)
--(axis cs:0.768,0)
--(axis cs:0.769,0)
--(axis cs:0.77,0)
--(axis cs:0.771,0)
--(axis cs:0.772,0)
--(axis cs:0.773,0)
--(axis cs:0.774,0)
--(axis cs:0.775,0)
--(axis cs:0.776,0)
--(axis cs:0.777,0)
--(axis cs:0.778,0)
--(axis cs:0.779,0)
--(axis cs:0.78,0)
--(axis cs:0.781,0)
--(axis cs:0.782,0)
--(axis cs:0.783,0)
--(axis cs:0.784,0)
--(axis cs:0.785,0)
--(axis cs:0.786,0)
--(axis cs:0.787,0)
--(axis cs:0.788,0)
--(axis cs:0.789,0)
--(axis cs:0.79,0)
--(axis cs:0.791,0)
--(axis cs:0.792,0)
--(axis cs:0.793,0)
--(axis cs:0.794,0)
--(axis cs:0.795,0)
--(axis cs:0.796,0)
--(axis cs:0.797,0)
--(axis cs:0.798,0)
--(axis cs:0.799,0)
--(axis cs:0.8,0)
--(axis cs:0.801,0)
--(axis cs:0.802,0)
--(axis cs:0.803,0)
--(axis cs:0.804,0)
--(axis cs:0.805,0)
--(axis cs:0.806,0)
--(axis cs:0.807,0)
--(axis cs:0.808,0)
--(axis cs:0.809,0)
--(axis cs:0.81,0)
--(axis cs:0.811,0)
--(axis cs:0.812,0)
--(axis cs:0.813,0)
--(axis cs:0.814,0)
--(axis cs:0.815,0)
--(axis cs:0.816,0)
--(axis cs:0.817,0)
--(axis cs:0.818,0)
--(axis cs:0.819,0)
--(axis cs:0.82,0)
--(axis cs:0.821,0)
--(axis cs:0.822,0)
--(axis cs:0.823,0)
--(axis cs:0.824,0)
--(axis cs:0.825,0)
--(axis cs:0.826,0)
--(axis cs:0.827,0)
--(axis cs:0.828,0)
--(axis cs:0.829,0)
--(axis cs:0.83,0)
--(axis cs:0.831,0)
--(axis cs:0.832,0)
--(axis cs:0.833,0)
--(axis cs:0.834,0)
--(axis cs:0.835,0)
--(axis cs:0.836,0)
--(axis cs:0.837,0)
--(axis cs:0.838,0)
--(axis cs:0.839,0)
--(axis cs:0.84,0)
--(axis cs:0.841,0)
--(axis cs:0.842,0)
--(axis cs:0.843,0)
--(axis cs:0.844,0)
--(axis cs:0.845,0)
--(axis cs:0.846,0)
--(axis cs:0.847,0)
--(axis cs:0.848,0)
--(axis cs:0.849,0)
--(axis cs:0.85,0)
--(axis cs:0.851,0)
--(axis cs:0.852,0)
--(axis cs:0.853,0)
--(axis cs:0.854,0)
--(axis cs:0.855,0)
--(axis cs:0.856,0)
--(axis cs:0.857,0)
--(axis cs:0.858,0)
--(axis cs:0.859,0)
--(axis cs:0.86,0)
--(axis cs:0.861,0)
--(axis cs:0.862,0)
--(axis cs:0.863,0)
--(axis cs:0.864,0)
--(axis cs:0.865,0)
--(axis cs:0.866,0)
--(axis cs:0.867,0)
--(axis cs:0.868,0)
--(axis cs:0.869,0)
--(axis cs:0.87,0)
--(axis cs:0.871,0)
--(axis cs:0.872,0)
--(axis cs:0.873,0)
--(axis cs:0.874,0)
--(axis cs:0.875,0)
--(axis cs:0.876,0)
--(axis cs:0.877,0)
--(axis cs:0.878,0)
--(axis cs:0.879,0)
--(axis cs:0.88,0)
--(axis cs:0.881,0)
--(axis cs:0.882,0)
--(axis cs:0.883,0)
--(axis cs:0.884,0)
--(axis cs:0.885,0)
--(axis cs:0.886,0)
--cycle;
\path [draw=none, fill=blue, fill opacity=0.5]
(axis cs:0,0)
--(axis cs:0.001,0)
--(axis cs:0.002,3.79114210771532e-305)
--(axis cs:0.003,2.27295435483742e-220)
--(axis cs:0.004,2.22041196223942e-164)
--(axis cs:0.005,3.59392471299512e-124)
--(axis cs:0.006,5.99467998609246e-94)
--(axis cs:0.007,1.26935745406657e-70)
--(axis cs:0.008,1.17955340981976e-68)
--(axis cs:0.009,1.59902350937739e-81)
--(axis cs:0.01,1.37827639915434e-94)
--(axis cs:0.011,8.18223547507063e-108)
--(axis cs:0.012,3.54880776405098e-121)
--(axis cs:0.013,1.17602663837293e-134)
--(axis cs:0.014,3.08309950875692e-148)
--(axis cs:0.015,6.5730467404995e-162)
--(axis cs:0.016,1.16520112860796e-175)
--(axis cs:0.017,1.7489381169991e-189)
--(axis cs:0.018,2.25638503050645e-203)
--(axis cs:0.019,2.53383682029332e-217)
--(axis cs:0.02,2.50314709766078e-231)
--(axis cs:0.021,2.19518644175894e-245)
--(axis cs:0.022,1.722324175685e-259)
--(axis cs:0.023,1.21714585515373e-273)
--(axis cs:0.024,7.79296272852925e-288)
--(axis cs:0.025,4.54387893249789e-302)
--(axis cs:0.026,2.42370933121564e-316)
--(axis cs:0.027,0)
--(axis cs:0.028,0)
--(axis cs:0.029,0)
--(axis cs:0.03,0)
--(axis cs:0.031,0)
--(axis cs:0.032,0)
--(axis cs:0.033,0)
--(axis cs:0.034,0)
--(axis cs:0.035,0)
--(axis cs:0.036,0)
--(axis cs:0.037,0)
--(axis cs:0.038,0)
--(axis cs:0.039,0)
--(axis cs:0.04,0)
--(axis cs:0.041,0)
--(axis cs:0.042,0)
--(axis cs:0.043,0)
--(axis cs:0.044,0)
--(axis cs:0.045,0)
--(axis cs:0.046,0)
--(axis cs:0.047,0)
--(axis cs:0.048,0)
--(axis cs:0.049,0)
--(axis cs:0.05,0)
--(axis cs:0.051,0)
--(axis cs:0.052,0)
--(axis cs:0.053,0)
--(axis cs:0.054,0)
--(axis cs:0.055,0)
--(axis cs:0.056,0)
--(axis cs:0.057,0)
--(axis cs:0.058,0)
--(axis cs:0.059,0)
--(axis cs:0.06,0)
--(axis cs:0.061,0)
--(axis cs:0.062,0)
--(axis cs:0.063,0)
--(axis cs:0.064,0)
--(axis cs:0.065,0)
--(axis cs:0.066,0)
--(axis cs:0.067,0)
--(axis cs:0.068,0)
--(axis cs:0.069,0)
--(axis cs:0.07,0)
--(axis cs:0.071,0)
--(axis cs:0.072,0)
--(axis cs:0.073,0)
--(axis cs:0.074,0)
--(axis cs:0.075,0)
--(axis cs:0.076,0)
--(axis cs:0.077,0)
--(axis cs:0.078,0)
--(axis cs:0.079,0)
--(axis cs:0.08,0)
--(axis cs:0.081,0)
--(axis cs:0.082,0)
--(axis cs:0.083,0)
--(axis cs:0.084,0)
--(axis cs:0.085,0)
--(axis cs:0.086,0)
--(axis cs:0.087,0)
--(axis cs:0.088,0)
--(axis cs:0.089,0)
--(axis cs:0.09,0)
--(axis cs:0.091,0)
--(axis cs:0.092,0)
--(axis cs:0.093,0)
--(axis cs:0.094,0)
--(axis cs:0.095,0)
--(axis cs:0.096,0)
--(axis cs:0.097,0)
--(axis cs:0.098,0)
--(axis cs:0.099,0)
--(axis cs:0.1,0)
--(axis cs:0.101,0)
--(axis cs:0.102,0)
--(axis cs:0.103,0)
--(axis cs:0.104,0)
--(axis cs:0.105,0)
--(axis cs:0.106,0)
--(axis cs:0.107,0)
--(axis cs:0.108,0)
--(axis cs:0.109,0)
--(axis cs:0.11,0)
--(axis cs:0.111,0)
--(axis cs:0.112,0)
--(axis cs:0.113,0)
--(axis cs:0.114,0)
--(axis cs:0.115,0)
--(axis cs:0.116,0)
--(axis cs:0.117,0)
--(axis cs:0.118,0)
--(axis cs:0.119,0)
--(axis cs:0.12,0)
--(axis cs:0.121,0)
--(axis cs:0.122,0)
--(axis cs:0.123,0)
--(axis cs:0.124,0)
--(axis cs:0.125,0)
--(axis cs:0.126,0)
--(axis cs:0.127,0)
--(axis cs:0.128,0)
--(axis cs:0.129,0)
--(axis cs:0.13,0)
--(axis cs:0.131,0)
--(axis cs:0.132,0)
--(axis cs:0.133,0)
--(axis cs:0.134,0)
--(axis cs:0.135,0)
--(axis cs:0.136,0)
--(axis cs:0.137,0)
--(axis cs:0.138,0)
--(axis cs:0.139,0)
--(axis cs:0.14,0)
--(axis cs:0.141,0)
--(axis cs:0.142,0)
--(axis cs:0.143,0)
--(axis cs:0.144,0)
--(axis cs:0.145,0)
--(axis cs:0.146,0)
--(axis cs:0.147,0)
--(axis cs:0.148,0)
--(axis cs:0.149,0)
--(axis cs:0.15,0)
--(axis cs:0.151,0)
--(axis cs:0.152,0)
--(axis cs:0.153,0)
--(axis cs:0.154,0)
--(axis cs:0.155,0)
--(axis cs:0.156,0)
--(axis cs:0.157,0)
--(axis cs:0.158,0)
--(axis cs:0.159,0)
--(axis cs:0.16,0)
--(axis cs:0.161,0)
--(axis cs:0.162,0)
--(axis cs:0.163,0)
--(axis cs:0.164,0)
--(axis cs:0.165,0)
--(axis cs:0.166,0)
--(axis cs:0.167,0)
--(axis cs:0.168,0)
--(axis cs:0.169,0)
--(axis cs:0.17,0)
--(axis cs:0.171,0)
--(axis cs:0.172,0)
--(axis cs:0.173,0)
--(axis cs:0.174,0)
--(axis cs:0.175,0)
--(axis cs:0.176,0)
--(axis cs:0.177,0)
--(axis cs:0.178,0)
--(axis cs:0.179,0)
--(axis cs:0.18,0)
--(axis cs:0.181,0)
--(axis cs:0.182,0)
--(axis cs:0.183,0)
--(axis cs:0.184,0)
--(axis cs:0.185,0)
--(axis cs:0.186,0)
--(axis cs:0.187,0)
--(axis cs:0.188,0)
--(axis cs:0.189,0)
--(axis cs:0.19,0)
--(axis cs:0.191,0)
--(axis cs:0.192,0)
--(axis cs:0.193,0)
--(axis cs:0.194,0)
--(axis cs:0.195,0)
--(axis cs:0.196,0)
--(axis cs:0.197,0)
--(axis cs:0.198,0)
--(axis cs:0.199,0)
--(axis cs:0.2,0)
--(axis cs:0.201,0)
--(axis cs:0.202,0)
--(axis cs:0.203,0)
--(axis cs:0.204,0)
--(axis cs:0.205,0)
--(axis cs:0.206,0)
--(axis cs:0.207,0)
--(axis cs:0.208,0)
--(axis cs:0.209,0)
--(axis cs:0.21,0)
--(axis cs:0.211,0)
--(axis cs:0.212,0)
--(axis cs:0.213,0)
--(axis cs:0.214,0)
--(axis cs:0.215,0)
--(axis cs:0.216,0)
--(axis cs:0.217,0)
--(axis cs:0.218,0)
--(axis cs:0.219,0)
--(axis cs:0.22,0)
--(axis cs:0.221,0)
--(axis cs:0.222,0)
--(axis cs:0.223,0)
--(axis cs:0.224,0)
--(axis cs:0.225,0)
--(axis cs:0.226,0)
--(axis cs:0.227,0)
--(axis cs:0.228,0)
--(axis cs:0.229,0)
--(axis cs:0.23,0)
--(axis cs:0.231,0)
--(axis cs:0.232,0)
--(axis cs:0.233,0)
--(axis cs:0.234,0)
--(axis cs:0.235,0)
--(axis cs:0.236,0)
--(axis cs:0.237,0)
--(axis cs:0.238,0)
--(axis cs:0.239,0)
--(axis cs:0.24,0)
--(axis cs:0.241,0)
--(axis cs:0.242,0)
--(axis cs:0.243,0)
--(axis cs:0.244,0)
--(axis cs:0.245,0)
--(axis cs:0.246,0)
--(axis cs:0.247,0)
--(axis cs:0.248,0)
--(axis cs:0.249,0)
--(axis cs:0.25,0)
--(axis cs:0.251,0)
--(axis cs:0.252,0)
--(axis cs:0.253,0)
--(axis cs:0.254,0)
--(axis cs:0.255,0)
--(axis cs:0.256,0)
--(axis cs:0.257,0)
--(axis cs:0.258,0)
--(axis cs:0.259,0)
--(axis cs:0.26,0)
--(axis cs:0.261,0)
--(axis cs:0.262,0)
--(axis cs:0.263,0)
--(axis cs:0.264,0)
--(axis cs:0.265,0)
--(axis cs:0.266,0)
--(axis cs:0.267,0)
--(axis cs:0.268,0)
--(axis cs:0.269,0)
--(axis cs:0.27,0)
--(axis cs:0.271,0)
--(axis cs:0.272,0)
--(axis cs:0.273,0)
--(axis cs:0.274,0)
--(axis cs:0.275,0)
--(axis cs:0.276,0)
--(axis cs:0.277,0)
--(axis cs:0.278,0)
--(axis cs:0.279,0)
--(axis cs:0.28,0)
--(axis cs:0.281,0)
--(axis cs:0.282,0)
--(axis cs:0.283,0)
--(axis cs:0.284,0)
--(axis cs:0.285,0)
--(axis cs:0.286,0)
--(axis cs:0.287,0)
--(axis cs:0.288,0)
--(axis cs:0.289,0)
--(axis cs:0.29,0)
--(axis cs:0.291,0)
--(axis cs:0.292,0)
--(axis cs:0.293,0)
--(axis cs:0.294,0)
--(axis cs:0.295,0)
--(axis cs:0.296,0)
--(axis cs:0.297,0)
--(axis cs:0.298,0)
--(axis cs:0.299,0)
--(axis cs:0.3,0)
--(axis cs:0.301,0)
--(axis cs:0.302,0)
--(axis cs:0.303,0)
--(axis cs:0.304,0)
--(axis cs:0.305,0)
--(axis cs:0.306,0)
--(axis cs:0.307,0)
--(axis cs:0.308,0)
--(axis cs:0.309,0)
--(axis cs:0.31,0)
--(axis cs:0.311,0)
--(axis cs:0.312,0)
--(axis cs:0.313,0)
--(axis cs:0.314,0)
--(axis cs:0.315,0)
--(axis cs:0.316,0)
--(axis cs:0.317,0)
--(axis cs:0.318,0)
--(axis cs:0.319,0)
--(axis cs:0.32,0)
--(axis cs:0.321,0)
--(axis cs:0.322,0)
--(axis cs:0.323,0)
--(axis cs:0.324,0)
--(axis cs:0.325,0)
--(axis cs:0.326,0)
--(axis cs:0.327,0)
--(axis cs:0.328,0)
--(axis cs:0.329,0)
--(axis cs:0.33,0)
--(axis cs:0.331,0)
--(axis cs:0.332,0)
--(axis cs:0.333,0)
--(axis cs:0.334,0)
--(axis cs:0.335,0)
--(axis cs:0.336,0)
--(axis cs:0.337,0)
--(axis cs:0.338,0)
--(axis cs:0.339,0)
--(axis cs:0.34,0)
--(axis cs:0.341,0)
--(axis cs:0.342,0)
--(axis cs:0.343,0)
--(axis cs:0.344,0)
--(axis cs:0.345,0)
--(axis cs:0.346,0)
--(axis cs:0.347,0)
--(axis cs:0.348,0)
--(axis cs:0.349,0)
--(axis cs:0.35,0)
--(axis cs:0.351,0)
--(axis cs:0.352,0)
--(axis cs:0.353,0)
--(axis cs:0.354,0)
--(axis cs:0.355,0)
--(axis cs:0.356,0)
--(axis cs:0.357,0)
--(axis cs:0.358,0)
--(axis cs:0.359,0)
--(axis cs:0.36,0)
--(axis cs:0.361,0)
--(axis cs:0.362,0)
--(axis cs:0.363,0)
--(axis cs:0.364,0)
--(axis cs:0.365,0)
--(axis cs:0.366,0)
--(axis cs:0.367,0)
--(axis cs:0.368,0)
--(axis cs:0.369,0)
--(axis cs:0.37,0)
--(axis cs:0.371,0)
--(axis cs:0.372,0)
--(axis cs:0.373,0)
--(axis cs:0.374,0)
--(axis cs:0.375,0)
--(axis cs:0.376,0)
--(axis cs:0.377,0)
--(axis cs:0.378,0)
--(axis cs:0.379,0)
--(axis cs:0.38,0)
--(axis cs:0.381,0)
--(axis cs:0.382,0)
--(axis cs:0.383,0)
--(axis cs:0.384,0)
--(axis cs:0.385,0)
--(axis cs:0.386,0)
--(axis cs:0.387,0)
--(axis cs:0.388,0)
--(axis cs:0.389,0)
--(axis cs:0.39,0)
--(axis cs:0.391,0)
--(axis cs:0.392,0)
--(axis cs:0.393,0)
--(axis cs:0.394,0)
--(axis cs:0.395,0)
--(axis cs:0.396,0)
--(axis cs:0.397,0)
--(axis cs:0.398,0)
--(axis cs:0.399,0)
--(axis cs:0.4,0)
--(axis cs:0.401,0)
--(axis cs:0.402,0)
--(axis cs:0.403,0)
--(axis cs:0.404,0)
--(axis cs:0.405,0)
--(axis cs:0.406,0)
--(axis cs:0.407,0)
--(axis cs:0.408,0)
--(axis cs:0.409,0)
--(axis cs:0.41,0)
--(axis cs:0.411,0)
--(axis cs:0.412,0)
--(axis cs:0.413,0)
--(axis cs:0.414,0)
--(axis cs:0.415,0)
--(axis cs:0.416,0)
--(axis cs:0.417,0)
--(axis cs:0.418,0)
--(axis cs:0.419,0)
--(axis cs:0.42,0)
--(axis cs:0.421,0)
--(axis cs:0.422,0)
--(axis cs:0.423,0)
--(axis cs:0.424,0)
--(axis cs:0.425,0)
--(axis cs:0.426,0)
--(axis cs:0.427,0)
--(axis cs:0.428,0)
--(axis cs:0.429,0)
--(axis cs:0.43,0)
--(axis cs:0.431,0)
--(axis cs:0.432,0)
--(axis cs:0.433,0)
--(axis cs:0.434,0)
--(axis cs:0.435,0)
--(axis cs:0.436,0)
--(axis cs:0.437,0)
--(axis cs:0.438,0)
--(axis cs:0.439,0)
--(axis cs:0.44,0)
--(axis cs:0.441,0)
--(axis cs:0.442,0)
--(axis cs:0.443,0)
--(axis cs:0.444,0)
--(axis cs:0.445,0)
--(axis cs:0.446,0)
--(axis cs:0.447,0)
--(axis cs:0.448,0)
--(axis cs:0.449,0)
--(axis cs:0.45,0)
--(axis cs:0.451,0)
--(axis cs:0.452,0)
--(axis cs:0.453,0)
--(axis cs:0.454,0)
--(axis cs:0.455,0)
--(axis cs:0.456,0)
--(axis cs:0.457,0)
--(axis cs:0.458,0)
--(axis cs:0.459,0)
--(axis cs:0.46,0)
--(axis cs:0.461,0)
--(axis cs:0.462,0)
--(axis cs:0.463,0)
--(axis cs:0.464,0)
--(axis cs:0.465,0)
--(axis cs:0.466,0)
--(axis cs:0.467,0)
--(axis cs:0.468,0)
--(axis cs:0.469,0)
--(axis cs:0.47,0)
--(axis cs:0.471,0)
--(axis cs:0.472,0)
--(axis cs:0.473,0)
--(axis cs:0.474,0)
--(axis cs:0.475,0)
--(axis cs:0.476,0)
--(axis cs:0.477,0)
--(axis cs:0.478,0)
--(axis cs:0.479,0)
--(axis cs:0.48,0)
--(axis cs:0.481,0)
--(axis cs:0.482,0)
--(axis cs:0.483,0)
--(axis cs:0.484,0)
--(axis cs:0.485,0)
--(axis cs:0.486,0)
--(axis cs:0.487,0)
--(axis cs:0.488,0)
--(axis cs:0.489,0)
--(axis cs:0.49,0)
--(axis cs:0.491,0)
--(axis cs:0.492,0)
--(axis cs:0.493,0)
--(axis cs:0.494,0)
--(axis cs:0.495,0)
--(axis cs:0.496,0)
--(axis cs:0.497,0)
--(axis cs:0.498,0)
--(axis cs:0.499,0)
--(axis cs:0.5,0)
--(axis cs:0.501,0)
--(axis cs:0.502,0)
--(axis cs:0.503,0)
--(axis cs:0.504,0)
--(axis cs:0.505,0)
--(axis cs:0.506,0)
--(axis cs:0.507,0)
--(axis cs:0.508,0)
--(axis cs:0.509,0)
--(axis cs:0.51,0)
--(axis cs:0.511,0)
--(axis cs:0.512,0)
--(axis cs:0.513,0)
--(axis cs:0.514,0)
--(axis cs:0.515,0)
--(axis cs:0.516,0)
--(axis cs:0.517,0)
--(axis cs:0.518,0)
--(axis cs:0.519,0)
--(axis cs:0.52,0)
--(axis cs:0.521,0)
--(axis cs:0.522,0)
--(axis cs:0.523,0)
--(axis cs:0.524,0)
--(axis cs:0.525,0)
--(axis cs:0.526,0)
--(axis cs:0.527,0)
--(axis cs:0.528,0)
--(axis cs:0.529,0)
--(axis cs:0.53,0)
--(axis cs:0.531,0)
--(axis cs:0.532,0)
--(axis cs:0.533,0)
--(axis cs:0.534,0)
--(axis cs:0.535,0)
--(axis cs:0.536,0)
--(axis cs:0.537,0)
--(axis cs:0.538,0)
--(axis cs:0.539,0)
--(axis cs:0.54,0)
--(axis cs:0.541,0)
--(axis cs:0.542,0)
--(axis cs:0.543,0)
--(axis cs:0.544,0)
--(axis cs:0.545,0)
--(axis cs:0.546,0)
--(axis cs:0.547,0)
--(axis cs:0.548,0)
--(axis cs:0.549,0)
--(axis cs:0.55,0)
--(axis cs:0.551,0)
--(axis cs:0.552,0)
--(axis cs:0.553,0)
--(axis cs:0.554,0)
--(axis cs:0.555,0)
--(axis cs:0.556,0)
--(axis cs:0.557,0)
--(axis cs:0.558,0)
--(axis cs:0.559,0)
--(axis cs:0.56,0)
--(axis cs:0.561,0)
--(axis cs:0.562,0)
--(axis cs:0.563,0)
--(axis cs:0.564,0)
--(axis cs:0.565,0)
--(axis cs:0.566,0)
--(axis cs:0.567,0)
--(axis cs:0.568,0)
--(axis cs:0.569,0)
--(axis cs:0.57,0)
--(axis cs:0.571,0)
--(axis cs:0.572,0)
--(axis cs:0.573,0)
--(axis cs:0.574,0)
--(axis cs:0.575,0)
--(axis cs:0.576,0)
--(axis cs:0.577,0)
--(axis cs:0.578,0)
--(axis cs:0.579,0)
--(axis cs:0.58,0)
--(axis cs:0.581,0)
--(axis cs:0.582,0)
--(axis cs:0.583,0)
--(axis cs:0.584,0)
--(axis cs:0.585,0)
--(axis cs:0.586,0)
--(axis cs:0.587,0)
--(axis cs:0.588,0)
--(axis cs:0.589,0)
--(axis cs:0.59,0)
--(axis cs:0.591,0)
--(axis cs:0.592,0)
--(axis cs:0.593,0)
--(axis cs:0.594,0)
--(axis cs:0.595,0)
--(axis cs:0.596,0)
--(axis cs:0.597,0)
--(axis cs:0.598,0)
--(axis cs:0.599,0)
--(axis cs:0.6,0)
--(axis cs:0.601,0)
--(axis cs:0.602,0)
--(axis cs:0.603,0)
--(axis cs:0.604,0)
--(axis cs:0.605,0)
--(axis cs:0.606,0)
--(axis cs:0.607,0)
--(axis cs:0.608,0)
--(axis cs:0.609,0)
--(axis cs:0.61,0)
--(axis cs:0.611,0)
--(axis cs:0.612,0)
--(axis cs:0.613,0)
--(axis cs:0.614,0)
--(axis cs:0.615,0)
--(axis cs:0.616,0)
--(axis cs:0.617,0)
--(axis cs:0.618,0)
--(axis cs:0.619,0)
--(axis cs:0.62,0)
--(axis cs:0.621,0)
--(axis cs:0.622,0)
--(axis cs:0.623,0)
--(axis cs:0.624,0)
--(axis cs:0.625,0)
--(axis cs:0.626,0)
--(axis cs:0.627,0)
--(axis cs:0.628,0)
--(axis cs:0.629,0)
--(axis cs:0.63,0)
--(axis cs:0.631,0)
--(axis cs:0.632,0)
--(axis cs:0.633,0)
--(axis cs:0.634,0)
--(axis cs:0.635,0)
--(axis cs:0.636,0)
--(axis cs:0.637,0)
--(axis cs:0.638,0)
--(axis cs:0.639,0)
--(axis cs:0.64,0)
--(axis cs:0.641,0)
--(axis cs:0.642,0)
--(axis cs:0.643,0)
--(axis cs:0.644,0)
--(axis cs:0.645,0)
--(axis cs:0.646,0)
--(axis cs:0.647,0)
--(axis cs:0.648,0)
--(axis cs:0.649,0)
--(axis cs:0.65,0)
--(axis cs:0.651,0)
--(axis cs:0.652,0)
--(axis cs:0.653,0)
--(axis cs:0.654,0)
--(axis cs:0.655,0)
--(axis cs:0.656,0)
--(axis cs:0.657,0)
--(axis cs:0.658,0)
--(axis cs:0.659,0)
--(axis cs:0.66,0)
--(axis cs:0.661,0)
--(axis cs:0.662,0)
--(axis cs:0.663,0)
--(axis cs:0.664,0)
--(axis cs:0.665,0)
--(axis cs:0.666,0)
--(axis cs:0.667,0)
--(axis cs:0.668,0)
--(axis cs:0.669,0)
--(axis cs:0.67,0)
--(axis cs:0.671,0)
--(axis cs:0.672,0)
--(axis cs:0.673,0)
--(axis cs:0.674,0)
--(axis cs:0.675,0)
--(axis cs:0.676,0)
--(axis cs:0.677,0)
--(axis cs:0.678,0)
--(axis cs:0.679,0)
--(axis cs:0.68,0)
--(axis cs:0.681,0)
--(axis cs:0.682,0)
--(axis cs:0.683,0)
--(axis cs:0.684,0)
--(axis cs:0.685,0)
--(axis cs:0.686,0)
--(axis cs:0.687,0)
--(axis cs:0.688,0)
--(axis cs:0.689,0)
--(axis cs:0.69,0)
--(axis cs:0.691,0)
--(axis cs:0.692,0)
--(axis cs:0.693,0)
--(axis cs:0.694,0)
--(axis cs:0.695,0)
--(axis cs:0.696,0)
--(axis cs:0.697,0)
--(axis cs:0.698,0)
--(axis cs:0.699,0)
--(axis cs:0.7,0)
--(axis cs:0.701,0)
--(axis cs:0.702,0)
--(axis cs:0.703,0)
--(axis cs:0.704,0)
--(axis cs:0.705,0)
--(axis cs:0.706,0)
--(axis cs:0.707,0)
--(axis cs:0.708,0)
--(axis cs:0.709,0)
--(axis cs:0.71,0)
--(axis cs:0.711,0)
--(axis cs:0.712,0)
--(axis cs:0.713,0)
--(axis cs:0.714,0)
--(axis cs:0.715,0)
--(axis cs:0.716,0)
--(axis cs:0.717,0)
--(axis cs:0.718,0)
--(axis cs:0.719,0)
--(axis cs:0.72,0)
--(axis cs:0.721,0)
--(axis cs:0.722,0)
--(axis cs:0.723,0)
--(axis cs:0.724,0)
--(axis cs:0.725,0)
--(axis cs:0.726,0)
--(axis cs:0.727,0)
--(axis cs:0.728,0)
--(axis cs:0.729,0)
--(axis cs:0.73,0)
--(axis cs:0.731,0)
--(axis cs:0.732,0)
--(axis cs:0.733,0)
--(axis cs:0.734,0)
--(axis cs:0.735,0)
--(axis cs:0.736,0)
--(axis cs:0.737,0)
--(axis cs:0.738,0)
--(axis cs:0.739,0)
--(axis cs:0.74,0)
--(axis cs:0.741,0)
--(axis cs:0.742,0)
--(axis cs:0.743,0)
--(axis cs:0.744,0)
--(axis cs:0.745,0)
--(axis cs:0.746,0)
--(axis cs:0.747,0)
--(axis cs:0.748,0)
--(axis cs:0.749,0)
--(axis cs:0.75,0)
--(axis cs:0.751,0)
--(axis cs:0.752,0)
--(axis cs:0.753,0)
--(axis cs:0.754,0)
--(axis cs:0.755,0)
--(axis cs:0.756,0)
--(axis cs:0.757,0)
--(axis cs:0.758,0)
--(axis cs:0.759,0)
--(axis cs:0.76,0)
--(axis cs:0.761,0)
--(axis cs:0.762,0)
--(axis cs:0.763,0)
--(axis cs:0.764,0)
--(axis cs:0.765,0)
--(axis cs:0.766,0)
--(axis cs:0.767,0)
--(axis cs:0.768,0)
--(axis cs:0.769,0)
--(axis cs:0.77,0)
--(axis cs:0.771,0)
--(axis cs:0.772,0)
--(axis cs:0.773,0)
--(axis cs:0.774,0)
--(axis cs:0.775,0)
--(axis cs:0.776,0)
--(axis cs:0.777,0)
--(axis cs:0.778,0)
--(axis cs:0.779,0)
--(axis cs:0.78,0)
--(axis cs:0.781,0)
--(axis cs:0.782,0)
--(axis cs:0.783,0)
--(axis cs:0.784,0)
--(axis cs:0.785,0)
--(axis cs:0.786,0)
--(axis cs:0.787,0)
--(axis cs:0.788,0)
--(axis cs:0.789,0)
--(axis cs:0.79,0)
--(axis cs:0.791,0)
--(axis cs:0.792,0)
--(axis cs:0.793,0)
--(axis cs:0.794,0)
--(axis cs:0.795,0)
--(axis cs:0.796,0)
--(axis cs:0.797,0)
--(axis cs:0.798,0)
--(axis cs:0.799,0)
--(axis cs:0.8,0)
--(axis cs:0.801,0)
--(axis cs:0.802,0)
--(axis cs:0.803,0)
--(axis cs:0.804,0)
--(axis cs:0.805,0)
--(axis cs:0.806,0)
--(axis cs:0.807,0)
--(axis cs:0.808,0)
--(axis cs:0.809,0)
--(axis cs:0.81,0)
--(axis cs:0.811,0)
--(axis cs:0.812,0)
--(axis cs:0.813,0)
--(axis cs:0.814,0)
--(axis cs:0.815,0)
--(axis cs:0.816,0)
--(axis cs:0.817,0)
--(axis cs:0.818,0)
--(axis cs:0.819,0)
--(axis cs:0.82,0)
--(axis cs:0.821,0)
--(axis cs:0.822,0)
--(axis cs:0.823,0)
--(axis cs:0.824,0)
--(axis cs:0.825,0)
--(axis cs:0.826,0)
--(axis cs:0.827,0)
--(axis cs:0.828,0)
--(axis cs:0.829,0)
--(axis cs:0.83,0)
--(axis cs:0.831,0)
--(axis cs:0.832,0)
--(axis cs:0.833,0)
--(axis cs:0.834,0)
--(axis cs:0.835,0)
--(axis cs:0.836,0)
--(axis cs:0.837,0)
--(axis cs:0.838,0)
--(axis cs:0.839,0)
--(axis cs:0.84,0)
--(axis cs:0.841,0)
--(axis cs:0.842,0)
--(axis cs:0.843,0)
--(axis cs:0.844,0)
--(axis cs:0.845,0)
--(axis cs:0.846,0)
--(axis cs:0.847,0)
--(axis cs:0.848,0)
--(axis cs:0.849,0)
--(axis cs:0.85,0)
--(axis cs:0.851,0)
--(axis cs:0.852,0)
--(axis cs:0.853,0)
--(axis cs:0.854,0)
--(axis cs:0.855,0)
--(axis cs:0.856,0)
--(axis cs:0.857,0)
--(axis cs:0.858,0)
--(axis cs:0.859,0)
--(axis cs:0.86,0)
--(axis cs:0.861,0)
--(axis cs:0.862,0)
--(axis cs:0.863,0)
--(axis cs:0.864,0)
--(axis cs:0.865,0)
--(axis cs:0.866,0)
--(axis cs:0.867,0)
--(axis cs:0.868,0)
--(axis cs:0.869,0)
--(axis cs:0.87,0)
--(axis cs:0.871,0)
--(axis cs:0.872,0)
--(axis cs:0.873,0)
--(axis cs:0.874,0)
--(axis cs:0.875,0)
--(axis cs:0.876,0)
--(axis cs:0.877,0)
--(axis cs:0.878,0)
--(axis cs:0.879,0)
--(axis cs:0.88,0)
--(axis cs:0.881,0)
--(axis cs:0.882,0)
--(axis cs:0.883,0)
--(axis cs:0.884,0)
--(axis cs:0.885,0)
--(axis cs:0.886,0)
--cycle;
\path [draw=none, fill=blue, fill opacity=0.5]
(axis cs:0,0)
--(axis cs:0.001,0)
--(axis cs:0.002,0)
--(axis cs:0.003,0)
--(axis cs:0.004,0)
--(axis cs:0.005,0)
--(axis cs:0.006,0)
--(axis cs:0.007,0)
--(axis cs:0.008,0)
--(axis cs:0.009,0)
--(axis cs:0.01,0)
--(axis cs:0.011,0)
--(axis cs:0.012,0)
--(axis cs:0.013,0)
--(axis cs:0.014,0)
--(axis cs:0.015,0)
--(axis cs:0.016,0)
--(axis cs:0.017,0)
--(axis cs:0.018,0)
--(axis cs:0.019,0)
--(axis cs:0.02,0)
--(axis cs:0.021,0)
--(axis cs:0.022,0)
--(axis cs:0.023,0)
--(axis cs:0.024,0)
--(axis cs:0.025,0)
--(axis cs:0.026,0)
--(axis cs:0.027,0)
--(axis cs:0.028,0)
--(axis cs:0.029,0)
--(axis cs:0.03,0)
--(axis cs:0.031,0)
--(axis cs:0.032,0)
--(axis cs:0.033,0)
--(axis cs:0.034,0)
--(axis cs:0.035,0)
--(axis cs:0.036,0)
--(axis cs:0.037,0)
--(axis cs:0.038,0)
--(axis cs:0.039,0)
--(axis cs:0.04,0)
--(axis cs:0.041,0)
--(axis cs:0.042,0)
--(axis cs:0.043,0)
--(axis cs:0.044,0)
--(axis cs:0.045,0)
--(axis cs:0.046,0)
--(axis cs:0.047,0)
--(axis cs:0.048,0)
--(axis cs:0.049,0)
--(axis cs:0.05,0)
--(axis cs:0.051,0)
--(axis cs:0.052,0)
--(axis cs:0.053,0)
--(axis cs:0.054,0)
--(axis cs:0.055,0)
--(axis cs:0.056,0)
--(axis cs:0.057,0)
--(axis cs:0.058,0)
--(axis cs:0.059,0)
--(axis cs:0.06,0)
--(axis cs:0.061,0)
--(axis cs:0.062,0)
--(axis cs:0.063,0)
--(axis cs:0.064,0)
--(axis cs:0.065,0)
--(axis cs:0.066,0)
--(axis cs:0.067,0)
--(axis cs:0.068,0)
--(axis cs:0.069,0)
--(axis cs:0.07,0)
--(axis cs:0.071,0)
--(axis cs:0.072,0)
--(axis cs:0.073,0)
--(axis cs:0.074,0)
--(axis cs:0.075,0)
--(axis cs:0.076,0)
--(axis cs:0.077,0)
--(axis cs:0.078,0)
--(axis cs:0.079,0)
--(axis cs:0.08,0)
--(axis cs:0.081,0)
--(axis cs:0.082,0)
--(axis cs:0.083,0)
--(axis cs:0.084,0)
--(axis cs:0.085,0)
--(axis cs:0.086,0)
--(axis cs:0.087,0)
--(axis cs:0.088,0)
--(axis cs:0.089,0)
--(axis cs:0.09,0)
--(axis cs:0.091,0)
--(axis cs:0.092,0)
--(axis cs:0.093,0)
--(axis cs:0.094,0)
--(axis cs:0.095,0)
--(axis cs:0.096,0)
--(axis cs:0.097,0)
--(axis cs:0.098,0)
--(axis cs:0.099,0)
--(axis cs:0.1,0)
--(axis cs:0.101,0)
--(axis cs:0.102,0)
--(axis cs:0.103,0)
--(axis cs:0.104,0)
--(axis cs:0.105,0)
--(axis cs:0.106,0)
--(axis cs:0.107,0)
--(axis cs:0.108,0)
--(axis cs:0.109,0)
--(axis cs:0.11,0)
--(axis cs:0.111,0)
--(axis cs:0.112,0)
--(axis cs:0.113,0)
--(axis cs:0.114,0)
--(axis cs:0.115,0)
--(axis cs:0.116,0)
--(axis cs:0.117,0)
--(axis cs:0.118,0)
--(axis cs:0.119,0)
--(axis cs:0.12,0)
--(axis cs:0.121,0)
--(axis cs:0.122,0)
--(axis cs:0.123,0)
--(axis cs:0.124,0)
--(axis cs:0.125,0)
--(axis cs:0.126,0)
--(axis cs:0.127,0)
--(axis cs:0.128,0)
--(axis cs:0.129,0)
--(axis cs:0.13,0)
--(axis cs:0.131,0)
--(axis cs:0.132,0)
--(axis cs:0.133,0)
--(axis cs:0.134,0)
--(axis cs:0.135,0)
--(axis cs:0.136,0)
--(axis cs:0.137,0)
--(axis cs:0.138,0)
--(axis cs:0.139,0)
--(axis cs:0.14,0)
--(axis cs:0.141,0)
--(axis cs:0.142,0)
--(axis cs:0.143,0)
--(axis cs:0.144,0)
--(axis cs:0.145,0)
--(axis cs:0.146,0)
--(axis cs:0.147,0)
--(axis cs:0.148,0)
--(axis cs:0.149,0)
--(axis cs:0.15,0)
--(axis cs:0.151,0)
--(axis cs:0.152,0)
--(axis cs:0.153,0)
--(axis cs:0.154,0)
--(axis cs:0.155,0)
--(axis cs:0.156,0)
--(axis cs:0.157,0)
--(axis cs:0.158,0)
--(axis cs:0.159,0)
--(axis cs:0.16,0)
--(axis cs:0.161,0)
--(axis cs:0.162,0)
--(axis cs:0.163,0)
--(axis cs:0.164,0)
--(axis cs:0.165,0)
--(axis cs:0.166,0)
--(axis cs:0.167,0)
--(axis cs:0.168,0)
--(axis cs:0.169,0)
--(axis cs:0.17,0)
--(axis cs:0.171,0)
--(axis cs:0.172,0)
--(axis cs:0.173,0)
--(axis cs:0.174,0)
--(axis cs:0.175,0)
--(axis cs:0.176,0)
--(axis cs:0.177,0)
--(axis cs:0.178,0)
--(axis cs:0.179,0)
--(axis cs:0.18,0)
--(axis cs:0.181,0)
--(axis cs:0.182,0)
--(axis cs:0.183,0)
--(axis cs:0.184,0)
--(axis cs:0.185,0)
--(axis cs:0.186,0)
--(axis cs:0.187,0)
--(axis cs:0.188,0)
--(axis cs:0.189,0)
--(axis cs:0.19,0)
--(axis cs:0.191,0)
--(axis cs:0.192,0)
--(axis cs:0.193,0)
--(axis cs:0.194,0)
--(axis cs:0.195,0)
--(axis cs:0.196,0)
--(axis cs:0.197,0)
--(axis cs:0.198,0)
--(axis cs:0.199,0)
--(axis cs:0.2,0)
--(axis cs:0.201,0)
--(axis cs:0.202,0)
--(axis cs:0.203,0)
--(axis cs:0.204,0)
--(axis cs:0.205,0)
--(axis cs:0.206,0)
--(axis cs:0.207,0)
--(axis cs:0.208,0)
--(axis cs:0.209,0)
--(axis cs:0.21,0)
--(axis cs:0.211,0)
--(axis cs:0.212,0)
--(axis cs:0.213,0)
--(axis cs:0.214,0)
--(axis cs:0.215,0)
--(axis cs:0.216,0)
--(axis cs:0.217,0)
--(axis cs:0.218,0)
--(axis cs:0.219,0)
--(axis cs:0.22,0)
--(axis cs:0.221,0)
--(axis cs:0.222,0)
--(axis cs:0.223,0)
--(axis cs:0.224,0)
--(axis cs:0.225,0)
--(axis cs:0.226,0)
--(axis cs:0.227,0)
--(axis cs:0.228,0)
--(axis cs:0.229,0)
--(axis cs:0.23,0)
--(axis cs:0.231,0)
--(axis cs:0.232,0)
--(axis cs:0.233,0)
--(axis cs:0.234,0)
--(axis cs:0.235,0)
--(axis cs:0.236,0)
--(axis cs:0.237,0)
--(axis cs:0.238,0)
--(axis cs:0.239,0)
--(axis cs:0.24,0)
--(axis cs:0.241,0)
--(axis cs:0.242,0)
--(axis cs:0.243,0)
--(axis cs:0.244,0)
--(axis cs:0.245,0)
--(axis cs:0.246,0)
--(axis cs:0.247,0)
--(axis cs:0.248,0)
--(axis cs:0.249,0)
--(axis cs:0.25,0)
--(axis cs:0.251,0)
--(axis cs:0.252,0)
--(axis cs:0.253,0)
--(axis cs:0.254,0)
--(axis cs:0.255,0)
--(axis cs:0.256,0)
--(axis cs:0.257,0)
--(axis cs:0.258,0)
--(axis cs:0.259,0)
--(axis cs:0.26,0)
--(axis cs:0.261,0)
--(axis cs:0.262,0)
--(axis cs:0.263,0)
--(axis cs:0.264,0)
--(axis cs:0.265,0)
--(axis cs:0.266,0)
--(axis cs:0.267,0)
--(axis cs:0.268,0)
--(axis cs:0.269,0)
--(axis cs:0.27,0)
--(axis cs:0.271,0)
--(axis cs:0.272,0)
--(axis cs:0.273,0)
--(axis cs:0.274,0)
--(axis cs:0.275,0)
--(axis cs:0.276,0)
--(axis cs:0.277,0)
--(axis cs:0.278,0)
--(axis cs:0.279,0)
--(axis cs:0.28,0)
--(axis cs:0.281,0)
--(axis cs:0.282,0)
--(axis cs:0.283,0)
--(axis cs:0.284,0)
--(axis cs:0.285,0)
--(axis cs:0.286,0)
--(axis cs:0.287,0)
--(axis cs:0.288,0)
--(axis cs:0.289,0)
--(axis cs:0.29,0)
--(axis cs:0.291,0)
--(axis cs:0.292,0)
--(axis cs:0.293,0)
--(axis cs:0.294,0)
--(axis cs:0.295,0)
--(axis cs:0.296,0)
--(axis cs:0.297,0)
--(axis cs:0.298,0)
--(axis cs:0.299,0)
--(axis cs:0.3,0)
--(axis cs:0.301,0)
--(axis cs:0.302,0)
--(axis cs:0.303,0)
--(axis cs:0.304,0)
--(axis cs:0.305,0)
--(axis cs:0.306,0)
--(axis cs:0.307,0)
--(axis cs:0.308,0)
--(axis cs:0.309,0)
--(axis cs:0.31,0)
--(axis cs:0.311,0)
--(axis cs:0.312,0)
--(axis cs:0.313,0)
--(axis cs:0.314,0)
--(axis cs:0.315,0)
--(axis cs:0.316,0)
--(axis cs:0.317,0)
--(axis cs:0.318,0)
--(axis cs:0.319,0)
--(axis cs:0.32,0)
--(axis cs:0.321,0)
--(axis cs:0.322,0)
--(axis cs:0.323,0)
--(axis cs:0.324,0)
--(axis cs:0.325,0)
--(axis cs:0.326,0)
--(axis cs:0.327,0)
--(axis cs:0.328,0)
--(axis cs:0.329,0)
--(axis cs:0.33,0)
--(axis cs:0.331,0)
--(axis cs:0.332,0)
--(axis cs:0.333,0)
--(axis cs:0.334,0)
--(axis cs:0.335,0)
--(axis cs:0.336,0)
--(axis cs:0.337,0)
--(axis cs:0.338,0)
--(axis cs:0.339,0)
--(axis cs:0.34,0)
--(axis cs:0.341,0)
--(axis cs:0.342,0)
--(axis cs:0.343,0)
--(axis cs:0.344,0)
--(axis cs:0.345,0)
--(axis cs:0.346,0)
--(axis cs:0.347,0)
--(axis cs:0.348,0)
--(axis cs:0.349,0)
--(axis cs:0.35,0)
--(axis cs:0.351,0)
--(axis cs:0.352,0)
--(axis cs:0.353,0)
--(axis cs:0.354,0)
--(axis cs:0.355,0)
--(axis cs:0.356,0)
--(axis cs:0.357,0)
--(axis cs:0.358,0)
--(axis cs:0.359,0)
--(axis cs:0.36,0)
--(axis cs:0.361,0)
--(axis cs:0.362,0)
--(axis cs:0.363,0)
--(axis cs:0.364,0)
--(axis cs:0.365,0)
--(axis cs:0.366,0)
--(axis cs:0.367,0)
--(axis cs:0.368,0)
--(axis cs:0.369,0)
--(axis cs:0.37,0)
--(axis cs:0.371,0)
--(axis cs:0.372,0)
--(axis cs:0.373,0)
--(axis cs:0.374,0)
--(axis cs:0.375,0)
--(axis cs:0.376,0)
--(axis cs:0.377,0)
--(axis cs:0.378,0)
--(axis cs:0.379,0)
--(axis cs:0.38,0)
--(axis cs:0.381,0)
--(axis cs:0.382,0)
--(axis cs:0.383,0)
--(axis cs:0.384,0)
--(axis cs:0.385,0)
--(axis cs:0.386,0)
--(axis cs:0.387,0)
--(axis cs:0.388,0)
--(axis cs:0.389,0)
--(axis cs:0.39,0)
--(axis cs:0.391,0)
--(axis cs:0.392,0)
--(axis cs:0.393,0)
--(axis cs:0.394,0)
--(axis cs:0.395,0)
--(axis cs:0.396,0)
--(axis cs:0.397,0)
--(axis cs:0.398,0)
--(axis cs:0.399,0)
--(axis cs:0.4,0)
--(axis cs:0.401,0)
--(axis cs:0.402,0)
--(axis cs:0.403,0)
--(axis cs:0.404,0)
--(axis cs:0.405,0)
--(axis cs:0.406,0)
--(axis cs:0.407,0)
--(axis cs:0.408,0)
--(axis cs:0.409,0)
--(axis cs:0.41,0)
--(axis cs:0.411,0)
--(axis cs:0.412,0)
--(axis cs:0.413,0)
--(axis cs:0.414,0)
--(axis cs:0.415,0)
--(axis cs:0.416,0)
--(axis cs:0.417,0)
--(axis cs:0.418,0)
--(axis cs:0.419,0)
--(axis cs:0.42,0)
--(axis cs:0.421,0)
--(axis cs:0.422,0)
--(axis cs:0.423,0)
--(axis cs:0.424,0)
--(axis cs:0.425,0)
--(axis cs:0.426,0)
--(axis cs:0.427,0)
--(axis cs:0.428,0)
--(axis cs:0.429,0)
--(axis cs:0.43,0)
--(axis cs:0.431,0)
--(axis cs:0.432,0)
--(axis cs:0.433,0)
--(axis cs:0.434,0)
--(axis cs:0.435,0)
--(axis cs:0.436,0)
--(axis cs:0.437,0)
--(axis cs:0.438,0)
--(axis cs:0.439,0)
--(axis cs:0.44,0)
--(axis cs:0.441,0)
--(axis cs:0.442,0)
--(axis cs:0.443,0)
--(axis cs:0.444,0)
--(axis cs:0.445,0)
--(axis cs:0.446,0)
--(axis cs:0.447,0)
--(axis cs:0.448,0)
--(axis cs:0.449,0)
--(axis cs:0.45,0)
--(axis cs:0.451,0)
--(axis cs:0.452,0)
--(axis cs:0.453,0)
--(axis cs:0.454,0)
--(axis cs:0.455,0)
--(axis cs:0.456,0)
--(axis cs:0.457,0)
--(axis cs:0.458,0)
--(axis cs:0.459,0)
--(axis cs:0.46,0)
--(axis cs:0.461,0)
--(axis cs:0.462,0)
--(axis cs:0.463,0)
--(axis cs:0.464,0)
--(axis cs:0.465,0)
--(axis cs:0.466,0)
--(axis cs:0.467,0)
--(axis cs:0.468,0)
--(axis cs:0.469,0)
--(axis cs:0.47,0)
--(axis cs:0.471,0)
--(axis cs:0.472,0)
--(axis cs:0.473,0)
--(axis cs:0.474,0)
--(axis cs:0.475,0)
--(axis cs:0.476,0)
--(axis cs:0.477,0)
--(axis cs:0.478,0)
--(axis cs:0.479,0)
--(axis cs:0.48,0)
--(axis cs:0.481,0)
--(axis cs:0.482,0)
--(axis cs:0.483,0)
--(axis cs:0.484,0)
--(axis cs:0.485,0)
--(axis cs:0.486,0)
--(axis cs:0.487,0)
--(axis cs:0.488,0)
--(axis cs:0.489,0)
--(axis cs:0.49,0)
--(axis cs:0.491,0)
--(axis cs:0.492,0)
--(axis cs:0.493,0)
--(axis cs:0.494,0)
--(axis cs:0.495,0)
--(axis cs:0.496,0)
--(axis cs:0.497,0)
--(axis cs:0.498,0)
--(axis cs:0.499,0)
--(axis cs:0.5,0)
--(axis cs:0.501,0)
--(axis cs:0.502,0)
--(axis cs:0.503,0)
--(axis cs:0.504,0)
--(axis cs:0.505,0)
--(axis cs:0.506,0)
--(axis cs:0.507,0)
--(axis cs:0.508,0)
--(axis cs:0.509,0)
--(axis cs:0.51,0)
--(axis cs:0.511,0)
--(axis cs:0.512,0)
--(axis cs:0.513,0)
--(axis cs:0.514,0)
--(axis cs:0.515,0)
--(axis cs:0.516,0)
--(axis cs:0.517,0)
--(axis cs:0.518,0)
--(axis cs:0.519,0)
--(axis cs:0.52,0)
--(axis cs:0.521,0)
--(axis cs:0.522,0)
--(axis cs:0.523,0)
--(axis cs:0.524,0)
--(axis cs:0.525,0)
--(axis cs:0.526,0)
--(axis cs:0.527,0)
--(axis cs:0.528,0)
--(axis cs:0.529,0)
--(axis cs:0.53,0)
--(axis cs:0.531,0)
--(axis cs:0.532,0)
--(axis cs:0.533,0)
--(axis cs:0.534,0)
--(axis cs:0.535,0)
--(axis cs:0.536,0)
--(axis cs:0.537,0)
--(axis cs:0.538,0)
--(axis cs:0.539,0)
--(axis cs:0.54,0)
--(axis cs:0.541,0)
--(axis cs:0.542,0)
--(axis cs:0.543,0)
--(axis cs:0.544,0)
--(axis cs:0.545,0)
--(axis cs:0.546,0)
--(axis cs:0.547,0)
--(axis cs:0.548,0)
--(axis cs:0.549,0)
--(axis cs:0.55,0)
--(axis cs:0.551,0)
--(axis cs:0.552,0)
--(axis cs:0.553,0)
--(axis cs:0.554,0)
--(axis cs:0.555,0)
--(axis cs:0.556,0)
--(axis cs:0.557,0)
--(axis cs:0.558,0)
--(axis cs:0.559,0)
--(axis cs:0.56,0)
--(axis cs:0.561,0)
--(axis cs:0.562,0)
--(axis cs:0.563,0)
--(axis cs:0.564,0)
--(axis cs:0.565,0)
--(axis cs:0.566,0)
--(axis cs:0.567,0)
--(axis cs:0.568,0)
--(axis cs:0.569,0)
--(axis cs:0.57,0)
--(axis cs:0.571,0)
--(axis cs:0.572,0)
--(axis cs:0.573,0)
--(axis cs:0.574,0)
--(axis cs:0.575,0)
--(axis cs:0.576,0)
--(axis cs:0.577,0)
--(axis cs:0.578,0)
--(axis cs:0.579,0)
--(axis cs:0.58,0)
--(axis cs:0.581,0)
--(axis cs:0.582,0)
--(axis cs:0.583,0)
--(axis cs:0.584,0)
--(axis cs:0.585,0)
--(axis cs:0.586,0)
--(axis cs:0.587,0)
--(axis cs:0.588,0)
--(axis cs:0.589,0)
--(axis cs:0.59,0)
--(axis cs:0.591,0)
--(axis cs:0.592,0)
--(axis cs:0.593,0)
--(axis cs:0.594,0)
--(axis cs:0.595,0)
--(axis cs:0.596,0)
--(axis cs:0.597,0)
--(axis cs:0.598,0)
--(axis cs:0.599,0)
--(axis cs:0.6,0)
--(axis cs:0.601,0)
--(axis cs:0.602,0)
--(axis cs:0.603,0)
--(axis cs:0.604,0)
--(axis cs:0.605,0)
--(axis cs:0.606,0)
--(axis cs:0.607,0)
--(axis cs:0.608,0)
--(axis cs:0.609,0)
--(axis cs:0.61,0)
--(axis cs:0.611,0)
--(axis cs:0.612,0)
--(axis cs:0.613,0)
--(axis cs:0.614,0)
--(axis cs:0.615,0)
--(axis cs:0.616,0)
--(axis cs:0.617,0)
--(axis cs:0.618,0)
--(axis cs:0.619,0)
--(axis cs:0.62,0)
--(axis cs:0.621,0)
--(axis cs:0.622,0)
--(axis cs:0.623,0)
--(axis cs:0.624,0)
--(axis cs:0.625,0)
--(axis cs:0.626,0)
--(axis cs:0.627,0)
--(axis cs:0.628,0)
--(axis cs:0.629,0)
--(axis cs:0.63,0)
--(axis cs:0.631,0)
--(axis cs:0.632,0)
--(axis cs:0.633,0)
--(axis cs:0.634,0)
--(axis cs:0.635,0)
--(axis cs:0.636,0)
--(axis cs:0.637,0)
--(axis cs:0.638,0)
--(axis cs:0.639,0)
--(axis cs:0.64,0)
--(axis cs:0.641,0)
--(axis cs:0.642,0)
--(axis cs:0.643,0)
--(axis cs:0.644,0)
--(axis cs:0.645,0)
--(axis cs:0.646,0)
--(axis cs:0.647,0)
--(axis cs:0.648,0)
--(axis cs:0.649,0)
--(axis cs:0.65,0)
--(axis cs:0.651,0)
--(axis cs:0.652,0)
--(axis cs:0.653,0)
--(axis cs:0.654,0)
--(axis cs:0.655,0)
--(axis cs:0.656,0)
--(axis cs:0.657,0)
--(axis cs:0.658,0)
--(axis cs:0.659,0)
--(axis cs:0.66,0)
--(axis cs:0.661,0)
--(axis cs:0.662,0)
--(axis cs:0.663,0)
--(axis cs:0.664,0)
--(axis cs:0.665,0)
--(axis cs:0.666,0)
--(axis cs:0.667,0)
--(axis cs:0.668,0)
--(axis cs:0.669,0)
--(axis cs:0.67,0)
--(axis cs:0.671,0)
--(axis cs:0.672,0)
--(axis cs:0.673,0)
--(axis cs:0.674,0)
--(axis cs:0.675,0)
--(axis cs:0.676,0)
--(axis cs:0.677,0)
--(axis cs:0.678,0)
--(axis cs:0.679,0)
--(axis cs:0.68,0)
--(axis cs:0.681,0)
--(axis cs:0.682,0)
--(axis cs:0.683,0)
--(axis cs:0.684,0)
--(axis cs:0.685,0)
--(axis cs:0.686,0)
--(axis cs:0.687,0)
--(axis cs:0.688,0)
--(axis cs:0.689,0)
--(axis cs:0.69,0)
--(axis cs:0.691,0)
--(axis cs:0.692,0)
--(axis cs:0.693,0)
--(axis cs:0.694,0)
--(axis cs:0.695,0)
--(axis cs:0.696,0)
--(axis cs:0.697,0)
--(axis cs:0.698,0)
--(axis cs:0.699,0)
--(axis cs:0.7,0)
--(axis cs:0.701,0)
--(axis cs:0.702,0)
--(axis cs:0.703,0)
--(axis cs:0.704,0)
--(axis cs:0.705,0)
--(axis cs:0.706,0)
--(axis cs:0.707,0)
--(axis cs:0.708,0)
--(axis cs:0.709,0)
--(axis cs:0.71,0)
--(axis cs:0.711,0)
--(axis cs:0.712,0)
--(axis cs:0.713,0)
--(axis cs:0.714,0)
--(axis cs:0.715,0)
--(axis cs:0.716,0)
--(axis cs:0.717,0)
--(axis cs:0.718,0)
--(axis cs:0.719,0)
--(axis cs:0.72,0)
--(axis cs:0.721,0)
--(axis cs:0.722,0)
--(axis cs:0.723,0)
--(axis cs:0.724,0)
--(axis cs:0.725,0)
--(axis cs:0.726,0)
--(axis cs:0.727,0)
--(axis cs:0.728,0)
--(axis cs:0.729,0)
--(axis cs:0.73,0)
--(axis cs:0.731,0)
--(axis cs:0.732,0)
--(axis cs:0.733,0)
--(axis cs:0.734,0)
--(axis cs:0.735,0)
--(axis cs:0.736,0)
--(axis cs:0.737,0)
--(axis cs:0.738,0)
--(axis cs:0.739,0)
--(axis cs:0.74,0)
--(axis cs:0.741,0)
--(axis cs:0.742,0)
--(axis cs:0.743,0)
--(axis cs:0.744,0)
--(axis cs:0.745,0)
--(axis cs:0.746,0)
--(axis cs:0.747,0)
--(axis cs:0.748,0)
--(axis cs:0.749,0)
--(axis cs:0.75,0)
--(axis cs:0.751,0)
--(axis cs:0.752,0)
--(axis cs:0.753,0)
--(axis cs:0.754,0)
--(axis cs:0.755,0)
--(axis cs:0.756,0)
--(axis cs:0.757,0)
--(axis cs:0.758,0)
--(axis cs:0.759,0)
--(axis cs:0.76,0)
--(axis cs:0.761,0)
--(axis cs:0.762,0)
--(axis cs:0.763,0)
--(axis cs:0.764,0)
--(axis cs:0.765,0)
--(axis cs:0.766,0)
--(axis cs:0.767,0)
--(axis cs:0.768,0)
--(axis cs:0.769,0)
--(axis cs:0.77,0)
--(axis cs:0.771,0)
--(axis cs:0.772,0)
--(axis cs:0.773,0)
--(axis cs:0.774,0)
--(axis cs:0.775,0)
--(axis cs:0.776,0)
--(axis cs:0.777,0)
--(axis cs:0.778,0)
--(axis cs:0.779,0)
--(axis cs:0.78,0)
--(axis cs:0.781,0)
--(axis cs:0.782,0)
--(axis cs:0.783,0)
--(axis cs:0.784,0)
--(axis cs:0.785,0)
--(axis cs:0.786,0)
--(axis cs:0.787,0)
--(axis cs:0.788,0)
--(axis cs:0.789,0)
--(axis cs:0.79,0)
--(axis cs:0.791,0)
--(axis cs:0.792,0)
--(axis cs:0.793,0)
--(axis cs:0.794,0)
--(axis cs:0.795,0)
--(axis cs:0.796,0)
--(axis cs:0.797,0)
--(axis cs:0.798,0)
--(axis cs:0.799,0)
--(axis cs:0.8,0)
--(axis cs:0.801,0)
--(axis cs:0.802,0)
--(axis cs:0.803,0)
--(axis cs:0.804,0)
--(axis cs:0.805,0)
--(axis cs:0.806,0)
--(axis cs:0.807,0)
--(axis cs:0.808,0)
--(axis cs:0.809,0)
--(axis cs:0.81,0)
--(axis cs:0.811,0)
--(axis cs:0.812,0)
--(axis cs:0.813,0)
--(axis cs:0.814,0)
--(axis cs:0.815,0)
--(axis cs:0.816,0)
--(axis cs:0.817,0)
--(axis cs:0.818,0)
--(axis cs:0.819,0)
--(axis cs:0.82,0)
--(axis cs:0.821,0)
--(axis cs:0.822,0)
--(axis cs:0.823,0)
--(axis cs:0.824,0)
--(axis cs:0.825,0)
--(axis cs:0.826,0)
--(axis cs:0.827,0)
--(axis cs:0.828,0)
--(axis cs:0.829,0)
--(axis cs:0.83,0)
--(axis cs:0.831,0)
--(axis cs:0.832,0)
--(axis cs:0.833,0)
--(axis cs:0.834,0)
--(axis cs:0.835,0)
--(axis cs:0.836,0)
--(axis cs:0.837,0)
--(axis cs:0.838,0)
--(axis cs:0.839,0)
--(axis cs:0.84,0)
--(axis cs:0.841,0)
--(axis cs:0.842,0)
--(axis cs:0.843,0)
--(axis cs:0.844,0)
--(axis cs:0.845,0)
--(axis cs:0.846,0)
--(axis cs:0.847,0)
--(axis cs:0.848,0)
--(axis cs:0.849,0)
--(axis cs:0.85,0)
--(axis cs:0.851,0)
--(axis cs:0.852,0)
--(axis cs:0.853,0)
--(axis cs:0.854,0)
--(axis cs:0.855,0)
--(axis cs:0.856,0)
--(axis cs:0.857,0)
--(axis cs:0.858,0)
--(axis cs:0.859,0)
--(axis cs:0.86,0)
--(axis cs:0.861,0)
--(axis cs:0.862,0)
--(axis cs:0.863,0)
--(axis cs:0.864,0)
--(axis cs:0.865,0)
--(axis cs:0.866,0)
--(axis cs:0.867,0)
--(axis cs:0.868,0)
--(axis cs:0.869,0)
--(axis cs:0.87,0)
--(axis cs:0.871,0)
--(axis cs:0.872,0)
--(axis cs:0.873,0)
--(axis cs:0.874,0)
--(axis cs:0.875,0)
--(axis cs:0.876,0)
--(axis cs:0.877,0)
--(axis cs:0.878,0)
--(axis cs:0.879,0)
--(axis cs:0.88,0)
--(axis cs:0.881,0)
--(axis cs:0.882,0)
--(axis cs:0.883,0)
--(axis cs:0.884,0)
--(axis cs:0.885,0)
--(axis cs:0.886,0)
--cycle;
\path [draw=none, fill=blue, fill opacity=0.5]
(axis cs:0,0)
--(axis cs:0.001,0)
--(axis cs:0.002,0)
--(axis cs:0.003,0)
--(axis cs:0.004,0)
--(axis cs:0.005,0)
--(axis cs:0.006,0)
--(axis cs:0.007,0)
--(axis cs:0.008,0)
--(axis cs:0.009,0)
--(axis cs:0.01,0)
--(axis cs:0.011,0)
--(axis cs:0.012,0)
--(axis cs:0.013,0)
--(axis cs:0.014,0)
--(axis cs:0.015,0)
--(axis cs:0.016,0)
--(axis cs:0.017,0)
--(axis cs:0.018,0)
--(axis cs:0.019,0)
--(axis cs:0.02,0)
--(axis cs:0.021,0)
--(axis cs:0.022,0)
--(axis cs:0.023,0)
--(axis cs:0.024,0)
--(axis cs:0.025,0)
--(axis cs:0.026,0)
--(axis cs:0.027,0)
--(axis cs:0.028,0)
--(axis cs:0.029,0)
--(axis cs:0.03,0)
--(axis cs:0.031,0)
--(axis cs:0.032,0)
--(axis cs:0.033,0)
--(axis cs:0.034,0)
--(axis cs:0.035,0)
--(axis cs:0.036,0)
--(axis cs:0.037,0)
--(axis cs:0.038,0)
--(axis cs:0.039,0)
--(axis cs:0.04,0)
--(axis cs:0.041,0)
--(axis cs:0.042,0)
--(axis cs:0.043,0)
--(axis cs:0.044,0)
--(axis cs:0.045,0)
--(axis cs:0.046,0)
--(axis cs:0.047,0)
--(axis cs:0.048,0)
--(axis cs:0.049,0)
--(axis cs:0.05,0)
--(axis cs:0.051,0)
--(axis cs:0.052,0)
--(axis cs:0.053,0)
--(axis cs:0.054,0)
--(axis cs:0.055,0)
--(axis cs:0.056,0)
--(axis cs:0.057,0)
--(axis cs:0.058,0)
--(axis cs:0.059,0)
--(axis cs:0.06,0)
--(axis cs:0.061,0)
--(axis cs:0.062,0)
--(axis cs:0.063,0)
--(axis cs:0.064,0)
--(axis cs:0.065,0)
--(axis cs:0.066,0)
--(axis cs:0.067,0)
--(axis cs:0.068,0)
--(axis cs:0.069,0)
--(axis cs:0.07,0)
--(axis cs:0.071,0)
--(axis cs:0.072,0)
--(axis cs:0.073,0)
--(axis cs:0.074,0)
--(axis cs:0.075,0)
--(axis cs:0.076,0)
--(axis cs:0.077,0)
--(axis cs:0.078,0)
--(axis cs:0.079,0)
--(axis cs:0.08,0)
--(axis cs:0.081,0)
--(axis cs:0.082,0)
--(axis cs:0.083,0)
--(axis cs:0.084,0)
--(axis cs:0.085,0)
--(axis cs:0.086,0)
--(axis cs:0.087,0)
--(axis cs:0.088,0)
--(axis cs:0.089,0)
--(axis cs:0.09,0)
--(axis cs:0.091,0)
--(axis cs:0.092,0)
--(axis cs:0.093,0)
--(axis cs:0.094,0)
--(axis cs:0.095,0)
--(axis cs:0.096,0)
--(axis cs:0.097,0)
--(axis cs:0.098,0)
--(axis cs:0.099,0)
--(axis cs:0.1,0)
--(axis cs:0.101,0)
--(axis cs:0.102,0)
--(axis cs:0.103,0)
--(axis cs:0.104,0)
--(axis cs:0.105,0)
--(axis cs:0.106,0)
--(axis cs:0.107,0)
--(axis cs:0.108,0)
--(axis cs:0.109,0)
--(axis cs:0.11,0)
--(axis cs:0.111,0)
--(axis cs:0.112,0)
--(axis cs:0.113,0)
--(axis cs:0.114,0)
--(axis cs:0.115,0)
--(axis cs:0.116,0)
--(axis cs:0.117,0)
--(axis cs:0.118,0)
--(axis cs:0.119,0)
--(axis cs:0.12,0)
--(axis cs:0.121,0)
--(axis cs:0.122,0)
--(axis cs:0.123,0)
--(axis cs:0.124,0)
--(axis cs:0.125,0)
--(axis cs:0.126,0)
--(axis cs:0.127,0)
--(axis cs:0.128,0)
--(axis cs:0.129,0)
--(axis cs:0.13,0)
--(axis cs:0.131,0)
--(axis cs:0.132,0)
--(axis cs:0.133,0)
--(axis cs:0.134,0)
--(axis cs:0.135,0)
--(axis cs:0.136,0)
--(axis cs:0.137,0)
--(axis cs:0.138,0)
--(axis cs:0.139,0)
--(axis cs:0.14,0)
--(axis cs:0.141,0)
--(axis cs:0.142,0)
--(axis cs:0.143,0)
--(axis cs:0.144,0)
--(axis cs:0.145,0)
--(axis cs:0.146,0)
--(axis cs:0.147,0)
--(axis cs:0.148,0)
--(axis cs:0.149,0)
--(axis cs:0.15,0)
--(axis cs:0.151,0)
--(axis cs:0.152,0)
--(axis cs:0.153,0)
--(axis cs:0.154,0)
--(axis cs:0.155,0)
--(axis cs:0.156,0)
--(axis cs:0.157,0)
--(axis cs:0.158,0)
--(axis cs:0.159,0)
--(axis cs:0.16,0)
--(axis cs:0.161,0)
--(axis cs:0.162,0)
--(axis cs:0.163,0)
--(axis cs:0.164,0)
--(axis cs:0.165,0)
--(axis cs:0.166,0)
--(axis cs:0.167,0)
--(axis cs:0.168,0)
--(axis cs:0.169,0)
--(axis cs:0.17,0)
--(axis cs:0.171,0)
--(axis cs:0.172,0)
--(axis cs:0.173,0)
--(axis cs:0.174,0)
--(axis cs:0.175,0)
--(axis cs:0.176,0)
--(axis cs:0.177,0)
--(axis cs:0.178,0)
--(axis cs:0.179,0)
--(axis cs:0.18,0)
--(axis cs:0.181,0)
--(axis cs:0.182,0)
--(axis cs:0.183,0)
--(axis cs:0.184,0)
--(axis cs:0.185,0)
--(axis cs:0.186,0)
--(axis cs:0.187,0)
--(axis cs:0.188,0)
--(axis cs:0.189,0)
--(axis cs:0.19,0)
--(axis cs:0.191,0)
--(axis cs:0.192,0)
--(axis cs:0.193,0)
--(axis cs:0.194,0)
--(axis cs:0.195,0)
--(axis cs:0.196,0)
--(axis cs:0.197,0)
--(axis cs:0.198,0)
--(axis cs:0.199,0)
--(axis cs:0.2,0)
--(axis cs:0.201,0)
--(axis cs:0.202,0)
--(axis cs:0.203,0)
--(axis cs:0.204,0)
--(axis cs:0.205,0)
--(axis cs:0.206,0)
--(axis cs:0.207,0)
--(axis cs:0.208,0)
--(axis cs:0.209,0)
--(axis cs:0.21,0)
--(axis cs:0.211,0)
--(axis cs:0.212,0)
--(axis cs:0.213,0)
--(axis cs:0.214,0)
--(axis cs:0.215,0)
--(axis cs:0.216,0)
--(axis cs:0.217,0)
--(axis cs:0.218,0)
--(axis cs:0.219,0)
--(axis cs:0.22,0)
--(axis cs:0.221,0)
--(axis cs:0.222,0)
--(axis cs:0.223,0)
--(axis cs:0.224,0)
--(axis cs:0.225,0)
--(axis cs:0.226,0)
--(axis cs:0.227,0)
--(axis cs:0.228,0)
--(axis cs:0.229,0)
--(axis cs:0.23,0)
--(axis cs:0.231,0)
--(axis cs:0.232,0)
--(axis cs:0.233,0)
--(axis cs:0.234,0)
--(axis cs:0.235,0)
--(axis cs:0.236,0)
--(axis cs:0.237,0)
--(axis cs:0.238,0)
--(axis cs:0.239,0)
--(axis cs:0.24,0)
--(axis cs:0.241,0)
--(axis cs:0.242,0)
--(axis cs:0.243,0)
--(axis cs:0.244,0)
--(axis cs:0.245,0)
--(axis cs:0.246,0)
--(axis cs:0.247,0)
--(axis cs:0.248,0)
--(axis cs:0.249,0)
--(axis cs:0.25,0)
--(axis cs:0.251,0)
--(axis cs:0.252,0)
--(axis cs:0.253,0)
--(axis cs:0.254,0)
--(axis cs:0.255,0)
--(axis cs:0.256,0)
--(axis cs:0.257,0)
--(axis cs:0.258,0)
--(axis cs:0.259,0)
--(axis cs:0.26,0)
--(axis cs:0.261,0)
--(axis cs:0.262,0)
--(axis cs:0.263,0)
--(axis cs:0.264,0)
--(axis cs:0.265,0)
--(axis cs:0.266,0)
--(axis cs:0.267,0)
--(axis cs:0.268,0)
--(axis cs:0.269,0)
--(axis cs:0.27,0)
--(axis cs:0.271,0)
--(axis cs:0.272,0)
--(axis cs:0.273,0)
--(axis cs:0.274,0)
--(axis cs:0.275,0)
--(axis cs:0.276,0)
--(axis cs:0.277,0)
--(axis cs:0.278,0)
--(axis cs:0.279,0)
--(axis cs:0.28,0)
--(axis cs:0.281,0)
--(axis cs:0.282,0)
--(axis cs:0.283,0)
--(axis cs:0.284,0)
--(axis cs:0.285,0)
--(axis cs:0.286,0)
--(axis cs:0.287,0)
--(axis cs:0.288,0)
--(axis cs:0.289,0)
--(axis cs:0.29,0)
--(axis cs:0.291,0)
--(axis cs:0.292,0)
--(axis cs:0.293,0)
--(axis cs:0.294,0)
--(axis cs:0.295,0)
--(axis cs:0.296,0)
--(axis cs:0.297,0)
--(axis cs:0.298,0)
--(axis cs:0.299,0)
--(axis cs:0.3,0)
--(axis cs:0.301,0)
--(axis cs:0.302,0)
--(axis cs:0.303,0)
--(axis cs:0.304,0)
--(axis cs:0.305,0)
--(axis cs:0.306,0)
--(axis cs:0.307,0)
--(axis cs:0.308,0)
--(axis cs:0.309,0)
--(axis cs:0.31,0)
--(axis cs:0.311,0)
--(axis cs:0.312,0)
--(axis cs:0.313,0)
--(axis cs:0.314,0)
--(axis cs:0.315,0)
--(axis cs:0.316,0)
--(axis cs:0.317,0)
--(axis cs:0.318,0)
--(axis cs:0.319,0)
--(axis cs:0.32,0)
--(axis cs:0.321,0)
--(axis cs:0.322,0)
--(axis cs:0.323,0)
--(axis cs:0.324,0)
--(axis cs:0.325,0)
--(axis cs:0.326,0)
--(axis cs:0.327,0)
--(axis cs:0.328,0)
--(axis cs:0.329,0)
--(axis cs:0.33,0)
--(axis cs:0.331,0)
--(axis cs:0.332,0)
--(axis cs:0.333,0)
--(axis cs:0.334,0)
--(axis cs:0.335,0)
--(axis cs:0.336,0)
--(axis cs:0.337,0)
--(axis cs:0.338,0)
--(axis cs:0.339,0)
--(axis cs:0.34,0)
--(axis cs:0.341,0)
--(axis cs:0.342,0)
--(axis cs:0.343,0)
--(axis cs:0.344,0)
--(axis cs:0.345,0)
--(axis cs:0.346,0)
--(axis cs:0.347,0)
--(axis cs:0.348,0)
--(axis cs:0.349,0)
--(axis cs:0.35,0)
--(axis cs:0.351,0)
--(axis cs:0.352,0)
--(axis cs:0.353,0)
--(axis cs:0.354,0)
--(axis cs:0.355,0)
--(axis cs:0.356,0)
--(axis cs:0.357,0)
--(axis cs:0.358,0)
--(axis cs:0.359,0)
--(axis cs:0.36,0)
--(axis cs:0.361,0)
--(axis cs:0.362,0)
--(axis cs:0.363,0)
--(axis cs:0.364,0)
--(axis cs:0.365,0)
--(axis cs:0.366,0)
--(axis cs:0.367,0)
--(axis cs:0.368,0)
--(axis cs:0.369,0)
--(axis cs:0.37,0)
--(axis cs:0.371,0)
--(axis cs:0.372,0)
--(axis cs:0.373,0)
--(axis cs:0.374,0)
--(axis cs:0.375,0)
--(axis cs:0.376,0)
--(axis cs:0.377,0)
--(axis cs:0.378,0)
--(axis cs:0.379,0)
--(axis cs:0.38,0)
--(axis cs:0.381,0)
--(axis cs:0.382,0)
--(axis cs:0.383,0)
--(axis cs:0.384,0)
--(axis cs:0.385,0)
--(axis cs:0.386,0)
--(axis cs:0.387,0)
--(axis cs:0.388,0)
--(axis cs:0.389,0)
--(axis cs:0.39,0)
--(axis cs:0.391,0)
--(axis cs:0.392,0)
--(axis cs:0.393,0)
--(axis cs:0.394,0)
--(axis cs:0.395,0)
--(axis cs:0.396,0)
--(axis cs:0.397,0)
--(axis cs:0.398,0)
--(axis cs:0.399,0)
--(axis cs:0.4,0)
--(axis cs:0.401,0)
--(axis cs:0.402,0)
--(axis cs:0.403,0)
--(axis cs:0.404,0)
--(axis cs:0.405,0)
--(axis cs:0.406,0)
--(axis cs:0.407,0)
--(axis cs:0.408,0)
--(axis cs:0.409,0)
--(axis cs:0.41,0)
--(axis cs:0.411,0)
--(axis cs:0.412,0)
--(axis cs:0.413,0)
--(axis cs:0.414,0)
--(axis cs:0.415,0)
--(axis cs:0.416,0)
--(axis cs:0.417,0)
--(axis cs:0.418,0)
--(axis cs:0.419,0)
--(axis cs:0.42,0)
--(axis cs:0.421,0)
--(axis cs:0.422,0)
--(axis cs:0.423,0)
--(axis cs:0.424,0)
--(axis cs:0.425,0)
--(axis cs:0.426,0)
--(axis cs:0.427,0)
--(axis cs:0.428,0)
--(axis cs:0.429,0)
--(axis cs:0.43,0)
--(axis cs:0.431,0)
--(axis cs:0.432,0)
--(axis cs:0.433,0)
--(axis cs:0.434,0)
--(axis cs:0.435,0)
--(axis cs:0.436,0)
--(axis cs:0.437,0)
--(axis cs:0.438,0)
--(axis cs:0.439,0)
--(axis cs:0.44,0)
--(axis cs:0.441,0)
--(axis cs:0.442,0)
--(axis cs:0.443,0)
--(axis cs:0.444,0)
--(axis cs:0.445,0)
--(axis cs:0.446,0)
--(axis cs:0.447,0)
--(axis cs:0.448,0)
--(axis cs:0.449,0)
--(axis cs:0.45,0)
--(axis cs:0.451,0)
--(axis cs:0.452,0)
--(axis cs:0.453,0)
--(axis cs:0.454,0)
--(axis cs:0.455,0)
--(axis cs:0.456,0)
--(axis cs:0.457,0)
--(axis cs:0.458,0)
--(axis cs:0.459,0)
--(axis cs:0.46,0)
--(axis cs:0.461,0)
--(axis cs:0.462,0)
--(axis cs:0.463,0)
--(axis cs:0.464,0)
--(axis cs:0.465,0)
--(axis cs:0.466,0)
--(axis cs:0.467,0)
--(axis cs:0.468,0)
--(axis cs:0.469,0)
--(axis cs:0.47,0)
--(axis cs:0.471,0)
--(axis cs:0.472,0)
--(axis cs:0.473,0)
--(axis cs:0.474,0)
--(axis cs:0.475,0)
--(axis cs:0.476,0)
--(axis cs:0.477,0)
--(axis cs:0.478,0)
--(axis cs:0.479,0)
--(axis cs:0.48,0)
--(axis cs:0.481,0)
--(axis cs:0.482,0)
--(axis cs:0.483,0)
--(axis cs:0.484,0)
--(axis cs:0.485,0)
--(axis cs:0.486,0)
--(axis cs:0.487,0)
--(axis cs:0.488,0)
--(axis cs:0.489,0)
--(axis cs:0.49,0)
--(axis cs:0.491,0)
--(axis cs:0.492,0)
--(axis cs:0.493,0)
--(axis cs:0.494,0)
--(axis cs:0.495,0)
--(axis cs:0.496,0)
--(axis cs:0.497,0)
--(axis cs:0.498,0)
--(axis cs:0.499,0)
--(axis cs:0.5,0)
--(axis cs:0.501,0)
--(axis cs:0.502,0)
--(axis cs:0.503,0)
--(axis cs:0.504,0)
--(axis cs:0.505,0)
--(axis cs:0.506,0)
--(axis cs:0.507,0)
--(axis cs:0.508,0)
--(axis cs:0.509,0)
--(axis cs:0.51,0)
--(axis cs:0.511,0)
--(axis cs:0.512,0)
--(axis cs:0.513,0)
--(axis cs:0.514,0)
--(axis cs:0.515,0)
--(axis cs:0.516,0)
--(axis cs:0.517,0)
--(axis cs:0.518,0)
--(axis cs:0.519,0)
--(axis cs:0.52,0)
--(axis cs:0.521,0)
--(axis cs:0.522,0)
--(axis cs:0.523,0)
--(axis cs:0.524,0)
--(axis cs:0.525,0)
--(axis cs:0.526,0)
--(axis cs:0.527,0)
--(axis cs:0.528,0)
--(axis cs:0.529,0)
--(axis cs:0.53,0)
--(axis cs:0.531,0)
--(axis cs:0.532,0)
--(axis cs:0.533,0)
--(axis cs:0.534,0)
--(axis cs:0.535,0)
--(axis cs:0.536,0)
--(axis cs:0.537,0)
--(axis cs:0.538,0)
--(axis cs:0.539,0)
--(axis cs:0.54,0)
--(axis cs:0.541,0)
--(axis cs:0.542,0)
--(axis cs:0.543,0)
--(axis cs:0.544,0)
--(axis cs:0.545,0)
--(axis cs:0.546,0)
--(axis cs:0.547,0)
--(axis cs:0.548,0)
--(axis cs:0.549,0)
--(axis cs:0.55,0)
--(axis cs:0.551,0)
--(axis cs:0.552,0)
--(axis cs:0.553,0)
--(axis cs:0.554,0)
--(axis cs:0.555,0)
--(axis cs:0.556,0)
--(axis cs:0.557,0)
--(axis cs:0.558,0)
--(axis cs:0.559,0)
--(axis cs:0.56,0)
--(axis cs:0.561,0)
--(axis cs:0.562,0)
--(axis cs:0.563,0)
--(axis cs:0.564,0)
--(axis cs:0.565,0)
--(axis cs:0.566,0)
--(axis cs:0.567,0)
--(axis cs:0.568,0)
--(axis cs:0.569,0)
--(axis cs:0.57,0)
--(axis cs:0.571,0)
--(axis cs:0.572,0)
--(axis cs:0.573,0)
--(axis cs:0.574,0)
--(axis cs:0.575,0)
--(axis cs:0.576,0)
--(axis cs:0.577,0)
--(axis cs:0.578,0)
--(axis cs:0.579,0)
--(axis cs:0.58,0)
--(axis cs:0.581,0)
--(axis cs:0.582,0)
--(axis cs:0.583,0)
--(axis cs:0.584,0)
--(axis cs:0.585,0)
--(axis cs:0.586,0)
--(axis cs:0.587,0)
--(axis cs:0.588,0)
--(axis cs:0.589,0)
--(axis cs:0.59,0)
--(axis cs:0.591,0)
--(axis cs:0.592,0)
--(axis cs:0.593,0)
--(axis cs:0.594,0)
--(axis cs:0.595,0)
--(axis cs:0.596,0)
--(axis cs:0.597,0)
--(axis cs:0.598,0)
--(axis cs:0.599,0)
--(axis cs:0.6,0)
--(axis cs:0.601,0)
--(axis cs:0.602,0)
--(axis cs:0.603,0)
--(axis cs:0.604,0)
--(axis cs:0.605,0)
--(axis cs:0.606,0)
--(axis cs:0.607,0)
--(axis cs:0.608,0)
--(axis cs:0.609,0)
--(axis cs:0.61,0)
--(axis cs:0.611,0)
--(axis cs:0.612,0)
--(axis cs:0.613,0)
--(axis cs:0.614,0)
--(axis cs:0.615,0)
--(axis cs:0.616,0)
--(axis cs:0.617,0)
--(axis cs:0.618,0)
--(axis cs:0.619,0)
--(axis cs:0.62,0)
--(axis cs:0.621,0)
--(axis cs:0.622,0)
--(axis cs:0.623,0)
--(axis cs:0.624,0)
--(axis cs:0.625,0)
--(axis cs:0.626,0)
--(axis cs:0.627,0)
--(axis cs:0.628,0)
--(axis cs:0.629,0)
--(axis cs:0.63,0)
--(axis cs:0.631,0)
--(axis cs:0.632,0)
--(axis cs:0.633,0)
--(axis cs:0.634,0)
--(axis cs:0.635,0)
--(axis cs:0.636,0)
--(axis cs:0.637,0)
--(axis cs:0.638,0)
--(axis cs:0.639,0)
--(axis cs:0.64,0)
--(axis cs:0.641,0)
--(axis cs:0.642,0)
--(axis cs:0.643,0)
--(axis cs:0.644,0)
--(axis cs:0.645,0)
--(axis cs:0.646,0)
--(axis cs:0.647,0)
--(axis cs:0.648,0)
--(axis cs:0.649,0)
--(axis cs:0.65,0)
--(axis cs:0.651,0)
--(axis cs:0.652,0)
--(axis cs:0.653,0)
--(axis cs:0.654,0)
--(axis cs:0.655,0)
--(axis cs:0.656,0)
--(axis cs:0.657,0)
--(axis cs:0.658,0)
--(axis cs:0.659,0)
--(axis cs:0.66,0)
--(axis cs:0.661,0)
--(axis cs:0.662,0)
--(axis cs:0.663,0)
--(axis cs:0.664,0)
--(axis cs:0.665,0)
--(axis cs:0.666,0)
--(axis cs:0.667,0)
--(axis cs:0.668,0)
--(axis cs:0.669,0)
--(axis cs:0.67,0)
--(axis cs:0.671,0)
--(axis cs:0.672,0)
--(axis cs:0.673,0)
--(axis cs:0.674,0)
--(axis cs:0.675,0)
--(axis cs:0.676,0)
--(axis cs:0.677,0)
--(axis cs:0.678,0)
--(axis cs:0.679,0)
--(axis cs:0.68,0)
--(axis cs:0.681,0)
--(axis cs:0.682,0)
--(axis cs:0.683,0)
--(axis cs:0.684,0)
--(axis cs:0.685,0)
--(axis cs:0.686,0)
--(axis cs:0.687,0)
--(axis cs:0.688,0)
--(axis cs:0.689,0)
--(axis cs:0.69,0)
--(axis cs:0.691,0)
--(axis cs:0.692,0)
--(axis cs:0.693,0)
--(axis cs:0.694,0)
--(axis cs:0.695,0)
--(axis cs:0.696,0)
--(axis cs:0.697,0)
--(axis cs:0.698,0)
--(axis cs:0.699,0)
--(axis cs:0.7,0)
--(axis cs:0.701,0)
--(axis cs:0.702,0)
--(axis cs:0.703,0)
--(axis cs:0.704,0)
--(axis cs:0.705,0)
--(axis cs:0.706,0)
--(axis cs:0.707,0)
--(axis cs:0.708,0)
--(axis cs:0.709,0)
--(axis cs:0.71,0)
--(axis cs:0.711,0)
--(axis cs:0.712,0)
--(axis cs:0.713,0)
--(axis cs:0.714,0)
--(axis cs:0.715,0)
--(axis cs:0.716,0)
--(axis cs:0.717,0)
--(axis cs:0.718,0)
--(axis cs:0.719,0)
--(axis cs:0.72,0)
--(axis cs:0.721,0)
--(axis cs:0.722,0)
--(axis cs:0.723,0)
--(axis cs:0.724,0)
--(axis cs:0.725,0)
--(axis cs:0.726,0)
--(axis cs:0.727,0)
--(axis cs:0.728,0)
--(axis cs:0.729,0)
--(axis cs:0.73,0)
--(axis cs:0.731,0)
--(axis cs:0.732,0)
--(axis cs:0.733,0)
--(axis cs:0.734,0)
--(axis cs:0.735,0)
--(axis cs:0.736,0)
--(axis cs:0.737,0)
--(axis cs:0.738,0)
--(axis cs:0.739,0)
--(axis cs:0.74,0)
--(axis cs:0.741,0)
--(axis cs:0.742,0)
--(axis cs:0.743,0)
--(axis cs:0.744,0)
--(axis cs:0.745,0)
--(axis cs:0.746,0)
--(axis cs:0.747,0)
--(axis cs:0.748,0)
--(axis cs:0.749,0)
--(axis cs:0.75,0)
--(axis cs:0.751,0)
--(axis cs:0.752,0)
--(axis cs:0.753,0)
--(axis cs:0.754,0)
--(axis cs:0.755,0)
--(axis cs:0.756,0)
--(axis cs:0.757,0)
--(axis cs:0.758,0)
--(axis cs:0.759,0)
--(axis cs:0.76,0)
--(axis cs:0.761,0)
--(axis cs:0.762,0)
--(axis cs:0.763,0)
--(axis cs:0.764,0)
--(axis cs:0.765,0)
--(axis cs:0.766,0)
--(axis cs:0.767,0)
--(axis cs:0.768,0)
--(axis cs:0.769,0)
--(axis cs:0.77,0)
--(axis cs:0.771,0)
--(axis cs:0.772,0)
--(axis cs:0.773,0)
--(axis cs:0.774,0)
--(axis cs:0.775,0)
--(axis cs:0.776,0)
--(axis cs:0.777,0)
--(axis cs:0.778,0)
--(axis cs:0.779,0)
--(axis cs:0.78,0)
--(axis cs:0.781,0)
--(axis cs:0.782,0)
--(axis cs:0.783,0)
--(axis cs:0.784,0)
--(axis cs:0.785,0)
--(axis cs:0.786,0)
--(axis cs:0.787,0)
--(axis cs:0.788,0)
--(axis cs:0.789,0)
--(axis cs:0.79,0)
--(axis cs:0.791,0)
--(axis cs:0.792,0)
--(axis cs:0.793,0)
--(axis cs:0.794,0)
--(axis cs:0.795,0)
--(axis cs:0.796,0)
--(axis cs:0.797,0)
--(axis cs:0.798,0)
--(axis cs:0.799,0)
--(axis cs:0.8,0)
--(axis cs:0.801,0)
--(axis cs:0.802,0)
--(axis cs:0.803,0)
--(axis cs:0.804,0)
--(axis cs:0.805,0)
--(axis cs:0.806,0)
--(axis cs:0.807,0)
--(axis cs:0.808,0)
--(axis cs:0.809,0)
--(axis cs:0.81,0)
--(axis cs:0.811,0)
--(axis cs:0.812,0)
--(axis cs:0.813,0)
--(axis cs:0.814,0)
--(axis cs:0.815,0)
--(axis cs:0.816,0)
--(axis cs:0.817,0)
--(axis cs:0.818,0)
--(axis cs:0.819,0)
--(axis cs:0.82,0)
--(axis cs:0.821,0)
--(axis cs:0.822,0)
--(axis cs:0.823,0)
--(axis cs:0.824,0)
--(axis cs:0.825,0)
--(axis cs:0.826,0)
--(axis cs:0.827,0)
--(axis cs:0.828,0)
--(axis cs:0.829,0)
--(axis cs:0.83,0)
--(axis cs:0.831,0)
--(axis cs:0.832,0)
--(axis cs:0.833,0)
--(axis cs:0.834,0)
--(axis cs:0.835,0)
--(axis cs:0.836,0)
--(axis cs:0.837,0)
--(axis cs:0.838,0)
--(axis cs:0.839,0)
--(axis cs:0.84,0)
--(axis cs:0.841,0)
--(axis cs:0.842,0)
--(axis cs:0.843,0)
--(axis cs:0.844,0)
--(axis cs:0.845,0)
--(axis cs:0.846,0)
--(axis cs:0.847,0)
--(axis cs:0.848,0)
--(axis cs:0.849,0)
--(axis cs:0.85,0)
--(axis cs:0.851,0)
--(axis cs:0.852,0)
--(axis cs:0.853,0)
--(axis cs:0.854,0)
--(axis cs:0.855,0)
--(axis cs:0.856,0)
--(axis cs:0.857,0)
--(axis cs:0.858,0)
--(axis cs:0.859,0)
--(axis cs:0.86,0)
--(axis cs:0.861,0)
--(axis cs:0.862,0)
--(axis cs:0.863,0)
--(axis cs:0.864,0)
--(axis cs:0.865,0)
--(axis cs:0.866,0)
--(axis cs:0.867,0)
--(axis cs:0.868,0)
--(axis cs:0.869,0)
--(axis cs:0.87,0)
--(axis cs:0.871,0)
--(axis cs:0.872,0)
--(axis cs:0.873,0)
--(axis cs:0.874,0)
--(axis cs:0.875,0)
--(axis cs:0.876,0)
--(axis cs:0.877,0)
--(axis cs:0.878,0)
--(axis cs:0.879,0)
--(axis cs:0.88,0)
--(axis cs:0.881,0)
--(axis cs:0.882,0)
--(axis cs:0.883,0)
--(axis cs:0.884,0)
--(axis cs:0.885,0)
--(axis cs:0.886,0)
--cycle;
\addplot [semithick, color0, opacity=0.8]
table {%
0 0
0.001 322.58271626249
0.002 4.5485439001751e-06
0.003 1.08331330228178e-16
0.004 1.85338944612998e-28
0.005 7.37735891461139e-41
0.006 1.15314008937915e-53
0.007 9.35747394104202e-67
0.008 4.65685972914961e-80
0.009 1.58252448966646e-93
0.01 3.95169462233851e-107
0.011 7.64016500319958e-121
0.012 1.18869226225084e-134
0.013 1.53251456003575e-148
0.014 1.67490368336033e-162
0.015 1.57998245946121e-176
0.016 1.30525638137975e-190
0.017 9.55588966251468e-205
0.018 6.26093883958636e-219
0.019 3.70139413359828e-233
0.02 1.98819757673142e-247
0.021 9.76084292719275e-262
0.022 4.40201301358189e-276
0.023 1.83170729292297e-290
0.024 7.05925369579821e-305
0.025 2.52818331633424e-319
0.026 0
0.027 0
0.028 0
0.029 0
0.03 0
0.031 0
0.032 0
0.033 0
0.034 0
0.035 0
0.036 0
0.037 0
0.038 0
0.039 0
0.04 0
0.041 0
0.042 0
0.043 0
0.044 0
0.045 0
0.046 0
0.047 0
0.048 0
0.049 0
0.05 0
0.051 0
0.052 0
0.053 0
0.054 0
0.055 0
0.056 0
0.057 0
0.058 0
0.059 0
0.06 0
0.061 0
0.062 0
0.063 0
0.064 0
0.065 0
0.066 0
0.067 0
0.068 0
0.069 0
0.07 0
0.071 0
0.072 0
0.073 0
0.074 0
0.075 0
0.076 0
0.077 0
0.078 0
0.079 0
0.08 0
0.081 0
0.082 0
0.083 0
0.084 0
0.085 0
0.086 0
0.087 0
0.088 0
0.089 0
0.09 0
0.091 0
0.092 0
0.093 0
0.094 0
0.095 0
0.096 0
0.097 0
0.098 0
0.099 0
0.1 0
0.101 0
0.102 0
0.103 0
0.104 0
0.105 0
0.106 0
0.107 0
0.108 0
0.109 0
0.11 0
0.111 0
0.112 0
0.113 0
0.114 0
0.115 0
0.116 0
0.117 0
0.118 0
0.119 0
0.12 0
0.121 0
0.122 0
0.123 0
0.124 0
0.125 0
0.126 0
0.127 0
0.128 0
0.129 0
0.13 0
0.131 0
0.132 0
0.133 0
0.134 0
0.135 0
0.136 0
0.137 0
0.138 0
0.139 0
0.14 0
0.141 0
0.142 0
0.143 0
0.144 0
0.145 0
0.146 0
0.147 0
0.148 0
0.149 0
0.15 0
0.151 0
0.152 0
0.153 0
0.154 0
0.155 0
0.156 0
0.157 0
0.158 0
0.159 0
0.16 0
0.161 0
0.162 0
0.163 0
0.164 0
0.165 0
0.166 0
0.167 0
0.168 0
0.169 0
0.17 0
0.171 0
0.172 0
0.173 0
0.174 0
0.175 0
0.176 0
0.177 0
0.178 0
0.179 0
0.18 0
0.181 0
0.182 0
0.183 0
0.184 0
0.185 0
0.186 0
0.187 0
0.188 0
0.189 0
0.19 0
0.191 0
0.192 0
0.193 0
0.194 0
0.195 0
0.196 0
0.197 0
0.198 0
0.199 0
0.2 0
0.201 0
0.202 0
0.203 0
0.204 0
0.205 0
0.206 0
0.207 0
0.208 0
0.209 0
0.21 0
0.211 0
0.212 0
0.213 0
0.214 0
0.215 0
0.216 0
0.217 0
0.218 0
0.219 0
0.22 0
0.221 0
0.222 0
0.223 0
0.224 0
0.225 0
0.226 0
0.227 0
0.228 0
0.229 0
0.23 0
0.231 0
0.232 0
0.233 0
0.234 0
0.235 0
0.236 0
0.237 0
0.238 0
0.239 0
0.24 0
0.241 0
0.242 0
0.243 0
0.244 0
0.245 0
0.246 0
0.247 0
0.248 0
0.249 0
0.25 0
0.251 0
0.252 0
0.253 0
0.254 0
0.255 0
0.256 0
0.257 0
0.258 0
0.259 0
0.26 0
0.261 0
0.262 0
0.263 0
0.264 0
0.265 0
0.266 0
0.267 0
0.268 0
0.269 0
0.27 0
0.271 0
0.272 0
0.273 0
0.274 0
0.275 0
0.276 0
0.277 0
0.278 0
0.279 0
0.28 0
0.281 0
0.282 0
0.283 0
0.284 0
0.285 0
0.286 0
0.287 0
0.288 0
0.289 0
0.29 0
0.291 0
0.292 0
0.293 0
0.294 0
0.295 0
0.296 0
0.297 0
0.298 0
0.299 0
0.3 0
0.301 0
0.302 0
0.303 0
0.304 0
0.305 0
0.306 0
0.307 0
0.308 0
0.309 0
0.31 0
0.311 0
0.312 0
0.313 0
0.314 0
0.315 0
0.316 0
0.317 0
0.318 0
0.319 0
0.32 0
0.321 0
0.322 0
0.323 0
0.324 0
0.325 0
0.326 0
0.327 0
0.328 0
0.329 0
0.33 0
0.331 0
0.332 0
0.333 0
0.334 0
0.335 0
0.336 0
0.337 0
0.338 0
0.339 0
0.34 0
0.341 0
0.342 0
0.343 0
0.344 0
0.345 0
0.346 0
0.347 0
0.348 0
0.349 0
0.35 0
0.351 0
0.352 0
0.353 0
0.354 0
0.355 0
0.356 0
0.357 0
0.358 0
0.359 0
0.36 0
0.361 0
0.362 0
0.363 0
0.364 0
0.365 0
0.366 0
0.367 0
0.368 0
0.369 0
0.37 0
0.371 0
0.372 0
0.373 0
0.374 0
0.375 0
0.376 0
0.377 0
0.378 0
0.379 0
0.38 0
0.381 0
0.382 0
0.383 0
0.384 0
0.385 0
0.386 0
0.387 0
0.388 0
0.389 0
0.39 0
0.391 0
0.392 0
0.393 0
0.394 0
0.395 0
0.396 0
0.397 0
0.398 0
0.399 0
0.4 0
0.401 0
0.402 0
0.403 0
0.404 0
0.405 0
0.406 0
0.407 0
0.408 0
0.409 0
0.41 0
0.411 0
0.412 0
0.413 0
0.414 0
0.415 0
0.416 0
0.417 0
0.418 0
0.419 0
0.42 0
0.421 0
0.422 0
0.423 0
0.424 0
0.425 0
0.426 0
0.427 0
0.428 0
0.429 0
0.43 0
0.431 0
0.432 0
0.433 0
0.434 0
0.435 0
0.436 0
0.437 0
0.438 0
0.439 0
0.44 0
0.441 0
0.442 0
0.443 0
0.444 0
0.445 0
0.446 0
0.447 0
0.448 0
0.449 0
0.45 0
0.451 0
0.452 0
0.453 0
0.454 0
0.455 0
0.456 0
0.457 0
0.458 0
0.459 0
0.46 0
0.461 0
0.462 0
0.463 0
0.464 0
0.465 0
0.466 0
0.467 0
0.468 0
0.469 0
0.47 0
0.471 0
0.472 0
0.473 0
0.474 0
0.475 0
0.476 0
0.477 0
0.478 0
0.479 0
0.48 0
0.481 0
0.482 0
0.483 0
0.484 0
0.485 0
0.486 0
0.487 0
0.488 0
0.489 0
0.49 0
0.491 0
0.492 0
0.493 0
0.494 0
0.495 0
0.496 0
0.497 0
0.498 0
0.499 0
0.5 0
0.501 0
0.502 0
0.503 0
0.504 0
0.505 0
0.506 0
0.507 0
0.508 0
0.509 0
0.51 0
0.511 0
0.512 0
0.513 0
0.514 0
0.515 0
0.516 0
0.517 0
0.518 0
0.519 0
0.52 0
0.521 0
0.522 0
0.523 0
0.524 0
0.525 0
0.526 0
0.527 0
0.528 0
0.529 0
0.53 0
0.531 0
0.532 0
0.533 0
0.534 0
0.535 0
0.536 0
0.537 0
0.538 0
0.539 0
0.54 0
0.541 0
0.542 0
0.543 0
0.544 0
0.545 0
0.546 0
0.547 0
0.548 0
0.549 0
0.55 0
0.551 0
0.552 0
0.553 0
0.554 0
0.555 0
0.556 0
0.557 0
0.558 0
0.559 0
0.56 0
0.561 0
0.562 0
0.563 0
0.564 0
0.565 0
0.566 0
0.567 0
0.568 0
0.569 0
0.57 0
0.571 0
0.572 0
0.573 0
0.574 0
0.575 0
0.576 0
0.577 0
0.578 0
0.579 0
0.58 0
0.581 0
0.582 0
0.583 0
0.584 0
0.585 0
0.586 0
0.587 0
0.588 0
0.589 0
0.59 0
0.591 0
0.592 0
0.593 0
0.594 0
0.595 0
0.596 0
0.597 0
0.598 0
0.599 0
0.6 0
0.601 0
0.602 0
0.603 0
0.604 0
0.605 0
0.606 0
0.607 0
0.608 0
0.609 0
0.61 0
0.611 0
0.612 0
0.613 0
0.614 0
0.615 0
0.616 0
0.617 0
0.618 0
0.619 0
0.62 0
0.621 0
0.622 0
0.623 0
0.624 0
0.625 0
0.626 0
0.627 0
0.628 0
0.629 0
0.63 0
0.631 0
0.632 0
0.633 0
0.634 0
0.635 0
0.636 0
0.637 0
0.638 0
0.639 0
0.64 0
0.641 0
0.642 0
0.643 0
0.644 0
0.645 0
0.646 0
0.647 0
0.648 0
0.649 0
0.65 0
0.651 0
0.652 0
0.653 0
0.654 0
0.655 0
0.656 0
0.657 0
0.658 0
0.659 0
0.66 0
0.661 0
0.662 0
0.663 0
0.664 0
0.665 0
0.666 0
0.667 0
0.668 0
0.669 0
0.67 0
0.671 0
0.672 0
0.673 0
0.674 0
0.675 0
0.676 0
0.677 0
0.678 0
0.679 0
0.68 0
0.681 0
0.682 0
0.683 0
0.684 0
0.685 0
0.686 0
0.687 0
0.688 0
0.689 0
0.69 0
0.691 0
0.692 0
0.693 0
0.694 0
0.695 0
0.696 0
0.697 0
0.698 0
0.699 0
0.7 0
0.701 0
0.702 0
0.703 0
0.704 0
0.705 0
0.706 0
0.707 0
0.708 0
0.709 0
0.71 0
0.711 0
0.712 0
0.713 0
0.714 0
0.715 0
0.716 0
0.717 0
0.718 0
0.719 0
0.72 0
0.721 0
0.722 0
0.723 0
0.724 0
0.725 0
0.726 0
0.727 0
0.728 0
0.729 0
0.73 0
0.731 0
0.732 0
0.733 0
0.734 0
0.735 0
0.736 0
0.737 0
0.738 0
0.739 0
0.74 0
0.741 0
0.742 0
0.743 0
0.744 0
0.745 0
0.746 0
0.747 0
0.748 0
0.749 0
0.75 0
0.751 0
0.752 0
0.753 0
0.754 0
0.755 0
0.756 0
0.757 0
0.758 0
0.759 0
0.76 0
0.761 0
0.762 0
0.763 0
0.764 0
0.765 0
0.766 0
0.767 0
0.768 0
0.769 0
0.77 0
0.771 0
0.772 0
0.773 0
0.774 0
0.775 0
0.776 0
0.777 0
0.778 0
0.779 0
0.78 0
0.781 0
0.782 0
0.783 0
0.784 0
0.785 0
0.786 0
0.787 0
0.788 0
0.789 0
0.79 0
0.791 0
0.792 0
0.793 0
0.794 0
0.795 0
0.796 0
0.797 0
0.798 0
0.799 0
0.8 0
0.801 0
0.802 0
0.803 0
0.804 0
0.805 0
0.806 0
0.807 0
0.808 0
0.809 0
0.81 0
0.811 0
0.812 0
0.813 0
0.814 0
0.815 0
0.816 0
0.817 0
0.818 0
0.819 0
0.82 0
0.821 0
0.822 0
0.823 0
0.824 0
0.825 0
0.826 0
0.827 0
0.828 0
0.829 0
0.83 0
0.831 0
0.832 0
0.833 0
0.834 0
0.835 0
0.836 0
0.837 0
0.838 0
0.839 0
0.84 0
0.841 0
0.842 0
0.843 0
0.844 0
0.845 0
0.846 0
0.847 0
0.848 0
0.849 0
0.85 0
0.851 0
0.852 0
0.853 0
0.854 0
0.855 0
0.856 0
0.857 0
0.858 0
0.859 0
0.86 0
0.861 0
0.862 0
0.863 0
0.864 0
0.865 0
0.866 0
0.867 0
0.868 0
0.869 0
0.87 0
0.871 0
0.872 0
0.873 0
0.874 0
0.875 0
0.876 0
0.877 0
0.878 0
0.879 0
0.88 0
0.881 0
0.882 0
0.883 0
0.884 0
0.885 0
0.886 0
};
\addlegendentry{monarch}
\addplot [semithick, color1, opacity=0.8]
table {%
0 0
0.001 2228.86834404618
0.002 0.0194954987105334
0.003 2.00479132047621e-11
0.004 4.97379362068654e-22
0.005 1.57902148944301e-33
0.006 1.3486800860086e-45
0.007 4.6067588085775e-58
0.008 7.97243351467814e-71
0.009 8.14247048492395e-84
0.01 5.44658606605447e-97
0.011 2.57009820807579e-110
0.012 9.03732129532875e-124
0.013 2.46869121432058e-137
0.014 5.41093175319026e-151
0.015 9.76293036005148e-165
0.016 1.48031238377385e-178
0.017 1.91829987292819e-192
0.018 2.15442792335058e-206
0.019 2.1216464556491e-220
0.02 1.85023475651152e-234
0.021 1.44094734573719e-248
0.022 1.0094313702969e-262
0.023 6.40073259964631e-277
0.024 3.69379232441528e-291
0.025 1.94929302582764e-305
0.026 9.44653514848463e-320
0.027 0
0.028 0
0.029 0
0.03 0
0.031 0
0.032 0
0.033 0
0.034 0
0.035 0
0.036 0
0.037 0
0.038 0
0.039 0
0.04 0
0.041 0
0.042 0
0.043 0
0.044 0
0.045 0
0.046 0
0.047 0
0.048 0
0.049 0
0.05 0
0.051 0
0.052 0
0.053 0
0.054 0
0.055 0
0.056 0
0.057 0
0.058 0
0.059 0
0.06 0
0.061 0
0.062 0
0.063 0
0.064 0
0.065 0
0.066 0
0.067 0
0.068 0
0.069 0
0.07 0
0.071 0
0.072 0
0.073 0
0.074 0
0.075 0
0.076 0
0.077 0
0.078 0
0.079 0
0.08 0
0.081 0
0.082 0
0.083 0
0.084 0
0.085 0
0.086 0
0.087 0
0.088 0
0.089 0
0.09 0
0.091 0
0.092 0
0.093 0
0.094 0
0.095 0
0.096 0
0.097 0
0.098 0
0.099 0
0.1 0
0.101 0
0.102 0
0.103 0
0.104 0
0.105 0
0.106 0
0.107 0
0.108 0
0.109 0
0.11 0
0.111 0
0.112 0
0.113 0
0.114 0
0.115 0
0.116 0
0.117 0
0.118 0
0.119 0
0.12 0
0.121 0
0.122 0
0.123 0
0.124 0
0.125 0
0.126 0
0.127 0
0.128 0
0.129 0
0.13 0
0.131 0
0.132 0
0.133 0
0.134 0
0.135 0
0.136 0
0.137 0
0.138 0
0.139 0
0.14 0
0.141 0
0.142 0
0.143 0
0.144 0
0.145 0
0.146 0
0.147 0
0.148 0
0.149 0
0.15 0
0.151 0
0.152 0
0.153 0
0.154 0
0.155 0
0.156 0
0.157 0
0.158 0
0.159 0
0.16 0
0.161 0
0.162 0
0.163 0
0.164 0
0.165 0
0.166 0
0.167 0
0.168 0
0.169 0
0.17 0
0.171 0
0.172 0
0.173 0
0.174 0
0.175 0
0.176 0
0.177 0
0.178 0
0.179 0
0.18 0
0.181 0
0.182 0
0.183 0
0.184 0
0.185 0
0.186 0
0.187 0
0.188 0
0.189 0
0.19 0
0.191 0
0.192 0
0.193 0
0.194 0
0.195 0
0.196 0
0.197 0
0.198 0
0.199 0
0.2 0
0.201 0
0.202 0
0.203 0
0.204 0
0.205 0
0.206 0
0.207 0
0.208 0
0.209 0
0.21 0
0.211 0
0.212 0
0.213 0
0.214 0
0.215 0
0.216 0
0.217 0
0.218 0
0.219 0
0.22 0
0.221 0
0.222 0
0.223 0
0.224 0
0.225 0
0.226 0
0.227 0
0.228 0
0.229 0
0.23 0
0.231 0
0.232 0
0.233 0
0.234 0
0.235 0
0.236 0
0.237 0
0.238 0
0.239 0
0.24 0
0.241 0
0.242 0
0.243 0
0.244 0
0.245 0
0.246 0
0.247 0
0.248 0
0.249 0
0.25 0
0.251 0
0.252 0
0.253 0
0.254 0
0.255 0
0.256 0
0.257 0
0.258 0
0.259 0
0.26 0
0.261 0
0.262 0
0.263 0
0.264 0
0.265 0
0.266 0
0.267 0
0.268 0
0.269 0
0.27 0
0.271 0
0.272 0
0.273 0
0.274 0
0.275 0
0.276 0
0.277 0
0.278 0
0.279 0
0.28 0
0.281 0
0.282 0
0.283 0
0.284 0
0.285 0
0.286 0
0.287 0
0.288 0
0.289 0
0.29 0
0.291 0
0.292 0
0.293 0
0.294 0
0.295 0
0.296 0
0.297 0
0.298 0
0.299 0
0.3 0
0.301 0
0.302 0
0.303 0
0.304 0
0.305 0
0.306 0
0.307 0
0.308 0
0.309 0
0.31 0
0.311 0
0.312 0
0.313 0
0.314 0
0.315 0
0.316 0
0.317 0
0.318 0
0.319 0
0.32 0
0.321 0
0.322 0
0.323 0
0.324 0
0.325 0
0.326 0
0.327 0
0.328 0
0.329 0
0.33 0
0.331 0
0.332 0
0.333 0
0.334 0
0.335 0
0.336 0
0.337 0
0.338 0
0.339 0
0.34 0
0.341 0
0.342 0
0.343 0
0.344 0
0.345 0
0.346 0
0.347 0
0.348 0
0.349 0
0.35 0
0.351 0
0.352 0
0.353 0
0.354 0
0.355 0
0.356 0
0.357 0
0.358 0
0.359 0
0.36 0
0.361 0
0.362 0
0.363 0
0.364 0
0.365 0
0.366 0
0.367 0
0.368 0
0.369 0
0.37 0
0.371 0
0.372 0
0.373 0
0.374 0
0.375 0
0.376 0
0.377 0
0.378 0
0.379 0
0.38 0
0.381 0
0.382 0
0.383 0
0.384 0
0.385 0
0.386 0
0.387 0
0.388 0
0.389 0
0.39 0
0.391 0
0.392 0
0.393 0
0.394 0
0.395 0
0.396 0
0.397 0
0.398 0
0.399 0
0.4 0
0.401 0
0.402 0
0.403 0
0.404 0
0.405 0
0.406 0
0.407 0
0.408 0
0.409 0
0.41 0
0.411 0
0.412 0
0.413 0
0.414 0
0.415 0
0.416 0
0.417 0
0.418 0
0.419 0
0.42 0
0.421 0
0.422 0
0.423 0
0.424 0
0.425 0
0.426 0
0.427 0
0.428 0
0.429 0
0.43 0
0.431 0
0.432 0
0.433 0
0.434 0
0.435 0
0.436 0
0.437 0
0.438 0
0.439 0
0.44 0
0.441 0
0.442 0
0.443 0
0.444 0
0.445 0
0.446 0
0.447 0
0.448 0
0.449 0
0.45 0
0.451 0
0.452 0
0.453 0
0.454 0
0.455 0
0.456 0
0.457 0
0.458 0
0.459 0
0.46 0
0.461 0
0.462 0
0.463 0
0.464 0
0.465 0
0.466 0
0.467 0
0.468 0
0.469 0
0.47 0
0.471 0
0.472 0
0.473 0
0.474 0
0.475 0
0.476 0
0.477 0
0.478 0
0.479 0
0.48 0
0.481 0
0.482 0
0.483 0
0.484 0
0.485 0
0.486 0
0.487 0
0.488 0
0.489 0
0.49 0
0.491 0
0.492 0
0.493 0
0.494 0
0.495 0
0.496 0
0.497 0
0.498 0
0.499 0
0.5 0
0.501 0
0.502 0
0.503 0
0.504 0
0.505 0
0.506 0
0.507 0
0.508 0
0.509 0
0.51 0
0.511 0
0.512 0
0.513 0
0.514 0
0.515 0
0.516 0
0.517 0
0.518 0
0.519 0
0.52 0
0.521 0
0.522 0
0.523 0
0.524 0
0.525 0
0.526 0
0.527 0
0.528 0
0.529 0
0.53 0
0.531 0
0.532 0
0.533 0
0.534 0
0.535 0
0.536 0
0.537 0
0.538 0
0.539 0
0.54 0
0.541 0
0.542 0
0.543 0
0.544 0
0.545 0
0.546 0
0.547 0
0.548 0
0.549 0
0.55 0
0.551 0
0.552 0
0.553 0
0.554 0
0.555 0
0.556 0
0.557 0
0.558 0
0.559 0
0.56 0
0.561 0
0.562 0
0.563 0
0.564 0
0.565 0
0.566 0
0.567 0
0.568 0
0.569 0
0.57 0
0.571 0
0.572 0
0.573 0
0.574 0
0.575 0
0.576 0
0.577 0
0.578 0
0.579 0
0.58 0
0.581 0
0.582 0
0.583 0
0.584 0
0.585 0
0.586 0
0.587 0
0.588 0
0.589 0
0.59 0
0.591 0
0.592 0
0.593 0
0.594 0
0.595 0
0.596 0
0.597 0
0.598 0
0.599 0
0.6 0
0.601 0
0.602 0
0.603 0
0.604 0
0.605 0
0.606 0
0.607 0
0.608 0
0.609 0
0.61 0
0.611 0
0.612 0
0.613 0
0.614 0
0.615 0
0.616 0
0.617 0
0.618 0
0.619 0
0.62 0
0.621 0
0.622 0
0.623 0
0.624 0
0.625 0
0.626 0
0.627 0
0.628 0
0.629 0
0.63 0
0.631 0
0.632 0
0.633 0
0.634 0
0.635 0
0.636 0
0.637 0
0.638 0
0.639 0
0.64 0
0.641 0
0.642 0
0.643 0
0.644 0
0.645 0
0.646 0
0.647 0
0.648 0
0.649 0
0.65 0
0.651 0
0.652 0
0.653 0
0.654 0
0.655 0
0.656 0
0.657 0
0.658 0
0.659 0
0.66 0
0.661 0
0.662 0
0.663 0
0.664 0
0.665 0
0.666 0
0.667 0
0.668 0
0.669 0
0.67 0
0.671 0
0.672 0
0.673 0
0.674 0
0.675 0
0.676 0
0.677 0
0.678 0
0.679 0
0.68 0
0.681 0
0.682 0
0.683 0
0.684 0
0.685 0
0.686 0
0.687 0
0.688 0
0.689 0
0.69 0
0.691 0
0.692 0
0.693 0
0.694 0
0.695 0
0.696 0
0.697 0
0.698 0
0.699 0
0.7 0
0.701 0
0.702 0
0.703 0
0.704 0
0.705 0
0.706 0
0.707 0
0.708 0
0.709 0
0.71 0
0.711 0
0.712 0
0.713 0
0.714 0
0.715 0
0.716 0
0.717 0
0.718 0
0.719 0
0.72 0
0.721 0
0.722 0
0.723 0
0.724 0
0.725 0
0.726 0
0.727 0
0.728 0
0.729 0
0.73 0
0.731 0
0.732 0
0.733 0
0.734 0
0.735 0
0.736 0
0.737 0
0.738 0
0.739 0
0.74 0
0.741 0
0.742 0
0.743 0
0.744 0
0.745 0
0.746 0
0.747 0
0.748 0
0.749 0
0.75 0
0.751 0
0.752 0
0.753 0
0.754 0
0.755 0
0.756 0
0.757 0
0.758 0
0.759 0
0.76 0
0.761 0
0.762 0
0.763 0
0.764 0
0.765 0
0.766 0
0.767 0
0.768 0
0.769 0
0.77 0
0.771 0
0.772 0
0.773 0
0.774 0
0.775 0
0.776 0
0.777 0
0.778 0
0.779 0
0.78 0
0.781 0
0.782 0
0.783 0
0.784 0
0.785 0
0.786 0
0.787 0
0.788 0
0.789 0
0.79 0
0.791 0
0.792 0
0.793 0
0.794 0
0.795 0
0.796 0
0.797 0
0.798 0
0.799 0
0.8 0
0.801 0
0.802 0
0.803 0
0.804 0
0.805 0
0.806 0
0.807 0
0.808 0
0.809 0
0.81 0
0.811 0
0.812 0
0.813 0
0.814 0
0.815 0
0.816 0
0.817 0
0.818 0
0.819 0
0.82 0
0.821 0
0.822 0
0.823 0
0.824 0
0.825 0
0.826 0
0.827 0
0.828 0
0.829 0
0.83 0
0.831 0
0.832 0
0.833 0
0.834 0
0.835 0
0.836 0
0.837 0
0.838 0
0.839 0
0.84 0
0.841 0
0.842 0
0.843 0
0.844 0
0.845 0
0.846 0
0.847 0
0.848 0
0.849 0
0.85 0
0.851 0
0.852 0
0.853 0
0.854 0
0.855 0
0.856 0
0.857 0
0.858 0
0.859 0
0.86 0
0.861 0
0.862 0
0.863 0
0.864 0
0.865 0
0.866 0
0.867 0
0.868 0
0.869 0
0.87 0
0.871 0
0.872 0
0.873 0
0.874 0
0.875 0
0.876 0
0.877 0
0.878 0
0.879 0
0.88 0
0.881 0
0.882 0
0.883 0
0.884 0
0.885 0
0.886 0
};
\addlegendentry{space bar}
\addplot [semithick, color2, opacity=0.8]
table {%
0 0
0.001 2281.97034558976
0.002 0.104408969856964
0.003 2.8290780686059e-10
0.004 1.39671764561713e-20
0.005 7.56563147323885e-32
0.006 1.00033335623812e-43
0.007 4.9459381555954e-56
0.008 1.17955340981976e-68
0.009 1.59902350937739e-81
0.01 1.37827639915434e-94
0.011 8.18223547507063e-108
0.012 3.54880776405098e-121
0.013 1.17602663837293e-134
0.014 3.08309950875692e-148
0.015 6.5730467404995e-162
0.016 1.16520112860796e-175
0.017 1.7489381169991e-189
0.018 2.25638503050645e-203
0.019 2.53383682029332e-217
0.02 2.50314709766078e-231
0.021 2.19518644175894e-245
0.022 1.722324175685e-259
0.023 1.21714585515373e-273
0.024 7.79296272852925e-288
0.025 4.54387893249789e-302
0.026 2.42370933121564e-316
0.027 0
0.028 0
0.029 0
0.03 0
0.031 0
0.032 0
0.033 0
0.034 0
0.035 0
0.036 0
0.037 0
0.038 0
0.039 0
0.04 0
0.041 0
0.042 0
0.043 0
0.044 0
0.045 0
0.046 0
0.047 0
0.048 0
0.049 0
0.05 0
0.051 0
0.052 0
0.053 0
0.054 0
0.055 0
0.056 0
0.057 0
0.058 0
0.059 0
0.06 0
0.061 0
0.062 0
0.063 0
0.064 0
0.065 0
0.066 0
0.067 0
0.068 0
0.069 0
0.07 0
0.071 0
0.072 0
0.073 0
0.074 0
0.075 0
0.076 0
0.077 0
0.078 0
0.079 0
0.08 0
0.081 0
0.082 0
0.083 0
0.084 0
0.085 0
0.086 0
0.087 0
0.088 0
0.089 0
0.09 0
0.091 0
0.092 0
0.093 0
0.094 0
0.095 0
0.096 0
0.097 0
0.098 0
0.099 0
0.1 0
0.101 0
0.102 0
0.103 0
0.104 0
0.105 0
0.106 0
0.107 0
0.108 0
0.109 0
0.11 0
0.111 0
0.112 0
0.113 0
0.114 0
0.115 0
0.116 0
0.117 0
0.118 0
0.119 0
0.12 0
0.121 0
0.122 0
0.123 0
0.124 0
0.125 0
0.126 0
0.127 0
0.128 0
0.129 0
0.13 0
0.131 0
0.132 0
0.133 0
0.134 0
0.135 0
0.136 0
0.137 0
0.138 0
0.139 0
0.14 0
0.141 0
0.142 0
0.143 0
0.144 0
0.145 0
0.146 0
0.147 0
0.148 0
0.149 0
0.15 0
0.151 0
0.152 0
0.153 0
0.154 0
0.155 0
0.156 0
0.157 0
0.158 0
0.159 0
0.16 0
0.161 0
0.162 0
0.163 0
0.164 0
0.165 0
0.166 0
0.167 0
0.168 0
0.169 0
0.17 0
0.171 0
0.172 0
0.173 0
0.174 0
0.175 0
0.176 0
0.177 0
0.178 0
0.179 0
0.18 0
0.181 0
0.182 0
0.183 0
0.184 0
0.185 0
0.186 0
0.187 0
0.188 0
0.189 0
0.19 0
0.191 0
0.192 0
0.193 0
0.194 0
0.195 0
0.196 0
0.197 0
0.198 0
0.199 0
0.2 0
0.201 0
0.202 0
0.203 0
0.204 0
0.205 0
0.206 0
0.207 0
0.208 0
0.209 0
0.21 0
0.211 0
0.212 0
0.213 0
0.214 0
0.215 0
0.216 0
0.217 0
0.218 0
0.219 0
0.22 0
0.221 0
0.222 0
0.223 0
0.224 0
0.225 0
0.226 0
0.227 0
0.228 0
0.229 0
0.23 0
0.231 0
0.232 0
0.233 0
0.234 0
0.235 0
0.236 0
0.237 0
0.238 0
0.239 0
0.24 0
0.241 0
0.242 0
0.243 0
0.244 0
0.245 0
0.246 0
0.247 0
0.248 0
0.249 0
0.25 0
0.251 0
0.252 0
0.253 0
0.254 0
0.255 0
0.256 0
0.257 0
0.258 0
0.259 0
0.26 0
0.261 0
0.262 0
0.263 0
0.264 0
0.265 0
0.266 0
0.267 0
0.268 0
0.269 0
0.27 0
0.271 0
0.272 0
0.273 0
0.274 0
0.275 0
0.276 0
0.277 0
0.278 0
0.279 0
0.28 0
0.281 0
0.282 0
0.283 0
0.284 0
0.285 0
0.286 0
0.287 0
0.288 0
0.289 0
0.29 0
0.291 0
0.292 0
0.293 0
0.294 0
0.295 0
0.296 0
0.297 0
0.298 0
0.299 0
0.3 0
0.301 0
0.302 0
0.303 0
0.304 0
0.305 0
0.306 0
0.307 0
0.308 0
0.309 0
0.31 0
0.311 0
0.312 0
0.313 0
0.314 0
0.315 0
0.316 0
0.317 0
0.318 0
0.319 0
0.32 0
0.321 0
0.322 0
0.323 0
0.324 0
0.325 0
0.326 0
0.327 0
0.328 0
0.329 0
0.33 0
0.331 0
0.332 0
0.333 0
0.334 0
0.335 0
0.336 0
0.337 0
0.338 0
0.339 0
0.34 0
0.341 0
0.342 0
0.343 0
0.344 0
0.345 0
0.346 0
0.347 0
0.348 0
0.349 0
0.35 0
0.351 0
0.352 0
0.353 0
0.354 0
0.355 0
0.356 0
0.357 0
0.358 0
0.359 0
0.36 0
0.361 0
0.362 0
0.363 0
0.364 0
0.365 0
0.366 0
0.367 0
0.368 0
0.369 0
0.37 0
0.371 0
0.372 0
0.373 0
0.374 0
0.375 0
0.376 0
0.377 0
0.378 0
0.379 0
0.38 0
0.381 0
0.382 0
0.383 0
0.384 0
0.385 0
0.386 0
0.387 0
0.388 0
0.389 0
0.39 0
0.391 0
0.392 0
0.393 0
0.394 0
0.395 0
0.396 0
0.397 0
0.398 0
0.399 0
0.4 0
0.401 0
0.402 0
0.403 0
0.404 0
0.405 0
0.406 0
0.407 0
0.408 0
0.409 0
0.41 0
0.411 0
0.412 0
0.413 0
0.414 0
0.415 0
0.416 0
0.417 0
0.418 0
0.419 0
0.42 0
0.421 0
0.422 0
0.423 0
0.424 0
0.425 0
0.426 0
0.427 0
0.428 0
0.429 0
0.43 0
0.431 0
0.432 0
0.433 0
0.434 0
0.435 0
0.436 0
0.437 0
0.438 0
0.439 0
0.44 0
0.441 0
0.442 0
0.443 0
0.444 0
0.445 0
0.446 0
0.447 0
0.448 0
0.449 0
0.45 0
0.451 0
0.452 0
0.453 0
0.454 0
0.455 0
0.456 0
0.457 0
0.458 0
0.459 0
0.46 0
0.461 0
0.462 0
0.463 0
0.464 0
0.465 0
0.466 0
0.467 0
0.468 0
0.469 0
0.47 0
0.471 0
0.472 0
0.473 0
0.474 0
0.475 0
0.476 0
0.477 0
0.478 0
0.479 0
0.48 0
0.481 0
0.482 0
0.483 0
0.484 0
0.485 0
0.486 0
0.487 0
0.488 0
0.489 0
0.49 0
0.491 0
0.492 0
0.493 0
0.494 0
0.495 0
0.496 0
0.497 0
0.498 0
0.499 0
0.5 0
0.501 0
0.502 0
0.503 0
0.504 0
0.505 0
0.506 0
0.507 0
0.508 0
0.509 0
0.51 0
0.511 0
0.512 0
0.513 0
0.514 0
0.515 0
0.516 0
0.517 0
0.518 0
0.519 0
0.52 0
0.521 0
0.522 0
0.523 0
0.524 0
0.525 0
0.526 0
0.527 0
0.528 0
0.529 0
0.53 0
0.531 0
0.532 0
0.533 0
0.534 0
0.535 0
0.536 0
0.537 0
0.538 0
0.539 0
0.54 0
0.541 0
0.542 0
0.543 0
0.544 0
0.545 0
0.546 0
0.547 0
0.548 0
0.549 0
0.55 0
0.551 0
0.552 0
0.553 0
0.554 0
0.555 0
0.556 0
0.557 0
0.558 0
0.559 0
0.56 0
0.561 0
0.562 0
0.563 0
0.564 0
0.565 0
0.566 0
0.567 0
0.568 0
0.569 0
0.57 0
0.571 0
0.572 0
0.573 0
0.574 0
0.575 0
0.576 0
0.577 0
0.578 0
0.579 0
0.58 0
0.581 0
0.582 0
0.583 0
0.584 0
0.585 0
0.586 0
0.587 0
0.588 0
0.589 0
0.59 0
0.591 0
0.592 0
0.593 0
0.594 0
0.595 0
0.596 0
0.597 0
0.598 0
0.599 0
0.6 0
0.601 0
0.602 0
0.603 0
0.604 0
0.605 0
0.606 0
0.607 0
0.608 0
0.609 0
0.61 0
0.611 0
0.612 0
0.613 0
0.614 0
0.615 0
0.616 0
0.617 0
0.618 0
0.619 0
0.62 0
0.621 0
0.622 0
0.623 0
0.624 0
0.625 0
0.626 0
0.627 0
0.628 0
0.629 0
0.63 0
0.631 0
0.632 0
0.633 0
0.634 0
0.635 0
0.636 0
0.637 0
0.638 0
0.639 0
0.64 0
0.641 0
0.642 0
0.643 0
0.644 0
0.645 0
0.646 0
0.647 0
0.648 0
0.649 0
0.65 0
0.651 0
0.652 0
0.653 0
0.654 0
0.655 0
0.656 0
0.657 0
0.658 0
0.659 0
0.66 0
0.661 0
0.662 0
0.663 0
0.664 0
0.665 0
0.666 0
0.667 0
0.668 0
0.669 0
0.67 0
0.671 0
0.672 0
0.673 0
0.674 0
0.675 0
0.676 0
0.677 0
0.678 0
0.679 0
0.68 0
0.681 0
0.682 0
0.683 0
0.684 0
0.685 0
0.686 0
0.687 0
0.688 0
0.689 0
0.69 0
0.691 0
0.692 0
0.693 0
0.694 0
0.695 0
0.696 0
0.697 0
0.698 0
0.699 0
0.7 0
0.701 0
0.702 0
0.703 0
0.704 0
0.705 0
0.706 0
0.707 0
0.708 0
0.709 0
0.71 0
0.711 0
0.712 0
0.713 0
0.714 0
0.715 0
0.716 0
0.717 0
0.718 0
0.719 0
0.72 0
0.721 0
0.722 0
0.723 0
0.724 0
0.725 0
0.726 0
0.727 0
0.728 0
0.729 0
0.73 0
0.731 0
0.732 0
0.733 0
0.734 0
0.735 0
0.736 0
0.737 0
0.738 0
0.739 0
0.74 0
0.741 0
0.742 0
0.743 0
0.744 0
0.745 0
0.746 0
0.747 0
0.748 0
0.749 0
0.75 0
0.751 0
0.752 0
0.753 0
0.754 0
0.755 0
0.756 0
0.757 0
0.758 0
0.759 0
0.76 0
0.761 0
0.762 0
0.763 0
0.764 0
0.765 0
0.766 0
0.767 0
0.768 0
0.769 0
0.77 0
0.771 0
0.772 0
0.773 0
0.774 0
0.775 0
0.776 0
0.777 0
0.778 0
0.779 0
0.78 0
0.781 0
0.782 0
0.783 0
0.784 0
0.785 0
0.786 0
0.787 0
0.788 0
0.789 0
0.79 0
0.791 0
0.792 0
0.793 0
0.794 0
0.795 0
0.796 0
0.797 0
0.798 0
0.799 0
0.8 0
0.801 0
0.802 0
0.803 0
0.804 0
0.805 0
0.806 0
0.807 0
0.808 0
0.809 0
0.81 0
0.811 0
0.812 0
0.813 0
0.814 0
0.815 0
0.816 0
0.817 0
0.818 0
0.819 0
0.82 0
0.821 0
0.822 0
0.823 0
0.824 0
0.825 0
0.826 0
0.827 0
0.828 0
0.829 0
0.83 0
0.831 0
0.832 0
0.833 0
0.834 0
0.835 0
0.836 0
0.837 0
0.838 0
0.839 0
0.84 0
0.841 0
0.842 0
0.843 0
0.844 0
0.845 0
0.846 0
0.847 0
0.848 0
0.849 0
0.85 0
0.851 0
0.852 0
0.853 0
0.854 0
0.855 0
0.856 0
0.857 0
0.858 0
0.859 0
0.86 0
0.861 0
0.862 0
0.863 0
0.864 0
0.865 0
0.866 0
0.867 0
0.868 0
0.869 0
0.87 0
0.871 0
0.872 0
0.873 0
0.874 0
0.875 0
0.876 0
0.877 0
0.878 0
0.879 0
0.88 0
0.881 0
0.882 0
0.883 0
0.884 0
0.885 0
0.886 0
};
\addlegendentry{hand-held computer}
\addplot [semithick, color3, opacity=0.8]
table {%
0 0
0.001 0
0.002 3.79114210771532e-305
0.003 2.27295435483742e-220
0.004 2.22041196223942e-164
0.005 3.59392471299512e-124
0.006 5.99467998609246e-94
0.007 1.26935745406657e-70
0.008 2.38670914413019e-52
0.009 6.17051050012563e-38
0.01 1.42429686065929e-26
0.011 1.11459755044725e-17
0.012 7.92014039067391e-11
0.013 1.0798380997675e-05
0.014 0.0505300022795885
0.015 12.8678756861954
0.016 258.594119847395
0.017 555.755305739387
0.018 164.300925711563
0.019 8.25033713777615
0.02 0.0841113418290838
0.021 0.000202727336064833
0.022 1.31683270664984e-07
0.023 2.58226511532717e-11
0.024 1.68777213874407e-15
0.025 4.01041442798873e-20
0.026 3.74027139203671e-25
0.027 1.46540882813021e-30
0.028 2.56233679582637e-36
0.029 2.1107868898396e-42
0.03 8.59997936559661e-49
0.031 1.81062547496178e-55
0.032 2.04953332677924e-62
0.033 1.29300636510445e-69
0.034 4.69772206477151e-77
0.035 1.01273716684543e-84
0.036 1.3314063621793e-92
0.037 1.09453864743292e-100
0.038 5.75818064933999e-109
0.039 1.98023258758548e-117
0.04 4.54006624907773e-126
0.041 7.06686031121748e-135
0.042 7.59518845267878e-144
0.043 5.72542707415952e-153
0.044 3.07165205892928e-162
0.045 1.18888227324024e-171
0.046 3.36220693450641e-181
0.047 7.03050787100005e-191
0.048 1.09914644642867e-200
0.049 1.2982550370239e-210
0.05 1.16990995166428e-220
0.051 8.11765269717308e-231
0.052 4.37478606583363e-241
0.053 1.84619325520785e-251
0.054 6.14803493893646e-262
0.055 1.62739535226688e-272
0.056 3.44773167804332e-283
0.057 5.88414126958083e-294
0.058 8.13988478711462e-305
0.059 9.18070737670431e-316
0.06 0
0.061 0
0.062 0
0.063 0
0.064 0
0.065 0
0.066 0
0.067 0
0.068 0
0.069 0
0.07 0
0.071 0
0.072 0
0.073 0
0.074 0
0.075 0
0.076 0
0.077 0
0.078 0
0.079 0
0.08 0
0.081 0
0.082 0
0.083 0
0.084 0
0.085 0
0.086 0
0.087 0
0.088 0
0.089 0
0.09 0
0.091 0
0.092 0
0.093 0
0.094 0
0.095 0
0.096 0
0.097 0
0.098 0
0.099 0
0.1 0
0.101 0
0.102 0
0.103 0
0.104 0
0.105 0
0.106 0
0.107 0
0.108 0
0.109 0
0.11 0
0.111 0
0.112 0
0.113 0
0.114 0
0.115 0
0.116 0
0.117 0
0.118 0
0.119 0
0.12 0
0.121 0
0.122 0
0.123 0
0.124 0
0.125 0
0.126 0
0.127 0
0.128 0
0.129 0
0.13 0
0.131 0
0.132 0
0.133 0
0.134 0
0.135 0
0.136 0
0.137 0
0.138 0
0.139 0
0.14 0
0.141 0
0.142 0
0.143 0
0.144 0
0.145 0
0.146 0
0.147 0
0.148 0
0.149 0
0.15 0
0.151 0
0.152 0
0.153 0
0.154 0
0.155 0
0.156 0
0.157 0
0.158 0
0.159 0
0.16 0
0.161 0
0.162 0
0.163 0
0.164 0
0.165 0
0.166 0
0.167 0
0.168 0
0.169 0
0.17 0
0.171 0
0.172 0
0.173 0
0.174 0
0.175 0
0.176 0
0.177 0
0.178 0
0.179 0
0.18 0
0.181 0
0.182 0
0.183 0
0.184 0
0.185 0
0.186 0
0.187 0
0.188 0
0.189 0
0.19 0
0.191 0
0.192 0
0.193 0
0.194 0
0.195 0
0.196 0
0.197 0
0.198 0
0.199 0
0.2 0
0.201 0
0.202 0
0.203 0
0.204 0
0.205 0
0.206 0
0.207 0
0.208 0
0.209 0
0.21 0
0.211 0
0.212 0
0.213 0
0.214 0
0.215 0
0.216 0
0.217 0
0.218 0
0.219 0
0.22 0
0.221 0
0.222 0
0.223 0
0.224 0
0.225 0
0.226 0
0.227 0
0.228 0
0.229 0
0.23 0
0.231 0
0.232 0
0.233 0
0.234 0
0.235 0
0.236 0
0.237 0
0.238 0
0.239 0
0.24 0
0.241 0
0.242 0
0.243 0
0.244 0
0.245 0
0.246 0
0.247 0
0.248 0
0.249 0
0.25 0
0.251 0
0.252 0
0.253 0
0.254 0
0.255 0
0.256 0
0.257 0
0.258 0
0.259 0
0.26 0
0.261 0
0.262 0
0.263 0
0.264 0
0.265 0
0.266 0
0.267 0
0.268 0
0.269 0
0.27 0
0.271 0
0.272 0
0.273 0
0.274 0
0.275 0
0.276 0
0.277 0
0.278 0
0.279 0
0.28 0
0.281 0
0.282 0
0.283 0
0.284 0
0.285 0
0.286 0
0.287 0
0.288 0
0.289 0
0.29 0
0.291 0
0.292 0
0.293 0
0.294 0
0.295 0
0.296 0
0.297 0
0.298 0
0.299 0
0.3 0
0.301 0
0.302 0
0.303 0
0.304 0
0.305 0
0.306 0
0.307 0
0.308 0
0.309 0
0.31 0
0.311 0
0.312 0
0.313 0
0.314 0
0.315 0
0.316 0
0.317 0
0.318 0
0.319 0
0.32 0
0.321 0
0.322 0
0.323 0
0.324 0
0.325 0
0.326 0
0.327 0
0.328 0
0.329 0
0.33 0
0.331 0
0.332 0
0.333 0
0.334 0
0.335 0
0.336 0
0.337 0
0.338 0
0.339 0
0.34 0
0.341 0
0.342 0
0.343 0
0.344 0
0.345 0
0.346 0
0.347 0
0.348 0
0.349 0
0.35 0
0.351 0
0.352 0
0.353 0
0.354 0
0.355 0
0.356 0
0.357 0
0.358 0
0.359 0
0.36 0
0.361 0
0.362 0
0.363 0
0.364 0
0.365 0
0.366 0
0.367 0
0.368 0
0.369 0
0.37 0
0.371 0
0.372 0
0.373 0
0.374 0
0.375 0
0.376 0
0.377 0
0.378 0
0.379 0
0.38 0
0.381 0
0.382 0
0.383 0
0.384 0
0.385 0
0.386 0
0.387 0
0.388 0
0.389 0
0.39 0
0.391 0
0.392 0
0.393 0
0.394 0
0.395 0
0.396 0
0.397 0
0.398 0
0.399 0
0.4 0
0.401 0
0.402 0
0.403 0
0.404 0
0.405 0
0.406 0
0.407 0
0.408 0
0.409 0
0.41 0
0.411 0
0.412 0
0.413 0
0.414 0
0.415 0
0.416 0
0.417 0
0.418 0
0.419 0
0.42 0
0.421 0
0.422 0
0.423 0
0.424 0
0.425 0
0.426 0
0.427 0
0.428 0
0.429 0
0.43 0
0.431 0
0.432 0
0.433 0
0.434 0
0.435 0
0.436 0
0.437 0
0.438 0
0.439 0
0.44 0
0.441 0
0.442 0
0.443 0
0.444 0
0.445 0
0.446 0
0.447 0
0.448 0
0.449 0
0.45 0
0.451 0
0.452 0
0.453 0
0.454 0
0.455 0
0.456 0
0.457 0
0.458 0
0.459 0
0.46 0
0.461 0
0.462 0
0.463 0
0.464 0
0.465 0
0.466 0
0.467 0
0.468 0
0.469 0
0.47 0
0.471 0
0.472 0
0.473 0
0.474 0
0.475 0
0.476 0
0.477 0
0.478 0
0.479 0
0.48 0
0.481 0
0.482 0
0.483 0
0.484 0
0.485 0
0.486 0
0.487 0
0.488 0
0.489 0
0.49 0
0.491 0
0.492 0
0.493 0
0.494 0
0.495 0
0.496 0
0.497 0
0.498 0
0.499 0
0.5 0
0.501 0
0.502 0
0.503 0
0.504 0
0.505 0
0.506 0
0.507 0
0.508 0
0.509 0
0.51 0
0.511 0
0.512 0
0.513 0
0.514 0
0.515 0
0.516 0
0.517 0
0.518 0
0.519 0
0.52 0
0.521 0
0.522 0
0.523 0
0.524 0
0.525 0
0.526 0
0.527 0
0.528 0
0.529 0
0.53 0
0.531 0
0.532 0
0.533 0
0.534 0
0.535 0
0.536 0
0.537 0
0.538 0
0.539 0
0.54 0
0.541 0
0.542 0
0.543 0
0.544 0
0.545 0
0.546 0
0.547 0
0.548 0
0.549 0
0.55 0
0.551 0
0.552 0
0.553 0
0.554 0
0.555 0
0.556 0
0.557 0
0.558 0
0.559 0
0.56 0
0.561 0
0.562 0
0.563 0
0.564 0
0.565 0
0.566 0
0.567 0
0.568 0
0.569 0
0.57 0
0.571 0
0.572 0
0.573 0
0.574 0
0.575 0
0.576 0
0.577 0
0.578 0
0.579 0
0.58 0
0.581 0
0.582 0
0.583 0
0.584 0
0.585 0
0.586 0
0.587 0
0.588 0
0.589 0
0.59 0
0.591 0
0.592 0
0.593 0
0.594 0
0.595 0
0.596 0
0.597 0
0.598 0
0.599 0
0.6 0
0.601 0
0.602 0
0.603 0
0.604 0
0.605 0
0.606 0
0.607 0
0.608 0
0.609 0
0.61 0
0.611 0
0.612 0
0.613 0
0.614 0
0.615 0
0.616 0
0.617 0
0.618 0
0.619 0
0.62 0
0.621 0
0.622 0
0.623 0
0.624 0
0.625 0
0.626 0
0.627 0
0.628 0
0.629 0
0.63 0
0.631 0
0.632 0
0.633 0
0.634 0
0.635 0
0.636 0
0.637 0
0.638 0
0.639 0
0.64 0
0.641 0
0.642 0
0.643 0
0.644 0
0.645 0
0.646 0
0.647 0
0.648 0
0.649 0
0.65 0
0.651 0
0.652 0
0.653 0
0.654 0
0.655 0
0.656 0
0.657 0
0.658 0
0.659 0
0.66 0
0.661 0
0.662 0
0.663 0
0.664 0
0.665 0
0.666 0
0.667 0
0.668 0
0.669 0
0.67 0
0.671 0
0.672 0
0.673 0
0.674 0
0.675 0
0.676 0
0.677 0
0.678 0
0.679 0
0.68 0
0.681 0
0.682 0
0.683 0
0.684 0
0.685 0
0.686 0
0.687 0
0.688 0
0.689 0
0.69 0
0.691 0
0.692 0
0.693 0
0.694 0
0.695 0
0.696 0
0.697 0
0.698 0
0.699 0
0.7 0
0.701 0
0.702 0
0.703 0
0.704 0
0.705 0
0.706 0
0.707 0
0.708 0
0.709 0
0.71 0
0.711 0
0.712 0
0.713 0
0.714 0
0.715 0
0.716 0
0.717 0
0.718 0
0.719 0
0.72 0
0.721 0
0.722 0
0.723 0
0.724 0
0.725 0
0.726 0
0.727 0
0.728 0
0.729 0
0.73 0
0.731 0
0.732 0
0.733 0
0.734 0
0.735 0
0.736 0
0.737 0
0.738 0
0.739 0
0.74 0
0.741 0
0.742 0
0.743 0
0.744 0
0.745 0
0.746 0
0.747 0
0.748 0
0.749 0
0.75 0
0.751 0
0.752 0
0.753 0
0.754 0
0.755 0
0.756 0
0.757 0
0.758 0
0.759 0
0.76 0
0.761 0
0.762 0
0.763 0
0.764 0
0.765 0
0.766 0
0.767 0
0.768 0
0.769 0
0.77 0
0.771 0
0.772 0
0.773 0
0.774 0
0.775 0
0.776 0
0.777 0
0.778 0
0.779 0
0.78 0
0.781 0
0.782 0
0.783 0
0.784 0
0.785 0
0.786 0
0.787 0
0.788 0
0.789 0
0.79 0
0.791 0
0.792 0
0.793 0
0.794 0
0.795 0
0.796 0
0.797 0
0.798 0
0.799 0
0.8 0
0.801 0
0.802 0
0.803 0
0.804 0
0.805 0
0.806 0
0.807 0
0.808 0
0.809 0
0.81 0
0.811 0
0.812 0
0.813 0
0.814 0
0.815 0
0.816 0
0.817 0
0.818 0
0.819 0
0.82 0
0.821 0
0.822 0
0.823 0
0.824 0
0.825 0
0.826 0
0.827 0
0.828 0
0.829 0
0.83 0
0.831 0
0.832 0
0.833 0
0.834 0
0.835 0
0.836 0
0.837 0
0.838 0
0.839 0
0.84 0
0.841 0
0.842 0
0.843 0
0.844 0
0.845 0
0.846 0
0.847 0
0.848 0
0.849 0
0.85 0
0.851 0
0.852 0
0.853 0
0.854 0
0.855 0
0.856 0
0.857 0
0.858 0
0.859 0
0.86 0
0.861 0
0.862 0
0.863 0
0.864 0
0.865 0
0.866 0
0.867 0
0.868 0
0.869 0
0.87 0
0.871 0
0.872 0
0.873 0
0.874 0
0.875 0
0.876 0
0.877 0
0.878 0
0.879 0
0.88 0
0.881 0
0.882 0
0.883 0
0.884 0
0.885 0
0.886 0
};
\addlegendentry{notebook}
\addplot [semithick, color4, opacity=0.8]
table {%
0 0
0.001 0
0.002 0
0.003 0
0.004 0
0.005 0
0.006 0
0.007 0
0.008 0
0.009 0
0.01 0
0.011 0
0.012 0
0.013 0
0.014 0
0.015 0
0.016 0
0.017 0
0.018 0
0.019 0
0.02 0
0.021 0
0.022 0
0.023 0
0.024 0
0.025 0
0.026 0
0.027 0
0.028 0
0.029 0
0.03 0
0.031 0
0.032 0
0.033 0
0.034 0
0.035 0
0.036 0
0.037 0
0.038 0
0.039 0
0.04 0
0.041 0
0.042 0
0.043 0
0.044 0
0.045 0
0.046 0
0.047 0
0.048 0
0.049 0
0.05 0
0.051 0
0.052 0
0.053 0
0.054 0
0.055 0
0.056 0
0.057 0
0.058 0
0.059 0
0.06 0
0.061 0
0.062 0
0.063 0
0.064 0
0.065 0
0.066 0
0.067 0
0.068 0
0.069 0
0.07 0
0.071 0
0.072 0
0.073 0
0.074 0
0.075 0
0.076 0
0.077 0
0.078 0
0.079 0
0.08 0
0.081 0
0.082 0
0.083 0
0.084 0
0.085 0
0.086 0
0.087 0
0.088 0
0.089 0
0.09 0
0.091 0
0.092 0
0.093 0
0.094 0
0.095 0
0.096 0
0.097 0
0.098 0
0.099 0
0.1 0
0.101 0
0.102 0
0.103 0
0.104 0
0.105 0
0.106 0
0.107 0
0.108 0
0.109 0
0.11 0
0.111 0
0.112 0
0.113 0
0.114 0
0.115 0
0.116 0
0.117 0
0.118 0
0.119 0
0.12 0
0.121 0
0.122 0
0.123 0
0.124 0
0.125 0
0.126 0
0.127 0
0.128 0
0.129 0
0.13 0
0.131 0
0.132 0
0.133 0
0.134 0
0.135 0
0.136 0
0.137 0
0.138 0
0.139 0
0.14 0
0.141 0
0.142 0
0.143 0
0.144 0
0.145 0
0.146 0
0.147 0
0.148 0
0.149 0
0.15 0
0.151 0
0.152 0
0.153 0
0.154 0
0.155 0
0.156 0
0.157 0
0.158 0
0.159 0
0.16 0
0.161 0
0.162 0
0.163 0
0.164 0
0.165 0
0.166 0
0.167 0
0.168 0
0.169 0
0.17 0
0.171 0
0.172 0
0.173 0
0.174 0
0.175 0
0.176 0
0.177 0
0.178 0
0.179 0
0.18 0
0.181 0
0.182 0
0.183 0
0.184 0
0.185 0
0.186 0
0.187 0
0.188 0
0.189 0
0.19 0
0.191 0
0.192 0
0.193 0
0.194 0
0.195 0
0.196 0
0.197 0
0.198 0
0.199 0
0.2 0
0.201 0
0.202 0
0.203 0
0.204 0
0.205 0
0.206 0
0.207 0
0.208 0
0.209 0
0.21 0
0.211 0
0.212 0
0.213 0
0.214 0
0.215 0
0.216 0
0.217 0
0.218 0
0.219 0
0.22 0
0.221 0
0.222 0
0.223 0
0.224 0
0.225 0
0.226 0
0.227 0
0.228 0
0.229 0
0.23 0
0.231 0
0.232 0
0.233 0
0.234 0
0.235 0
0.236 0
0.237 0
0.238 0
0.239 0
0.24 0
0.241 0
0.242 0
0.243 0
0.244 0
0.245 0
0.246 0
0.247 0
0.248 0
0.249 0
0.25 0
0.251 0
0.252 0
0.253 0
0.254 0
0.255 0
0.256 0
0.257 0
0.258 0
0.259 0
0.26 0
0.261 0
0.262 0
0.263 0
0.264 0
0.265 0
0.266 0
0.267 0
0.268 0
0.269 0
0.27 0
0.271 0
0.272 0
0.273 0
0.274 0
0.275 0
0.276 0
0.277 0
0.278 0
0.279 0
0.28 0
0.281 0
0.282 0
0.283 0
0.284 0
0.285 0
0.286 0
0.287 0
0.288 0
0.289 0
0.29 0
0.291 0
0.292 0
0.293 0
0.294 0
0.295 0
0.296 0
0.297 0
0.298 0
0.299 0
0.3 0
0.301 0
0.302 0
0.303 0
0.304 0
0.305 0
0.306 0
0.307 0
0.308 0
0.309 0
0.31 0
0.311 0
0.312 0
0.313 0
0.314 0
0.315 0
0.316 0
0.317 0
0.318 0
0.319 0
0.32 0
0.321 0
0.322 0
0.323 0
0.324 0
0.325 0
0.326 0
0.327 0
0.328 0
0.329 0
0.33 0
0.331 0
0.332 0
0.333 0
0.334 0
0.335 0
0.336 0
0.337 0
0.338 0
0.339 0
0.34 0
0.341 0
0.342 0
0.343 0
0.344 0
0.345 0
0.346 0
0.347 0
0.348 0
0.349 0
0.35 0
0.351 0
0.352 0
0.353 0
0.354 0
0.355 0
0.356 0
0.357 0
0.358 0
0.359 0
0.36 0
0.361 0
0.362 0
0.363 0
0.364 0
0.365 0
0.366 0
0.367 0
0.368 0
0.369 0
0.37 0
0.371 0
0.372 0
0.373 0
0.374 0
0.375 0
0.376 0
0.377 0
0.378 0
0.379 0
0.38 0
0.381 0
0.382 0
0.383 0
0.384 0
0.385 0
0.386 0
0.387 0
0.388 0
0.389 0
0.39 0
0.391 0
0.392 0
0.393 0
0.394 0
0.395 0
0.396 0
0.397 0
0.398 0
0.399 0
0.4 0
0.401 0
0.402 0
0.403 0
0.404 0
0.405 0
0.406 0
0.407 0
0.408 0
0.409 0
0.41 0
0.411 0
0.412 0
0.413 0
0.414 0
0.415 0
0.416 0
0.417 0
0.418 0
0.419 0
0.42 0
0.421 0
0.422 0
0.423 0
0.424 0
0.425 0
0.426 0
0.427 0
0.428 0
0.429 0
0.43 0
0.431 0
0.432 0
0.433 0
0.434 0
0.435 0
0.436 0
0.437 0
0.438 0
0.439 0
0.44 0
0.441 0
0.442 0
0.443 0
0.444 0
0.445 0
0.446 0
0.447 0
0.448 0
0.449 0
0.45 0
0.451 0
0.452 0
0.453 0
0.454 0
0.455 0
0.456 0
0.457 0
0.458 0
0.459 0
0.46 0
0.461 0
0.462 0
0.463 0
0.464 0
0.465 0
0.466 0
0.467 0
0.468 0
0.469 0
0.47 0
0.471 0
0.472 0
0.473 0
0.474 0
0.475 0
0.476 0
0.477 0
0.478 0
0.479 0
0.48 0
0.481 0
0.482 0
0.483 0
0.484 0
0.485 0
0.486 0
0.487 0
0.488 0
0.489 0
0.49 0
0.491 0
0.492 0
0.493 0
0.494 0
0.495 0
0.496 0
0.497 0
0.498 0
0.499 0
0.5 0
0.501 0
0.502 0
0.503 0
0.504 0
0.505 0
0.506 0
0.507 0
0.508 0
0.509 0
0.51 0
0.511 0
0.512 0
0.513 0
0.514 0
0.515 0
0.516 0
0.517 0
0.518 0
0.519 0
0.52 0
0.521 0
0.522 0
0.523 0
0.524 0
0.525 0
0.526 0
0.527 0
0.528 0
0.529 0
0.53 0
0.531 0
0.532 0
0.533 0
0.534 0
0.535 0
0.536 0
0.537 0
0.538 0
0.539 0
0.54 0
0.541 0
0.542 0
0.543 0
0.544 0
0.545 0
0.546 0
0.547 0
0.548 0
0.549 0
0.55 0
0.551 0
0.552 0
0.553 0
0.554 0
0.555 0
0.556 0
0.557 0
0.558 0
0.559 0
0.56 0
0.561 0
0.562 0
0.563 0
0.564 0
0.565 0
0.566 0
0.567 0
0.568 0
0.569 0
0.57 0
0.571 0
0.572 0
0.573 0
0.574 0
0.575 0
0.576 0
0.577 0
0.578 0
0.579 0
0.58 0
0.581 0
0.582 0
0.583 0
0.584 0
0.585 0
0.586 0
0.587 0
0.588 0
0.589 0
0.59 0
0.591 0
0.592 0
0.593 0
0.594 0
0.595 0
0.596 0
0.597 0
0.598 0
0.599 0
0.6 0
0.601 0
0.602 0
0.603 0
0.604 0
0.605 0
0.606 0
0.607 0
0.608 0
0.609 0
0.61 0
0.611 0
0.612 0
0.613 0
0.614 0
0.615 0
0.616 0
0.617 0
0.618 0
0.619 0
0.62 0
0.621 0
0.622 0
0.623 0
0.624 0
0.625 0
0.626 0
0.627 0
0.628 0
0.629 0
0.63 0
0.631 0
0.632 0
0.633 0
0.634 0
0.635 0
0.636 0
0.637 0
0.638 0
0.639 0
0.64 0
0.641 0
0.642 0
0.643 0
0.644 0
0.645 0
0.646 0
0.647 0
0.648 0
0.649 0
0.65 0
0.651 0
0.652 0
0.653 0
0.654 0
0.655 0
0.656 0
0.657 0
0.658 0
0.659 0
0.66 0
0.661 0
0.662 0
0.663 0
0.664 0
0.665 0
0.666 0
0.667 0
0.668 0
0.669 0
0.67 0
0.671 0
0.672 0
0.673 0
0.674 0
0.675 0
0.676 0
0.677 0
0.678 0
0.679 0
0.68 0
0.681 0
0.682 0
0.683 0
0.684 0
0.685 0
0.686 0
0.687 0
0.688 0
0.689 0
0.69 0
0.691 0
0.692 0
0.693 0
0.694 0
0.695 0
0.696 0
0.697 0
0.698 0
0.699 0
0.7 0
0.701 0
0.702 0
0.703 0
0.704 0
0.705 0
0.706 0
0.707 0
0.708 0
0.709 0
0.71 0
0.711 0
0.712 0
0.713 0
0.714 0
0.715 0
0.716 0
0.717 0
0.718 0
0.719 0
0.72 0
0.721 0
0.722 0
0.723 0
0.724 0
0.725 0
0.726 0
0.727 0
0.728 0
0.729 0
0.73 0
0.731 0
0.732 0
0.733 0
0.734 0
0.735 0
0.736 0
0.737 0
0.738 0
0.739 0
0.74 0
0.741 0
0.742 0
0.743 0
0.744 0
0.745 0
0.746 0
0.747 0
0.748 0
0.749 0
0.75 0
0.751 0
0.752 0
0.753 0
0.754 0
0.755 0
0.756 0
0.757 0
0.758 0
0.759 0
0.76 0
0.761 0
0.762 0
0.763 0
0.764 0
0.765 0
0.766 0
0.767 0
0.768 0
0.769 0
0.77 0
0.771 0
0.772 0
0.773 0
0.774 0
0.775 0
0.776 0
0.777 0
0.778 0
0.779 4.94065645841247e-324
0.78 4.25491014021677e-317
0.781 3.53381649321468e-310
0.782 2.54420624099887e-303
0.783 1.5866658159489e-296
0.784 8.56461884490139e-290
0.785 3.99833368979171e-283
0.786 1.61306458891609e-276
0.787 5.61918992311433e-270
0.788 1.6888322152547e-263
0.789 4.3754803571598e-257
0.79 9.76384543446471e-251
0.791 1.87497391915346e-244
0.792 3.09574838834201e-238
0.793 4.39078951838754e-232
0.794 5.34478191265313e-226
0.795 5.57859623250057e-220
0.796 4.98790037438595e-214
0.797 3.81673390694148e-208
0.798 2.49702875031027e-202
0.799 1.395345625355e-196
0.8 6.65316533172589e-191
0.801 2.7040658749984e-185
0.802 9.35826098449554e-180
0.803 2.75486717239213e-174
0.804 6.89071882174844e-169
0.805 1.46288218682303e-163
0.806 2.63299146654857e-158
0.807 4.013198998803e-153
0.808 5.17404648263784e-148
0.809 5.63581577741579e-143
0.81 5.18023603184757e-138
0.811 4.01309006985814e-133
0.812 2.61701259474567e-128
0.813 1.43476794123977e-123
0.814 6.60462419694282e-119
0.815 2.54939335491226e-114
0.816 8.24080064074958e-110
0.817 2.2277011700033e-105
0.818 5.0292122501029e-101
0.819 9.46865315545411e-97
0.82 1.4845686929154e-92
0.821 1.93554625504211e-88
0.822 2.09533630300071e-84
0.823 1.88059288341279e-80
0.824 1.397199967879e-76
0.825 8.57955142135212e-73
0.826 4.34730098186531e-69
0.827 1.81475461792263e-65
0.828 6.23074434535115e-62
0.829 1.75651962494063e-58
0.83 4.05892041865657e-55
0.831 7.67452976634479e-52
0.832 1.18522688136398e-48
0.833 1.49234638655822e-45
0.834 1.52915315355625e-42
0.835 1.27269626150659e-39
0.836 8.58725413456208e-37
0.837 4.68798980272875e-34
0.838 2.06657959125617e-31
0.839 7.34114815653018e-29
0.84 2.09709417766619e-26
0.841 4.80719493777429e-24
0.842 8.82357825834798e-22
0.843 1.29394985467866e-19
0.844 1.51262400140624e-17
0.845 1.40632813948237e-15
0.846 1.03744585012788e-13
0.847 6.05797364478302e-12
0.848 2.79325866162844e-10
0.849 1.01445653815584e-08
0.85 2.89459881640702e-07
0.851 6.4721254714704e-06
0.852 0.000113098353752267
0.853 0.00154042026845579
0.854 0.016307727690588
0.855 0.133810717794225
0.856 0.848549219527334
0.857 4.14638978237826
0.858 15.5654062170328
0.859 44.7517049690791
0.86 98.2312089985834
0.861 164.0898604239
0.862 207.910609216928
0.863 199.146333763628
0.864 143.705610866307
0.865 77.8487722773576
0.866 31.5458493437938
0.867 9.52674899837384
0.868 2.13610033355251
0.869 0.354239524102041
0.87 0.0432768161369737
0.871 0.00387918262436916
0.872 0.000254070055745373
0.873 1.21074709336991e-05
0.874 4.17978307824279e-07
0.875 1.04069135735529e-08
0.876 1.86027860290161e-10
0.877 2.37625300803626e-12
0.878 2.15866916209263e-14
0.879 1.38779681710989e-16
0.88 6.28240047138235e-19
0.881 1.9922455868413e-21
0.882 4.40227762200277e-24
0.883 6.74169746093345e-27
0.884 7.11535403585436e-30
0.885 5.14604240725244e-33
0.886 2.5354074677424e-36
};
\addlegendentry{laptop}
\end{groupplot}

\end{tikzpicture}

	\caption{\textbf{Upper row:} images from the ``laptop'' class of ImageNet. \textbf{Bottom row:} Beta marginal distributions of the top-$k$ predictions for the respective image. In the first column, the overlap between the marginal of all classes is large, signifying high uncertainty, i.e. the prediction is ``I do not know''. In the column, ``notebook'' and ``laptop'' have confident, yet overlapping marginal densities and we, therefore, have a top-$2$ prediction: ``either a notebook or a laptop''. In the third column ``desktop computer'', ``screen'' and ``monitor'' have overlapping marginal densities, yielding a top-$3$ estimate. The last case shows a top-$1$ estimate: the network is confident that ``laptop'' is the only correct label.
	}
	\label{fig:imagenet_betas}
\end{figure*}

Classification tasks on large datasets with many classes, like ImageNet, are not often done in a Bayesian fashion since the posterior inference and sampling are expensive. The Laplace Bridge, in conjunction with the last-layer Bayesian approximations, can be used to alleviate this problem. Furthermore, having a full distribution over the softmax outputs of a BNN gives rise to new possibilities. For example, one could subsume all classes which have sufficiently overlapping marginal distributions into one if they are semantically similar as illustrated in \Cref{fig:imagenet_betas}.


%experiment to introduce new top k.
Another possibility is to improve the standard classification metrics. Large classification tasks like ImageNet are often compared along a top-$5$ metric, i.e.~it is tested whether the correct class is within the five most probable estimates of the network. Although widely accepted, this metric has some pathologies. Consider two examples: i) Assume the network has to classify a hypothetical image of ``raptor'' and it is confident that the label is either a ``hawk'' or an ``eagle''. Then all probability mass should be distributed between those two classes. The three other classes within the top-$5$ are not needed to inform the decision. ii) Assume the network has to classify an image of which it is confident that it is a ``fish'' but it is uncertain between ten different possible fish species. Which five of the ten fish classes is within the top-$5$ is nearly arbitrary and so is the thereby following classification.

Leveraging the probabilistic output provided by the Laplace Bridge, we propose a simple decision rule that can handle both examples and is more fine-grained due to its awareness of uncertainty. One may call such a rule \emph{uncertainty-aware top-$k$}; it is shown in \Cref{alg:ua-top-k}. Instead of taking the top-$k$ as a decision threshold for an arbitrary $k$ we take the uncertainty/confidence of the model to inform the decision. This is more flexible and therefore able to handle situations in which different numbers of classes are plausible outcomes. The Dirichlet distribution obtained from the Laplace Bridge provides this capability. In particular, since the marginal distribution over each component of a Dirichlet distribution is a $\mathrm{Beta}(\alpha_i, \sum_{j\neq i} \alpha_j)$, this can be done analytically and efficiently. The proposed decision rule uses the area of overlap between the marginal distributions of the sorted outcomes. This is similar to hypotheses testing, i.e.~$t$-tests \cite{nickerson2000null} or its Bayesian alternatives \cite{BayesianAltTTest2011}. If, for example, two Beta densities overlap more than $5\%$, we cannot say that they are different distributions with high confidence. All distributions that have sufficient overlap should become the new top-$k$ estimate. Figure \ref{fig:imagenet_betas} shows four examples from the ``laptop'' class of ImageNet.

We evaluate this decision rule on the test set of ImageNet. The overlap is calculated through the inverse CDF\footnote{Also known as the quantile function or percent point function} of the respective Beta marginals. The original top-$1$ accuracy of DenseNet on ImageNet is $0.744$. Meanwhile, the uncertainty-aware top-$k$ accuracy is $0.797$, where $k$ is on average $1.688$. A more detailed analysis is shown in \Cref{fig:imagenet_counts}. Most of the predictions given by the uncertainty-aware metric still yielded a top-$1$ prediction. %, thereby \textit{not} adding meaningless classes to the prediction.
This shows that using uncertainty does not imply adding meaningless classes to the prediction.
However, there are some non-negligible cases where $k$ equals to $2$, $3$, or $10$. This indicates that whenever there is ambiguity in the class labels, our method is able to detect it, and thus yields a significantly higher accuracy.


\setlength{\figwidth}{0.8\textwidth}
\setlength{\figheight}{0.3\textheight}

\begin{figure}[h!]
    \centering
    \scriptsize

    \hspace{-2em}
    % This file was created by tikzplotlib v0.9.0.
\begin{tikzpicture}

\definecolor{color0}{rgb}{0.12156862745098,0.466666666666667,0.705882352941177}

\begin{axis}[
height=\figheight,
tick align=outside,
tick pos=both,
width=\figwidth,
x grid style={white!69.0196078431373!black},
xlabel={top-k},
xmin=0.55, xmax=10.45,
xtick align=inside,
xtick pos=left,
xtick style={color=black},
y grid style={white!69.0196078431373!black},
ylabel={count},
ymin=0, ymax=45287.55,
ytick align=inside,
ytick pos=left,
ytick style={color=black}
]
\draw[draw=none,fill=color0] (axis cs:1,0) rectangle (axis cs:1.9,43131);
\draw[draw=none,fill=color0] (axis cs:1.9,0) rectangle (axis cs:2.8,2940);
\draw[draw=none,fill=color0] (axis cs:2.8,0) rectangle (axis cs:3.7,418);
\draw[draw=none,fill=color0] (axis cs:3.7,0) rectangle (axis cs:4.6,117);
\draw[draw=none,fill=color0] (axis cs:4.6,0) rectangle (axis cs:5.5,38);
\draw[draw=none,fill=color0] (axis cs:5.5,0) rectangle (axis cs:6.4,11);
\draw[draw=none,fill=color0] (axis cs:6.4,0) rectangle (axis cs:7.3,6);
\draw[draw=none,fill=color0] (axis cs:7.3,0) rectangle (axis cs:8.2,3);
\draw[draw=none,fill=color0] (axis cs:8.2,0) rectangle (axis cs:9.1,1);
\draw[draw=none,fill=color0] (axis cs:9.1,0) rectangle (axis cs:10,3335);
\end{axis}

\end{tikzpicture}


    \caption{A histogram of ImageNet predictions' length using the proposed uncertainty-aware top-$k$. Most test images are a top-$1$ prediction, indicating high confidence. There are some top-$2$, top-$3$, and top-$10$ predictions, showing an increasing uncertainty.}
    \label{fig:imagenet_counts}
\end{figure}


\begin{algorithm}[tb]
   \caption{Uncertainty-aware top-$k$}
   \label{alg:ua-top-k}
    \begin{algorithmic}
        \REQUIRE A Dirichlet parameter $\valpha \in \R^K$ obtained by applying the Laplace Bridge to the Gaussian over the logit of an input, a percentile threshold $T$ e.g. $0.05$, a function $\mathrm{class\_of}$ that returns the underlying class of a sorted index.
        \STATE
        \STATE $\tilde{\valpha} = \mathrm{sort\_descending}(\valpha)$ \COMMENT{start with the highest confidence}
        \STATE $\alpha_0 = \sum_i \alpha_i$
        \STATE $\mathcal{C} = \{ \mathrm{class\_of}(1) \}$ \COMMENT{initialize top-$k$, must include at least one class}

        \FOR{$i = 2, \dots, K$}
            \STATE $F_{i-1} = \mathrm{Beta}(\tilde{\alpha}_{i-1}, \alpha_0 - \tilde{\alpha}_{i-1})$ \COMMENT{the previous marginal CDF}
            \STATE $F_{i} = \mathrm{Beta}(\tilde{\alpha}_i, \alpha_0 - \tilde{\alpha}_i)$ \COMMENT{the current marginal CDF}
            \STATE $l_{i-1} = F_{i-1}^\inv(T/2)$  \COMMENT{left $\frac{T}{2}$ percentile of the previous marginal}
            \STATE $r_{i} = F_i^\inv(1-T/2)$ \COMMENT{right $\frac{T}{2}$ percentile of the current marginal}

            \IF{$r_i > l_{i-1}$}
                \STATE $\mathcal{C} = \mathcal{C} \cup \{ \mathrm{class\_of}(i) \}$  \COMMENT{overlap detected, add the current class}
            \ELSE
                \BREAK \COMMENT{No more overlap, end the algorithm}
            \ENDIF
        \ENDFOR
        \STATE
        \ENSURE $\mathcal{C}$ \COMMENT{return the resulting top-$k$ prediction}
    \end{algorithmic}
\end{algorithm}



