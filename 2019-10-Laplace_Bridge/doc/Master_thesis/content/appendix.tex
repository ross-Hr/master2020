\section*{Appendix A: Background and Proofs}
\label{sec:appendix_A}

\subsection*{Change of Variable for pdf} 
Let $\rvx$ be an $n$-dimensional continuous random variable with joint density function $p_\rvx$. If $\rvy = G(\rvx)$, where $G$ is a differentiable function, then $\rvy$ has density $p_\rvy$:
\begin{equation}
g(\mathbf{y}) = f\Big(G^{-1}(\mathbf{y})\Big)\left\vert \det\left[\frac{dG^{-1}(\mathbf{z})}{d\mathbf{z}}\Bigg \vert_{\mathbf{z}=\mathbf{y}}\right]\right \vert
\end{equation}
where the differential is the Jacobian of the inverse of $G$ evaluated at $\rvy$. This procedure, also known as `change of basis', is at the core of the Laplace bridge since it is used to transform the Dirichlet into the softmax basis.

\subsection*{Proof for Proposition}
\begin{proof}
    Considering that $\alpha_k$ is a decreasing function of $\mSigma_{kk}$ by definition \eqref{eq:mapping_alpha}, it is sufficient to show that under the hypothesis, the derivative of $\frac{\partial}{\partial \alpha_k}\mathrm{Var}(\pi_k| \valpha)$ is negative.

    By definition, the variance $\mathrm{Var}(\pi_k | \valpha)$ is
    %
    \begin{equation*}
        \mathrm{Var}(\pi_k | \valpha) = \frac{\frac{\alpha_k}{\alpha_k + \alpha_{\neq k}} - \frac{\alpha_k^2}{(\alpha_k + \alpha_{\neq k})^2}}{\alpha_k + \alpha_{\neq k} + 1} \, .
    \end{equation*}
    %
    The derivative is therefore
    %
    \begin{align*}
        \frac{\partial}{\partial \alpha_k}\mathrm{Var}&(\pi_k | \valpha) = \\
            &\frac{\alpha_{\neq k} (\alpha_{\neq k}^2 - \alpha_{\neq k} \alpha_k + \alpha_{\neq k} - \alpha_k (2 \alpha_k + 1))}{(\alpha_k + \alpha_{\neq k})^3 (\alpha_k + \alpha_{\neq k} + 1)^2} \, .
    \end{align*}
    %
    Solving $\frac{\partial}{\partial \alpha_k}\mathrm{Var}(\pi_k| \valpha) < 0$ for $\alpha_k$ yields
    %
    \begin{equation*}
        \alpha_k > \frac{1}{4} \left(\sqrt{9\alpha_{\neq k}^2 + 10\alpha_{\neq k} + 1} - \alpha_{\neq k} - 1\right) \, .
    \end{equation*}
    %
    Therefore, under this hypothesis, $\mathrm{Var}(\pi_k| \valpha)$ is a decreasing function of $\alpha_k$.
\end{proof}

\subsection*{Experimental Evaluation of the Proposition}

To test how often the condition is fulfilled we count its frequency. The fact that the condition is fulfilled implies a good approximation. The fact that the condition is not fulfilled does not automatically imply a bad approximation.

\begin{table}[h!]
	\scriptsize
    \centering
    \begin{tabularx}{\textwidth}{@{}XX|| X@{}}%{l  l || c}
	     \toprule
         & & frequency \\
         \midrule
         MNIST & MNIST & - \\
         MNIST & FMNIST & -\\
         MNIST & notMNIST & -\\
         MNIST & KMNIST & - \\
         \midrule
         CIFAR-10 & CIFAR-10  & 0.998 \\
         CIFAR-10 & CIFAR-100 & 0.925 \\
         CIFAR-10 & SVHN      & 0.832 \\
         \midrule
         SVHN & SVHN       &  0.999 \\
         SVHN & CIFAR-100  &  0.668 \\
         SVHN & CIFAR-10   &  0.653 \\
         \midrule
         CIFAR-100 & CIFAR-100 &  0.662 \\
         CIFAR-100 & CIFAR-10  &  0.214\\
         CIFAR-100 & SVHN      &  0.166\\
         \bottomrule
    \end{tabularx}
    \caption{}
    \label{tab:conditions_table}
\end{table}