\documentclass{article}

% if you need to pass options to natbib, use, e.g.:
%     \PassOptionsToPackage{numbers, compress}{natbib}
% before loading neurips_2018

% ready for submission
% \usepackage{neurips_2018}

% to compile a preprint version, e.g., for submission to arXiv, add add the
% [preprint] option:
%     \usepackage[preprint]{neurips_2018}

% to compile a camera-ready version, add the [final] option, e.g.:
     \usepackage[final]{nips_2018}

% to avoid loading the natbib package, add option nonatbib:
%     \usepackage[nonatbib]{neurips_2018}

\usepackage[utf8]{inputenc} % allow utf-8 input
\usepackage[T1]{fontenc}    % use 8-bit T1 fonts
\usepackage{hyperref}       % hyperlinks
\usepackage{url}            % simple URL typesetting
\usepackage{booktabs}       % professional-quality tables
\usepackage{amsfonts}       % blackboard math symbols
\usepackage{nicefrac}       % compact symbols for 1/2, etc.
\usepackage{microtype}      % microtypography
\usepackage{graphicx}
\usepackage{subfig}
\usepackage{mathtools,amsthm,amssymb,amsfonts}

\usepackage{arydshln} %for dashed lines

%change caption styles
\usepackage{caption} 
\captionsetup{labelfont=bf, font={small,sf}}

%change margins for the tables in the appendix
\usepackage{chngpage}

\title{Changing the Basis of distributions within the exponential family}

% The \author macro works with any number of authors. There are two commands
% used to separate the names and addresses of multiple authors: \And and \AND.
%
% Using \And between authors leaves it to LaTeX to determine where to break the
% lines. Using \AND forces a line break at that point. So, if LaTeX puts 3 of 4
% authors names on the first line, and the last on the second line, try using
% \AND instead of \And before the third author name.

\author{%
  Marius Hobbhahn \\
  Department of Computer Science\\
  University of Tübingen\\
  Tübingen, Germany \\
  \texttt{marius.hobbhahn@gmail.com} \\
  % examples of more authors
  % \And
  % Coauthor \\
  % Affiliation \\
  % Address \\
  % \texttt{email} \\
  % \AND
  % Coauthor \\
  % Affiliation \\
  % Address \\
  % \texttt{email} \\
  % \And
  % Coauthor \\
  % Affiliation \\
  % Address \\
  % \texttt{email} \\
  % \And
  % Coauthor \\
  % Affiliation \\
  % Address \\
  % \texttt{email} \\
}

\begin{document}

\maketitle

\section{Introduction}

normal distributions are the best, lets transform everything to normal

\section{General Information}

\subsection{Change of Variable for Probability Density Function}
\label{subsec:variable_change_pdf}

This is also referred to as change of basis. Let $X$ be a continuous random variable with PDF $f_X(x)$ over $c_1 < x < c_2$. And let $Y = g(x)$ be a monotonic differentiable function with Inverse $X = g^{-1}(Y)$. Then the PDF of $Y$ is
$$f_Y(y) = f_X(g^{-1}(y)) \cdot \left | \frac{d g^{-1}(y)}{dy} \right| 
= f_X(g^{-1}(y)) \cdot \left | \frac{d g(y)}{dy} \right|^{-1} $$

\subsection{Laplace Approximation}

The Laplace approximation (LPA) is a tool to fit a normal distribution to the PDF of a given other distribution. The only constraints for the other distribution are: one peak (mode/ point of maximum) and twice differentiable. Laplace proposed a simple 2-term Taylor expansion on the log pdf. If $\hat{\theta}$ denotes the mode of a pdf $h(\theta)$, then it is also the mode of the log-pdf $q(\theta) = \log h(\theta)$. The 2-term Taylor expansion of $q(\theta)$ therefore is:

\begin{align}
	q(\theta) &\approx q(\hat{\theta}) + q'(\hat{\theta})(\theta - \hat{\theta}) + \frac{1}{2}(\theta- \hat{\theta})q''(\hat{\theta}) (\theta - \hat{\theta})\\
	&= 	q(\hat{\theta}) + 0 +  \frac{1}{2}(\theta- \hat{\theta})q''(\hat{\theta}) (\theta - \hat{\theta}) \qquad \text{[since } q'(\theta) = 0]\\
	&= c - \frac{(\theta - \mu)^2}{2\sigma^2}
\end{align}
where $c$ is a constant, $\mu = \hat{\theta}$ and $\sigma^2 = \{-q''(\hat{\theta})\}^{-1}$. The right hand side of the last line matches the log-pdf of a normal distribution $N(\mu, \sigma^2)$. Therefore the pdf $h(\theta)$ is approximated by the pdf of the normal distribution $N(\mu, \sigma^2)$ where $\mu = \hat{\theta}$ and $\sigma^2 = \{-q''(\hat{\theta})\}^{-1}$. Note, that even though this derivation is done for the one dimensional case only, it is also true for the multidimensional case. The second derivative just becomes the Hessian of the pdf at the mode.

\subsection{Exponential Family}

exponential family form

\begin{equation}
 f(x|\theta) = h(x)\exp[\eta(\theta) t(x) - A(\theta)]
 \label{eq:exp_family}
\end{equation}

for sufficient statistics $t:X \rightarrow \mathbb{R}$, natural parameters $\eta: \Theta \rightarrow \mathbb{R}$, and functions $A: \Theta \rightarrow \mathbb{R}$ and $h: X \rightarrow \mathbb{R}_+$

\section{General overview}

\begin{table}[htb]
	\centering
	\caption{Overview}
	\begin{tabular}{lcccc}
		\toprule
		\textbf{Distribution}	&\textbf{$g(x)$} &\textbf{$g^{-1}(x)$} &\textbf{$T$} & \text{$T$'}	\\
		\midrule
		Exponential	& log & exp & $x$ & $(x,\exp(x))$ \\
		Gamma		& log & exp & $(\log(x), x)$ & $(x,\exp(x))$ \\
		Gamma		& log & exp & $(\log(x), x)$ & $(\log(x),x^2)$ \\
		Inverse Gamma & log & exp& $(\log(x), x)$ & $(x,\exp(x))$ \\
		Inverse Gamma & log & exp& $(\log(x), x)$ & $(\log(x),x^2)$ \\
		Chi2        & log & exp &  $(\log(x), x)$ & $(x,\exp(x))$ \\
		Chi2        & log & exp &  $(\log(x), x)$ & $(\log(x),x^2)$ \\
		Beta		& logit & logistic&  $(\log(x),\log(1-x))$ & ? \\
		Dirichlet 	& - & softmax &  ? & ? \\
		Wishart	    & mlog & mexp & $(\log(X), X)$ & $(X, \exp(X))$ \\
		Wishart	    & msqrt & msqr & $(\log(X), X)$ & $(\log(X), X^2)$ \\
		Inverse Wishart & mlog & mexp & ? & ? \\		
		\bottomrule
	\end{tabular}
\end{table}

NOTE:MOST DISTRIBUTIONS ACTUALLY HAVE TWO VALID TRANSFORMS. ONE FOR $X$ AND ONE FOR $X^2$ AS SUFFICIENT STATISTICS


\section{Exponential Distribution}

\subsection{Standard Exponential Distribution}

The PDF of the exponential distribution is

\begin{equation}
	p(x| \lambda) = \lambda \exp(-\lambda x)
	\label{eq:exponential_pdf}
\end{equation}

which can be written as

\begin{equation}
	p(x|\lambda) = \exp\left[-\lambda x + \log\lambda\right]
	\label{eq:exponential_exp_family}
\end{equation}

with $T=x$, $\eta = -\lambda$ and $A(\lambda) = -\log\lambda$

\subsubsection{Laplace Approximation of the Exponential Distribution}

\begin{align*}
\text{log-pdf: } &\log\left( \log \lambda - \lambda x\right) \\
\text{1st derivative: }& - \lambda \\
\text{2nd derivative: }& 0
\end{align*}
The Laplace Approximation is not defined since the second derivative is not positive. 

\subsection{Log-transformed Exponential Distribution}

We choose $g(x) = \log(x)$, and thereby $g^{-1}(x) = \exp(x)$. It follows that the new pdf is 

\begin{align}
	f(x|\lambda) &= \lambda \exp(-\lambda \exp(x)) \cdot \exp(x) \\ \nonumber
	&= \lambda \exp(-\lambda \exp(x) + x)
\end{align}

which can be written as 


\begin{equation}
	p(x|\lambda) = \exp\left[-\lambda \exp(x) + x + \log\lambda\right]
	\label{eq:exponential_exp_family_trans}
\end{equation}

with $T= x, \exp(x)$, $\eta = 1, -\lambda$ and $A(\lambda)=\log\lambda$

\subsubsection{Laplace Approximation of the log-transformed Exponential Distribution}

\begin{align*}
\text{log-pdf: } & -\lambda \exp(x) + x + \log\lambda \\
\text{1st derivative: }& \lambda -\exp(x) + 1 \\
\text{mode: } & x = \log(1/\lambda) \\
\text{2nd derivative: }& \lambda -\exp(x)\\
\text{insert mode: }& -\lambda\exp(1/\lambda) = -1\\
\text{invert: }&\sigma^2 = 1
\end{align*}

Therefore the Laplace approximation in the transformed basis is given by $\mathcal{N}(x, \log(1/\lambda), 1)$. 

\subsection{The Bridge}

We have already found $\mu$ and $\sigma$. The inverse transformation is easily found through the mode $x = \log(1/\lambda) \Leftrightarrow \lambda = 1/\exp(x)$. In summary:

\begin{align}
	\mu &= \log(1/\lambda) \\
	\sigma &= 1 \\
	\lambda &= 1/\exp(x)
\end{align}

\begin{figure}
	\centering
	\includegraphics[width=\textwidth]{figures/exponential_playground.pdf}
	\caption{exponential comparison}
	\label{fig:exponential_comparison}
\end{figure}

\section{Gamma Distribution}

\subsection{Standard Gamma Distribution}

\begin{equation}
	f(x, \alpha, \lambda) = \frac{\lambda^\alpha}{\Gamma(\alpha)} \cdot x^{(\alpha - 1)} \cdot e^{(-\lambda x)}
	\label{eq:gamma_pdf}
\end{equation}

where $\Gamma(\alpha)$ is the Gamma function. This can be written as

\begin{equation}
	f(x, \alpha, \lambda) = \exp \left[(\alpha -1)\log(x) - \lambda x + \alpha \log(\lambda) - \log(\Gamma(\alpha))\right]
	\label{eq:gamma_exp_family}
\end{equation}

with $T=(\log x, x), \eta=(\alpha, -\lambda)$ and $A(\alpha, \lambda) = \log(\Gamma(\alpha)) - \alpha  \log(\lambda)$. 

\subsubsection{Laplace Approximation of the Gamma Distribution}

To get the LPA of the Gamma function in the standard basis we need its mode and the second derivative of the log-pdf. The mode is already known to be $\hat{\theta} = \frac{\alpha -1}{\lambda}$. For the second derivative of the log-pdf we take the log-pdf and derive it twice and insert the mode for $x$:
\begin{align*}
	\text{log-pdf: } &\log\left( \frac{\lambda^\alpha}{\Gamma(\alpha)} \cdot x^{(\alpha - 1)} \cdot e^{(-\lambda x)}\right) \\
	&= \alpha \cdot \log(\lambda) - \log(\Gamma(\alpha)) + (\alpha -1)\log(x) -\lambda x\\
	\text{1st derivative: }& \frac{(\alpha-1)}{x} - \lambda \\
	\text{mode: }&  \frac{(\alpha-1)}{x} - \lambda = 0 \Leftrightarrow x=\frac{\alpha -1}{\lambda}\\
	\text{2nd derivative: }& -\frac{(\alpha-1)}{x^2}\\
	\text{insert mode: }& -\frac{(\alpha-1)}{(\frac{\alpha -1}{\lambda})^2} = -\frac{\lambda^2}{\alpha - 1} \\
	\text{invert and times -1: }&\sigma^2 = \frac{\alpha-1}{\lambda^2}
\end{align*}

The LPA of the Gamma distribution is therefore approximately distributed according to the pdf of $\mathcal{N}(\frac{\alpha - 1}{\lambda}, \frac{\lambda^2}{\alpha-1})$.

\subsection{Sqrt-Transform of the Gamma Distribution}

\subsubsection{Sqrt-Transformation}

We transform the Gamma Distribution with the Log-Transformation, i.e. $Y = \sqrt(X), g(x) = \sqrt(x), g^{-1}(x) = x^2$. The transformed pdf is

\begin{align}
f_t(x, \alpha, \lambda) &= \frac{\lambda^\alpha}{\Gamma(\alpha)} \cdot x^{2(\alpha - 1)} \cdot e^{(-\lambda x^2)} \cdot 2x \\ \nonumber
&=\frac{\lambda^\alpha}{\Gamma(\alpha)} \cdot x^{2\alpha} \cdot e^{(-\lambda x^2)}
\end{align}

which can be rewritten as

\begin{equation}
f_t(x, \alpha, \lambda) = \exp \left[2\alpha \log(x) - \lambda 2x + \alpha \log(\lambda) - \log(\Gamma(\alpha))\right]	
\label{eq:square_gamma_trans}
\end{equation}

with $T = (\log(x), x^2), \eta = (2\alpha, -\lambda)$ and $A(\alpha, \lambda) = \log(\Gamma(\alpha)) - \alpha  \log(\lambda)$. 


\subsubsection{Laplace Approximation of the sqrt-transformed Gamma Distribution}

To get the LPA of the Gamma distribution in the transformed basis we need to calculate its mode and the second derivative of the log-pdf. To get the mode we take the first derivative and set it to zero. 

\begin{align*}
\text{log-pdf: } &2\alpha \log(x) - \lambda 2x + \alpha \log(\lambda) - \log(\Gamma(\alpha)) \\
\text{1st derivative: }&  \frac{2\alpha}{x} - 2\lambda x\\
\text{mode: }& \frac{2\alpha}{x} - 2\lambda x = 0 \Leftrightarrow x = \sqrt{\frac{\alpha}{\lambda}}\\
\text{2nd derivative: }&  -\frac{2\alpha}{x^2} - 2\lambda\\
\text{insert mode: }& -\frac{2\alpha}{\frac{\alpha}{\lambda}} - 2\lambda = -4\lambda\\
\text{invert and times -1: }& \frac{1}{4\lambda}
\end{align*}

Therefore the LPA now is $\mathcal{N}\left(\sqrt{\frac{\alpha}{\lambda}}, \frac{1}{4\lambda} \right)$.

\subsubsection{The bridge for the sqrt-transformation}

We already know how to get $\mu$ and $\sigma$ from $\lambda$ and $\alpha$. To invert we calculate $\mu = \sqrt{\frac{\alpha}{\lambda}} \Leftrightarrow \alpha = \frac{\mu^2}{\lambda}$ and insert $\lambda=\frac{4}{\sigma^2}$. In summary we have

\begin{align}
\mu &= \sqrt{\frac{\alpha}{\lambda}} \\
\sigma^2 &= \frac{1}{4\lambda} \\
\lambda &= \frac{4}{\sigma^2} \\
\alpha &= \frac{(\sigma\mu)^2}{4} 
\end{align}

\begin{figure}[!htb]
	\centering
	\includegraphics[width=\textwidth]{figures/gamma_playground_sqrt.pdf}
	\caption{gamma comparison square}
	\label{fig:gamma_comparison_square}
\end{figure}

\subsection{Log-Transform of the Gamma Distribution}

\subsubsection{Log-Transformation}

We transform the Gamma Distribution with the Log-Transformation, i.e. $Y = \log(X), g(x) = \log(x), g^{-1}(x) = \exp(x)$. The transformed pdf is

\begin{align}
f_t(x, \alpha, \lambda) &= \frac{\lambda^\alpha}{\Gamma(\alpha)} \cdot \exp(x)^{(\alpha - 1)} \cdot e^{(-\lambda \exp(x))} \cdot \exp(x) \\ \nonumber
&=\frac{\lambda^\alpha}{\Gamma(\alpha)} \cdot \exp(x)^{\alpha} \cdot e^{(-\lambda \exp(x))}
\end{align}

which can be rewritten as

\begin{equation}
	f_t(x, \alpha, \lambda) = \exp \left[\alpha x - \lambda \exp(x) \alpha \log(\lambda) - \log(\Gamma(\alpha))\right]	
	\label{eq:exp_gamma_trans}
\end{equation}

with $T = (x, \exp(x)), \eta = (\alpha, -\lambda)$ and $A(\alpha, \lambda) = \log(\Gamma(\alpha)) - \alpha  \log(\lambda)$. 


\subsubsection{Laplace Approximation of the log-transformed Gamma Distribution}

To get the LPA of the Gamma distribution in the transformed basis we need to calculate its mode and the second derivative of the log-pdf. To get the mode we take the first derivative and set it to zero. 

\begin{align*}
\text{log-pdf: } &\log\left(\frac{\lambda^\alpha}{\Gamma(\alpha)} \cdot \exp(x)^{\alpha} \cdot exp(-\lambda \exp(x)) \right) \\
&= \alpha \log(\lambda) - \log(\Gamma(\alpha)) + \alpha x - \lambda \exp(x)\\
\text{1st derivative: }&  \alpha - \lambda \exp(x)\\
\text{mode: }& \alpha - \lambda \exp(x) = 0 \Leftrightarrow x = \log\left(\frac{\alpha}{\lambda}\right)\\
\text{2nd derivative: }&  -\lambda \exp(x)\\
\text{insert mode: }& -\lambda \exp(\log\left(\frac{\alpha}{\lambda}\right)) = -\frac{1}{\alpha} \\
\text{invert and times -1: }&\sigma^2 = \alpha 
\end{align*}

Therefore the LPA now is $N(\log\left(\frac{\alpha}{\lambda}\right), \alpha)$.

\subsubsection{The bridge for the log-transformation}

We already know how to get $\mu$ and $\sigma$ from $\lambda$ and $\alpha$. To invert we calculate $\mu = \log(\alpha/\lambda) \Leftrightarrow \lambda= \alpha/\exp(\mu)$ and insert $\alpha=\sigma^2$. In summary we have

\begin{align}
	\mu &= \log(\alpha/\lambda) \\
	\sigma^2 &= \alpha \\
	\lambda &= \alpha/\exp(\mu) \\
	\alpha &= \sigma^2
\end{align}

\begin{figure}[!htb]
	\centering
	\includegraphics[width=\textwidth]{figures/gamma_playground_log.pdf}
	\caption{gamma comparison}
	\label{fig:gamma_comparison}
\end{figure}

\section{Inverse Gamma Distribution}

\subsection{Standard Inverse Gamma Distribution}

The pdf of the inverse gamma is 

\begin{equation}
	f(x, \alpha, \lambda) = \frac{\lambda^{\alpha}}{\Gamma(\alpha)} x^{-\alpha-1} \exp(-\frac{\lambda}{x})
	\label{eq:pdf_inverse_gamma}
\end{equation}

where $\Gamma$ is the Gamma function. This can be rewritten as

\begin{equation}
	f(x, \alpha, \lambda) = \exp \left[(-\alpha-1)\log(x) - \lambda/x + \alpha \log(\lambda) -\log\Gamma(\alpha)\right]
	\label{eq:exp_inverse_gamma}
\end{equation}

where $T=(\log(x), x), \eta= (-\alpha-1, -\lambda)$ and $A(\alpha,\lambda) = \log\Gamma(\alpha) - \alpha\log\lambda$.

\subsubsection{Laplace Approximation of the standard inverse gamma distribution}

\begin{align*}
\text{log-pdf: } &(-\alpha-1)\log(x) - \lambda/x + \alpha \log(\lambda) -\log\Gamma(\alpha) \\
\text{1st derivative: }&  \frac{-\alpha-1}{x} + \frac{\lambda}{x^2}\\
\text{mode: }& \frac{-\alpha-1}{x} + \frac{\lambda}{x^2} = 0 \Leftrightarrow x = \frac{\lambda}{a+1}\\
\text{2nd derivative: }&  \frac{\alpha+1}{x^2} - 2\frac{\lambda}{x^3}\\
\text{insert mode: }& \frac{\alpha+1}{\frac{\lambda}{a+1}^2} - 2\frac{\lambda}{\frac{\lambda}{a+1}^3} = -\frac{(\alpha +1)^3}{\lambda^2} \\
\text{invert and times -1: }&\sigma^2 = \frac{\lambda^2}{(\alpha +1)^3}
\end{align*}

\subsection{Sqrt-Transform of the inverse Gamma distribution}

TODO: Double-Check this whole sqrt business.

\subsubsection{Laplace Approximation of the sqrt-transformed Inverse Gamma Distribution}

\begin{align*}
\text{log-pdf: } &-2\alpha\log(x) - \frac{\lambda}{x^2} + \alpha \log \lambda - \log\Lambda(\alpha) \\
\text{1st derivative: }& -\frac{2\alpha}{x^2} + 2\frac{\lambda}{x^3} \\
\text{mode: }&  x = \sqrt{\frac{\lambda}{\alpha}}\\
\text{2nd derivative: }&  \frac{2\alpha}{x^2} - 6\frac{\lambda}{x^4}\\
\text{insert mode: }&  -4\frac{\alpha^2}{\lambda}\\
\text{invert and times -1: }&\sigma^2 = \frac{\lambda}{4 \alpha^2}
\end{align*}

\subsubsection{The Bridge for the sqrt-transformed Inverse Gamma Distribution}

\begin{align}
\mu &= \sqrt\frac{\lambda}{\alpha} \\
\sigma^2 &= \frac{\lambda}{4\alpha^2} \\
\alpha &= \frac{\mu^2}{4\sigma^2}\\
\lambda &= \frac{\mu^4}{4\sigma^2}\\
\end{align}

\subsection{Log-Transform of the inverse Gamma distribution}

We choose $g(x) = \log(x)$, and thereby $g^{-1}(x) = \exp(x)$. It follows that the new pdf is 

\begin{equation}
	f_t(x, \alpha, \lambda) = \frac{\lambda^{\alpha}}{\Gamma(\alpha)} \exp(x)^{-\alpha} \exp(-\lambda/\exp(x))
	\label{eq:inv_gamma_trans_pdf}
\end{equation}

which can be written as 

\begin{equation}
	f_t(x, \alpha, \lambda) = \exp \left[-\alpha x - \frac{\lambda}{\exp(x)} + \alpha \log \lambda - \log\Lambda(\alpha)\right]
\end{equation}

with $T=(x, 1/\exp(x)), \eta(-\alpha, \lambda)$ and $A(\alpha, \lambda) = \log\Gamma(\alpha) - \alpha \log \lambda$.

\subsubsection{Laplace Approximation of the log-transformed Inverse Gamma Distribution}


\begin{align*}
\text{log-pdf: } &-\alpha x - \frac{\lambda}{\exp(x)} + \alpha \log \lambda - \log\Lambda(\alpha) \\
\text{1st derivative: }&  -\alpha + \frac{\lambda}{\exp(x)}\\
\text{mode: }&  -\alpha + \frac{\lambda}{\exp(x)} = 0 \Leftrightarrow x = \log(\lambda/\alpha)\\
\text{2nd derivative: }&  -\frac{\lambda}{\exp(x)}\\
\text{insert mode: }&  -\frac{\lambda}{\exp(\log(\lambda/\alpha))} = -\alpha\\
\text{invert and times -1: }&\sigma^2 = \frac{1}{\alpha}
\end{align*}

\subsubsection{The Bridge for the log-transformed Inverse Gamma Distribution}

\begin{align}
	\mu &= \log\left(\frac{\lambda}{\alpha}\right) \\
	\sigma^2 &= \frac{1}{\alpha} \\
	\alpha &= \frac{1}{\sigma^2}\\
	\lambda &= \frac{\exp(\mu)}{\sigma^2}\\
\end{align}

\begin{figure}
	\centering
	\includegraphics[width=\textwidth]{figures/inverse_gamma_playground.pdf}
	\caption{inverse gamma comparison}
	\label{fig:inverse_gamma_comparison}
\end{figure}

\subsection{Standard Chi-squared distribution}

The pdf of the Chi-squared distribution in the standard basis is

\begin{align}\label{eq:chi2_pdf}
	\mathcal{C}_X(x, k) &= \frac{1}{2^{k/2}\Gamma(k/2)}  x^{k/2 -1} \exp(-x/2) \\
	&= \frac{1}{x}\exp \left[(k/2)\log(x) - x/2 - \log(2^{k/2}\Gamma(k/2))\right]
\end{align} 

with $h(x)=\frac{1}{x}$, $\phi(x) = (\log(x), x), w = k/2$ and $Z(k) = \log(2^{k/2}\Gamma(k/2))$.

\subsubsection{Laplace approximation of the standard Chi-squared distribution}

\begin{align*}
\text{log-pdf: } &(k/2-1)\log(x) - x/2 - \log(2^{k/2}\Gamma(k/2)) \\
\text{1st derivative: }&  \frac{k/2-1}{x} - \frac{1}{2} \\
\text{mode: }&  \frac{k/2-1}{x} - \frac{1}{2} = 0 \Leftrightarrow x = k-2\\
\text{2nd derivative: }&  -\frac{k/2-1}{x^2}\\
\text{insert mode: }& -\frac{k/2-1}{(k-2)^2} = -\frac{(k-2)}{2(k-2)^2}\\
\text{invert and times -1: }&\sigma^2 = 2(k-2)
\end{align*}

\subsection{Log-Transformed Chi-squared distribution}

we transform the distribution with $g(x) = \log(x)$, i.e. $x(y) = g^{-1}(x) = \exp(y)$. The new pdf becomes

\begin{align}
	\mathcal{C}_{Y\_\log}(y,k) &= \frac{1}{2^{k/2}\Gamma(k/2)}  \exp(y)^{k/2 -1} \exp(-\exp(y)/2) \cdot \exp(y) \\
	&= \frac{1}{2^{k/2}\Gamma(k/2)}  \exp(y)^{k/2} \exp(-\exp(y)/2)
	&= \exp\left[\frac{k}{2}y - \frac{\exp(y)}{x} - \log(2^{k/2}\Gamma(k/2))\right]
\end{align}

meaning $h(y) = 1, \phi(y) =(y, \exp(y)), \eta=(k/2)$ and $Z(k) =  \log(2^{k/2}\Gamma(k/2))$. 

\subsubsection{Laplace approximation of the log-transformed Chi-squared distribution}

\begin{align*}
\text{log-pdf: } &\frac{k}{2}y - \frac{\exp(y)}{2} - \log(2^{k/2}\Gamma(k/2)) \\
\text{1st derivative: }&  \frac{k}{2} - \frac{\exp(y)}{2} \\
\text{mode: }& k/2 - \frac{\exp(y)}{2} = 0 \Leftrightarrow y = \log(k)\\
\text{2nd derivative: }&  -\frac{\exp(y)}{2}\\
\text{insert mode: }& -\frac{\exp(y)}{2} = -k/2\\
\text{invert and times -1: }&\sigma^2 = 2/k
\end{align*}

which yields the Laplace Approximation $q(y) = \mathcal{N}(y; \mu= \log(k), \sigma^2 = 2/k)$.

\subsubsection{The Bridge for log-transform}

\begin{align}
	\mu &= \log(k) \\
	\sigma^2 &= 2/k \\
	k &= \exp(\mu)
\end{align}

\begin{figure}[!htb]
	\centering
	\includegraphics[width=\textwidth]{figures/chi2_playground_log.pdf}
	\caption{chi2 comparison log transform}
	\label{fig:chi2_log_comparison}
\end{figure}

\subsection{Sqrt-Transformed Chi-squared distribution}

we transform the distribution with $g(x) = \sqrt{x}$, i.e. $x(y) = g^{-1}(x) = y^2$. The new pdf becomes

TODO: nasty transform as in beginning.

\begin{align}
\mathcal{C}_{Y_sqrt}(y,k) &= \frac{1}{2^{k/2}\Gamma(k/2)}  y^{2(k/2 -1)} \exp(-y^2/2) \cdot 2y \\
		 &= \frac{1}{2^{k/2}\Gamma(k/2)}  y^{k} \exp(-\frac{y^2}{2}) \\
		 &= \exp \left[k\log(y) - \frac{y^2}{2} - \log(2^{k/2}\Gamma(k/2))\right]
\end{align}


meaning $h(y) = 1, \phi(y)=(\log(y), y^2), w=(k, 1/2)$ and $Z(k) =  \log(2^{k/2}\Gamma(k/2))$. 

\subsubsection{Laplace approximation of the sqrt-transformed Chi-squared distribution}

\begin{align*}
\text{log-pdf: } &(k\log(x) - \frac{x^2}{2} - \log(2^{k/2}\Gamma(k/2)) \\
\text{1st derivative: }&  \frac{k}{x} -x \\
\text{mode: }& \frac{k}{x} -x = 0 \Leftrightarrow x = \sqrt{k}\\
\text{2nd derivative: }&  -\frac{k}{x^2} - 1\\
\text{insert mode: }& -\frac{k}{k}-1\\
\text{invert and times -1: }&\sigma^2 = 1/2
\end{align*}

\subsubsection{The Bridge for sqrt-transform}

\begin{align}
\mu &= \sqrt{k} \\
\sigma^2 &= 1/2 \\
k &= \mu^2
\end{align}

TODO:THE BRIDGE BACK LOOKS A BIT WEIRD\\

\begin{figure}[!htb]
	\centering
	\includegraphics[width=\textwidth]{figures/chi2_playground_sqrt.pdf}
	\caption{chi2 sqrt comparison}
	\label{fig:chi2_sqrt_comparison}
\end{figure}

\subsection{Standard Beta Distribution}

The pdf of the Beta distribution in the standard basis is

\begin{subequations}
\begin{align}
	\mathcal{B}(x, \alpha, \beta) &= \frac{x^{(\alpha - 1)} \cdot (1-x)^{(\beta-1)}}{B(\alpha, \beta)} \\
	 &=  \exp\left[(\alpha-1) \log(x) + (\beta-1)\log(1-x) - \log(B(\alpha,\beta)))\right]\\
	&= \frac{1}{x(1-x)}\exp\left[\alpha\log(x) + \beta\log(1-x) - \log(B(\alpha,\beta)))\right]
\end{align}
\end{subequations}

with $h(x) = \frac{1}{x(1-x)}, \phi(x)=(\log(x), \log(1-x), w = (\alpha, \beta)$ and $Z(\alpha, \beta) = \log(B(\alpha,\beta)))$ where $B(\alpha, \beta) = \frac{\Gamma(\alpha)\Gamma(\beta)}{\Gamma(\alpha + \beta)}$ and $\Gamma(x)$ is the Gamma function.

\subsubsection{Laplace approximation of the standard Beta distribution}

To get the Laplace approximation we need the mode and Hessian. To get the mode we use the first derivative of the log-pdf and set it to zero. To get the Covariance we use the Hessian at the mode, multiply it with -1 and invert it. 

\begin{align*}
\text{log-pdf: } &\log\left( \frac{x^{(\alpha - 1)} \cdot (1-x)^{(\beta-1)}}{B(\alpha, \beta)} \right) \\
&= (\alpha-1) \log(x) + (\beta-1)\log(1-x) - \log(B(\alpha,\beta)))\\
\text{1st derivative: }& \frac{(\alpha-1)}{x} - \frac{(\beta-1)}{1-x}  \\
\text{mode: }& \frac{(\alpha-1)}{x} - \frac{(\beta-1)}{1-x}  = 0 \Leftrightarrow x = \frac{\alpha-1}{\alpha + \beta - 2} \\
\text{2nd derivative: }& \frac{\alpha -1}{x^2} + \frac{\beta - 1}{(1 - x)^2} \\
\text{insert mode: }& \frac{\alpha -1}{\frac{\alpha-1}{\alpha + \beta - 2}^2} + \frac{\beta - 1}{(1 - \frac{\alpha-1}{\alpha + \beta - 2})^2} = \frac{(\alpha + \beta - 2)^3}{(\alpha-1)(\beta-1)}\\
\text{invert: }& \frac{(\alpha -1)(\beta-1)}{(\alpha + \beta - 2)^3}
\end{align*}

The Beta distribution in standard basis is therefore approximated by $N(\mu = \frac{\alpha-1}{\alpha + \beta - 2}, \sigma^2 = \frac{(\alpha -1)(\beta-1)}{(\alpha + \beta - 2)^3})$.

\subsection{Logit-Transform of the Beta distribution}

We transform the Beta distribution using $g(x) = \log(\frac{x}{1-x})$. Therefore $x(y) = g^{-1}(y) = \sigma(y) = \frac{1}{1+ \exp(-y)}$. This yields the following pdf

\begin{subequations}
\begin{align}
	\mathcal{B}_{Y_\text{logit}}(y, \alpha, \beta) &= \frac{1}{\sigma(y)(1-\sigma(y))}\exp\left[\alpha\log(\sigma(y))) + \beta\log(1-\sigma(y)) - \log(B(\alpha,\beta)))\right] \cdot (\sigma(y)(1-\sigma(y)) \\
	&= \exp\left[\alpha\log(\sigma(y)) + \beta\log(1-\sigma(y)) - \log(B(\alpha,\beta)))\right]
	\label{eq:beta_logit_trans_pdf}
\end{align}
\end{subequations}

Which has $h(y) = 1, \phi(y)=(\log(\sigma(y)), \log(1-\sigma(y)), w = (\alpha, \beta)$ and $Z(\alpha, \beta) = \log(B(\alpha, \beta))$.

\subsubsection{Laplace approximation of the logit transformed Beta distribution}

\begin{align*}
\text{log-pdf: } &\log\left( \frac{\sigma(y)^{\alpha} \cdot (1 - \sigma(y)^{\beta})}{B(\alpha, \beta)} \right) \\
&= \alpha \log(\sigma(y)) + \beta \log(1 - \sigma(x)) - \log(B(\alpha, \beta))\\
\text{1st derivative: }&  \alpha (1 - \sigma(y)) - \beta \sigma(y)\\
\text{mode: }& \alpha (1 - \sigma(y)) - \beta \sigma(y) = 0 \Leftrightarrow y = \log(\frac{\alpha}{\beta}) \\
\text{2nd derivative: }& (\alpha + \beta)\sigma(y)(1 - \sigma(y))  \\
\text{insert mode: }& (\alpha + \beta)\sigma(-\log(\frac{\beta}{\alpha}))(1 - \sigma(-\log(\frac{\beta}{\alpha}))) = \frac{\alpha\beta}{\alpha + \beta}  \\
\text{invert: }& \frac{\alpha + \beta}{\alpha \beta}
\end{align*}


The Laplace approximation is therefore given by $\mathcal{N}(y; \mu=\log(\frac{\alpha}{\beta}), \sigma^2 = \frac{\alpha + \beta}{\alpha \beta})$.

\subsubsection{The Bridge for the logit transformation}

\begin{subequations}
\begin{align}
	\mu &=\log(\frac{\alpha}{\beta} \\
	\sigma^2 &= \frac{\alpha + \beta}{\alpha \beta} \\
	\alpha &= \frac{\exp(\mu) + 1}{\sigma^2} \\
	\alpha &= \frac{\exp(-\mu) + 1}{\sigma^2} 
\end{align}
\end{subequations}

\begin{figure}[!htb]
	\centering
	\includegraphics[width=\textwidth]{figures/beta_logit_bridge.pdf}
	\caption{beta logit bridge}
	\label{fig:beta_logit_bridge}
\end{figure}

\section{Dirichlet Distribution}

\subsection{Standard Dirichlet distribution}

The pdf for the Dirichlet distribution in the standard basis (i.e. probability space) is

\begin{align}
\mathrm{Dir}(\vpi | \valpha) &= \frac{\Gamma \left( \sum_{k=1}^K \alpha_k \right)}{\prod_{k=1}^K \Gamma(\alpha_k)} \prod_{k=1}^K \pi_k^{\alpha_k-1} \\
&= \frac{1}{B(\alpha)} \prod_{k=1}^K \pi_k^{\alpha_k-1} \\
&= \exp\left[\sum_k(\alpha_k-1)\log(\pi_k) - \log(B(\alpha))\right] \\
&= \frac{1}{\prod_k \pi_k} \exp\left[\sum_k\alpha_k\log(\pi_k) - \log(B(\alpha))\right] \\
\end{align}

with sufficient statistics $\phi(x_i) = \log(x_i)$, natural parameters $w_i=\alpha_i$, base measure $h(x) = \prod_k x_k$, and partition function $Z(w) = \log(B(\alpha))$.

\subsubsection{Laplace approximation of the standard Dirichlet distribution}

\begin{align*}
\text{log-pdf: } & f = \sum_k\alpha_k\log(\pi_k) - \log(B(\alpha)) \\
\text{1st derivative: }&  \frac{\partial f}{\partial x_i} =  \frac{\alpha_i-1}{x_i}\\
\text{mode: }& x_i = \frac{(\alpha_i - 1)}{\sum_k \alpha_k} \\
\text{2nd derivative: }&  \frac{\partial^2 f}{\partial x_i \partial x_j} = - \delta_{ij} \frac{(\alpha_i - 1)}{x_i^2} \\
\text{insert mode: }& - \delta_{ij}\frac{(\sum_k \alpha_k)^2}{(\alpha_i - 1)} \\
\text{invert and times -1: }&\Sigma_{ij} = \delta_{ij} \frac{\alpha_i - 1}{(\sum_k \alpha_k)^2}
\end{align*}

Which yields a diagonal Covariance matrix for the Laplace approximation.

\subsection{Softmax-Transform of the Dirichlet distribution}

We aim to transform the basis of this distribution from base $\vy$ via the softmax transform to be in the new base $\pi$:

\begin{equation}\label{eq:softmax}
\pi_k(\vy) := \frac{\exp(y_k)}{\sum_{l=1}^K \exp(y_l)} \, ,
\end{equation}

%Usually, to transform the basis we would need the inverse transformation $G^{-1}(\vy)$ as described in TODO:sectionhref. However, the softmax does not have an analytic inverse. Therefore, David JC MacKay uses the following trick. Assume we know that the distribution in the transformed basis is:

TODO:David J. MacKay has already done a transformation for the determinant but in a slightly different fashion. 

The softmax transform has no analytic inverse $\pi_k^{-1}(y)$ but it is not necessary for our computation since we assume $\pi(y)$ to be the inverse transformation already (i.e. $g^{-1}(y)$). However, our transformation is from a variable in $\mathbb{R}^d$ (which has $d$ degrees of freedom) to a variable that is in $\mathbb{P}^d$ (which has $d-1$ degrees of freedom). To account for the difference in size of the two spaces we create a helper variable for the transformation as described in the following.

We want to transform $K$ variables $y_i$ from $\mathbb{R}^d$ to $\tau_i = \exp(y_i)$. For $\tau_i$ to be equal to $\pi_i$ we need to ensure that it sums to 1, $u = \sum_i \tau_i = 1$. With the helper-variable $u$ our variable transform $g(\pi, u)$ becomes

\begin{align}
	p_{y,u}(\pi(y), u) &= p_{\pi, u}(\pi(y), u) |\det g(\pi(y), u)|
\end{align}

with 

\begin{align}
	|\det g(\pi(y), u)| &= \left|\det\begin{pmatrix}
		\frac{\partial \tau_1}{\partial y_1} & \cdots & \frac{\partial \tau_k}{\partial y_1} & \frac{\partial u}{\partial y_1} \\
		\vdots & \ddots & \vdots & \vdots \\
		\frac{\partial \tau_1}{\partial y_k} & \cdots & \frac{\partial \tau_k}{\partial y_k} & \frac{\partial u}{\partial y_k} \\
		\frac{\partial \tau_1}{\partial u} & \cdots & \frac{\partial \tau_k}{\partial u} & \frac{\partial u}{\partial u}
	\end{pmatrix} \right| \\
	&= \left|\det\begin{pmatrix}
	\tau_1 = \exp(y_1) & \cdots & 0 & \tau_1 \\
	\vdots & \ddots & \vdots & \vdots \\
	0 & \cdots & \tau_k & \tau_k \\
	0 & \cdots & 0 & 1
	\end{pmatrix} \right| \\
	&= \prod_i^{K} \tau_i 
\end{align}

To get $p_y(\pi(y))$ we have to integrate out $u$. 

\begin{align}
	p_y(\pi(y)) &=\int_{-\infty}^{\infty} p_{\pi, u}(\pi(y), u) |\det g(\pi, u)| du \\
	&= \int_{-\infty}^{\infty} p_{\pi}(\pi(y))  \prod_i^{K} \tau_i  du \\
	&= p_{\pi}(\pi(y)) \int_{0}^{\infty}  \prod_i^{K} \tau_i \delta(u-1) du \\
	&= p_{\pi}(\pi(y)) \int_{0}^{\infty}  \prod_i^{K} \tau_i \frac{u}{u} \delta(u-1) du \\
	&= p_{\pi}(\pi(y)) \int_{0}^{\infty}  \underbrace{\prod_i^{K} \pi_i u}_{f(u)} \delta(u-1) du \\
	&= p_{\pi}(\pi(y)) \cdot \prod_i^{K} \pi_i(y) \\
	&= \frac{1}{\prod_k \pi_k(y)} \exp\left[\sum_k\alpha_k\log(\pi_k(y)) - \log(B(\alpha))\right] \prod_k^{K} \pi_k(y) \\
	&= \exp\left[\sum_k\alpha_k\log(\pi(y_k)) - \log(B(\alpha))\right]
\end{align}

where we used the fact that $u > 0$ since its a sum of exponentials, $\frac{\tau_i(y)}{u} = \pi_i(y)$, and multiplied with $\delta(u-1)$ since this transformation is only valid if $\sum_i \tau_i = u = 1$ because otherwise it is not a probability space. Additionally, we use

\begin{equation}
	\int_{-\infty}^{\infty} f(x)\delta(x-t)dx = f(t)
\end{equation}

which is known as the shifting property or sampling property of the Dirac delta function $\delta$. Using all of the above we get the pdf of the Dirichlet distribution in the new basis $\vy$: 

\begin{align}\label{eq:dirichlet_softmax}
\mathrm{Dir}_{\vy}(\vpi(\vy) | \valpha) &:= \frac{\Gamma \left( \sum_{k=1}^K \alpha_k \right)}{\prod_{k=1}^K \Gamma(\alpha_k)} \prod_{k=1}^K \pi_k(\vy)^{\alpha_k}  \\
&= \exp\left[\sum_k\alpha_k\log(\pi(y_k)) - \log(B(\alpha))\right]
\end{align}

with sufficient statistics $\phi(y_i) = \log(\pi_i(y))$, natural parameters $w_i = \alpha_i$, base measure $h(y) = 1$ and normalizing constant $Z = \log(B(\alpha))$.

\subsubsection{Laplace approximation of the softmax-transformed Dirichlet distribution}

TODO: mention that this has been done by Philipp in his PhD thesis.

Through the figures of the 1D Dirichlet approximation in the main paper we have already established that the mode of the Dirichlet lies at the mean of the Gaussian distribution and therefore $\vpi(\vy) = \frac{\mathbf{\alpha}}{\sum_i \alpha_i}$. Additionally, the elements of $\vy$ must sum to zero. These two constraints combined yield only one possible solution for $\vmu$.

\begin{equation}
\mu_k = \log \alpha_k  - \frac{1}{K} \sum_{l=1}^{K} \log \alpha_l
\label{eq:mu_k}
\end{equation}

Calculating the covariance matrix $\vSigma$ is more complicated but layed out in the following. The logarithm of the Dirichlet is, up to additive constants

\begin{equation}
\log p_y(y|\alpha) = \sum_k \alpha_k \pi_k 
\end{equation}

Using $\pi_k$ as the softmax of $\vy$ as shown in Equation \ref{eq:softmax} we can find the elements of the Hessian $\vL$

\begin{equation}
L_{kl} = \hat{\alpha}(\delta_{kl}\hat{\pi_k} - \hat{\pi_k} \hat{\pi_l})
\end{equation}

where $\hat{\valpha} := \sum_k \alpha_k$ and $\hat{\vpi} = \frac{\alpha_k}{\hat{\alpha}}$ for the value
of $\vpi$ at the mode. Analytically inverting $\vL$ is done via a lengthy derivation using the fact that we can write $\vL = \mA + \mX\mB\mX^\top$ and inverting it with the Schur-complement. This process results in the inverse of the Hessian

%TODO: can I just omit so much of the derivation? I mean it's in Philipp Hennigs thesis. People can just download it. 

\begin{equation}
L_{kl}^{-1} = \delta_{kl} \frac{1}{\alpha_k} - \frac{1}{K} \left[\frac{1}{\alpha_k} + \frac{1}{\alpha_l} - \frac{1}{K}\left(\sum_u^K \frac{1}{\alpha_u}\right) \right]
\end{equation}

We are mostly interested in the diagonal elements, since we desire a sparse encoding for computational reasons and we otherwise needed to map a $K \times K$ covariance matrix to a $K\times 1$ Dirichlet parameter vector which would be a very overdetermined mapping. Note that $K$ is a scalar not a matrix. The diagonal elements of $\vSigma = \vL^{-1}$ can be calculated as

\begin{equation}
\label{eq:Hessian_diag}
\Sigma_{kk} = \frac{1}{\alpha_k} \left(1 - \frac{2}{K}\right)  + \frac{1}{K^2} \sum_{l}^{k} \frac{1}{\alpha_l}.
\end{equation}

To invert this mapping we transform Equation \ref{eq:mu_k} to 

\begin{equation}
\label{eq:reform_mu_k}
\alpha_k = e^{\mu_k} \prod_l^{K} \alpha_l^{1/K}
\end{equation}

by applying the logarithm and re-ordering some parts. Inserting this into Equation \ref{eq:Hessian_diag} and re-arranging yields

\begin{equation}
\prod_l^K \alpha_l^{1/K} = \frac{1}{\vSigma_{kk}} \left[e^{-\mu}\left(1 - \frac{2}{K}\right)  + \frac{1}{K^2} \sum_u^K e^{-\mu_u} \right]
\end{equation}

which can be re-inserted into Equation \ref{eq:reform_mu_k} to give

\begin{equation}
\label{eq:mapping_alpha}
\alpha_k = \frac{1}{\Sigma_kk} \left(1 - \frac{2}{K} + \frac{e^{-\mu_k}}{K^2} \sum_l^K e^{-\mu_k} \right)
\end{equation}

which is the final mapping. With Equations \ref{eq:mu_k} and \ref{eq:Hessian_diag} we are able to map from Dirichlet to Gaussian and with Equation \ref{eq:mapping_alpha} we are able to map the inverse direction. 

\subsubsection{The Bridge for the inverse-softmax transform}

In summary we get the following forward and backward transformations between $\vy \in \mathbb{R}^d$ and $\pi \in \mathbb{P}^d$.

\begin{align}
	\mu_k &= \log \alpha_k  - \frac{1}{K} \sum_{l=1}^{K} \log \alpha_l \, , \label{eq:mubridge}\\
	\Sigma_{k\ell} &= \delta_{k\ell}\frac{1}{\alpha_k} - \frac{1}{K}\left[\frac{1}{\alpha_k} + \frac{1}{\alpha_\ell} - \frac{1}{K}\sum_{u=1} ^K \frac{1}{\alpha_u} \right].
	\label{eq:Sigmabridge} 
\end{align}

The corresponding derivations require care because the Gaussian parameter space is evidently larger than that of the Dirichlet and not fully identified by the transformation.
A pseudo-inverse of this map was provided by \citet{KernelTopicModels2012}. It maps the Gaussian parameters to those of the Dirichlet as

\begin{equation} \label{eq:alpha_transform}
	\alpha_k = \frac{1}{\Sigma_{kk}}\left(1 - \frac{2}{K} + \frac{e^{\mu_k}}{K^2}\sum_{l=1}^K e^{-\mu_l} \right) \,
\end{equation}

\section{Wishart Distribution}

\subsection{Interlude: Box-product and Kronecker-product}

Kronecker-product: $A \otimes B \in \mathbb{R}^{(m_1m_2) \times (n_1n_2)}$ is defined by $(A \otimes B)_{(i - 1)m_2+j,(k - 1)n_2+l} = a_{il}b_{jk} = (A \otimes B)_{(ij)(kl)}$.

Box-product: $A \boxtimes B \in \mathbb{R}^{(m_1m_2) \times (n_1n_2)}$ is defined by $(A \boxtimes B)_{(i - 1)m_2+j,(k - 1)n_1+l} = a_{ik}b_{jl} = (A \boxtimes B)_{(ij)(kl)}$.

I found this box-product only in two sources, one of which is this: \url{https://researcher.watson.ibm.com/researcher/files/us-pederao/ADTalk.pdf} but it generally seems to be very helpful for matrix derivations with transposed matrices.

\subsection{Standard Wishart distribution}

the pdf of the Wishart is

\begin{equation}
f(X; n,p,V) = \frac{1}{2^{np/2} \left|{\mathbf V}\right|^{n/2} \Gamma_p\left(\frac {n}{2}\right ) }{\left|\mathbf{X}\right|}^{(n-p-1)/2} e^{-(1/2)\operatorname{tr}({\mathbf V}^{-1}\mathbf{X})}
\label{eq:wishart_pdf}
\end{equation}

which can be written as

\begin{equation}
f(X; n,p,V) = \exp \left[(n-p-1)/2 \log(|X|) -(1/2)\operatorname{tr}({\mathbf V}^{-1}\mathbf{X}) - \log\left(2^{np/2} \left|{\mathbf V}\right|^{n/2} \Gamma_p\left(\frac {n}{2}\right )\right) \right]
\end{equation}

with $T=(\log(X), X), \eta=((n-p-1)/2, V^{-1})$ and $A(n,p,V)=\log\left(2^{np/2} \left|{\mathbf V}\right|^{n/2} \Gamma_p\left(\frac {n}{2}\right )\right)$

\subsubsection{Laplace Approximation of the standard Wishart distribution}

Using $\frac{\partial \det(X)}{\partial X} = \det(X)(X^{-1})^\top$ and $\frac{\partial}{\partial X} Tr(AX^\top) = A$ we can calculate the mode by setting the first derivative of the log-pdf to zero

\begin{align*}
\frac{\partial \log f(X; n,p,V)}{\partial X} &= \frac{(n-p-1)\det(X)(X^{-\top})}{2\det(X)} - \frac{V^{-1}}{2} \\
\Rightarrow 0 &= \frac{(n-p-1)X^{-1}}{2} - \frac{V^{-1}}{2} \\
\Leftrightarrow  \frac{(n-p-1)X^{-1}}{2} &= \frac{V^{-1}}{2} \\
\Leftrightarrow X &= (n-p-1)V
\end{align*}

Using the fact that $\frac{\partial X^{-T}}{\partial X} = X^{-T} \boxtimes X^{-1}$ where $\boxtimes$ is the Box-product we compute the second derivative as

\begin{align*}
\frac{\partial^2 \log f(X; n,p,V)}{\partial^2 X} &= -\frac{(n-p-1)}{2} X^{-\top} \boxtimes X^{-1}
\end{align*}

Using $(\alpha A)^{-1} = \alpha^{-1}A^{-1}$, the linearity of the Kronecker product to pull out scalars and $X^{-1} \boxtimes X^{-1} = (X \boxtimes X)^{-1}$ to insert the mode and invert we get:

\begin{align*}
-\frac{(n-p-1)}{2} X^{-1} \boxtimes X^{-1} &= -\frac{(n-p-1)}{2} \frac{1}{(n-p-1)} V^{-1} \otimes \frac{1}{(n-p-1)} V^{-1} \\
&= -\frac{1}{2(n-p-1)}(V \boxtimes V)^{-1} \\
\Rightarrow \Sigma &= 2(n-p-1)(V \boxtimes V)
\end{align*}

In summary, the Laplace approximation of a Wishart distribution in the standard basis is $\mathcal{N}\left(X; (n-p-1)V, 2(n-p-1)(V \boxtimes V)\right)$, where the representation of the symmetric positive definite matrices has been changed from $\mathbb{R}^{n\times n}$ to $\mathbb{R}^{n^2}$.

\subsection{Logm-Transformed Wishart distribution}

we transform the distribution with $g(X) = \text{logm}(X)$, i.e. $g^{-1}(X) = \text{expm}(X)$, where $\text{expm}(X)$ is the matrix exponential and $\text{logm}(X)$ is the matrix logarithm of $X$. The new pdf becomes

\begin{align*}
f(X; n, p, V) &= \frac{1}{2^{np/2} \left|{\mathbf V}\right|^{n/2} \Gamma_p\left(\frac {n}{2}\right ) }{\left|\mathbf{\operatorname{expm}X}\right|}^{(n-p-1)/2} e^{-(1/2)\operatorname{tr}({\mathbf V}^{-1}\mathbf{\operatorname{expm}X})} \cdot |\operatorname{expm}X| \\ 
&= \frac{1}{2^{np/2} \left|{\mathbf V}\right|^{n/2} \Gamma_p\left(\frac {n}{2}\right ) }{\left|\mathbf{\operatorname{expm}X}\right|}^{(n-p+1)/2} e^{-(1/2)\operatorname{tr}({\mathbf V}^{-1}\mathbf{\operatorname{expm}X})} \\ 
&= \exp \left[C + \frac{(n-p+1)}{2} \log(\left|\mathbf{\operatorname{expm}X}\right|)  - \frac{1}{2}\operatorname{tr}({\mathbf V}^{-1}\mathbf{\operatorname{expm}X}) \right]
\end{align*}

with blablabla as expfam values. 

\subsubsection{Laplace Approximation of the logm-transformed Wishart distribution}

To compute the first derivative we use the following

\begin{align}
	\frac{\partial \log(\det(\operatorname{expm}(X)))}{\partial X} 
	&= \frac{\partial \log(\det(\operatorname{expm}(X)))}{\partial \det(\operatorname{expm}(X))} \cdot \frac{\partial \det(\operatorname{expm}(X))}{\partial \operatorname{expm}(X)} \cdot \frac{\partial \operatorname{expm}(X)}{\partial X} \\
	&= \frac{1}{\det(\operatorname{expm}(X))} \cdot \det(\operatorname{expm}(X)) \operatorname{expm}(X)^{-\top} \cdot \operatorname{expm}(X) \\
	&= I_p
\end{align}

where $I_p$ is the identity matrix of size $p$ and we use the fact that the matrix logaritm of a symmetric matrix is symmetric, implying $\operatorname{expm}(X)^{-\top} = \operatorname{expm}(X)^{-1}$. With this we get the 

\begin{align}
	\frac{\partial \log W_{log}}{\partial X} &= \frac{\partial}{\partial X} \left[C + \frac{(n-p+1)}{2} \log(\left|\mathbf{\operatorname{expm}X}\right|)  - \frac{1}{2}\operatorname{tr}({\mathbf V}^{-1}\mathbf{\operatorname{expm}X}) \right] \\
	&=  \frac{(n-p+1)}{2} I_p - \frac{1}{2}V^{-1}\mathbf{\operatorname{expm}X} 
\end{align}

By setting this to zero we get a mode of

\begin{align}
	0 &=  \frac{(n-p+1)}{2} I_p - \frac{1}{2}V^{-1}\mathbf{\operatorname{expm}X} \\
	\Leftrightarrow (n-p+1)I_p &= V^{-1}\mathbf{\operatorname{expm}X} \\
	\Leftrightarrow X &= \operatorname{logm}((n-p+1)V)
\end{align}

For the second derivative we use the fact that

\begin{align}
	\frac{\partial (B\operatorname{expm}(X))_{kl}}{\partial X_{ij}} &= ... \\
	\Leftrightarrow \frac{\partial B\operatorname{expm}(X)}{\partial X} &= \left(B\operatorname{expm}(X) \otimes I_p\right)
\end{align}

yielding

\begin{align}
	\frac{\partial^2 \log W_{log}}{\partial^2 X} &= \frac{\partial \log W_{log}}{\partial X}  \frac{(n-p+1)}{2} I_p - \frac{1}{2}V^{-1}\mathbf{\operatorname{expm}X} \\
	&= -\frac{1}{2}(V^{-1}\mathbf{\operatorname{expm}X} \otimes I_p) \\
	&\overset{\text{mode}}{\Rightarrow} -\frac{1}{2}((n-p+1)V^{-1}V \otimes I_p) \\
	&= -\frac{(n-p+1)}{2} (I_p \otimes I_p)\\
	\Leftrightarrow \Sigma &= \frac{2}{n-p+1} I_{p^2}
\end{align}

where $I_{p^2}$ is an Identity matrix of size $p^2$. 

\subsubsection{The Bridge for logm-tranform}

$\mu$ and $\Sigma$ are already given by the Laplace approximation. Inverting the mode yields an estimate for $V$. 
\begin{align}
	\mu &= \operatorname{logm}((n-p+1)V) \Leftrightarrow \operatorname{expm}(\mu) = (n-p+1)V \Leftrightarrow V = \frac{\operatorname{expm}(\mu)}{(n-p+1)}
\end{align}

where $\mu$ and $V$ are reshaped to a matrix of size $p\times p$. In summary this yields

\begin{align}
	\mu &= \operatorname{logm}((n-p+1)V) \\
	\Sigma &= \frac{2}{n-p+1} I_{p^2} \\
	V &= \frac{\operatorname{expm}(\mu)}{(n-p+1)}
\end{align} 

\subsection{Sqrtm-Transformed Wishart distribution}

we transform the distribution with $g(X) = \text{sqrtm}(X) = X^{\frac{1}{2}}$, i.e. $g^{-1}(X) = X^2$, where $\text{sqrtm}(X)$ is the square root of the matrix. The new pdf becomes

\begin{align}
f_t(X; n,p,V) &= \frac{1}{2^{np/2} \left|{\mathbf V}\right|^{n/2} \Gamma_p\left(\frac {n}{2}\right ) }{\left|\mathbf{X^2}\right|}^{(n-p-1)/2} e^{-(1/2)\operatorname{tr}({\mathbf V}^{-1}\mathbf{X^2})} \cdot |2X| \\ 
&= \frac{1}{2^{np/2} \left|{\mathbf V}\right|^{n/2} \Gamma_p\left(\frac {n}{2}\right ) }{\left|\mathbf{X}\right|}^{2(n-p-1)/2} e^{-(1/2)\operatorname{tr}({\mathbf V}^{-1}\mathbf{X^2})} \cdot 2^p|X| \\
&= \frac{1}{2^{np/2} \left|{\mathbf V}\right|^{n/2} \Gamma_p\left(\frac {n}{2}\right ) }{\left|\mathbf{X}\right|}^{(n-p)} e^{-(1/2)\operatorname{tr}({\mathbf V}^{-1}\mathbf{X^2})} 
\label{eq:wishart_trans_sqrtm_pdf}
\end{align}

where we drop the $2^p$ in line (4) because there are $2^p$ matrices that are a root of $X$ (I have explained this in more detailed in another version of the current draft). This can be rewritten as 

\begin{equation}
f_t(X; n,p,V) = \exp \left[(n-p) \log(|X|) - (1/2)\operatorname{tr}({\mathbf V}^{-1}\mathbf{X^2}) - \log\left(2^{np/2} \left|{\mathbf V}\right|^{n/2} \Gamma_p\left(\frac {n}{2}\right )\right)\right]
\end{equation}

with $T=(\log(X), X^2), \eta=((n-p), V^{-1})$ and $A(n,p,V)=\log\left(2^{np/2} \left|{\mathbf V}\right|^{n/2} \Gamma_p\left(\frac {n}{2}\right )\right)$


\subsubsection{Laplace Approximation of the sqrtm-transformed Wishart distribution}

Using $\frac{\partial \det(X)}{\partial X} = \det(X)(X^{-1})^\top$ and $\frac{\partial}{\partial X} Tr(AX^2) = (AX + XA)^T$ we can calculate the mode by setting the first derivative of the log-pdf to zero

\begin{align*}
\frac{\partial \log f_t(X; n,p,V)}{\partial X} &= \frac{(n-p)\det(X)(X^{-\top})}{\det(X)} - \frac{(V^{-1}X + XV^{-1})^\top}{2} \\
\Rightarrow 0 &= (n-p)X^{-\top} - \frac{(V^{-1}X + XV^{-1})^\top}{2} \\
\Leftrightarrow  (n-p)X^{-\top} &=  \frac{(V^{-1}X + XV^{-1})^\top}{2} \\
\Leftrightarrow  (n-p)X^{-1} &=  \frac{(V^{-1}X + XV^{-1})}{2} \\
\Leftrightarrow X &= ???
\end{align*}

THIS IS WHERE SOLVING FOR X GETS COMPLICATED. Maybe we can rewrite it with Kronecker products and vectorized matrices like for the Sylvester equation and these laws \url{https://en.wikipedia.org/wiki/Vectorization_(mathematics)#Compatibility_with_Kronecker_products}.

So far I have found the following relationships that don't get me any further to the solution of $X$:

\begin{align*}
(n-p)X^{-1} &=  \frac{(V^{-1}X + XV^{-1})}{2} \\
\Leftrightarrow C &= BXX + XBX \\
\Leftrightarrow C &= (I_p \otimes BX)\vec{X} + (B^TX^T \otimes I_p)\vec{X} \\
\Leftrightarrow C &= (B^TX^T \oplus BX)\vec{X} \\
\Leftrightarrow C &= (BX \oplus BX)\vec{X}
\end{align*}

Computing the second derivative by using $\frac{\partial}{\partial X}X^{-1} = -X^{-1} \otimes X^{-1}$, $\frac{\partial}{\partial X} (AX + XA)^\top = I \boxtimes A + A \boxtimes I$:

\begin{align*}
\frac{\partial^2 \log f_t(X; n,p,V)}{\partial^2 X} &= \frac{\partial}{\partial X} \left[(n-p)X^{-\top} - \frac{(V^{-1}X + XV^{-1})^\top}{2}\right] \\
&= -(n-p) (X^{-\top} \otimes X^{-1}) + \frac{1}{2}(I_p \boxtimes V^{-1} + V^{-1} \boxtimes I_p) \\
&\Rightarrow -(n-p) \left[\sqrt{\frac{1}{(n-p)}}V^{-\frac{1}{2}} \otimes \sqrt{\frac{1}{(n-p)}}V^{-\frac{1}{2}}\right] + \frac{1}{2}\left[I_p \boxtimes V^{-1} + V^{-1} \boxtimes I_p\right] \\
&= - \left(V^{-\frac{1}{2}} \otimes V^{-1}\right) + \frac{1}{2}\left[I_p \boxtimes V^{-1} + V^{-1} \boxtimes I_p\right] \\
&\overset{\cdot -1}{\Rightarrow} \left(V^{-\frac{1}{2}} \otimes V^{-\frac{1}{2}}\right) - \frac{1}{2}\left[I_p \boxtimes V^{-1} + V^{-1} \boxtimes I_p\right] \\
\Rightarrow \Sigma &= \left[V^{-\frac{1}{2}} \otimes V^{-\frac{1}{2}} - \frac{1}{2}\left(I_p \boxtimes V^{-1} + V^{-1} \boxtimes I_p\right)\right]^{-1} \\
\end{align*}

We can assume that $X$ is symmetric because the root of a symmetric positive definite matrix is symmetric. 
This is the solution if we assume that the mode is given by $X = \sqrt{(n-p)}V^{\frac{1}{2}}$. 

\subsubsection{The Bridge for sqrtm-tranform}

we use $\mu =  ((n-p)V)^{\frac{1}{2}} \Leftrightarrow \mu^2 = (n-p)V \Leftrightarrow V = \frac{\mu^2}{(n-p)}$. Remember that $\mu$ is reshaped to be the same size as $V$ even though we usually think of it in vector-form. 

\begin{align}
	\mu &=  ((n-p)V)^{\frac{1}{2}} \\
	\Sigma &= (V \otimes V)^{\frac{1}{2}} - \tilde{V^{-1}} \\
	V &= \frac{\mu^2}{(n-p)}
\end{align}

QUESTION: DO WE ASSUME WE KNOW THE $n$? 

\begin{figure}[!htb]
	\centering
	\includegraphics[width=\textwidth]{figures/wishart_playground_sqrtm.pdf}
	\caption{wishart comparison for sqrtm}
	\label{fig:wishart_comparison}
\end{figure}





\section{Inverse Wishart Distribution}

TODO: see above

%\begin{figure}[!htb]
%	\centering
%	\includegraphics[width=\textwidth]{inverse_wishart_playground.pdf}
%	\caption{inverse wishart comparison}
%	\label{fig:inverse_wishart_comparison}
%\end{figure}

\end{document}

