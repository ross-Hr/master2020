\documentclass{article}

% if you need to pass options to natbib, use, e.g.:
%     \PassOptionsToPackage{numbers, compress}{natbib}
% before loading neurips_2018

% ready for submission
% \usepackage{neurips_2018}

% to compile a preprint version, e.g., for submission to arXiv, add add the
% [preprint] option:
%     \usepackage[preprint]{neurips_2018}

% to compile a camera-ready version, add the [final] option, e.g.:
     \usepackage[final]{nips_2018}

% to avoid loading the natbib package, add option nonatbib:
%     \usepackage[nonatbib]{neurips_2018}

\usepackage[utf8]{inputenc} % allow utf-8 input
\usepackage[T1]{fontenc}    % use 8-bit T1 fonts
\usepackage{hyperref}       % hyperlinks
\usepackage{url}            % simple URL typesetting
\usepackage{booktabs}       % professional-quality tables
\usepackage{amsfonts}       % blackboard math symbols
\usepackage{nicefrac}       % compact symbols for 1/2, etc.
\usepackage{microtype}      % microtypography
\usepackage{graphicx}
\usepackage{subfig}
\usepackage{mathtools,amsthm,amssymb,amsfonts}

\usepackage{arydshln} %for dashed lines
\usepackage{marvosym}

%change caption styles
\usepackage{caption} 
\captionsetup{labelfont=bf, font={small,sf}}

%change margins for the tables in the appendix
\usepackage{chngpage}

% The \author macro works with any number of authors. There are two commands
% used to separate the names and addresses of multiple authors: \And and \AND.
%
% Using \And between authors leaves it to LaTeX to determine where to break the
% lines. Using \AND forces a line break at that point. So, if LaTeX puts 3 of 4
% authors names on the first line, and the last on the second line, try using
% \AND instead of \And before the third author name.

\author{%
  Marius Hobbhahn \\
  Department of Computer Science\\
  University of Tübingen\\
  Tübingen, Germany \\
  \texttt{marius.hobbhahn@gmail.com} \\
  % examples of more authors
  % \And
  % Coauthor \\
  % Affiliation \\
  % Address \\
  % \texttt{email} \\
  % \AND
  % Coauthor \\
  % Affiliation \\
  % Address \\
  % \texttt{email} \\
  % \And
  % Coauthor \\
  % Affiliation \\
  % Address \\
  % \texttt{email} \\
  % \And
  % Coauthor \\
  % Affiliation \\
  % Address \\
  % \texttt{email} \\
}

\begin{document}

\section*{Comments}

\begin{itemize}
	\item The derivation of the Wishart in the standard basis seems to work just fine. If you don't care about it you can skip to 1.2. 
	\item For simplicity I just replaced all $X$ with $X^2$ for the transformed case. Even though this is not clean I found it the easiest way to calculate with. 
	\item The derivations for the first and second derivative for the sqrtm-transformed case are correct. I have double-checked them numerically. The Problem is just to get the mode by solving for $X$. 
\end{itemize}

\textbf{Interlude:} Definition of the Box-product compared to the Kronecker-Product: \\

Kronecker-product: $A \otimes B \in \mathbb{R}^{(m_1m_2) \times (n_1n_2)}$ is defined by $(A \otimes B)_{(i - 1)m_2+j,(k - 1)n_2+l} = a_{il}b_{jk} = (A \otimes B)_{(ij)(kl)}$.

Box-product: $A \boxtimes B \in \mathbb{R}^{(m_1m_2) \times (n_1n_2)}$ is defined by $(A \boxtimes B)_{(i - 1)m_2+j,(k - 1)n_1+l} = a_{ik}b_{jl} = (A \boxtimes B)_{(ij)(kl)}$.

I found this box-product only in two sources, one of which is this: \url{https://researcher.watson.ibm.com/researcher/files/us-pederao/ADTalk.pdf} but it generally seems to be very helpful for matrix derivations with transposed matrices.

\section{Wishart Distribution}

\subsection{Wishart distribution in Standard Basis}

the pdf of the Wishart is

\begin{equation}
f(X; n,p,V) = \frac{1}{2^{np/2} \left|{\mathbf V}\right|^{n/2} \Gamma_p\left(\frac {n}{2}\right ) }{\left|\mathbf{X}\right|}^{(n-p-1)/2} e^{-(1/2)\operatorname{tr}({\mathbf V}^{-1}\mathbf{X})}
\label{eq:wishart_pdf}
\end{equation}

which can be written as

\begin{equation}
f(X; n,p,V) = \exp \left[(n-p-1)/2 \log(|X|) -(1/2)\operatorname{tr}({\mathbf V}^{-1}\mathbf{X}) - \log\left(2^{np/2} \left|{\mathbf V}\right|^{n/2} \Gamma_p\left(\frac {n}{2}\right )\right) \right]
\end{equation}

with $T=(\log(X), X), \eta=((n-p-1)/2, V^{-1})$ and $A(n,p,V)=\log\left(2^{np/2} \left|{\mathbf V}\right|^{n/2} \Gamma_p\left(\frac {n}{2}\right )\right)$

\subsubsection{Laplace Approximation of the standard Wishart distribution}

Using $\frac{\partial \det(X)}{\partial X} = \det(X)(X^{-1})^\top$ and $\frac{\partial}{\partial X} Tr(AX^\top) = A$ we can calculate the mode by setting the first derivative of the log-pdf to zero

\begin{align*}
\frac{\partial \log f(X; n,p,V)}{\partial X} &= \frac{(n-p-1)\det(X)(X^{-\top})}{2\det(X)} - \frac{V^{-1}}{2} \\
\Rightarrow 0 &= \frac{(n-p-1)X^{-1}}{2} - \frac{V^{-1}}{2} \\
\Leftrightarrow  \frac{(n-p-1)X^{-1}}{2} &= \frac{V^{-1}}{2} \\
\Leftrightarrow X &= (n-p-1)V
\end{align*}

Using the fact that $\frac{\partial X^{-T}}{\partial X} = X^{-T} \boxtimes X^{-1}$ where $\boxtimes$ is the Box-product we compute the second derivative as

\begin{align*}
\frac{\partial^2 \log f(X; n,p,V)}{\partial^2 X} &= -\frac{(n-p-1)}{2} X^{-\top} \boxtimes X^{-1}
\end{align*}

Using $(\alpha A)^{-1} = \alpha^{-1}A^{-1}$, the linearity of the Kronecker product to pull out scalars and $X^{-1} \boxtimes X^{-1} = (X \boxtimes X)^{-1}$ to insert the mode and invert we get:

\begin{align*}
-\frac{(n-p-1)}{2} X^{-1} \boxtimes X^{-1} &= -\frac{(n-p-1)}{2} \frac{1}{(n-p-1)} V^{-1} \otimes \frac{1}{(n-p-1)} V^{-1} \\
&= -\frac{1}{2(n-p-1)}(V \boxtimes V)^{-1} \\
\Rightarrow \Sigma &= 2(n-p-1)(V \boxtimes V)
\end{align*}

In summary, the Laplace approximation of a Wishart distribution in the standard basis is $\mathcal{N}\left(X; (n-p-1)V, 2(n-p-1)(V \boxtimes V)\right)$, where the representation of the symmetric positive definite matrices has been changed from $\mathbb{R}^{n\times n}$ to $\mathbb{R}^{n^2}$.


\subsection{Sqrtm-Transformed Wishart distribution}

we transform the distribution with $g(X) = \text{sqrtm}(X) = X^{\frac{1}{2}}$, i.e. $g^{-1}(X) = X^2$, where $\text{sqrtm}(X)$ is the square root of the matrix. The new pdf becomes

\begin{align}
f_t(X; n,p,V) &= \frac{1}{2^{np/2} \left|{\mathbf V}\right|^{n/2} \Gamma_p\left(\frac {n}{2}\right ) }{\left|\mathbf{X^2}\right|}^{(n-p-1)/2} e^{-(1/2)\operatorname{tr}({\mathbf V}^{-1}\mathbf{X^2})} \cdot |2X| \\ 
&= \frac{1}{2^{np/2} \left|{\mathbf V}\right|^{n/2} \Gamma_p\left(\frac {n}{2}\right ) }{\left|\mathbf{X}\right|}^{2(n-p-1)/2} e^{-(1/2)\operatorname{tr}({\mathbf V}^{-1}\mathbf{X^2})} \cdot 2^p|X| \\
&= \frac{1}{2^{np/2} \left|{\mathbf V}\right|^{n/2} \Gamma_p\left(\frac {n}{2}\right ) }{\left|\mathbf{X}\right|}^{(n-p)} e^{-(1/2)\operatorname{tr}({\mathbf V}^{-1}\mathbf{X^2})} 
\label{eq:wishart_trans_sqrtm_pdf}
\end{align}

where we drop the $2^p$ in line (4) because there are $2^p$ matrices that are a root of $X$ (I have explained this in more detailed in another version of the current draft). This can be rewritten as 

\begin{equation}
f_t(X; n,p,V) = \exp \left[(n-p) \log(|X|) - (1/2)\operatorname{tr}({\mathbf V}^{-1}\mathbf{X^2}) - \log\left(2^{np/2} \left|{\mathbf V}\right|^{n/2} \Gamma_p\left(\frac {n}{2}\right )\right)\right]
\end{equation}

with $T=(\log(X), X^2), \eta=((n-p), V^{-1})$ and $A(n,p,V)=\log\left(2^{np/2} \left|{\mathbf V}\right|^{n/2} \Gamma_p\left(\frac {n}{2}\right )\right)$


\subsubsection{Laplace Approximation of the sqrtm-transformed Wishart distribution}

Using $\frac{\partial \det(X)}{\partial X} = \det(X)(X^{-1})^\top$ and $\frac{\partial}{\partial X} Tr(AX^2) = (AX + XA)^T$ we can calculate the mode by setting the first derivative of the log-pdf to zero

\begin{align*}
\frac{\partial \log f_t(X; n,p,V)}{\partial X} &= \frac{(n-p)\det(X)(X^{-\top})}{\det(X)} - \frac{(V^{-1}X + XV^{-1})^\top}{2} \\
\Rightarrow 0 &= (n-p)X^{-\top} - \frac{(V^{-1}X + XV^{-1})^\top}{2} \\
\Leftrightarrow  (n-p)X^{-\top} &=  \frac{(V^{-1}X + XV^{-1})^\top}{2} \\
\Leftrightarrow  (n-p)X^{-1} &=  \frac{(V^{-1}X + XV^{-1})}{2} \\
\Leftrightarrow X &= ???
\end{align*}

THIS IS WHERE SOLVING FOR X GETS COMPLICATED. Maybe we can rewrite it with Kronecker products and vectorized matrices like for the Sylvester equation and these laws \url{https://en.wikipedia.org/wiki/Vectorization_(mathematics)#Compatibility_with_Kronecker_products}.

So far I have found the following relationships that don't get me any further to the solution of $X$:

\begin{align*}
 (n-p)X^{-1} &=  \frac{(V^{-1}X + XV^{-1})}{2} \\
\Leftrightarrow C &= BXX + XBX \\
\Leftrightarrow C &= (I_p \otimes BX)\vec{X} + (B^TX^T \otimes I_p)\vec{X} \\
\Leftrightarrow C &= (B^TX^T \oplus BX)\vec{X} \\
\Leftrightarrow C &= (BX \oplus BX)\vec{X}
\end{align*}

Computing the second derivative by using $\frac{\partial}{\partial X}X^{-\top} = -X^{-\top} \boxtimes X^{-1}$, $\frac{\partial}{\partial X} (AX + XA)^\top = I \boxtimes A + A \boxtimes I$:

\begin{align*}
\frac{\partial^2 \log f_t(X; n,p,V)}{\partial^2 X} &= \frac{\partial}{\partial X} \left[(n-p)X^{-\top} - \frac{(V^{-1}X + XV^{-1})^\top}{2}\right] \\
&= -(n-p) (X^{-\top} \boxtimes X^{-1}) -\frac{1}{2}(I_p \boxtimes V^{-1} + V^{-1} \boxtimes I_p) \\
\end{align*}

Now we would want to multiply with -1 and insert the mode for X. However, this hinges on the problem of actually solving the first derivative for X. 


\end{document}