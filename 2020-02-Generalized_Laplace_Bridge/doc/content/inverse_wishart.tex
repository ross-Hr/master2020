\subsection{Standard Inverse Wishart distribution}

the pdf of the Inverse Wishart is

\begin{subequations}
\begin{align}
\mathcal{IW}_{\mathbf x}({\mathbf x}; {\mathbf \Psi}, \nu) &= \frac{\left|{\mathbf\Psi}\right|^{\nu/2}}{2^{\nu p/2}\Gamma_p(\frac \nu 2)} \left|\mathbf{x}\right|^{-(\nu+p+1)/2} e^{-\frac{1}{2}\operatorname{tr}(\mathbf\Psi\mathbf{x}^{-1})}
\label{eq:inverse_wishart_pdf} \\
&= \exp \left[-(\nu + p + 1)/2 \log(|x|) - \frac{1}{2} \text{tr}(\Psi x^{-1}) + \log(\frac{\left|{\mathbf\Psi}\right|^{\nu/2}}{2^{\nu p/2}\Gamma_p(\frac \nu 2)}) \right]
\end{align}
\end{subequations}

with $h(X) = 1/X^\frac{1}{2}, \phi(X)=(\log(X), X^{-1}), w=(-(\nu+p)/2, \Psi)$ and $Z(n,p,V)=- \log(\frac{\left|{\mathbf\Psi}\right|^{\nu/2}}{2^{\nu p/2}\Gamma_p(\frac \nu 2)})$.

\subsubsection{Laplace Approximation of the standard inverse Wishart distribution}

Using $\frac{\partial \det(X)}{\partial X} = \det(X)(X^{-1})^\top$ and $\frac{\partial}{\partial X} \operatorname{tr}(AX)^{-1} = (X^{-1}AX^{-1})^\top$ we can calculate the mode by setting the first derivative of the log-pdf to zero:

\begin{subequations}
\begin{align}
	\frac{\partial \log \mathcal{IW}_{\mathbf X}({\mathbf X}; {\mathbf \Psi}, \nu)}{\partial X} &= \frac{-(\nu + p + 1) \det(X) X^{-\top}}{2\det(X)} + \frac{(X^{-1} \Psi X^{-1})^\top}{2} \\
	&=\frac{-(\nu + p + 1) X^{-\top}}{2} + \frac{(X^{-1} \Psi X^{-1})^\top}{2} \\
	\Rightarrow 0 &=\frac{-(\nu + p + 1) X^{-\top}}{2} + \frac{(X^{-1} \Psi X^{-1})^\top}{2} \\
	\Leftrightarrow (\nu + p + 1) X^{-1} &= X^{-1} \Psi X^{-1} \\
	\Leftrightarrow (\nu + p + 1) &= X^{-1} \Psi\\
	\Leftrightarrow X &= \frac{1}{\nu + p + 1} \Psi
\end{align}
\end{subequations}

Using 

\begin{align}
	\frac{\partial(XBX)_{kl}}{\partial X_{ij}} &= \delta_{ki}(BX)_{lj} + \delta_{lj}(XB)_{ki} \\
	\frac{\partial X^{-1}}{\partial X} &= -(X^{-1} \otimes X^{-1}) \\
	\frac{\partial (X^{-1}B X^{-1})}{\partial X} &= \frac{\partial (X^{-1}B X^{-1})}{\partial X^{-1}} \frac{\partial X^{-1}}{\partial X} = -[\delta_{ki}(BX^{-1})_{lj} + \delta_{lj}(X^{-1}B)_{ki}] (X^{-1} \otimes X^{-1})\\
	(AB)^{-1} &= B^{-1} A^{-1}
\end{align}

we can get the covariance matrix by inverting the Hessian and multiplying with -1. 

\begin{subequations}
\begin{align}
		\frac{\partial^2 \log f_{\mathbf X}({\mathbf X}; {\mathbf \Psi}, \nu)}{\partial^2 X} &= \frac{(\nu + p + 1)}{2}(X^{-1} \otimes X^{-1})^\top - \frac{[\delta_{ki}(\Psi X^{-1})_{lj} + \delta_{lj}(X^{-1}\Psi)_{ki}]}{2} (X^{-1} \otimes X^{-1})^\top \\
		&= \left\{\frac{(\nu + p + 1)}{2} I_{n^2}- \frac{[\delta_{ki}(\Psi X^{-1})_{lj} + \delta_{lj}(X^{-1}\Psi)_{ki}]}{2}\right\} (X^{-1} \otimes X^{-1})^\top \\
		&\overset{\text{insert mode}}{=} \left\{\frac{(\nu + p + 1)}{2} I_{n^2}- \frac{[\delta_{ki}((\nu + p + 1)\Psi \Psi^{-1})_{lj} + \delta_{lj}((\nu + p + 1)\Psi^{-1}\Psi)_{ki}]}{2}\right\} (\nu + p + 1)^2(\psi \otimes \psi)^{-\top} \\
		&= \left\{\frac{(\nu + p + 1)}{2} I_{n^2}- \underbrace{\frac{[\delta_{ki}((\nu + p + 1)I_n)_{lj} + \delta_{lj}((\nu + p + 1)I_n)_{ki}]}{2}}_{=(\nu + p + 1)I_{n^2}}\right\} (\nu + p + 1)^2(\psi \otimes \psi)^{-\top} \\
		&= -\underbrace{\frac{1}{2}(\nu + p + 1) I_{n^2}}_{A} \underbrace{(\nu + p + 1)^2(\psi \otimes \psi)^{-\top}}_{B} \\
		&\overset{\text{invert}}{\Rightarrow} -\frac{1}{(\nu + p + 1)^2}(\psi \otimes \psi)^{\top} \frac{2}{(\nu + p + 1)}I_{n^2} \\
		&=\overset{\cdot -1}{\Rightarrow} \Sigma = \frac{2}{(\nu + p + 1)^3}(\Psi \otimes \Psi)^\top
\end{align}
\end{subequations}

where $I_{n}$ and $I_n^2$ are the identity matrix of size $n$ and $n^2$ respectively. We can also ignore the transpose since we are dealing with symmetric positive definite matrices when it comes to the inverse Wishart distribution. This yields a multivariate Normal of the form $\mathcal{N}\left(X; \mu = \frac{1}{\nu + p + 1} \Psi, \Sigma = \frac{2}{(\nu + p + 1)^3}(\Psi \otimes \Psi)^\top\right)$. 

\subsection{Logm-Transformed inverse Wishart distribution}

we transform the distribution with $g(X) = \text{logm}(X)$, i.e. $X(Y) = g^{-1}(X) = \text{expm}(Y)$, where $\operatorname{expm}(Y)$ is the matrix exponential. The new pdf becomes 

\begin{subequations}
\begin{align}
	\mathcal{IW}_{logm}({\mathbf Y}; {\mathbf \Psi}, \nu) &= \frac{\left|{\mathbf\Psi}\right|^{\nu/2}}{2^{\nu p/2}\Gamma_p(\frac \nu 2)} \left|\operatorname{expm}\mathbf{Y}\right|^{-(\nu+p+1)/2} e^{-\frac{1}{2}\operatorname{tr}(\mathbf\Psi(\operatorname{expm}\mathbf{Y})^{-1})} \cdot |\operatorname{expm}(\mathbf{Y})|\\
	&=  \frac{\left|{\mathbf\Psi}\right|^{\nu/2}}{2^{\nu p/2}\Gamma_p(\frac \nu 2)} \left|\operatorname{expm}\mathbf{Y}\right|^{-(\nu+p-1)/2} e^{-\frac{1}{2}\operatorname{tr}(\mathbf\Psi(\operatorname{expm}\mathbf{Y})^{-1})} \\
	&= \exp\left[C - (\nu+p-1)/2\log  \left|\operatorname{expm}\mathbf{Y}\right| - -\frac{1}{2}\operatorname{tr}(\mathbf\Psi\operatorname{expm}(\mathbf{-Y}))\right]
\end{align}
\end{subequations}

with exponential family statistics $h(Y) = Y^\frac{1}{2}, \phi(Y) = (\log(\det(\operatorname{expm}(Y))), \operatorname{expm}(Y)), w = ((\nu + p)/2, \Psi)$ and $Z(n,p,V)=- \log\left(\frac{\left|{\mathbf\Psi}\right|^{\nu/2}}{2^{\nu p/2}\Gamma_p(\frac \nu 2)}\right)$

\subsubsection{Laplace Approximation of the logm-transformed inverse Wishart distribution}

For the first derivative we use the same ideas as for the Wishart (TODO:link). 

\begin{subequations}
\begin{align}
	\frac{\partial \log \mathcal{IW}_{logm}}{\partial Y} &= \frac{\partial}{\partial Y} -(\nu+p-1)/2\log  \left|\operatorname{expm}\mathbf{Y}\right| - -\frac{1}{2}\operatorname{tr}(\mathbf\Psi\operatorname{expm}(\mathbf{-Y}))\\
	&= -\frac{(\nu+p-1)}{2} + \frac{1}{2}\mathbf\Psi\operatorname{expm}(\mathbf{-Y})
\end{align}
\end{subequations}

which yields the mode by setting it to zero and solving for $Y$.

\begin{subequations}
\begin{align}
	0 &= -\frac{(\nu+p-1)}{2} + \frac{1}{2}\mathbf\Psi\operatorname{expm}(\mathbf{-Y}) \\
	\Leftrightarrow (\nu+p-1) I_p &=  \mathbf\Psi(\operatorname{expm}(\mathbf{Y}))^{-1} \\
	\Leftrightarrow Y &= \operatorname{logm}\left(\frac{\mathbf\Psi}{n+p-1}\right)
\end{align}
\end{subequations}

For the second derivative we also use the same ideas as for the Wishart (TODO: link + rewrite). 

\begin{subequations}
\begin{align}
	\frac{\partial^2 \log \mathcal{IW}_{logm}}{\partial^2 Y} &= \frac{\partial }{\partial Y} \left[-\frac{(\nu+p-1)}{2} + \frac{1}{2}\mathbf\Psi\operatorname{expm}(\mathbf{-Y}) \right] \\
	&= -\frac{1}{2}\left[\mathbf\Psi\operatorname{expm}(\mathbf{Y})^{-1} \otimes I_p\right] \\
	&\overset{\text{mode}}{\Rightarrow} -\frac{1}{2}\left[\mathbf\Psi\frac{\mathbf\Psi^{-1}}{(\nu+p-1)} \otimes I_p\right] \\
	&= -\frac{1}{2(\nu+p-1)} I_{p\times p} \\
	\Rightarrow \Sigma &= 2(\nu+p-1) I_{p\times p}
\end{align}
\end{subequations}

This yields a multivariate Normal distribution $\mathcal{N}\left(Y; \mu = \operatorname{logm}\left(\frac{\mathbf\Psi}{n+p-1}\right), \Sigma = 2(\nu+p-1) I_{p\times p}\right)$. 

\subsubsection{The Bridge for the logm-transformed inverse Wishart distribution}

We get $\mu$ and $\Psi$ from the Laplace approximation and determine $V$ by inverting $\mu$

\begin{subequations}
\begin{align}
	\mu = \operatorname{logm}\left(\frac{\mathbf\Psi}{n+p-1}\right) \Leftrightarrow \operatorname{expm}(\mu) = \frac{\mathbf\Psi}{n+p-1} \Leftrightarrow \mathbf\Psi = \operatorname{expm}(\mu)(n+p-1)
\end{align}
\end{subequations}

In summary: 

\begin{subequations}
\begin{align}
	\mu &= \operatorname{logm}\left(\frac{\mathbf\Psi}{n+p-1}\right) \\
	\Sigma &= 2(\nu+p-1) I_{p\times p} \\
	\mathbf{\Psi} &=  (\nu+p-1)\operatorname{expm}(\mu) \\
	\nu &= ?
\end{align}
\end{subequations}

where $\mathbf{\Psi}$ is reshaped to a $p \times p$ matrix. 

\subsection{Sqrtm-Transformed inverse Wishart distribution}

we transform the distribution with $g(X) = \text{sqrtm}(X) = X^{\frac{1}{2}}$, i.e. $X(Y) = g^{-1}(X) = Y^2$, where $\text{sqrtm}(Y)$ is the square root of the matrix. The new pdf becomes 

\begin{subequations}
\begin{align}
	\mathcal{IW}_{\operatorname{sqrtm}}({\mathbf Y}; {\mathbf \Psi}, \nu) &= \frac{\left|{\mathbf\Psi}\right|^{\nu/2}}{2^{\nu p/2}\Gamma_p(\frac \nu 2)} \left|\mathbf{Y^2}\right|^{-(\nu+p+1)/2} e^{-\frac{1}{2}\operatorname{tr}(\mathbf\Psi\mathbf{Y}^{-2})} |2Y| \\
	&= \frac{\left|{\mathbf\Psi}\right|^{\nu/2}}{2^{\nu p/2}\Gamma_p(\frac \nu 2)} \left|\mathbf{Y}\right|^{-(\nu+p+1)} e^{-\frac{1}{2}\operatorname{tr}(\mathbf\Psi\mathbf{Y}^{-2})} 2^p|Y| \\
	&= \frac{\left|{\mathbf\Psi}\right|^{\nu/2}}{2^{\nu p/2}\Gamma_p(\frac \nu 2)} \left|\mathbf{Y}\right|^{-(\nu+p)} e^{-\frac{1}{2}\operatorname{tr}(\mathbf\Psi\mathbf{Y}^{-2})} 2^p \\ 
	&= \exp\left[-(\nu + p) \log(|Y|) - \frac{1}{2}\text{tr}(\Psi Y^{-2}) + \log(C)\right]
\end{align}
\end{subequations}

with $h(Y) = 1, \phi(Y) = (\log(|Y|), Y^{-2}), w = (-(\nu + p), \Psi)$ and $Z(w) = \log\left( \frac{\left|{\mathbf\Psi}\right|^{\nu/2}}{2^{\nu p/2}\Gamma_p(\frac \nu 2)}\right)$. 

\subsubsection{Laplace Approximation of the sqrtm-transformed inverse Wishart distribution}

Using 

\begin{align}
	\frac{\partial \operatorname{tr}(\Psi Y^{-2})}{\partial Y} &= \frac{1}{2}[Y^{-1}\Psi Y^{-2} + Y^{-2}\Psi Y^{-1}]^\top 
\end{align}

we can calculate the mode by setting the derivative of the log-pdf to zero:

\begin{subequations}
\begin{align}
	\frac{\partial \log \mathcal{IW}_{\text{sqrtm}}(Y, \Psi, \nu)}{\partial Y} &= \frac{-(\nu + p )\det(Y) Y^{-\top}}{\det(Y)} + \frac{1}{2}[Y^{-1}\Psi Y^{-2} + Y^{-2}\Psi Y^{-1}]^\top \\
	&= -(\nu + p)Y^{-\top} + \frac{1}{2}[Y^{-1}\Psi Y^{-2} + Y^{-2}\Psi Y^{-1}]^\top \\
	\Rightarrow (\nu + p)Y^{-1} &= \frac{1}{2}[Y^{-1}\Psi Y^{-2} + Y^{-2}\Psi Y^{-1}] \\
	\Rightarrow 2(\nu + p)I_p &= [Y^{-1}\Psi Y^{-1} + Y^{-2}\Psi] \\
	\Leftrightarrow Y = ?
\end{align}
\end{subequations}

Using

\begin{align}
	\frac{\partial(XVXX)_{kl}}{\partial X_{ij}} &= \frac{\partial}{\partial X_{ij}} \sum_{u_1, u_2, u_3} X_{k,u_1} V_{u_1, u_2} X_{u_2, u_3} X_{u_3, l} \\
	&= \delta_{i,k} \delta_{j,u_1} V_{u_1, u_2} X_{u_2, u_3} X_{u_3, l} \\
	&+ X_{k,u_1} V_{u_1, u_2} \delta_{i,u_2} \delta_{j,u_3}X_{u_3, l} \\
	&+ X_{k,u_1} V_{u_1, u_2} X_{u_2, u_3}\delta_{i,u_3} \delta_{j,l} \\
	&= I_{k,i} \cdot (VXX)_{j,l} + (XV)_{k,i} \cdot (X)_{j,l} + (XVX)_{k,i} \cdot I_{j,l} \\
	\Rightarrow \frac{\partial XVXX}{\partial X} &=  I \otimes VXX + XV \otimes X + XVX \otimes I \\
	\frac{\partial (X^{-1} V X^{-1} X^{-1})^\top}{\partial X} &= -[I \boxtimes VX^{-1}X^{-1} + (X^{-1}V)^\top \boxtimes X^{-1} + (X^{-1}VX^{-1})^\top \boxtimes I]^T (X^{-1} \boxtimes X^{-\top}) \\
	\frac{\partial (X^{-1} X^{-1} V X^{-1})^\top}{\partial X} &= -[I \boxtimes X^{-1}VX^{-1} + X^{-^\top} \boxtimes VX^{-1} + (X^{-1}X^{-1}V)^\top \boxtimes I]^T (X^{-1} \boxtimes X^{-\top})
\end{align}

With these rules we can calculate the Hessian

\begin{subequations}
\begin{align}
	\frac{\partial^2 \log \mathcal{IW}_{\text{sqrtm}}(Y, \Psi, \nu)}{\partial^2 Y} &= \frac{\partial}{\partial Y} -(\nu + p)Y^{-\top} + \frac{1}{2}[Y^{-1}\Psi Y^{-2} + Y^{-2}\Psi Y^{-1}]^\top \\ 
	&\overset{\text{symmetry}}{=}\frac{\partial}{\partial Y} -(\nu + p)Y^{-1} + \frac{1}{2}[Y^{-1}\Psi Y^{-2} + Y^{-2}\Psi Y^{-1}]^\top \\
	&= (\nu + p)(Y^{-1} \otimes Y^{-\top}) \\
	&- \frac{1}{2}\{[I \boxtimes \Psi Y^{-1}Y^{-1} + (Y^{-1}\Psi)^\top \boxtimes Y^{-1} + (Y^{-1}\Psi Y^{-1})^\top \boxtimes I]^T (Y^{-1} \boxtimes Y^{-\top}) \\
	&+ [I \boxtimes Y^{-1}\Psi Y^{-1} + Y^{-^\top} \boxtimes \Psi Y^{-1} + (Y^{-1}Y^{-1}\Psi)^\top \boxtimes I]^T (Y^{-1} \boxtimes Y^{-\top})\} \\
\end{align}
\end{subequations}

However, since we don't know the solution to the mode this result is of limited value to us. 

However, if we choose the the approximation 
\begin{align}
\frac{\partial \operatorname{tr}(\Psi Y^{-2})}{\partial Y} &= \frac{1}{2}[Y^{-1}\Psi Y^{-2} + Y^{-2}\Psi Y^{-1}]^\top \approx 2\Psi X^{-3}
\end{align}
we can find explicit formulations of the mode and covariance matrix. The mode is

\begin{subequations}
\begin{align}
\frac{\partial \log \mathcal{IW}'_{\text{sqrtm}}(Y, \Psi, \nu)}{\partial Y} &= \frac{-(\nu + p )\det(Y) Y^{-\top}}{\det(Y)} + \frac{1}{2} 2 \Psi Y^{-3}
&= -(\nu + p)Y^{-\top} + \Psi Y^{-3} \\
\Rightarrow (\nu + p)Y^{-1} &= \Psi Y^{-3} \\
\Leftrightarrow (\nu + p)\Psi^{-1} &= Y^{-2} \\
\Leftrightarrow Y &= \operatorname{sqrtm}\left(\frac{1}{(\nu+p)} \Psi^{-1}\right)
\end{align}
\end{subequations}

It also means we can compute the Hessian and insert the mode

\begin{subequations}
\begin{align}
	\frac{\partial^2 \log \mathcal{IW}_{\text{sqrtm}}(Y, \Psi, \nu)}{\partial^2 Y} &= \frac{\partial}{\partial Y} -(\nu + p)Y^{-\top} + (\Psi Y^{-3})^\top \\ 
	&= \frac{\partial}{\partial Y} -(\nu + p)Y^{-\top} + (Y^{-\top}Y^{-\top}Y^{-\top}\Psi^\top Y^{-3}) \\
	&\overset{\text{symmetry}}{=}\frac{\partial}{\partial Y} -(\nu + p)Y^{-1} + Y^{-3} \Psi \\
	&= (\nu + p)(X^{-1} \otimes X^{-1}) - [I_p \otimes X^{-2}\Psi + X^{-1} \otimes X^{-1}\Psi + X^{-2} \otimes V](X^{-1} \otimes X^{-1}) \\
	&= (\nu + p)(X^{-1} \otimes X^{-1}) - [X^{-1} \otimes X^{-3}\Psi + X^{-2} \otimes X^{-2}V + X^{-3} \otimes VX^{-1}]\\
	&\overset{\text{mode}}{\Rightarrow} (\nu + p)^2 (\Psi^{-\frac{1}{2}} \otimes \Psi^{-\frac{1}{2}}) - (\nu+p)^2[\Psi^{-\frac{1}{2}} \otimes \Psi^{-\frac{1}{2}} + \Psi^{-1} \otimes I_p + \Psi^{-\frac{3}{2}} \otimes \Psi^{\frac{1}{2}}] \\
	&= - (\nu+p)^2[\Psi^{-1} \otimes I_p + \Psi^{-\frac{3}{2}} \otimes \Psi^{\frac{1}{2}}] \\
	&= - (\nu+p)^2[(\Psi^{-\frac{3}{2}} \otimes \Psi^{\frac{1}{2}})(\Psi^{\frac{1}{2}} \otimes \Psi^{\frac{1}{2}} + I_{p^2})] \\
	\Rightarrow \Sigma &= 1/(\nu+p)^2 [(\Psi^{-\frac{3}{2}} \otimes \Psi^{\frac{1}{2}})(\Psi^{\frac{1}{2}} \otimes \Psi^{\frac{1}{2}} + I_{p^2})]^{-1}
\end{align}
\end{subequations}

which could be inverted more easily using equation 5 of \url{http://papers.nips.cc/paper/4281-efficient-inference-in-matrix-variate-gaussian-models-with-iid-observation-noise.pdf}

\subsubsection{The Bridge for the sqrtm-transformed inverse Wishart distribution}

TODO
